\documentclass[a4paper,
fontsize=11pt,
%headings=small,
oneside,
numbers=noperiodatend,
parskip=half-,
bibliography=totoc,
final
]{scrartcl}

\usepackage{synttree}
\usepackage{graphicx}
\setkeys{Gin}{width=.4\textwidth} %default pics size

\graphicspath{{./plots/}}
\usepackage[ngerman]{babel}
\usepackage[T1]{fontenc}
%\usepackage{amsmath}
\usepackage[utf8x]{inputenc}
\usepackage [hyphens]{url}
\usepackage{booktabs} 
\usepackage[left=2.4cm,right=2.4cm,top=2.3cm,bottom=2cm,includeheadfoot]{geometry}
\usepackage{eurosym}
\usepackage{multirow}
\usepackage[ngerman]{varioref}
\setcapindent{1em}
\renewcommand{\labelitemi}{--}
\usepackage{paralist}
\usepackage{pdfpages}
\usepackage{lscape}
\usepackage{float}
\usepackage{acronym}
\usepackage{eurosym}
\usepackage[babel]{csquotes}
\usepackage{longtable,lscape}
\usepackage{mathpazo}
\usepackage[normalem]{ulem} %emphasize weiterhin kursiv
\usepackage[flushmargin,ragged]{footmisc} % left align footnote
\usepackage{ccicons} 

%%%% fancy LIBREAS URL color 
\usepackage{xcolor}
\definecolor{libreas}{RGB}{112,0,0}

\usepackage{listings}

\urlstyle{same}  % don't use monospace font for urls

\usepackage[fleqn]{amsmath}

%adjust fontsize for part

\usepackage{sectsty}
\partfont{\large}

%Das BibTeX-Zeichen mit \BibTeX setzen:
\def\symbol#1{\char #1\relax}
\def\bsl{{\tt\symbol{'134}}}
\def\BibTeX{{\rm B\kern-.05em{\sc i\kern-.025em b}\kern-.08em
    T\kern-.1667em\lower.7ex\hbox{E}\kern-.125emX}}

\usepackage{fancyhdr}
\fancyhf{}
\pagestyle{fancyplain}
\fancyhead[R]{\thepage}

% make sure bookmarks are created eventough sections are not numbered!
% uncommend if sections are numbered (bookmarks created by default)
\makeatletter
\renewcommand\@seccntformat[1]{}
\makeatother


\usepackage{hyperxmp}
\usepackage[colorlinks, linkcolor=black,citecolor=black, urlcolor=libreas,
breaklinks= true,bookmarks=true,bookmarksopen=true]{hyperref}
%URLs hart brechen
\makeatletter 
\g@addto@macro\UrlBreaks{ 
  \do\a\do\b\do\c\do\d\do\e\do\f\do\g\do\h\do\i\do\j 
  \do\k\do\l\do\m\do\n\do\o\do\p\do\q\do\r\do\s\do\t 
  \do\u\do\v\do\w\do\x\do\y\do\z\do\&\do\1\do\2\do\3 
  \do\4\do\5\do\6\do\7\do\8\do\9\do\0} 
% \def\do@url@hyp{\do\-} 
\makeatother 

%meta
%meta

\fancyhead[L]{U. Herb \\ %author
LIBREAS. Library Ideas, 32 (2017). % journal, issue, volume.
\href{http://nbn-resolving.de/}
{}} % urn 
% recommended use
%\href{http://nbn-resolving.de/}{\color{black}{urn:nbn:de...}}
\fancyhead[R]{\thepage} %page number
\fancyfoot[L] {\ccLogo \ccAttribution\ \href{https://creativecommons.org/licenses/by/3.0/}{\color{black}Creative Commons BY 3.0}}  %licence
\fancyfoot[R] {ISSN: 1860-7950}

\title{\LARGE{Ist Open Access an ein Ende gelangt? \\ Ein Interview}} % title
\author{Ulrich Herb} % author

\setcounter{page}{1}

\hypersetup{%
      pdftitle={Ist Open Access an ein Ende gelangt? \\ Ein Interview},
      pdfauthor={Ulrich Herb},
      pdfcopyright={CC BY 3.0 Unported},
      pdfsubject={LIBREAS. Library Ideas, 32 (2017).},
      pdfkeywords={Open Access, OA Gold, Transformation},
      pdflicenseurl={https://creativecommons.org/licenses/by/3.0/},
      pdfcontacturl={http://libreas.eu},
      baseurl={http://libreas.eu},
      pdflang={de},
      pdfmetalang={de}
     }



\date{}
\begin{document}

\maketitle
\thispagestyle{fancyplain} 

%abstracts

%body
\emph{Fragen für LIBREAS von Karsten Schuldt und Linda Freyberg}

Ulrich Herb beschäftigt sich als, in einer Bibliothek tätiger,
Informationswissenschaftler seit Jahren praktisch und forschend mit Open
Access. Seine Dissertation lieferte einen Überblick zum tatsächlichen
Handeln von Forschenden eines Felder, der deutschsprachigen Soziologie,
in Bezug auf Open Access (Herb 2015). Zudem hat er als langer Begleiter
der Bewegung erst kürzlich eine Kritik der Entwicklungen der letzten
Jahre vorgelegt (Herb 2017). Das folgende Interview mit LIBREAS. Library
Ideas schließt an diese Äußerungen an.

\emph{\textbf{LIBREAS:}Wenn man die Effekte und Entwicklungen im Open
Access-Feld in den letzten Jahren betrachtet -- insbesondere die
Durchsetzung von Gold-OA als Publikationsstandard für
Zeitschriftenaufsätze -- erscheint es dann nicht so, als wäre das
eigentliche Transformationspotential von Open Access bestenfalls in
verwässerter Form spürbar? Haben wir heute nicht ganz ähnliche
Abhängigkeiten und Probleme wie in der Wissenschaftskommunikation vor
Open Access?}

\textbf{Herb:}Meiner Meinung nach kannte Open Access in seinen
Kindertagen zwei prägende Elemente: Zum einen sollte er Wissenschaft
transparent und überprüfbar machen, beschleunigen und wissenschaftliches
Arbeiten erleichtern. Dieses Versprechen hält er, zumal er ein
quantitativer Erfolg ist: Sein Anteil an der Gesamtproduktion
wissenschaftlicher Texte wächst. Das andere Element war eher
sozial-romantisch oder sozial-utopisch. Die Budapest Open Access
Initiative (BOAI) formulierte euphorisch, dass Open Access das Teilen
von Wissen zwischen den Armen und Reichen ermöglichen werde und dass er
der Grundstein für die Vereinigung der Menschheit in einer gemeinsamen
intellektuellen Konversation und bei der Suche nach Wissen sei. Man
dachte diese gemeinsame intellektuelle Konversation benötige die
kommerziellen Verlage nicht, rein infrastrukturell, weil man die
Distribution via Internet als Wissenschaftler oder
non-profit-Einrichtung selbst in die Hand nehmen kann, rein produktiv,
weil der Content bekanntermaßen von Wissenschaftlern erstellt, editiert
und qualitativ geprüft wird. So erklärt sich auch, dass die BOAI jedes
kommerzielle Element verneint. Der Begriff \emph{publisher} findet nur
einmal in ihrer Erklärung Verwendung.

Die Regeln und Strukturen in der internen Wissenschaftskommunikation,
die vor allem durch wissenschaftliche Publikationen und dabei in vielen
Fächern vorrangig in Journalen erfolgt, sind aber eben persistent. Zu
diesen Regeln gehört, dass in der Wissenschaft zwei Arten Kapital
zirkulieren: ökonomisches und soziales. Die Crux bei der Finanzierung
wissenschaftlicher Publikationen ist leider, dass die öffentliche Hand,
genauer die Bibliotheken, das ökonomische Kapital verteilen, die Verlage
jedoch das soziale, das für das Renommee eines Wissenschaftlers
ausschlaggebend ist. Dazu profitieren auch Bibliotheken und
Universitäten von diesem sozialen Kapital, das die Verlage vermitteln:
Auf der einen Seite kann man damit protzen, dass man sich die
hochpreisigen High-Impact-Journale leisten kann, auf der anderen Seite
kann man damit prahlen, dass man obszön hohe APCs für
Open-Access-Artikel zahlen kann -- beides macht einen Standort natürlich
attraktiv und exzellent. Die Klagen der Bibliotheken, dass
Wissenschaftler beim Publizieren nur auf Impact und Renommee achten,
sind daher ein wenig halbseiden, nutzen sie dieses Renommee doch auch
für die eigene Stilisierung, andererseits können sie sich schwer aus
ihrer Rolle befreien.

Langer Rede kurzer Sinn: Ich stimme zu, es hat sich nicht so viel
geändert -- abgesehen davon, dass es mehr Open Access als früher gibt.
Der Grund liegt darin, dass die das Renommee oder soziale Kapital
verleihenden Marken in den Händen der großen Publisher sind. Es ist eine
Binse unter Soziologen: Soziales Kapital schlägt ökonomisches, denn es
lässt sich recht einfach in ökonomisches ummünzen, der umgekehrte
Prozess hingegen ist sehr aufwändig, sein Erfolg ungewiss.

\emph{\textbf{LIBREAS:}In Ihrem Artikel \enquote{Open Access zwischen
Revolution und Goldesel} (Herb 2017) zeichnen Sie ein eher negatives
Bild. Die einzigen Akteure im Wissenschaftssystem, die mit Open Access
gewonnen hätten, wären die Verlage. Dabei seien am Anfang der
Diskussionen vor allem Forschende, Forschungsfördereinrichtungen und
Bibliotheken aktiv gewesen. Sehen Sie heute noch Möglichkeiten, diesen
Trend umzudrehen?}

\textbf{Herb:}Das ist schwierig, aber vielleicht bin ich in Sachen Open
Access desillusioniert. Entwicklungen verlaufen in aller Regel
pfadabhängig. Bis in die 1960er Jahre wurden wissenschaftliche Journale
meist noch in Eigeninitiative der Wissenschaftler verlegt, die Expansion
der Wissenschaft und die rapide steigende Zahl der Wissenschaftler
überforderte dieses System aber. Die Wissenschaftler wollen forschen und
keine Distribuenten oder Verleger sein -- das ist etwas, was die
Open-Access-Befürworter oft vergessen: Also begann das Outsourcing der
Journale zu kommerziellen Verlagen. Seither, besonders aber seit Beginn
der Digitalisierung, hat sich der Konzentrationsprozess am
Publikationsmarkt enorm beschleunigt und auch im Open Access halten,
gemessen an den publizierten Artikeln, die großen drei Verlage,
Elsevier, Springer Nature und Wiley, die Hälfte des Volumens. Ich sehe
keine Tendenz zu gravierenden Veränderungen. Ich freue mich über jeden
kleinen Sieg des nicht-kommerziellen Open Access, zum Beispiel wenn die
Herausgeber des Closed-Access-Journals Lingua ihre Tätigkeiten
niederlegen und das Open-Access-Journal Glossa gründen. Gemessen an
nationalen Open-Access-Konsortien, die mit Elsevier, Springer Nature und
Wiley getroffen oder geplant werden, ist dies natürlich nur eine
Fußnote.

\emph{\textbf{LIBREAS:}Insbesondere interessant scheint ja, dass
Forschende -- für die Open Access ja eigentlich die meisten Vorteile
bringen soll -- wenig aktiv scheinen. Wurden deren Interessen die ganze
Zeit falsch eingeschätzt?}

\textbf{Herb:}Dem stimme ich leider zu, aus mehreren Gründen: Zum einen
aus der oben beschriebenen Bedeutung des sozialen Kapitals oder
Renommees. Eine andere Soziologen-Binse ist, dass eine Technik nie ein
soziales Problem löst. Ob man die Bindung der Wissenschaftler an die
Verlage als Problem ansieht, muss jeder wissen, ich formuliere es so:
Open Access vertraute sehr darauf, dass eine Technik, das Internet,
diese Bindung auflöst -- das funktioniert jedoch nicht und man hätte es
ahnen können. Zum anderen sehen sich die Wissenschaftler nicht in der
Rolle von Verlegern, sondern ziehen es vor, bestehende
Publikationsoptionen zu nutzen, die bieten ihnen nunmal meist die
kommerziellen Verlage. Man steht hinter dem Motto \emph{science as a
public good}, man freut sich auch über höhere Zitationsraten und liebt
offene Kommunikation.

Allerdings liegt in der anderen Waagschale die Reputation des Verlages
oder Journals, man sollte diesen Faktor nicht abtun: An deutschen
Hochschulen sitzen 90\,\% der Nicht-Professoren auf befristeten Stellen
und leben prekär, sie arbeiten oft 100\,\% auf einer 50\,\% Stelle,
unbezahlte Lehraufträge sind keine Seltenheit. Das Publizieren von
Artikeln bei Verlagen, die der Open-Access-Community unsympathisch sind,
kann diesen Leuten das finanzielle Überleben sichern, eine
Open-Access-Publikation dürfte für manchen von ihnen Luxus sein, selbst
wenn sie keine APC kostet. Und natürlich freuen sich die Wissenschaftler
am meisten, wenn sie alles bekommen: Open Access, science as a public
good, höhere Zitationsraten, offene Kommunikation \emph{und} Reputation
-- ohne selbst als Verleger tätig sein zu müssen. Dieses
All-Inclusive-Paket bekommen sie am einfachsten bei bekannten,
kommerziellen Verlagen.

\emph{\textbf{LIBREAS:}Sie stellen die nationalen Vereinbarungen zu Open
Access, die in der Literatur gemeinhin positiv bewertet werden, als vor
allem für die großen Verlage vorteilhaft dar. Würde Sie stattdessen für
etwas anderes plädieren? Wären die mit diesen Vereinbarungen
geschaffenen Strukturen nicht auch ein Ansatzpunkt, später eine andere
Form von Open Access durchzusetzen?}

\textbf{Herb:}Ich hoffe ausdrücklich, dass dies gelingt und wünsche mir
sehr, dass zum Beispiel DEAL seine Ziele erreicht. Dennoch bin ich
skeptisch, denn ich sehe hier keine neuen Strukturen. Die öffentliche
Hand zahlt bald vielleicht anstelle vieler großer Rechnungen eine
\emph{sehr} große Rechnung an einen Verlag, das ist kein struktureller
Wandel. Die Marke, das woran sich Wissenschaftler orientieren, wenn sie
entscheiden, in welchem Journal sie publizieren, verbleibt bei den
kommerziellen Verlagen. Also werden die Wissenschaftler weiter bei den
Journalen dieser Verlage publizieren, zumal sie ja keine APCs zahlen
müssen. Warum sollte ich als Wissenschaftler in einer Nation, die ein
landesweites Konsortium inklusive Open-Access-Klausel etwa mit Wiley
hat, bei einem wissenschaftlich guten, nicht-kommerziellen
Open-Access-Journal mit geringer Bekanntheit -- noch schlimmer: geringen
Zitationszahlen -- publizieren, wenn ich auch ohne Unkosten bei einem
Wiley-Journal veröffentlichen kann, dem meine Community zu Füßen liegt
und das seit 30~Jahren Zitationszahlen sammelt? Zumal die Entscheidung
für das nicht-kommerzielle Journal mich schlimmstenfalls um den nächsten
Zwei-Jahres-Vertrag in einem Projekt bringt.

Noch eins: Warum sollte ich als Herausgeberschaft eines, sagen wir,
Springer-Nature-Journals den Schritt der Lingua-Herausgeber gehen und
ein neues Open-Access-Journal gründen, wenn ich damit rechnen kann, dass
über kurz oder lang nationale Konsortien mit Open-Access-Klauseln Usus
sind und 90\,\% der Artikel oder mehr in meinem Journal Open Access
erscheinen? Dabei ist die Argumentation der Open-Access-Vertreter
durchaus heuchlerisch: Einerseits bedrängt man Herausgeber, dem
Lingua-Glossa-Beispiel zu folgen, andererseits fördert man die globale
Transformation der Closed-Access-Journale zu kommerziellen
Open-Access-Journalen. Als Herausgeber fühlte ich mich sehr veralbert,
wenn ich mein anerkanntes Closed-Access-Journal als Open-Access-Journal
neu gründe, dessen Impact Factor Score bei 0,0 beginnt, ich riesige Mühe
habe, Autoren zu akquirieren, mich um Submission-System,
Redaktionsworkflow und vieles mehr kümmern muss, nach drei Jahren einen
Impact-Factor von 0,2 habe und andere Journale, die einfach ein bisschen
warten, durch die Open-Access-Transformation und letztlich gegen Zahlung
von viel öffentlichem Geld urplötzlich Open Access erscheinen und einen
schönen Impact-Factor von 18,2 haben. Wer bitteschön soll sich auf so
etwas Selbstmörderisches einlassen?

Machen wir es kurz: Ich fürchte die nationalen Konsortien stärken den
kommerziellen Open Access. Warten wir ab.

\emph{\textbf{LIBREAS:}In Ihrer Dissertation (Herb 2015) haben Sie die
Soziologie und die dortigen OA-Publikationsstrukturen untersucht und
kamen zu weniger negativen Einschätzungen. Insbesondere sei die
Publikation als Green-OA verbreitet. Würden Sie die Entwicklungen bei OA
in unterschiedlichen Forschungsfeldern auch unterschiedlich beurteilen?
Welches Feld kommt den ursprünglichen Idealen von OA, wie sie Anfang der
2000er Jahre formuliert wurden, am nächsten?}

\textbf{Herb:}In der Tat, die Soziologie hat mich da recht positiv
überrascht. Man hat hier einen pragmatischen Umgang mit Open Access
entwickelt: Es gibt zwar teils respektable Gold-Open-Access-Journale,
allerdings nimmt der Green Open Access mehr Raum ein, vor allem weil
viele Subskriptionsjournale ihre Hefte mit Embargo selbst Open Access
stellen. Man betreibt Green Open Access als Journal. Man muss dazu
sagen, dass man in der Soziologie, was das Publizieren angeht, noch ein
wenig in der idyllischen Zeit der 1960er lebt. Zumindest im
deutschsprachigen Bereich, wo es noch angesehene Journale gibt, die in
Händen der Community sind. Dazu betreiben Autoren recht rege grünen Open
Access. Begünstigt wird dieses Klima bestimmt auch durch das geringere
ökonomische Potential der Soziologie-Journale, die Marktführer haben
hier nicht die gleiche Dominanz wie in anderen Fächern, kleinere Verlage
mit sehr moderaten Green-Open-Access-Modalitäten sind ebenfalls
bedeutsam. Mein Eindruck ist weiterhin, dass die Soziologen instinktiv
die Vorteile des Open Access erkennen: Die Verbreitung, die offene
Kommunikation, die Umsetzung des Mottos science as a public good. Ihre
Vorbehalte gegen Open Access zielen zudem seltener auf die
Renommee-Frage des Open Access, sie hegen ein anderes Misstrauen und
sehen Open Access als Impuls der Fremdsteuerung von Wissenschaft.

Ich sehe auch die Entwicklungen der Mathematik positiv, hier versuchen
Wissenschaftler das Renommee oder die Marke in die eigene Hand zu
nehmen, etwa mit den Epi Journals oder Discrete Analysis, dazu zum
Beispiel etwa die Sprachwissenschaften, etwa mit Glossa als leuchtendem
Beispiel oder Verlagen wie Language Science Press. Mein Eindruck ist,
dass sich nicht-kommerzieller Open Access in den Fächern besser
entwickelt, in denen nicht sehr viel Geld zirkuliert.

\emph{\textbf{LIBREAS:}Wenn man der Darstellung im oben genannten Text
folgt, haben sich mit Gold-OA und den Vereinbarungen auf nationaler
Ebene Strukturen der Wissenschaftskommunikation etabliert, die wieder
den globalen Norden bevorzugen und den globalen Süden weiter in
Abhängigkeit halten. In dem von Joachim Schöpfel herausgegeben Buch zu
Open Access im globalen Süden (Schöpfel 2015) scheint es aber so, als
wenn in diesem die Forschenden eigene Systeme etabliert haben, die -- im
Gegensatz zum Beispiel zur Situation in Deutschland -- wenig mit den
großen Verlagen zu tun hat. Sehen Sie das auch so? Wäre dieses Vorgehen
nicht ein Vorbild für Forschende in Europa?}

\textbf{Herb:}Bei der Diskussion um den globalen Süden wäre Joachim
sicher ein besserer Ansprechpartner als ich es bin. Allerdings erscheint
Anfang 2018 mit \emph{Open Divide? Critical Studies on Open Access} ein
Sammelband, den Joachim und ich herausgeben. Dort finden sich Artikel zu
diskussionswürdigen Eigenschaften und Entwicklungen des Open Access
allgemein und zu seinen Wirkungen und Funktionen im Nord-Süd-Verhältnis.
Autoren des Buches berichten tatsächlich, dass Open Access im globalen
Süden andere Formen annehmen kann als im Norden und Selbstorganisation
eine große Rolle spielt, man denke an SciELO oder Bioline International.
Ein Grund mag sein, dass der globale Süden für kommerzielle Verlage
wenig lukrativ ist. Das Aufkommen von Open Access im Süden war auch von
etwas anderen Motivationen als im Norden begleitet: In Open Divide
berichtet Leslie Chan, dass sein Interesse an Open Access nicht durch
den fehlenden \emph{Zugang} zu wissenschaftlichen Informationen aus
hochpreisigen Abonnementzeitschriften geweckt wurde, sondern durch das
fehlende \emph{Wissen} über hochwertige Literatur aus dem Süden der
Welt. Die Bemühungen, diesen Mangel zu beheben, führten zur Gründung von
Bioline International, einem Marktführer im Open Access für
biowissenschaftliche Fachzeitschriften, die in Entwicklungsländern
veröffentlicht werden.

Und ja: Genau diese Selbstorganisation täte dem Norden gut. Ich zweifle
leider, ob man sich die Mühen dieser Selbstorganisation dort auflastet
-- zumal die erwähnten All-Inclusive-Deals bereits alles für die
Wissenschaftler organisieren. Man publiziert ohne Mehrkosten in
Springer- oder Elsevier-Zeitschriften, dann ist die Indexierung in den
wichtigsten Fachdatenbanken, Plattformen und Zitationsdatenbanken schon
erledigt -- warum Selbstorganisation, wenn alles organisiert ist?

\emph{\textbf{LIBREAS:}Wie könnte man den gesamten OA-Diskurs wieder auf
ein sachliches Level bringen? Oder werden wir in absehbarer Zeit in
einer antagonistischen Situation zwischen Verlagen auf der einen Seite
und Bibliotheken und Forschungsförderung auf der anderen Seite
verbleiben?}

\textbf{Herb:}Ich weiß gar nicht, ob es einen Open-Access-Diskurs gibt.
Jeder will etwas anderes von Open Access, Bibliotheken wollen Geld
sparen, Verlage wollen Geld einnehmen, Wissenschaftler wollen Jobs und
Karriere, Forschungsförderer wollen Impact. Dazu franst Open Access auch
aus: Früher redeten wir von Gold und Grün, heute zusätzlich von Bronze
und Platinum, dazu vom Ablegen von Dokumenten auf Homepages, in Social
Networks oder von Angeboten wie Springers SharedIt, das es erlaubt, den
reinen Lesezugriff auf Dokumente, ohne Druck- oder Speicher-Option, zu
teilen. Diese Unschärfe spiegelt sich auch in vielen empirischen Studien
zu Open Access, es gibt wohl keine zwei Studien unterschiedlicher
Forscher, die zum Beispiel bei der Messung des Open-Access-Anteils am
Gesamtvolumen wissenschaftlicher Literatur Open Access gleich
definieren. Ist das ein Diskurs, wenn jeder von etwas anderem redet und
von diesen unterschiedlichen Wunschgestalten Unterschiedliches erwartet
wird?

Ich sehe auch keinen so großen Antagonismus, nehmen wir das Beispiel der
landesweiten Open-Access-Konsortien: Lassen die Verlage in den
DEAL-Verhandlungen kurzfristig beim Preis nach und erlauben die
Open-Access-Klauseln, sind doch alle zufrieden, die Förderer, die
Bibliotheken, die Verlage. Alles eine Preisfrage. Zudem ist Open Access
längst im Alltag angekommen und wird nicht mehr von Veränderern
betrieben. Bei den Open-Access-Tagen in Dresden hielt ich einen Vortrag
darüber, wie Open Access sich von seinen Idealen verabschiedet und
kommerzialisiert wird. Eine Kollegin, jünger als ich und noch nicht so
lange im Open Access aktiv wie ich, meinte anschließend, sie kenne diese
idealistische Komponente gar nicht und sie interessiere sie auch nicht.
Open Access ist, das meine ich ohne jede Larmoyanz oder ohne die
Kollegin in ein schlechtes Licht rücken zu wollen, bei den Technokraten
angekommen und die führen keine Diskurse.

\hypertarget{literatur}{%
\section*{Literatur}\label{literatur}}

Herb, Ulrich \& Schöpfel, Joachim (edit.) (2018). Open Divide? Critical
Studies on Open Access. Sacramento, CA : Litwin Books, 2018.

Herb, Ulrich (2017). Open Access zwischen Revolution und Goldesel: Eine
Bilanz fünfzehn Jahre nach der Erklärung der Budapest Open Access
Initiative. In: Information. Wissenschaft \& Praxis, 68 (2017) 1, 1-10.

Herb, Ulrich (2015). Open Science in der Soziologie: Eine
interdisziplinäre Bestandsaufnahme zur offenen Wissenschaft und eine
Untersuchung ihrer Verbreitung in der Soziologie. (Schriften zur
Informationswissenschaft, 67). Glückstadt: Verlag Werner Hülsbusch,
2015.

Schöpfel, Joachim (edit.) (2015). Learning from the BRICS: Open Access
to Scientific Information in Emerging Countries. Sacramento, CA : Litwin
Books, 2015.

%autor
\begin{center}\rule{0.5\linewidth}{\linethickness}\end{center}

\textbf{Dr.~Ulrich Herb}, Soziologe und Informationswissenschaftler,
tätig für die Saarländische Universitäts- und Landesbibliothek und
freiberuflich als Consultant.

\textbf{Karsten Schuldt}, Wissenschaftlicher Mitarbeiter Schweizerisches
Institut für Informationswissenschaft, HTW Chur. Redakteur LIBREAS.
Library Ideas.

\textbf{Linda Freyberg}, Doktorandin am Promotionskolleg Wissenskulturen
/ Digitale Medien der Leuphana Universität Lüneburg, Stipendiatin im
Rahmen des Professorinnenprogrammes am Urban Complexity Lab (FH Potsdam)
und Redakteurin der LIBREAS.Library Ideas.

\end{document}
