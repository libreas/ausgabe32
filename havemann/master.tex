\documentclass[a4paper,
fontsize=11pt,
%headings=small,
oneside,
numbers=noperiodatend,
parskip=half-,
bibliography=totoc,
final
]{scrartcl}

\usepackage{synttree}
\usepackage{graphicx}
\setkeys{Gin}{width=.4\textwidth} %default pics size

\graphicspath{{./plots/}}
\usepackage[ngerman]{babel}
\usepackage[T1]{fontenc}
%\usepackage{amsmath}
\usepackage[utf8x]{inputenc}
\usepackage [hyphens]{url}
\usepackage{booktabs} 
\usepackage[left=2.4cm,right=2.4cm,top=2.3cm,bottom=2cm,includeheadfoot]{geometry}
\usepackage{eurosym}
\usepackage{multirow}
\usepackage[ngerman]{varioref}
\setcapindent{1em}
\renewcommand{\labelitemi}{--}
\usepackage{paralist}
\usepackage{pdfpages}
\usepackage{lscape}
\usepackage{float}
\usepackage{acronym}
\usepackage{eurosym}
\usepackage[babel]{csquotes}
\usepackage{longtable,lscape}
\usepackage{mathpazo}
\usepackage[normalem]{ulem} %emphasize weiterhin kursiv
\usepackage[flushmargin,ragged]{footmisc} % left align footnote
\usepackage{ccicons} 

%%%% fancy LIBREAS URL color 
\usepackage{xcolor}
\definecolor{libreas}{RGB}{112,0,0}

\usepackage{listings}

\urlstyle{same}  % don't use monospace font for urls

\usepackage[fleqn]{amsmath}

%adjust fontsize for part

\usepackage{sectsty}
\partfont{\large}

%Das BibTeX-Zeichen mit \BibTeX setzen:
\def\symbol#1{\char #1\relax}
\def\bsl{{\tt\symbol{'134}}}
\def\BibTeX{{\rm B\kern-.05em{\sc i\kern-.025em b}\kern-.08em
    T\kern-.1667em\lower.7ex\hbox{E}\kern-.125emX}}

\usepackage{fancyhdr}
\fancyhf{}
\pagestyle{fancyplain}
\fancyhead[R]{\thepage}

% make sure bookmarks are created eventough sections are not numbered!
% uncommend if sections are numbered (bookmarks created by default)
\makeatletter
\renewcommand\@seccntformat[1]{}
\makeatother


\usepackage{hyperxmp}
\usepackage[colorlinks, linkcolor=black,citecolor=black, urlcolor=libreas,
breaklinks= true,bookmarks=true,bookmarksopen=true]{hyperref}

%meta
%meta

\fancyhead[L]{F. Havemann \\ %author
LIBREAS. Library Ideas, 32 (2017). % journal, issue, volume.
\href{http://nbn-resolving.de/}
{}} % urn 
% recommended use
%\href{http://nbn-resolving.de/}{\color{black}{urn:nbn:de...}}
\fancyhead[R]{\thepage} %page number
\fancyfoot[L] {\ccLogo \ccAttribution\ \href{https://creativecommons.org/licenses/by/3.0/}{\color{black}Creative Commons BY 3.0}}  %licence
\fancyfoot[R] {ISSN: 1860-7950}

\title{\LARGE{Erst veröffentlichen und diskutieren, dann begutachten lassen!\\ Wie die Wissenschaft mit \emph{Overlay Journals} ihre Kommunikation wieder zurückgewinnen kann
}} % title
\author{Frank Havemann} % author

\setcounter{page}{1}

\hypersetup{%
      pdftitle={Erst veröffentlichen und diskutieren, dann begutachten lassen! Wie die Wissenschaft mit \emph{Overlay Journals} ihre Kommunikation wieder zurückgewinnen kann},
      pdfauthor={Frank Havemann},
      pdfcopyright={CC BY 3.0 Unported},
      pdfsubject={LIBREAS. Library Ideas, 32 (2017).},
      pdfkeywords={Open Access, Overlay Journals, Open Peer Review},
      pdflicenseurl={https://creativecommons.org/licenses/by/3.0/},
      pdfcontacturl={http://libreas.eu},
      baseurl={http://libreas.eu},
      pdflang={de},
      pdfmetalang={de}
     }



\date{}
\begin{document}

\maketitle
\thispagestyle{fancyplain} 

%abstracts

%body
Der erste Teil des Titels meines Aufsatzes ist eine freie Übersetzung
von \enquote{publish and discuss first, referee then}, der kurzen
Formel, auf die Eberhard Hilf das seit langem von ihm favorisierte
Modell für Open Access (OA) zur Forschungsliteratur gebracht hat (Hilf
und Severiens 2013, S. 391). Es geht auf keinen geringeren als Enrico
Fermi zurück, der die Praxis des Versendens von Preprints schon 1932
einführte (Hilf und Severiens 2013, S. 381). Preprints sind Kopien von
Aufsätzen, die noch nicht als Zeitschriftenaufsätze gedruckt worden
sind. In der Elementarteilchenphysik etablierte sich schon im
Papierzeitalter ein System der Verteilung solcher Kopien. Wöchentlich
gab die Bibliothek des CERN, dem großen europäischen Forschungszentrum
in Genf, thematisch geordnete Listen aller an sie eingesandten Preprints
heraus, die interessierte Forscherinnen und Forscher dann von den
Autoren anfordern konnten. Es war unausweichlich, dass im
Internet-Zeitalter dieses System als über das Web frei zugängliches
Repositorium von Forschungsliteratur etabliert wurde: Bereits 1991 schuf
Paul Ginsparg das arXiv (vergleiche Wikipedia und
\url{http://arxiv.org}). Die Kosten pro Artikel im arXiv werden auf
weniger als zehn Dollar pro Artikel geschätzt (Ball 2015) und durch
öffentliche Gelder und Spenden aufgebracht.

Das Publizieren von Preprints hat mehrere Vorteile. Zum einen
beschleunigt es die Kommunikation in den Fachgebieten. Eine Studie von
Valeria Aman (2013) zeigte, dass die Mehrzahl der arXiv-Preprints
bereits in anderen arXiv-Preprints zitiert werden, bevor sie in einer
Zeitschrift erschienen sind. Die zitierenden Autorinnen und Autoren
wurden also ohne Hilfe einer Zeitschrift auf den zitierten Artikel
aufmerksam und benötigten offenbar kein Peer Review, um seinen Wert für
ihre Forschung einschätzen zu können.

Zum anderen bringen Preprints Vorteile für das Begutachtungsverfahren
selbst: \enquote{Die besonders renommierten Physik-Zeitschriften
\emph{Physical Review} und \emph{Physical Review Letters} der APS
(American Physical Society) erwarten inzwischen sogar, dass die
eingereichten Artikel bereits vorher im arXiv erschienen sind: Auf diese
Weise stehen ihre Gutachter unter einem geringeren Zeitdruck, schaden
dem Prioritätsanspruch des Autors im Falle der Ablehnung nicht, und
können die bereits einsetzende öffentliche Diskussion aller
interessierten einschlägigen Experten zum Preprint-Artikel in die eigene
Meinungsbildung einbeziehen (...).} (Hilf und Severiens 2013, S.
381/82). Autorinnen und Autoren werden auch mehr Sorgfalt auf ihre Texte
verwenden, wenn sie als Preprints öffentlich zugänglich sind, bevor
Gutachterinnen und Gutachter auf inhaltliche oder auch formale Fehler
hinweisen konnten. Das erleichtert und verkürzt den Peer-Review-Prozess.

Für die Kommunikation, wenigstens innerhalb eines Fachgebietes, werden
also wissenschaftliche Zeitschriften nicht mehr benötigt. Forschende
publizieren trotzdem immer noch in Zeitschriften, weil sie für ihre
wissenschaftlichen Karrieren Aufsätze in angesehenen Fachjournalen
vorweisen müssen. Dies sollte man auch nicht aufgeben, denn es macht den
Kern des Anreizsystems der neuzeitlichen Wissenschaft aus, in dem durch
Publizieren gewonnene Reputation zu guten Stellen und zu
Forschungsmitteln führt. Neues wissenschaftliches Wissen wird nicht auf
einem Markt verkauft, sondern der Fachgemeinschaft frei zur Verfügung
gestellt. Wissenschaftliches Wissen ist eine Allmende, zudem eine, die
durch Benutzung an Wert gewinnt.

Genau das ist der Inhalt der Norm des \emph{Kommunismus}, die der
bekannte US-amerikanische Soziologe Robert Merton bereits 1942 als ein
Charakteristikum von Wissenschaft benannte (Merton 1988, S. 620; vgl.
auch Havemann 2017, S. 193). Diese Mertonsche Norm wird durch den freien
Zugang zu wissenschaftlichen Aufsätzen in Repositorien auf eine ideale
Weise erfüllt. Davon konnte man im Papierzeitalter nur träumen. Auch für
einen bestellten Preprint musste viermal Papier in die Post gegeben
werden, oft auf lange Wege um den halben Erdball.

Das Anreizsystem der Wissenschaft hat über die Jahrhunderte gut
funktioniert und sollte beibehalten werden, wenn es auch heute in
manchen Fachgebieten in Bibliometrie-gestützen Exzesse ausartet, wo der
dazu ungeeignete \emph{Journal Impact Factor} zur Gewichtung von
Artikeln verwendet wird (vgl. Havemann 2016, S. 109--111). Was aber
aufgegeben werden sollte, ist das Vergeuden öffentlicher Mittel für die
Finanzierung der exorbitanten Renditen, die die großen
Wissenschaftsverlage mit Fachzeitschriften erzielen (Hofmann 2015, S.
4--5). Diese werden heute nicht nur durch den überteuerten Verkauf von
Abonnements oder von einzelnen Aufsätzen erzielt, sondern zunehmend auch
durch nicht gerade niedrige Autorengebühren für dann frei zugängliche
Zeitschriftenaufsätze. Für die von Expertinnen und Experten
üblicherweise kostenlos geleistete nachträgliche Begutachtung von
bereits als Preprint öffentlich gemachten Aufsätzen muss man nicht
Aktionäre und andere Anteilseignerinnen füttern, egal ob diese an
Lizenzen für Zeitschriften oder an Gebühren für das Lesen oder
Publizieren von Artikeln verdienen. Vom Verwertungs- zum
Verbreitungsparadigma überzugehen, das hat schon vor einem Jahrzehnt
Stefan Gradmann für die Wissenschaftskommunikation postuliert (Gradmann
2007, vgl. auch Mittler 2007, S. 168/69, Kaden 2010, S. 235, und
Havemann 2017). Wenn für Autoren und Leser elektronische Publikationen
kostenlos sind, spricht man auch vom \emph{diamantenen} Open Access
(Fuchs und Sandoval 2013). \emph{Libreas} ist also eine diamantene
Zeitschrift.

Es geht darum, dass die wissenschaftlichen Gemeinschaften ihre
Kommunikation wieder selbst organisieren, wie es bis zur Mitte des
vorigen Jahrhunderts vorherrschend war, bevor dann die
Wissenschaftsverlage viele Fachzeitschriften von den Fachgesellschaften
übernahmen und für neue Fachgebiete neue Zeitschriften gründeten
(Cassella und Calvi 2010, S. 7). Weil die privaten Verlage ihre
Anteilseigner mit Renditen befriedigen müssen und jede halbwegs
angesehene Zeitschrift quasi ein Monopol auf ihrem Gebiet hat, kam es
dann zur Zeitschriftenkrise mit überhöhten Preisen (Kuhlen 2010, S.
322--324). Die digitale Kommunikation über das Netz erlaubt es, diesen
Prozess hin zur Profitorientierung rückgängig zu machen (Steinberg
2015).

Die Kosten für die Herausgabe einer wissenschaftlichen Zeitschrift
reduzieren sich im Internetzeitalter auf die Finanzierung der
Layout-Dienstleister, des Server-Betriebs und der Organisation des Peer
Review, denn eine Printausgabe ist nicht mehr vonnöten. Von diesen drei
Kostenarten bleibt nur die letzte übrig, wenn man zu \emph{Overlay
Journals} übergeht, bei denen keine Server benötigt werden, weil ihre
Artikel in der akzeptierten Version im arXiv oder einem anderen
Fachrepositorium zugänglich sind. Das Layout machen die Autoren und
Autorinnen selber oder heuern dafür Dienstleister an. So kommt das vor
zwei Jahren gegründete Overlay Journal \emph{Discrete Analysis} auf
Kosten von zehn Dollar pro eingereichtem Artikel (Ball 2015), was um
Größenordnungen geringer ist als die Gebühren für Artikel in
OA-Zeitschriften, auch wenn diese von gemeinnützigen Organisationen
herausgegeben werden (PLOS One: 1500 Dollar). Das heißt, die
Begutachtung von hunderten von Artikeln kann durch eine bescheidene Gabe
einer Förderorganisation oder Fachgesellschaft ermöglicht werden.
Tatsächlich gibt die Deutsche Forschungsgemeinschaft (DFG)
\enquote{Starthilfen u.a. für den Aufbau von Open-Access-Zeitschriften}
und stellt auch für \enquote{die Überführung bestehender Zeitschriften
in den Open Access} (DFG 2015, S. 3) Mittel zur Verfügung.

Der Gedanke, über frei zugängliche Repositorien (wie das arXiv) eine
virtuelle Schicht von Zeitschriften mit Peer Review zu legen, wurde
schon früh von Ginsparg (1996) propagiert (einen historischen Abriss
findet man bei Brown 2010). Der Mathematiker und Open-Access-Pionier
Martin Grötschel (seit 2015 Präsident der Berlin-Brandenburgischen
Akademie der Wissenschaften) hat diese Idee kürzlich in die folgenden
Worte gefasst : \enquote{Das, was in Physik, Mathematik, Informatik und
angrenzenden Fachgebieten durch den E-Print-Service arXiv geschieht, hat
sicherlich Vorbildcharakter. Preprints können (nach einer ersten
Prüfung) auf dem arXiv-Server allgemein zugänglich abgelegt werden und
danach den üblichen Gang durch die Begutachtungsprozeduren gehen. Nach
positiver Evaluierung können sie in eigenständigen Zeitschriften oder
Overlay-Journalen als geprüfte Publikationen gefunden werden. Dieser
gesamte Prozess ist transparent und zitierfähig. Ähnliches lässt sich in
allen Fachgebieten einrichten oder auf institutioneller, regionaler bzw.
nationaler Ebene organisieren.} (Grötschel 2016, S. 255) Er weist dann
auf eine für das Betreiben von elektronischen Journale
\enquote{erfreuliche Entwicklung} hin: \enquote{Open Journal Systems
(OJS) ist eine Open-Source-Software zur Verwaltung und Veröffentlichung
von wissenschaftlichen Zeitschriften, die kontinuierlich von
verschiedenen Institutionen und Einzelpersonen weiterentwickelt wird.
Der Code ist frei zugänglich, und das Programm kann kostenfrei verwendet
werden.} (Grötschel 2016, S. 255, vergleiche
\url{http://www.ojs-de.net/index.html})

Ein Beispiel für ein seit 2005 funktionierendes und für Leser wie
Autoreninnen kostenloses (also diamantenes) Overlay Journal ist die
Zeitschrift \emph{Logical Methods in Computer Science}, die auf dem
arXiv aufliegt und auch im Web of Science indexiert wird. In Bezug auf
ihren Impact-Faktor befindet sie sich im Mittelfeld der Logik-Journale
im Web of Science (Rang 9 von 19). (vgl. Category LOGIC auf
\url{https://jcr.incites.thomsonreuters.com/JCRJournalHomeAction.action})
Sie wird von einem in Braunschweig eingetragenen gemeinnützigen Verein
herausgegeben und hat in den vergangenen zwölf Jahren 668 Artikel
publiziert (Stand vom August 2017). Es ist generell nicht leicht, eine
neue Fachzeitschrift zu etablieren, zumal wenn man dabei auf die
Rückendeckung durch einen bekannten Verlag verzichtet. Dass das bei
diesem Journal gelungen ist, lässt sich auf zwei Umstände zurückführen.
Zum Einen sind die Herausgeber und die Mitglieder des Editorial Board
offenbar namhafte Vertreterinnen und Vertreter ihres Fachgebiets.
Editor-in-Chief ist zum Beispiel Lars Birkedal, Professor an der Aarhus
University und Ko-Autor einer Reihe von hochzitierten Publikationen
(vgl. \url{http://cs.au.dk//~birke/}). Zum anderen werden hohe Maßstäbe
an die publizierten Artikel angelegt. (vgl.
\url{https://lmcs.episciences.org/page/authors-information}) Das
beteuern sicher alle Herausgeber von Fachzeitschriften, aber bei
diamantenen OA-Journalen steht dieser erklärten Absicht kein
unmittelbares wirtschaftliches Interesse entgegen (Steinberg 2015).

\emph{Logical Methods in Computer Science} ist eine von aktuell fünf
Overlay-OA-Journalen auf dem Gebiet der Informatik, die die 2013
geschaffene \emph{Episciences}-Plattform für ihren Peer Review und ihren
Internet-Auftritt nutzen. Dazu kommen noch zwei mathematische und drei
geisteswissenschaftliche \emph{epijournals} (das griechische \emph{epi}
bedeutet \emph{dazu}). Die Plattform wird vom \emph{Centre pour la
Communication Scientifique Directe} (CCSD) betrieben. Jeder
Wissenschaftler, der ein Epijournal gründen will, kann sie nutzen
(\url{https://www.episciences.org}).

Wissenschaft kann also heute ihre Kommunikation wieder zurückgewinnen
und von der Profitmacherei befreien. Fehlt es bloß noch an renommierten
Forschenden, die sich von ihrer Bindung an die Verlage befreien und frei
zugängliche Epijournale gründen. Dazu bedarf es der Unterstützung und
dem Drängen durch eine zeitgemäße Wissenschaftspolitik.

\emph{Polemische Nachbemerkung:} Kritikern der großen
Wissenschaftsverlage wird manchmal vorgeworfen, es ginge ihnen gar nicht
um eine optimale Wissenschaftskommunikation. Als verkappte Sozialisten
würden sie den Anteilseignern der Verlage bloß ihre Renditen nicht
gönnen. So schrieb z.B. der US-amerikanische Bibliothekar Jeffrey Beall
(2013, S. 590): \enquote{{[}..{]} a close analysis of the discourse of
the OA advocates reveals that the real goal of the open access movement
is to kill off the for-profit publishers and make scholarly publishing a
cooperative and socialistic enterprise. It's a negative movement.} Es
wird schon so sein, dass Menschen mit sozialistischen Ideen keine
Hemmungen haben, Profitorientierung in Frage zu stellen. Aber fallen
denn der Verlagslobby wirklich keine intelligenten Argumente mehr ein,
dass sie es schon nötig hat, die Motive ihrer Kritiker in Zweifel zu
ziehen?

\hypertarget{referenzen}{%
\section*{Referenzen}\label{referenzen}}

Aman, V. (2013, Juni). The potential of preprints to accelerate
scholarly communication -- A bibliometric analysis based on selected
journals. Master Thesis 2013, arXiv: 1306.4856.

Ball, P. (2015, Oktober). The journal that publishes no papers.
Mathematics journal \emph{overlays} arXiv preprint server. Nature 526,
S. 146, doi:10.1038/nature.2015.18351.

Beall, J. (2013, September) The Open-Access Movement is Not Really about
Open Access. tripleC: Communication, Capitalism \& Critique. Open Access
Journal for a Global Sustainable Information Society 11(2), S. 589--597.

Brown, J. (2009) An Introduction to Overlay Journals.
\url{http://discovery.ucl.ac.uk/19081/1/19081.pdf}

Cassella, M. und L. Calvi (2010). New journal models and publishing
perspectives in the evolving digital environment. IFLA journal 36(1), S.
7--15.

DFG (2015). Merkblatt Infrastruktur für elektronische Publikationen und
digitale Wissenschaftskommunikation. DFG-Vordruck 12.11 -- 09/15,
\url{http://www.dfg.de/formulare/12_11/}

Fuchs, C. und M. Sandoval (2013, September). The Diamond Model of Open
Access Publishing: Why Policy Makers, Scholars, Universities, Libraries,
Labour Unions and the Publishing World Need to Take Non-Commercial,
Non-Profit Open Access Serious. tripleC: Communication, Capitalism \&
Critique. Open Access Journal for a Global Sustainable Information
Society 11(2), S. 428--443.

Ginsparg, P. (1996). Winners and Losers in the Global Research Village.
Eingeladener Vortrag auf der Konferemz im UNESCO Hauptquartier, Paris,
19.-23. Februar 1996, in der Sitzung \emph{Scientist}s View of
Electronic Publishing and Issues Raised*, 21. Februar 1996. Vgl.
\url{http://openscience.ens.fr/OPEN_ACCESS_MODELS/GREEN_OPEN_ACCESS/ARXIV/}

Gradmann, S. (2007). Verbreitung vs.~Verwertung. Anmerkungen zu Open
Access, zum Warencharakter wissenschaftlicher Informationen und zur
Zukunft des elektronischen Publizierens. In: F. Havemann, H. Parthey,
und W. Umstätter (Hrsg.), Integrität wissenschaftlicher Publikationen in
der Digitalen Bibliothek -- Wissenschaftsforschung Jahrbuch 2007, S.
92--106. Berlin: Gesellschaft für Wissenschaftsforschung. 2.,
unveränderte Auflage 2012, frei verfügbar bei der Deutschen
Nationalbibliothek unter \url{http://d-nb.info/1021026476}.

Grötschel, M. (2016). Elektronisches Publizieren, Open Access, Open
Science und ähnliche Träume. In: Peter Weingart und Niels Taubert
(Hrsg.): Wissenschaftliches Publizieren: Zwischen Digitalisierung,
Leistungsmessung, Ökonomisierung. De Gruyter Akademie Forschung.

Havemann, F. (2016). Einführung in die Bibliometrie (2., erweiterte
Aufl.). Berlin: Gesellschaft für Wissenschaftsforschung.
\url{http://d-nb.info/1113795433}.

Havemann, F. (2017). Freier Zugang zu Wissen nach dem Papierzeitalter:
Fragen, Thesen und Vorschläge. In: Theorien und Konzepte des
wissenschaftlichen Erkennens: Festschrift zum 80. Geburtstag von
Heinrich Parthey, herausgegeben von Vivien Petras, Walther Umstätter und
Karl-Friedrich Wessel. Wissenschaftlicher Verlag Berlin, S. 193--204.
Preprint 2016 auf
\url{https://www.researchgate.net/publication/311470242_Freier_Zugang_zu_Wissen_nach_dem_Papierzeitalter_Fragen_Thesen_und_Vorschlage}.

Hofmann, J. (2015). Open Access: Ein Lackmustest. In: Thomas Dreier,
Veronika Fischer, Anne van Raay, Indra Spiecker gen. Döhmann (Hrsg.),
Zugang und Verwertung öffentlicher Informationen, Nomos Verlag
Baden-Baden. Preprint auf
\url{https://papers.ssrn.com/sol3/papers.cfm?abstract_id=2515844}

Hilf, E. R. und T. Severiens (2013). Vom Open Access für Dokumente und
Daten zu Open Content in der Wissenschaft. In: R. Kuhlen, W. Semar, und
D. Strauch (Hrsg.), Grundlagen der praktischen Information und
Dokumentation:s Handbuch zur Einführung in die Informationswissenschaft
und -praxis (völlig neu gefasste Ausg., 6. Aufl.)., S. 379--395. Berlin,
Boston: De Gruyter.

Kuhlen, R. (2010). Open Access -- eine elektronischen Umgebungen
angemessene Institutionalisierungsform für das Gemeingut
\enquote{Wissen}. Leviathan 38(3), S. 313--329.

Kaden, B. (2010). Unordnung des Diskurses. Bemerkungen zu Uwe Jochums
\enquote{Open Access}. Zur Korrektur einiger populärer Annahmen.
Göttingen 2009. 61 S. (Göttinger Sudelblätter), ISBN 978-3-8353-0618-9.
BIBLIOTHEK Forschung und Praxis 34(2), S. 232--237.

Merton, R. K. (1988). The Matthew effect in science, II: cumulative
advantage and the symbolism of intellectual property. Isis 79, S.
606--623.

Mittler, E. (2007). Open Access zwischen E-Commerce und E-Science:
Beobachtungen zu Entwicklung und Stand. Zeitschrift für Bibliothekswesen
und Bibliographie 54(4--5), S. 163--169.

Steinberg, P. E. (2015, Juli). Reclaiming Society Publishing.
Publications 3(3), S. 150--154.

%autor
\begin{center}\rule{0.5\linewidth}{\linethickness}\end{center}

\textbf{Frank Havemann}, Dr.~rer.nat., geboren 1949 in Berlin, Studium
der Physik an der Humboldt-Universität zu Berlin, Dissertation zur
Theorie der Elementarteilchen; ab 1990 Forschungen zur Bibliometrie und
Scientometrie, Dozent am Institut für Bibliotheks- und
Informationswissenschaft der Humboldt-Universität.

\end{document}
