\documentclass[a4paper,
fontsize=11pt,
%headings=small,
oneside,
numbers=noperiodatend,
parskip=half-,
bibliography=totoc,
final
]{scrartcl}

\usepackage{synttree}
\usepackage{graphicx}
\setkeys{Gin}{width=.4\textwidth} %default pics size

\graphicspath{{./plots/}}
\usepackage[ngerman]{babel}
\usepackage[T1]{fontenc}
%\usepackage{amsmath}
\usepackage[utf8x]{inputenc}
\usepackage [hyphens]{url}
\usepackage{booktabs} 
\usepackage[left=2.4cm,right=2.4cm,top=2.3cm,bottom=2cm,includeheadfoot]{geometry}
\usepackage{eurosym}
\usepackage{multirow}
\usepackage[ngerman]{varioref}
\setcapindent{1em}
\renewcommand{\labelitemi}{--}
\usepackage{paralist}
\usepackage{pdfpages}
\usepackage{lscape}
\usepackage{float}
\usepackage{acronym}
\usepackage{eurosym}
\usepackage[babel]{csquotes}
\usepackage{longtable,lscape}
\usepackage{mathpazo}
\usepackage[normalem]{ulem} %emphasize weiterhin kursiv
\usepackage[flushmargin,ragged]{footmisc} % left align footnote
\usepackage{ccicons} 

%%%% fancy LIBREAS URL color 
\usepackage{xcolor}
\definecolor{libreas}{RGB}{112,0,0}

\usepackage{listings}

\urlstyle{same}  % don't use monospace font for urls

\usepackage[fleqn]{amsmath}

%adjust fontsize for part

\usepackage{sectsty}
\partfont{\large}

%Das BibTeX-Zeichen mit \BibTeX setzen:
\def\symbol#1{\char #1\relax}
\def\bsl{{\tt\symbol{'134}}}
\def\BibTeX{{\rm B\kern-.05em{\sc i\kern-.025em b}\kern-.08em
    T\kern-.1667em\lower.7ex\hbox{E}\kern-.125emX}}

\usepackage{fancyhdr}
\fancyhf{}
\pagestyle{fancyplain}
\fancyhead[R]{\thepage}

% make sure bookmarks are created eventough sections are not numbered!
% uncommend if sections are numbered (bookmarks created by default)
\makeatletter
\renewcommand\@seccntformat[1]{}
\makeatother


\usepackage{hyperxmp}
\usepackage[colorlinks, linkcolor=black,citecolor=black, urlcolor=libreas,
breaklinks= true,bookmarks=true,bookmarksopen=true]{hyperref}
%URLs hart brechen
\makeatletter 
\g@addto@macro\UrlBreaks{ 
  \do\a\do\b\do\c\do\d\do\e\do\f\do\g\do\h\do\i\do\j 
  \do\k\do\l\do\m\do\n\do\o\do\p\do\q\do\r\do\s\do\t 
  \do\u\do\v\do\w\do\x\do\y\do\z\do\&\do\1\do\2\do\3 
  \do\4\do\5\do\6\do\7\do\8\do\9\do\0} 
% \def\do@url@hyp{\do\-} 
\makeatother 

%meta
%meta

\fancyhead[L]{S. Volkmann \\ %author
LIBREAS. Library Ideas, 32 (2017). % journal, issue, volume.
\href{http://nbn-resolving.de/}
{}} % urn 
% recommended use
%\href{http://nbn-resolving.de/}{\color{black}{urn:nbn:de...}}
\fancyhead[R]{\thepage} %page number
\fancyfoot[L] {\ccLogo \ccAttribution\ \href{https://creativecommons.org/licenses/by/3.0/}{\color{black}Creative Commons BY 3.0}}  %licence
\fancyfoot[R] {ISSN: 1860-7950}

\title{\LARGE{Learning Cities}} % title
\author{Sebastian Volkmann} % author

\setcounter{page}{1}

\hypersetup{%
      pdftitle={Learning Cities},
      pdfauthor={Sebastian Volkmann},
      pdfcopyright={CC BY 3.0 Unported},
      pdfsubject={LIBREAS. Library Ideas, 32 (2017).},
      pdfkeywords={Bibliothek, Stadt, Lernen},
      pdflicenseurl={https://creativecommons.org/licenses/by/3.0/},
      pdfcontacturl={http://libreas.eu},
      baseurl={http://libreas.eu},
      pdflang={de},
      pdfmetalang={de}
     }



\date{}
\begin{document}

\maketitle
\thispagestyle{fancyplain} 

%abstracts

%body
\hypertarget{einleitung}{%
\section*{Einleitung}\label{einleitung}}

Kulturelle Vielfalt, massive Zuwanderung und soziale Ungleichheit sind
charakteristisch für die australische Stadt Hume im Nordwesten von
Melbourne. Als die Kommunalverwaltung Mitte der 1990er an Stelle von
Entwicklungsmaßnahmen unter Anderem in ein größeres Polizeigebäude
investiert, verleitet das den Journalisten und Lokalpolitiker Frank
McGuire zu einer ungewöhnlichen Initiative. Anstatt die Bevölkerung für
ihre missliche Lage unter Generalverdacht zu stellen, wollte er
tatsächlichen Fortschritt. Sein Plan war, in dem besonders betroffenen
Stadtteil Broadmeadows eine erste Bibliothek zu bauen, die das Fundament
für ein \enquote{lokales Bildungsnetzwerk} formen würde. Lebenslanges
Lernen sollte Hume in eine bessere Zukunft bringen. (HGLV n.d.; Wheeler
et al.~2013, p.~35)

Über eine Entwicklungszeit von fast zehn Jahren wurden eine Vielzahl an
lokalen, und sogar nationalen und internationalen Partnern für das
Bildungsnetzwerk aktiviert, das als \enquote{Hume Global Learning
Village} (HGLV) bekannt ist. 2003 eröffnete dann das \enquote{Hume
Global Learning Center} (HGLC), das neben der geplanten Bibliothek auch
ein Café, Teile der Kommunalverwaltung und weitere Räumlichkeiten unter
modernem Stahl und Glas zusammen führte. Zum ersten Mal hatten die
Bewohner von Broadmeadows einen zugänglichen und nichtkommerziellen Ort
zum Verweilen und Lesen, für soziale Aktivitäten, freien Internetzugang,
oder eben zur Weiterbildung. Partnerschaften aus Bildung, regionalen
Arbeitgebern und der öffentlichen Hand steuerten Beträge hinzu, die
Angebote für unterschiedliche Alters- und Sozialgruppen ermöglichten.

Keine zehn Jahre nach dem Start des HGLC lag der Bibliotheksnutzeranteil
bereits über 50\,\% und damit über dem Durchschnitt des Bundesstaates
Victoria. (Wheeler et al.~2013, pp.~39--40) Auch wenn die
Durchschnittseinkommen und die Quote der Bildungsabschlüsse weiterhin
unter denen Melbournes sind, haben sie sich sichtbar verbessert.
(Wheeler et al.~2013, pp.~35--36) Entsprechend wird bereits an einem
dritten Lernzentrum gebaut. Der moderne, einladende Charakter dieser
Orte, sowie die vielen, kooperativ geschaffenen Angebote, die durch sie
entstanden sind, haben dem Thema Bildung eine neue Bedeutung für die
Menschen Humes verliehen. (Wheeler et al.~2013, p.~37)

Ausschlaggebend für den Erfolg des HGLV ist jedoch nicht die Idee an
sich, und die große Aufmerksamkeit, die sie mit sich brachte. Wie eine
genauere Betrachtung des Planungs- und Umsetzungsprozesses zeigt
(Volkmann 2016), sind es vor allem menschliche Faktoren, wie genug Zeit
und Kapital, die nicht nur in Hume, sondern auch in vielen anderen
\enquote{Lernenden Städten} weltweit zu positiven Veränderungen in der
Region geführt haben. Gerade weil jene Faktoren in Fachdiskussionen eher
selten zusammenhängend und kritisch behandelt werden (Kearns 2015),
lohnt es sich, den Werdegang Humes für den deutschsprachigen Raum einmal
aufzuarbeiten. Da man in Deutschland Bildungslandschaften tendenziell
auf Schulentwicklung einschränkt, kann ein erweitertes Verständnis von
umfassenden, lokalen Bildungsnetzwerken Kommunen und
Bildungseinrichtungen aller Art neuen Ansporn verleihen, sich gemeinsam
für die mittellosesten Mitbürger einzusetzen. (Aring 2014)

Ziel dieses Beitrags soll es sein, sechs Erfolgsfaktoren des Hume Global
Learning Village auf Grundlage von Volkmann (2016) vereinfacht
aufzuarbeiten. Hierdurch soll besonders der deutschsprachigen
Bibliothekswelt die Erfolgsgeschichte aus Hume zugänglich gemacht
werden. Das Thema Lernende Städte und Bildungsnetzwerke kann auch für
nicht-schulische Bildungseinrichtungen verständlich und praktikabel
sein.

Zunächst werden Bildungsnetzwerke in den heutigen Kontext gesetzt, indem
ihr historischer Werdegang sowie die Begriffe um Netzwerke und
Lebenslanges Lernen erläutert werden. Als nächstes wird der
demographische und gesellschaftliche Hintergrund Humes geschildert sowie
die Rolle verschiedener Einzelpersonen im Aufbau des HGLV. Dies mündet
in die Diskussion um die nötigen Kompetenzen und die interne
Lernfähigkeit der Netzwerkteilnehmer. Schließlich werden auch
organisationale Fragen um die Schaffung eines gemeinsamen Verständnisses
von Kommunikation und Öffentlichkeitsarbeit erarbeitet. Zuletzt wird auf
das Problem der Langfristigkeit von Bildungsnetzwerken und deren
Finanzierung eingegangen.

\hypertarget{lernende-stuxe4dte-und-bildungsnetzwerke-im-internationalen-und-deutschsprachigen-kontext}{%
\section*{Lernende Städte und Bildungsnetzwerke im internationalen
und deutschsprachigen
Kontext}\label{lernende-stuxe4dte-und-bildungsnetzwerke-im-internationalen-und-deutschsprachigen-kontext}}

Die Idee von lokalen Bildungsnetzwerken beruht auf der Grunderkenntnis,
dass man als einzelner Akteur, als einzelne Bildungseinrichtung, nur
eingeschränkten Zugang und Einfluss auf die regionale Bevölkerung hat.
Stattdessen könnte man doch viel umfassendere Ziele erreichen, und
nebenbei den eigenen Standort in der Kommune sichern oder stärken, wenn
man sich mit gleichgesinnten Einrichtungen zusammentut. Es geht also um
eine innovative Art über die eigenen organisatorischen Grenzen hinaus in
Netzwerken zu denken und zu handeln. Dieser Gedanke ist
interessanterweise weder neu, noch einzig auf den Bildungssektor
zutreffend. Er entstammt einem Neudenken der öffentlichen Verwaltung,
das heute auf fast jeder Agenda des öffentlichen Sektors zu finden ist.
(Cepiku 2017; Torfing 2016, pp.~7--10; Ramesh, Howlett 2017)

Die globalen Herausforderungen von heute reichen von Klimaveränderung
und Digitalisierung über Urbanisierung und Migration, Globalisierung und
wachsender Ungleichheit, bis zum demographischen, arbeitsweltlichen
Wandel. (Stang, Eigenbrodt 2014, pp.~233--235) Es sind Themen, die alle
Bevölkerungsschichten betreffen, für die aber nur bedingt politische
Lösungen vorliegen. Mit einem sinkenden Vertrauen in die politische
Führung wächst daher in den letzten Jahrzehnten die Zahl der Kommunen
und Regionen, die nicht länger auf eine Wegweisung von staatlicher Seite
warten wollen. (Goldsmith, Eggers 2004) Entgegen jeglicher
\enquote{Alternativlosigkeit} besinnen sie sich auf vorhandene Stärken
und Ressourcen, um ihre Probleme auf eigene Art zu lösen, nämlich in
Netzwerken.

Schon in den Siebzigern gab es Städte, die aus eigener Initiative
regionale Netzwerke aufgebaut haben, um ihre Eigenständigkeit und
Wirtschaftlichkeit zu bestärken. (Watson, Wu 2015, pp.~9--11) Heute gibt
es im Wettbewerb der internationalen Metropolen kaum noch eine, die
keine Strategie zur regionalen Innovationsförderung, oder zum Anlocken
der Bestausgebildetsten fährt. (Landry 2008; Florida 2003; Longworth
2006) Man nennt sie \enquote{Kreative Städte}, \enquote{Wissensstädte}
oder eben \enquote{Lernende Städte}.

Diese neuen Ansätze der Stadt- und Regionalentwicklung zeichnen sich
dadurch aus, dass man verschiedenste Akteure (kommunale Verwaltungen,
Bildungs- und Kultureinrichtungen, Firmen, Vereine, Bürgervertreter),
die sonst wenig miteinander zu tun haben, an einen Tisch bringt.
Gemeinsam sollen Wege gefunden werden, um lokale Probleme, von denen
alle betroffen sind, durch öffentliche Dienstleistungen und
Infrastrukturen neu anzugehen. Hierzu werden die existierenden Stärken
und Kapazitäten der Netzwerkteilnehmer kombiniert, um so die
Lebensqualität Aller zu verbessern. Es kann hier um Wirtschaftswachstum,
Nachhaltigkeit, Innovation und Technologie, aber auch soziale Inklusion
oder Bildung gehen. (Longworth {[}2014b{]}, pp.~4--5)

Essenziell ist jedoch, dass sich die Kooperation nach Zielerreichung
nicht auflöst, sondern weiterbesteht, sich eher sogar ausbaut, und sich
höhere Ziele steckt. Ebendas differenziert Bildungsnetzwerke von
projektbasierten Kooperationen. Sie sind auf Dauer angelegt und müssen
entsprechend belastbar und ressourcenstark sein. (Schäffter 2004, p.~32;
Tippelt 2015, p.~21) Nicht alle Bildungskooperationen schaffen den
Übergang zu einem langfristigen Netzwerk. Genau das ist aber nötig, um
die sozioökonomischen Probleme, vor denen Kommunen und ihre Bürger
stehen, ganzheitlich anzugehen.

Bemerkenswert ist dabei, dass es bei der Vielzahl der Probleme, vor
denen Städte und Regionen weltweit stehen, eine Herangehensweise gibt,
die als universeller Lösungsansatz gesehen wird: \emph{Lebenslanges
Lernen}. (Longworth {[}2014b{]}; Duke 2012; Brookfield 2012) Ansätze zum
Lebenslangen Lernen sollen eine Gesellschaft schaffen, in der Lernen als
fortlaufender Lebensinhalt akzeptiert wird, und die Bürger
Selbstverantwortung für ihren Lernfortschritt übernehmen. (Longworth
2006, p.~25) Der Schlüssel soll darin liegen, Menschen zu ermöglichen,
die Kontrolle über ihr Leben (wieder) zu übernehmen. (Schuller \& Watson
in Wheeler et al.~2013, p.~10) Ob eine Kommune nun ihr Wirtschafts- und
Innovationspotential stärken, die Umwelt schützen, Kulturen oder
Generationen zusammenbringen will -- jedes dieser Ziele wird bedient,
wenn das selbstgesteuerte, Lebenslange Lernen kooperativ gefördert wird.
Ein gemeinsames Ziel vor Augen verbinden solche Netzwerke lokale
Bildungseinrichtungen, -akteure, und -zielgruppen langfristig. Sie
stellen gegenseitig Mehrwert und greifbarere Bildungszugänge her.

Ebendieses Prinzip wurde auch im australischen Hume angewandt. Durch
eine kooperative Vernetzung kommunaler Bildungsangebote und -akteure
wurden Ressourcen und Kapazitäten gebündelt, und gleichzeitig neue
Zugänge und Synergien geschaffen. Schon in den ersten Jahren wuchs das
Netzwerk in Hume auf über 300 teilnehmende Vereine und Einzelpersonen an
und steht 2011 bei etwa 700. Es umschließt neben dem neu-gebauten Hume
Global Learning Center (HGLC) im Stadtteil Broadmeadows vier weitere
Bibliotheken, eine mobile Bibliothek, diverse Schulen, sieben
Erwachsenen-Lernzentren, sowie Partnerschaften mit wichtigen
Unternehmen\footnote{Die Lokalzeitung \enquote{The Age} unterstütze das
  Vorhaben zum Beispiel zuerst nur finanziell, weil sie die Werte und
  die Vision des Projekts teilten. Dann kam es zu engeren Kooperationen,
  in denen die Print- und e-Ressourcen des Zeitungsverlags über die
  Bibliothek zugänglich gemacht wurden. Lehrern und Schülern bringen sie
  dort auch den Umgang mit neuen Medien bei. Ressourcenschwache
  Grundschüler können ein Jahr lang einen Computer mit Internetzugang
  gesponsert bekommen. Motorenhersteller Ford betreibt ein
  Mentoring-Programm, in dem Angestellte einheimische
  Langzeitarbeitslose in Schlüsselkompetenzen schulen. Ford und The Time
  demonstrieren damit unternehmerische Gesellschaftsverantwortung und
  verbessern ihr Arbeitnehmer-Image. Sie zeigen aber auch kleineren
  Firmen der Region, dass sich die Teilnahme am Bildungsnetzwerk lohnt.}
und Hochschulen. (Wheeler, Osborne 2011, p.~536; Faris 2006)
Entsprechend hat sich die Reichweite und die Erreichbarkeit aller
Partner vergrößert. Darüber hinaus können bestehende und kollaborative
Angebote jetzt weitaus effektiver auf bestimmte Zielgruppen fokussiert
werden, die zuvor nur schwer anzusprechen waren. So wird Lernen und
soziale Teilnahme völlig neuen Teilen der Bevölkerung nähergebracht, was
für den gesellschaftlichen Kontext in Hume ein großes
Veränderungspotential bietet. Im Folgenden soll dieser Kontext und der
erste Erfolgsfaktor des HGLV näher betrachtet werden.

\hypertarget{fuxfchrung-und-die-erfolgsgeschichte-hume}{%
\subsection*{Führung und die Erfolgsgeschichte
Hume}\label{fuxfchrung-und-die-erfolgsgeschichte-hume}}

Hume ist eine von Industrie und anhaltender Migration geprägte Stadt am
nordwestlichen Rand von Melbourne, Australien. Wirtschaftliche Pfeiler
der Region sind Melbournes Flughafen und eine Automobilfabrik, die für
Arbeitsplätze und Investitionen sorgen (Bentley 2009). Angesichts eines
rasanten Bevölkerungszuwachses an vielen, unterschiedlichen Kulturen
profitieren davon nicht alle sozialen Gruppen. 2011 lebten in Hume
160.000 Einwohner aus über 140 Nationen, von denen etwa 30\,\% im
Ausland geboren waren und die über 125 Sprachen sprechen. Bis zum Jahre
2030 werden 240.000 Einwohner erwartet. Beträchtlich ist der Anteil an
jungen Menschen. 30\,\% der Bevölkerung ist 19 Jahre oder jünger,
wohingegen nur etwa 8\,\% über 65 sind. (Wheeler, Osborne 2011, p.~535;
O'Hagan 2014, pp.~2--3)

Obwohl Hume also eine wachsende und enorm junge Stadt mit Potenzial ist,
liegt sie im Bildungsdurchschnitt hinter Melbourne und anderen Kommunen.
Neben kulturellen und sprachlichen Barrieren machen sich auch höhere
Behinderungs- und Arbeitslosigkeitsraten, sowie niedrigere
Internet-Heimzugänge oder Durchschnittseinkommen in der Demographie
bemerkbar. Über die letzten Jahrzehnte wuchs die Zahl der sozial
ausgegrenzten Jugendlichen, aber die Politik reagierte nicht auf die
Ursachen der Entwicklung. Anstatt die Situation als Gelegenheit für die
Region zu nutzen und den Menschen mehr Chancen zu ermöglichen, wurde in
die Strafverfolgung investiert.

Diese Untätigkeit der Verwaltung verleitete den Journalisten und
Lokalpolitiker (später Bürgermeister, dann Landespolitiker) Frank
McGuire zu der Idee, in dem besonders betroffenen Stadtteil Broadmeadows
eine Bibliothek zu bauen, die die Grundlage für ein lokales
Bildungsnetzwerk sein sollte. Die Grundannahme war, dass Menschen mit
mehr Bildung zufriedener mit ihrer Arbeit und Freizeit sind, überhaupt
höhere Chancen auf Beschäftigung haben, gesünder und rechtskonformer
handeln, sowie mehr am demokratischen Leben teilnehmen. (Wilkinson und
Pickett in Wheeler et al.~2013, p.~22)

Als jemand, der selbst in Hume aufgewachsen ist, kannte McGuire die
Situation vieler Einwohnergruppen. In seinem weitreichenden, politischen
Netzwerk fand er Befürworter für seine Idee. Mit den richtigen Kontakten
stand bald sogar der Premier des Bundesstaates Victoria, Steve Bracks,
hinter der innovativen Strategie, und dessen Nachfolger, John Brumby,
sorgte für die nötigen Investitionen in die kommunale Infrastruktur.
Neben der australischen Regierung wurde auch die Aufmerksamkeit von
lokalen Unternehmen, Stiftungen, sowie den Hauptarbeitgebern der Region
geweckt, was McGuires Initiative den nötigen Schwung und die
Finanzierung verlieh.

Mitte der 1990er wurde Frank McGuire von der Stadt zum Vorsitzenden
einer unabhängigen \enquote{Safe City Taskforce} ernannt, die die
Vorarbeit für das HGLV-Netzwerk machte. 2003 wurde dann die geplante
Bibliothek eröffnet, die viele bildungsnahe Organisationen und Akteure
der Umgebung verbinden sollte. Der eigentliche Aufbau und das Management
des Netzwerkes übergab McGuire jedoch an Vanessa Little. Sie wurde wegen
ihrer Erfahrung in bibliothekarischen Kooperationen ausgewählt, und
sollte mit ihrem Team die Idee des HGLV der Bevölkerung
\enquote{verständlich und genießbar} machen. Die Aussicht auf ein
glücklicheres und erfolgreicheres Leben sollte vermittelt werden. Das
Team war der Stadtverwaltung unterstellt, was ihnen eine hohe
Steuerfähigkeit und Glaubwürdigkeit einbrachte. Littles externer Fokus,
aufbauend auf McGuires Vorarbeit, brachte die ersten Netzwerkteilnehmer
zusammen. Laut Little war besonders hilfreich, dass viele Leute, die
sich aus freien Stücken entschieden hatten, in Hume zu leben und zu
arbeiten, dies mit guten Intentionen taten. Little bot also denen, die
der Stadt und ihren Menschen ohnehin etwas Guten wollten, ein Forum.

McGuire und Litte sind nur zwei starke Beispiele für charismatische
Persönlichkeiten, die einen Beitrag zum HGLV leisteten oder es noch
heute tun. Die Rolle solcher Einzelpersonen ist für den Aufbau von
Bildungsnetzwerken essenziell. (Longworth 2006, p.~163) Auf die Agenda
der Lokalpolitik gelangen Netzwerkansätze schließlich nicht ohne
überzeugende Menschen und Visionen. (Brandt, Maas 2015, p.~24)
Entsprechend brauchte es Initiatoren, wie McGuire, Ausführende, wie
Little, und viele weitere, damit eine Lernende Stadt langfristig
Realität wird. Die unterschiedlichen Funktionen dieser
Führungspersönlichkeiten lässt sich wie folgt kategorisieren: Neben den
Initiatoren um McGuire, die die Vision schaffen oder anstoßen, bedarf es
charismatische oder autoritäre Schlüsselfiguren oder
\enquote{Champions}, die Begeisterung für jene Vision in bestimmten
Bevölkerungsgruppen oder zukünftigen Partnern zu schaffen vermögen.
Diese beiden Führungscharaktere müssen jedoch nicht zwangsweise das
tatsächliche Netzwerken oder das Netzwerkmanagement ausführen. Der
operative Betrieb eines Bildungsnetzwerks bedarf eher eines fähigen
Teams an Managern und Vermittlern, wie Vanessa Little und ihrer
Arbeitsgruppe. Sie müssen Partner identifizieren und engagieren und die
Zusammenarbeit der Netzwerkpartner ermöglichen oder gar steuern.
Letztendlich gilt die Notwendigkeit des stetigen Ausbaus des Netzwerkes
aber für alle drei Typen. Alle Führungscharaktere müssen fortlaufend
neue Schlüsselpersonen identifizieren, aktivieren und befähigen, um dem
Netzwerk ein nachhaltiges Fundament zu schaffen.

\hypertarget{kommunikation-und-ein-gemeinsames-verstuxe4ndnis-von-lebenslangem-lernen}{%
\subsection*{Kommunikation und ein gemeinsames Verständnis von
Lebenslangem
Lernen}\label{kommunikation-und-ein-gemeinsames-verstuxe4ndnis-von-lebenslangem-lernen}}

Der Werdegang des HGLV veranschaulicht, wie wichtig fähige
Einzelpersonen in der Schaffung eines Bildungsnetzwerkes sind. Eine der
kritischen Fähigkeiten, die sie beherrschen müssen, ist eine
charismatische und vertrauenserweckende Kommunikation. Denn was Andere
mit dem Begriff um Bildungsnetzwerke verbinden, kann stark variieren,
und mit schlechten Erfahrungen oder Vorurteilen belastet sein. Das kann
die Akzeptanz für derartige Projekte schon im Keim ersticken, wenn damit
keine konkrete Vision für die regionale Entwicklung verbunden ist. Die
Netzwerkführung muss also klar aufzeigen können, wie sich die Region
durch Bildungskooperationen machbar zum Besseren verändern kann.
(Laitinen 2015, p.~40)

Kommunikation wird aber auch von den unterschiedlichen Gruppen, aus
denen die Region und das Bildungsnetzwerk besteht, erschwert. Lehrer,
Bibliothekare, Planer, Manager, oder Beamte der öffentlichen Verwaltung,
sowie unterschiedliche soziale und kulturelle Hintergründe der Bürger
bedeuten verschiedene Verständnisse von Lernen und wie es organisiert
sein sollte. Bevor man also überhaupt über Bildungskooperationen
sprechen kann, muss erst einmal ein gemeinsames Verständnis von Lernen,
wie auch von lokalen Problemlagen und den Zusammenhängen aufgebaut
werden. (Johnson, Lomas 2005, p.~24; Longworth {[}2014a{]}, p.~3, 2006,
pp.~5--6)

In Hume konnten diese Sprach- und Verständnisbarrieren mit der Zeit
überwunden werden. Grundlegend war der kommunikative Fokus McGuires, weg
von der Betonung offensichtlicher, regionaler Defizite, hin zu
konstruktiven Möglichkeiten, die sich aus dem Netzwerk auftaten.
(Wheeler, Osborne 2011, p.~535) Stets das Positive hervorzuheben verband
auch die Bevölkerung mit der Vision von mehr Wohlstand für alle.
Besondere Bürgernähe wurde aber auch durch organisatorische Details
erzielt, wie die zentral organisierte Übersetzung der
Öffentlichkeitsarbeit in die meistverbreitetsten Sprachen Humes. Klare
und differenzierte Kommunikation der Führungspersonen und professionelle
Öffentlichkeitsarbeit ist also das zweite Erfolgskriterium zu einem
effektiven Bildungsnetzwerk.

\hypertarget{netzwerkmanagement--beteiligung-und-zusammenarbeit}{%
\subsection*{Netzwerkmanagement, -beteiligung und
Zusammenarbeit}\label{netzwerkmanagement--beteiligung-und-zusammenarbeit}}

Die rhetorische und emotionale Übersetzung der Idee des HGLV hat dem
Netzwerk eine solide Akzeptanz verschafft. Nun ergibt sich die
Fragestellung, wer überhaupt in dem Netzwerk teilnehmen kann, und in
welcher Form Entscheidungen getroffen werden. Es geht um die Methoden
und Prozesse, um unterschiedliche Menschen und Ansichten zur Diskussion
und zu einem Konsens zu bringen.

Ein lokales Netzwerk muss zunächst nicht bedeuten, dass jeder Bürger
seine Meinung kundtun kann. Bürgerbeteiligung \emph{kann} ein Werkzeug
sein, um die Wurzel lokaler Probleme zu identifizieren, Ideen zu
generieren, oder auch nur um Vertrauen zu schaffen. Praktisch sind es
aber eher ehrenamtlich zusammenkommende Vertreter verschiedener
Organisationen und Bürgergruppen, die gemeinsam Kooperationen aushandeln
und umsetzen. Je unsicherer oder umstrittener eine Idee jedoch ist, umso
mehr sollte ein solches Gremium die Meinung der Bevölkerung einbeziehen.
Um hier optimal zu differenzieren, hat das HLGV eine interne Richtlinie
entwickelt. (Hume City Council {[}2012{]}, pp.~13--18)

Auch das Vertretergremium muss in sich selbst eine gewisse Balance
haben, und alle Sektoren, die einen Einfluss auf die Lebensqualität in
der Region haben, einbeziehen. Dass Interessenkonflikte und
Meinungsverschiedenheiten einen Konsens erschweren, ist eine zentrale
Herausforderung für die Netzwerkleitung. Werden Diskussionen und
Entscheidungsprozesse aber zudem nicht gleichberechtigt geführt,
bestimmen schnell die etablierten und kapitalstärksten Teilnehmer, oder
eben wieder alteingesessene Politiker die Agenda und Beschlüsse des
Netzwerks. (Margerum 2011, 2002)

Bildungsnetzwerke stehen hier vor einer großen Herausforderung, weil der
effektive Austausch praktischer Erfahrungen häufig an die Teilnahme an
überregionalen Netzwerken wie Städtepartnerschaften gebunden ist.
(Longworth {[}2014b{]}, p.~13) Daneben ist es jedoch keine Seltenheit,
sich externe Hilfe zu holen, um die Aufbau- und Entscheidungsprozesse zu
begleiten. In Hume engagierte man deshalb eine Vielzahl an Experten aus
Bildung und öffentlicher Verwaltung, aber auch Hochschulen aus dem
nahegelegenen Melbourne, die das Projekt aus wissenschaftlicher
Perspektive begleiteten. Darüber hinaus nimmt man noch heute an
internationalen Programmen und -konferenzen teil, um Erfahrungen mit
anderen \enquote{Lernenden Städten} zu teilen, wie zum Beispiel im
UNESCO Global Network of Learning Cities (2013).

Um aber nicht nur auf Ideen von außerhalb zu setzen, wurde in Hume eine
Kultur des Lernens voneinander schon früh als eine zentrale Strategie
für richtig befunden. Auch wenn man klein anfangen musste, rentierte es
sich, die Netzwerkgruppen zum Teilen ihrer Erfahrungen und Ansichten um
Bildung und Kooperation zu bewegen. So kamen die Bildungsakteure in Hume
langsam zur Einsicht, dass zusammen tatsächlich mehr erreicht werden
kann. Damit war der Weg für größere Projekte geebnet.

Erfolgreiche Bildungsnetzwerke zeigen also, dass offene, integrative
Methoden zur Zusammenarbeit essenziell sind, aber auch ihre Grenzen
haben. Es setzt viel Expertise und Zeit voraus, alle Netzwerkteilnehmer
gleich stark einzubinden, und ihnen allen ein Podium, sowie Gestaltungs-
und Mitentscheidungsmöglichkeiten zu bieten. (Harrison, Hutton 2014,
p.~246) Dazu kommt häufig die ideelle Ambition, so viele regionale
Gruppen wie nur möglich zu beteiligen. Je größer aber der Unterschied
zwischen den Gruppen, desto größer auch der Aufwand, alle zu einem
Dialog und gegenseitigem Verständnis zu bringen. Methodischer Erfolg
setzt also Kompetenzen und Kapazitäten voraus. (Ramesh, Howlett 2017;
Goldsmith, Eggers 2004, p.~49)

Fehlen den Netzwerkmanagern diese Fähigkeiten, wird es schwierig, die
unterschiedlichen Gruppen über die lange Aufbauphase, in der noch längst
keine Erfolge abzusehen sind, zusammen und motiviert zu halten.
Kompetenzschulungen sind allerdings nicht immer eine Lösung (Ramesh,
Howlett 2017, p.~328), da jene Führungsfähigkeiten nicht allein aus
Büchern oder Workshops im Voraus erlernt werden können, sondern sich
eher erst durch den Anwendungsprozess selbst ausbilden.

Das hat zur Konsequenz, dass der einzelne Manager oder Teilnehmer ein
Bildungsnetzwerk als konstanten Lernprozess für sich selbst sehen muss.
Entsprechend dürfen sich Workshops, Trainings und Feedback nicht nur auf
die Anfangsphase beschränken. Vielmehr sollten die wichtigsten
Verantwortlichen, aber auch engagierte Teilnehmer und selbst zum
Beispiel Bibliotheksmitarbeiter fortlaufend in ihrer Selbst- und
Führungsentwicklung unterstützt werden. Plakativ kann man sagen, dass
ein Bildungsnetzwerk und seine Vertreter zuallererst selbst Lebenslange
Lerner werden müssen, bevor sie ihre Region dazu bewegen können. Im
organisationalen Kontext wird hier von \enquote{Lernenden
Organisationen} gesprochen. (Dollhausen 2007)

\hypertarget{lernende-organisationen}{%
\subsection*{\texorpdfstring{\enquote{Lernende
Organisationen}}{``Lernende Organisationen''}}\label{lernende-organisationen}}

Damit ein Bildungsnetzwerk die Werte und Kultur von Lebenslangem Lernen
durch lokale Kooperationen verbreiten kann, müssen die teilnehmenden
Bildungsinstitutionen dieses Lernen zuerst selbst in ihren Alltag
integrieren. Das betrifft sowohl Schulen, Bibliotheken, als auch
Kultureinrichtungen, Firmen und Verwaltungen. Als ein Beispiel dafür
soll an dieser Stelle nicht das HGLV, sondern die Geschichte der
nahegelegenen Gwydir Learning Region, New South Wales, herhalten.
(Wheeler et al.~2013, pp.~29--31)

Ähnlich wie in Hume wollte der Bürgermeister eine Lösung für die
niedrigen Bildungs- und Alphabetisierungsraten in der Region finden. Er
begann in der eigenen Verwaltung. Die Mitarbeiter erhielten gezielt
Kompetenzschulungen und es bildete sich eine Kultur des Lernens, die
sich in der gesamten Kommune verbreitete. So bekamen die Bürger Gwydirs
einen schärferen Sinn für ihre Zukunft und Karriere, heißt es. Lernende
Organisationen können also auch ohne Netzwerk schon einen positiven
Effekt auf ihre Umgebung haben. Wie Gwydir zeigt, sind es aber weniger
die organisatorischen Veränderungen selbst, die den Effekt
hervorbringen, sondern die Menschen und Mitarbeiter, die davon betroffen
sind.

Eine Lernende Organisation muss ihre Mitarbeiter also dazu motivieren,
selbst lernen und sich selbst verändern zu wollen. Dies setzt voraus,
dass Angestellte über die organisationalen Grenzen hinausschauen und im
Netzwerk neue Chancen sehen wollen. Intern müssen hier häufig Barrieren
überwunden werden, bevor sich die Belegschaft auf Weiterbildung oder die
Kooperation mit womöglich stereotypisierten oder als Kongruenz
angesehenen Gruppen und Institutionen einlässt. Es gilt also ähnliche
Überzeugungsarbeit und integrative Methoden anzuwenden wie im
Netzwerkmanagement oben. (Lepoluoto 2012; Dollhausen 2007, p.~6)

Entgegen den typischen Argumenten um \enquote{so haben wir das schon
immer\ldots{} ‚} oder \enquote{\ldots{} noch nie gemacht,} müssen
Betriebsstrukturen angepasst werden, schon um überhaupt Zeit und
Möglichkeit für ein Andersdenken einzurichten. Es geht weniger darum,
einfach noch mehr zu tun, als sich als Organisation auf diejenigen
Dienstleistungen und Prozesse zu fokussieren, die den stärksten
Unterschied machen. Die Bibliotheken in Hume haben zum Beispiel die
Arbeitsbeschreibungen der Mitarbeiter ausdrücklich um die Rolle des
\enquote{\emph{Learning Facilitator}} ergänzt. Anstatt nur
vorgeschriebene Dienstleistungen zu erbringen, soll der Bibliothekar
aktiv die Kommune zum Lernen bewegen und durch das Netzwerk
gemeinschaftliche Angebote vermitteln.

Veränderungen im Führungsstil und der Arbeitsstruktur können Mitarbeiter
dazu bewegen, selbst in Netzwerken zu denken und zu handeln, über den
eigenen Arbeitgeber und die Beschäftigung hinaus, hin zu einem größeren
Beitrag. (Laitinen 2015, p.~52) Das ist die Voraussetzung dafür, dass
Organisationen aus sich selbst heraus lernen, sich fortwährend
reflektieren und optimieren können. Nur diese Art von Bildungsanbieter
wird schließlich ein effektives und kooperationsbereites Mitglied eines
Bildungsnetzwerks.

\hypertarget{ein-langatmiger-prozess}{%
\subsection*{Ein langatmiger Prozess}\label{ein-langatmiger-prozess}}

Wie schon in der Einleitung dargestellt, haben sich der Bildungsstand
und die Lebensqualität in Hume verbessert, jedoch erst über ein
Jahrzehnt an Überzeugungs- und Netzwerkarbeit. (Wheeler et al.~2013,
p.~8) Obwohl man statistisch weiterhin unter dem regionalen Durchschnitt
bleibt, ist es den Schlüsselakteuren des HGLV gelungen, die politische
Kurzsichtigkeit zu überbrücken. Das ist nicht immer trivial.

Es ist schwierig, ein Netzwerk aus fast ausschließlich altruistisch
motivierten Teilnehmern über zehn Jahre mit der Vision einer besseren
und gerechteren Region zu engagieren. Vor allem die kurzen, politischen
Legislaturperioden können einem auf lange Zeit angesetzten Netzwerk
schnell den Rückhalt oder die Finanzierung zunichte machen (Longworth
{[}2014b{]}, p.~3; Goldsmith, Eggers 2004, p.~50). Warum sollte sich
ein/e Politiker/in auch für ein Projekt begeistern, das erst dann
Resultate abwirft, wenn die eigene Amtszeit schon vorüber ist!?

Auch das HGLV kam an dieser Konfrontation nicht vorbei. 2013 stellten
manche Geldgeber in Frage, weshalb die vielen kooperativen
Bildungsangebote keine erkennbare Verbesserung in der Arbeitslosenquote
bewirkt hätten. (Hume Learning Community 2015; Remplan Community 2017)
Eine Untersuchung ergab schließlich, dass die verschiedenen Programme
sehr wohl Menschen zu Bildung und Arbeit verhalfen. Das Problem lag in
der hohen Fluktuation in Hume. Wohlhabendere würden in andere Städte
abwandern, und neue Bedürftige zuziehen, sodass es immerzu einen relativ
hohen Anteil an Arbeitslosen gäbe, auch wenn es völlig andere Menschen
sind. Umstände dieser Art machen es schwierig, Erfolge nachzuweisen.

Kommunikation muss also auf Dauer überzeugen und darf die Langatmigkeit
eines Bildungsnetzwerks nicht herunterspielen. Das fällt zurück auf die
Schlüsselpersonen und die Netzwerkplanung. Sobald sich ein Wechsel von
wichtigen Politikern und anderen gesellschaftlichen Vertretern
abzeichnet, müssen potentielle Nachfolger bereits identifiziert und von
den richtigen Kanälen aus angesprochen werden. (Tippelt et al.~2014,
pp.~73--74) Idealziel ist, dass trotz Amts- oder Positionswechsel die
Unterstützung und die Kapitalflüsse für das Bildungsnetzwerk unberührt
weiterbestehen.

Für diese Art von Lobbyarbeit darf man sich nicht zu schade sein. So
positiv der Effekt von McGuires Initiative auch ist, Bildungsnetzwerke
und Lernende Städte sind keine Wunderwaffe, die soziale Ungleichheit
über Nacht beseitigen. Man muss langfristig Denken, um die Wirkung der
vielen Kleinsterfolge und Veränderungen zu bemerken, und politisch und
öffentlichkeitswirksam einzusetzen.

\hypertarget{finanzierung-von-lokalen-bildungsnetzwerken}{%
\subsection*{Finanzierung von lokalen
Bildungsnetzwerken}\label{finanzierung-von-lokalen-bildungsnetzwerken}}

So sehr sich Bildungsnetzwerke auf Ehrenamt stützen, kommen sie schon in
der Aufbauphase selten ohne externe Beratung, Managementkosten, und
Öffentlichkeitsarbeit aus. Kommunale Finanzierung und Fundraising sind
zudem häufig nur punktuell oder auf wenige Jahre ausgelegt. Sind
Strategien zur langfristigen finanziellen Nachhaltigkeit nicht schon
früh vorzeigbar, kann von Schlüsselgeldgebern und -partnern weder
Vertrauen erwartet werden, noch wird es sich unter den
Netzwerkmitgliedern selbst ausbilden. (Margerum 2002, pp.~245--246)

Das HGLV wurde durch McGuires Kontakte und Expertise auf ein solides
Fundament aus vielen unterschiedlichen Geldgebern und Unterstützern
gestellt. Neben der Stadt ist das zum Beispiel der Bundesstaat Victoria,
die Pratt Foundation, oder die in der Einleitung genannten Unternehmen
The Age und der Autohersteller Ford. (Wheeler, Osborne 2011, p.~535;
Phillips et al.~2005, p.~25) Es gilt also von Beginn an
Finanzierungsmöglichkeiten aller Art und auf allen Ebenen zu
identifizieren und auszuschöpfen. Neben Regierungsprogrammen und
Stiftungen geht es lokal um den Zugang zu Einzelpersonen mit den
richtigen Verbindungen, um Vereine oder um Partnerschaften mit
Arbeitnehmern. Zusätzlich zum Netzwerken mit Politik und Geldgebern
sollten strukturierte Evaluationsprozesse und Ergebnistransparenz das
Vertrauen in das Konzept erwecken. (Cavaye et al.~2013, p.~605)
Kosten-Nutzen-Analysen müssen aufzeigen können, dass die Investition in
Lebenslanges Lernen tatsächlich einen Unterschied für die Nötigsten
macht. (Valdes-Cotera 2015, p.~9; Hume Learning Community 2015,
pp.~4--22)

Andererseits muss eine Lernende Stadt auch kein alleiniger Kostenfaktor
sein. Nicht selten werden Bildungspartnerschaften gerade da angesetzt,
wo Sparzwang oder gar Schließung von Einrichtungen droht. Stang (2011,
pp.~19--20) berichtet zum Beispiel über deutschsprachige Bibliotheken
und Volkshochschulen, deren finanzielle Situation zu der Idee einer
gemeinsamen Raumnutzung und einer \enquote{kooperativen
Organisationsstruktur} führte. Zusammenarbeit kann also Betriebskosten
senken. Das kann für Personal und Management zutreffen, aber auch für
die Zusammenlegung von Leistungen, Infrastruktur und Räumlichkeiten.

Selbst bei der Entwicklung von Kooperationen lohnt es sich, auf
Kosteneffizienz zu pochen. Häufig resultieren Bildungspartnerschaften in
zusätzlichen Leistungen, die die Partner erbringen müssen, anstatt auf
existierenden Ressourcen und Stärken als Grundlage zu bauen. (Baumheier,
Warsewa 2010) Die Herausforderung ist, effektiv die Kostenverteilung im
Netzwerk zu moderieren, sodass der Aufwand und das Risiko nicht nur bei
wenigen Teilnehmern bleiben. Eine derartig klare Voraussicht, zusammen
mit dem richtigen Spürsinn und dem Netzwerk für Finanzierungsquellen
stellen den letzten Erfolgsfaktor dar, der aus dem HGLV und aus anderen
Lernenden Städten herauszulesen ist.

\hypertarget{fazit}{%
\section*{Fazit}\label{fazit}}

Die australische Vorstadt Hume hat es durch den Aufbau eines Netzwerks
aus unterschiedlichen Bildungsakteuren, Vereinen, Firmen, und
Privatpersonen geschafft, das Bildungsniveau ihrer multikulturellen,
häufig benachteiligten Bevölkerung anzuheben. Als Mittel und Ziel soll
fortlaufendes, Lebenslanges Lernen mehr in den Mittelpunkt der
Gesellschaft gebracht werden. Ein Bibliotheksneubau legte 2003 den
Grundstein für verschiedenste Bildungskooperationen und zieht bis heute
viel Aufmerksamkeit auf sich. Schulen, Kulturvereine und Arbeitgeber
engagieren sich in einem von der Stadt unterhaltenen Netzwerk. Es soll
die regional ärgsten Probleme durch besseren und dauerhaften Zugang zu
kooperativ gestalteten Bildungsangeboten bewältigen. Die Bibliothek
stellt dafür das Grundgerüst dar.

Aus diesem Werdegang lassen sich sechs Erfolgsfaktoren ableiten, die
auch für Bildungsnetzwerke in anderen Kontexten ausschlaggebend sind.
Die ersten zwei unabdingbaren Voraussetzungen sind lokal verankerte
Einzelpersonen mit einerseits ausgeprägten Führungsfähigkeiten und
andererseits überzeugender Kommunikationsstärke. Diese haben es nach
Jahren der Lobbyarbeit in Hume geschafft, viele Prominente, Politiker
und Fonds für ihre Idee zu gewinnen, den sozialen Herausforderungen in
Hume durch eine Bibliothek und einem Bildungsnetzwerk zu begegnen.

Neben selbstsicherer PR und der Fähigkeit, eine machbare Vision in die
Köpfe der Bürger zu setzten, will solch ein Netzwerk auch entsprechend
professionell geleitet werden. Dritter Erfolgsfaktor ist, dass das
\enquote{Hume Global Learning Village} (HGLV) bis heute Teil der
Langzeitstrategie des Stadtrats ist, organisatorisch etabliert ist und
ein fähiges Administrationsteam hat. Moderations- und Expertenkompetenz
wurde teils von außerhalb geholt, um Konsens und Vertrauen zwischen den
teilnehmenden Organisationen zu bilden. Schließlich müssen Letztere zu
aller erst lernen, von sich selbst aus in Netzwerken und Kooperationen
zu denken, da Zusammenarbeit zuvor eher selten vorkam. Vorurteile bei
Führungskräften und Mitarbeitern müssen abgebaut, und Raum zum Neudenken
von Dienstleistungen gegeben werden. \enquote{Lernende Organisationen}
sind also die vierte Voraussetzung für erfolgreiche Bildungsnetzwerke.

Diese Wandlungsprozesse brauchen viel Zeit und eine langfristige,
finanzielle Unterstützung. Das sind die letzten zwei Faktoren, denn
Lebenslanges Lernen nachhaltig in einer Region zu verbreiten und zu
verankern, kann eben nicht über Nacht und ohne Mittel geschehen.
Entsprechend müssen Bildungsnetzwerke wie das HGLV über politische
Perioden hinweg überzeugende Kosten-Nutzen-Bilanzen und den Rückhalt in
der Bevölkerung sicherstellen.

Nicht nur in Hume, sondern in vielen Regionen der Welt verändern
Bildungsnetzwerke nicht nur die Situation der Bevölkerung zum Besseren,
sondern auch die der Bildungsakteure. Das \enquote{Hume Global Learning
Center} (HGLC) und ihre Bibliothek ist in aller Munde und wurde zum
zentralen Treffpunkt vieler lokaler Einwohnergruppen. Humes Bibliotheken
wurden 2015 zu den besten Australiens gewählt. (Hume City Council 2015)
Dorthin gekommen sind sie jedoch nicht nur durch Eigeninitiative,
sondern vielmehr durch ein nachhaltiges Netzwerk. Die sechs diskutierten
Erfolgsfaktoren des HGLV sollen Bibliotheken und anderen bildungsnahen
Institutionen praktische Ansätze geben, um den Herausforderungen der
Lokalbevölkerung gemeinsam zu begegnen.

\hypertarget{bibliographie}{%
\section*{Bibliographie}\label{bibliographie}}

Aring, Jürgen (2014): Integration hoch zwei. Bildungslandschaften und
Stadtentwicklung verknüpfen. In \emph{Forum Wohnen und Stadtentwicklung}
(3), pp.~115--120. Online verfügbar unter:
\url{http://www.vhw.de/fileadmin/user_upload/08_publikationen/verbandszeitschrift/2000_2014/PDF_Dokumente/2014/3_2014/FWS_3_2014_Integration_hoch_zwei_-_Bildungsland___schaft.pdf}.

Baumheier, Ulrike; Warsewa, Günter (2010): Lokale Bildungslandschaften.
Stadtteilnetzwerke für Bildung und soziale Integration. In
\emph{sozialraum.de}. Online verfügbar unter:
\url{http://www.sozialraum.de/lokale-bildungslandschaften.php}.

Bentley, Tom (2009): Innovation and Diffusion as a Theory of Change. In
Andy Hargreaves, Ann Lieberman, Michael Fullan, David Hopkins (Eds.):
Second International Handbook of Educational Change. Dordrecht: Springer
Netherlands, pp.~29--46.

Brandt, Peter; Maas, Theresa (2015): \enquote{Bildungsmanagement zur
politischen Priorität erklären!}. im Gespräch mit den Architekten der
kommunalen Bildungslandschaft in Trier. Interview with Klaus Jensen,
Rudolf Fries. Trier. DIE Zeitschrift für Erwachsenenbildung.

Brookfield, Stephen (2012): The Impact of Lifelong Learning on
Communities. In David N. Aspin, Judith Chapman, Karen Evans, Richard
Bagnall (Eds.): Second International Handbook of Lifelong Learning.
Dordrecht: Springer Netherlands, pp.~875--886.

Cavaye, Jim; Wheeler, Leone; Wong, Shanti; Simmons, Jan; Herlihy, Paula;
Saleeba, Jim (2013): Evaluating the community outcomes of Australian
learning community initiatives. Innovative approaches to assessing
complex outcomes. In \emph{Community Development} 44 (5), pp.~597--607.
\url{https://doi.org/10.1080/15575330.2013.853681}.

Cepiku, Denita. (2017): Collaborative Governance. In Thomas Richard
Klassen, Denita Cepiku, T. J. Lah (Eds.): The Routledge handbook of
global public policy and administration. New York: Routledge (Routledge
international handbooks).

Dollhausen, Karin (2007): \enquote{Lernende Organisationen} als
Bezugspunkt der erwachsenenpädagogischen~Organisationsforschung? In
Karin Dollhausen (Ed.): Bildungseinrichtungen als \enquote{lernende
Organisationen}? Befunde aus der Weiterbildung. 1. Aufl. Wiesbaden:
Deutscher Universitäts-Verlag (Sozialwissenschaft), pp.~1--15.

Duke, Chris (2012): Networking and Partnerships. Another Road to
Lifelong Learning. In David N. Aspin, Judith Chapman, Karen Evans,
Richard Bagnall (Eds.): Second International Handbook of Lifelong
Learning. Dordrecht: Springer Netherlands, pp.~829--843.

Faris, Ron (2006): Learning Cities. Lessons Learned. In support of the
Vancouver Learning City Initiative. Available online at
\url{http://www.resdac.net/documentation/pdf/forum_aga/2012/en/Learning_Cities.pdf}
.

Florida, Richard (2003): Cities and the Creative Class. In \emph{City \&
Community} 2 (1), pp.~3--19.
\url{https://doi.org/10.1111/1540-6040.00034}.

Goldsmith, Stephen; Eggers, William D. (2004): Governing by network. The
new shape of the public sector. Washington, D.C.: Brookings Institution
Press.

Harrison, Andrew; Hutton, Les (2014): Design for the changing
educational landscape. Space, place and the future of learning. London:
Routledge.

HGLV (n.d.): Inclusion. Global Learning Village. Project History. Hume
Global Learning Village (HGLV). Melbourne. Online verfügbar unter:
\url{http://www.glv.org.au/learning-center/inclusion} .

Hume City Council ({[}2012{]}): Community Engagement Framework and
Planning Guide. Hume. Online verfügbar unter:
\url{http:://www.hume.vic.gov.au/files/51cbbebe-df6c-453c-ae26-9f2800edb59f/Community_Engagement_Framework.pdf}.

Hume City Council (2015): We're Australia's Favourite Library Service!
Hume City Council. Hume. Online verfügbar unter:
\url{https://www.hume.vic.gov.au/Libraries_Learning/Hume_Libraries/Were_Australias_Favourite_Library_Service}.

Hume Learning Community (2015): Demonstrating the Impact of Projects and
Programs. Broadmeadows. Online verfügbar unter:
\url{http://pascalobservatory.org/pascalnow/blogentry/measuring-impact-community-development-programs-city-hume}.

Johnson, Chris; Lomas, Cyprien (2005): Design of the Learning Space.
Learning and Design Principles. In \emph{EDUCAUSE Review} 40 (4),
pp.~16--28. Online verfügbar unter:
\url{http://er.educause.edu/articles/2005/1/design-of-the-learning-space-learning-and-design-principles}.

Kearns, Peter (2015): Learning cities on the move. In \emph{Australian
Journal of Adult Learning} 55 (1), pp.~153--168.

Laitinen, Ilpo (2015): Adult and Lifelong Learning as Tools for
Organisational Adaptation and Change Management. In \emph{Journal of
Adult and Continuing Education} 21 (2), pp.~38--54. DOI:
10.7227/JACE.21.2.4.

Landry, Charles (2008): The creative city. A toolkit for urban
innovators. 2nd ed. London: Earthscan.

Lepoluoto, Pälvi (2012): In the footsteps of a learning organization.
Finland. In \emph{Scandinavian Library Quaterly} 45 (1). Online
verfügbar unter: \url{http://slq.nu/?article=volume-45-no-1-2012-13}.

Longworth, Norman (2006): Learning cities, learning regions, learning
communities. Lifelong learning and local government. London, New York:
Routledge.

Longworth, Norman ({[}2014a{]}): Changing Perceptions, Changing Cities,
Changing Economies. A historical review of the development of learning
cities and learning regions. RSA, City Growth Commission. Online
verfügbar unter:
\url{https://www.thersa.org/globalassets/pdfs/city-growth-commission/evidence/learning-cities-and-regions-historical-review.pdf}.

Longworth, Norman ({[}2014b{]}): Learning City Dimensions. City Growth
Commission. Online verfügbar unter:
\url{https://www.thersa.org/globalassets/pdfs/city-growth-commission/evidence/learning-city-dimensions.pdf}.

Margerum, Richard D. (2002): Collaborative Planning. Building Consensus
and Building a Distinct Model for Practice. In \emph{Journal of Planning
Education and Research} 21 (3), pp.~237--253.

Margerum, Richard D. (2011): Beyond consensus. Improving collaborative
planning and management. Cambridge, Mass., London: MIT Press.

O'Hagan, Mathew (2014): Making a case for a Collective Impact approach
towards raising educational achievement in The City of Hume. EDUC90148:
Project in Educational Leadership. PASCAL International Observatory
(601181). Online verfügbar unter:
\url{http://pascalobservatory.org/sites/default/files/educ90148_project_in_educational_leadership_ohagan_601181-1_0.pdf}.

Phillips, Ian; Wheeler, Leone; White, Kimbra (2005): Hume Global
Learning Village. Learning Together Strategy 2004/2008. HUME GLOBAL
LEARNING VILLAGE EVALUATION REPORT Final. Adult Learning Australia
(ALA). Canberra. Online verfügbar unter:
\url{http://www.resdac.net/communautesapprenantes/documents/1-communautes-apprenantes-de-la-comprehension-aux-enjeux/e-quelques-exemples-de-projets-de-communautes-apprenantes/hume-global-learning-village-learning-together-strategy-2004-2008.pdf}.

Ramesh, M.; Howlett, M. (2017): The role of policy capacity in policy
success and failure. In Thomas Richard Klassen, Denita Cepiku, T. J. Lah
(Eds.): The Routledge handbook of global public policy and
administration. New York: Routledge (Routledge international handbooks),
pp.~319--334.

Remplan Community (2017): Hume Community Profile. Unemployment Rate.
Online verfügbar unter:
\url{http://www.communityprofile.com.au/hume/trends/unemployment-by-year/\#!trendline;i=0;b=AAgB}.

Schäffter, Ortfried (2004): Auf dem Weg zum Lernen in Netzwerken.
Institutionelle Voraussetzungen für lebensbegleitendes Lernen. In Rainer
Brödel (Ed.): Weiterbildung als Netzwerk des Lernens. Differenzierung
der Erwachsenenbildung. Bielefeld: W. Bertelsmann (Erwachsenenbildung
und lebensbegleitendes Lernen), pp.~29--48.

Stang, Richard (2011): Strukturen und Leistungen von Lernzentren.
Empirische Befunde und Perspektiven zur Entwicklung von kommunalen
Lernzentren als innovative Institutionalformen für Lebenslanges Lernen.
Deutsches Institut für Erwachsenenbildung - Leibniz-Zentrum für
Lebenslanges Lernen (DIE). Online verfügbar unter:
\url{http://www.die-bonn.de/doks/2011-lernzentrum-01.pdf}, checked on
12/04/2018.

Stang, Richard; Eigenbrodt, Olaf (2014): Informations- und Wissensräume
der Zukunft. Von Hochgefühlen und lernenden Städten. In Olaf Eigenbrodt,
Richard Stang (Eds.): Formierungen von Wissensräumen. Optionen des
Zugangs zu Information und Bildung. Berlin, Boston: DE GRUYTER (Age of
access? Grundfragen der Informationsgesellschaft, 3).

Tippelt, Rudolf (2015): Stichwort: \enquote{Bildungslandschaften}. In
\emph{DIE Zeitschrift für Erwachsenenbildung} (4), pp.~20--21.
\url{https://doi.org/10.3278/DIE1504W}.

Tippelt, Rudolf; Kadera, Stepanka; Buschle, Christina (2014):
Interorganisationale Kooperation zur Förderung des lebenslangen Lernens.
In \emph{Zeitschrift für Erziehungswissenschaft} 17 (S5), pp.~65--78.
\url{https://doi.org/10.1007/s11618-014-0548-8}.

Torfing, Jacob (2016): Collaborative innovation in the public sector.
Washington, DC: Georgetown University Press (Public management and
change series).

UNESCO Institute for Lifelong Learning (2013): Key Features of Learning
Cities. Adopted at the International Conference on Learning Cities
Beijing, China, October 21--23, 2013. Beijing. Available online at
\url{http://unesdoc.unesco.org/images/0022/002267/226756e.pdf}.

Valdes-Cotera, Raúl (2015): Guideline for building learning cities.
UNESCO Institute for Lifelong Learning. Available online at
\url{http://unesdoc.unesco.org/images/0023/002349/234987e.pdf}.

Volkmann, Stefan (2016): Stakeholder Interactions in Learning City
Projects. An investigation into good-case project practice and
stakeholder interaction of institutional learning space development and
urban planning for education-centered urban development. Bachelor
thesis. Media University, Stuttgart.

Watson, Connie; Wu, Aimee Tiu (2015): Evolution and Reconstruction of
Learning Cities for Sustainable Actions. In \emph{New Directions for
Adult and Continuing Education} 2015 (145), pp.~5--19.
https://doi.org/10.1002/ace.20119.

Wheeler, Leone; Osborne, George (2011): Hume global learning village. a
creative learning community. 30. Northern Partnerships Unit. Melbourne.
Online verfügbar unter:
\url{http://apo.org.au/system/files/27412/apo-nid27412-55616.pdf},
checked on 12/04/2018.

Wheeler, Leone; Wong, Shanti; Farrell, J.; Wong, I. (2013): Learning as
a Driver for Change. Australian Centre of Excellence for Local
Government. Sydney. Online verfügbar unter:
\url{https://www.uts.edu.au/sites/default/files/ACELG_Report_Learning_as_Driver_for_Change_03062013.pdf}.

%autor
\begin{center}\rule{0.5\linewidth}{\linethickness}\end{center}

\textbf{Stefan Volkmann} arbeitet für AIESEC in Denmark.

\end{document}
