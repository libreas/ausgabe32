\documentclass[a4paper,
fontsize=11pt,
%headings=small,
oneside,
numbers=noperiodatend,
parskip=half-,
bibliography=totoc,
final
]{scrartcl}

\usepackage{synttree}
\usepackage{graphicx}
\setkeys{Gin}{width=.4\textwidth} %default pics size

\graphicspath{{./plots/}}
\usepackage[ngerman]{babel}
\usepackage[T1]{fontenc}
%\usepackage{amsmath}
\usepackage[utf8x]{inputenc}
\usepackage [hyphens]{url}
\usepackage{booktabs} 
\usepackage[left=2.4cm,right=2.4cm,top=2.3cm,bottom=2cm,includeheadfoot]{geometry}
\usepackage{eurosym}
\usepackage{multirow}
\usepackage[ngerman]{varioref}
\setcapindent{1em}
\renewcommand{\labelitemi}{--}
\usepackage{paralist}
\usepackage{pdfpages}
\usepackage{lscape}
\usepackage{float}
\usepackage{acronym}
\usepackage{eurosym}
\usepackage[babel]{csquotes}
\usepackage{longtable,lscape}
\usepackage{mathpazo}
\usepackage[normalem]{ulem} %emphasize weiterhin kursiv
\usepackage[flushmargin,ragged]{footmisc} % left align footnote
\usepackage{ccicons} 

%%%% fancy LIBREAS URL color 
\usepackage{xcolor}
\definecolor{libreas}{RGB}{112,0,0}

\usepackage{listings}

\urlstyle{same}  % don't use monospace font for urls

\usepackage[fleqn]{amsmath}

%adjust fontsize for part

\usepackage{sectsty}
\partfont{\large}

%Das BibTeX-Zeichen mit \BibTeX setzen:
\def\symbol#1{\char #1\relax}
\def\bsl{{\tt\symbol{'134}}}
\def\BibTeX{{\rm B\kern-.05em{\sc i\kern-.025em b}\kern-.08em
    T\kern-.1667em\lower.7ex\hbox{E}\kern-.125emX}}

\usepackage{fancyhdr}
\fancyhf{}
\pagestyle{fancyplain}
\fancyhead[R]{\thepage}

% make sure bookmarks are created eventough sections are not numbered!
% uncommend if sections are numbered (bookmarks created by default)
\makeatletter
\renewcommand\@seccntformat[1]{}
\makeatother


\usepackage{hyperxmp}
\usepackage[colorlinks, linkcolor=black,citecolor=black, urlcolor=libreas,
breaklinks= true,bookmarks=true,bookmarksopen=true]{hyperref}
%URLs hart brechen
\makeatletter 
\g@addto@macro\UrlBreaks{ 
  \do\a\do\b\do\c\do\d\do\e\do\f\do\g\do\h\do\i\do\j 
  \do\k\do\l\do\m\do\n\do\o\do\p\do\q\do\r\do\s\do\t 
  \do\u\do\v\do\w\do\x\do\y\do\z\do\&\do\1\do\2\do\3 
  \do\4\do\5\do\6\do\7\do\8\do\9\do\0} 
% \def\do@url@hyp{\do\-} 
\makeatother 

%meta
%meta

\fancyhead[L]{Thomas Hartmann \\ %author
LIBREAS. Library Ideas, 32 (2017). % journal, issue, volume.
\href{http://nbn-resolving.de/}
{}} % urn 
% recommended use
%\href{http://nbn-resolving.de/}{\color{black}{urn:nbn:de...}}
\fancyhead[R]{\thepage} %page number
\fancyfoot[L] {\ccLogo \ccAttribution\ \href{https://creativecommons.org/licenses/by/3.0/}{\color{black}Creative Commons BY 3.0}}  %licence
\fancyfoot[R] {ISSN: 1860-7950}

\title{\LARGE{Zwang zum Open Access-Publizieren? Der rechtliche Präzedenzfall ist schon da!}} % title
\author{Thomas Hartmann} % author

\setcounter{page}{1}

\hypersetup{%
      pdftitle={Zwang zum Open Access-Publizieren? Der rechtliche Präzedenzfall ist schon da!},
      pdfauthor={Thomas Hartmann},
      pdfcopyright={CC BY 3.0 Unported},
      pdfsubject={LIBREAS. Library Ideas, 32 (2017).},
      pdfkeywords={Open Access, Zweit\-ver\-öffent\-lich\-ungs\-recht, Universität Konstanz, Mandate},
      pdflicenseurl={https://creativecommons.org/licenses/by/3.0/},
      pdfcontacturl={http://libreas.eu},
      baseurl={http://libreas.eu},
      pdflang={de},
      pdfmetalang={de}
     }



\date{}
\begin{document}

\maketitle
\thispagestyle{fancyplain} 

%abstracts

%body
\hypertarget{uxfcberblick}{%
\subsubsection{Überblick}\label{uxfcberblick}}

Lange Zeit war es hierzulande in Wissenschaft, Politik und Recht weithin
Konsens, dass Wissenschaftler/innen Open Access publizieren
\emph{können} aber nicht dazu von Rechts wegen verpflichtet sein sollen.
Diese Auffassung vertritt aus mannigfaltigen Gründen auch der Verfasser
dieses Beitrags.\footnote{\emph{Hartmann} (2017), Open Access rechtlich
  absichern -- warum es ein Opt-in braucht. In: Söllner/Mittermaier
  (Hrsg.), Praxishandbuch Open Access, S. 45-52, Berlin, De Gruyter
  Saur. Als Zweitveröffentlichung abrufbar unter
  \url{http://hdl.handle.net/11858/00-001M-0000-002D-6F30-3}.}

Neuland betreten hat im Frühjahr 2014 die Wissenschaftsministerin in
Baden-Württemberg. Nach dem neuen Landeshochschulgesetz\footnote{§ 44
  Abs. 6 Landeshochschulgesetz Baden-Württemberg (LHG BaWü) in der
  Fassung vom 01.04.2014 lautet: \enquote{Die Hochschulen sollen die
  Angehörigen ihres wissenschaftlichen Personals durch Satzung
  verpflichten, das Recht auf nichtkommerzielle Zweitveröffentlichung
  nach einer Frist von einem Jahr nach Erstveröffentlichung für
  wissenschaftliche Beiträge wahrzunehmen, die im Rahmen der
  Dienstaufgaben entstanden und in einer periodisch mindestens zweimal
  jährlich erscheinenden Sammlung erschienen sind. Die Satzung regelt
  die Fälle, in denen von der Erfüllung der Pflicht nach Satz 1
  ausnahmsweise abgesehen werden kann. Sie kann regeln, dass die
  Zweitveröffentlichung auf einem Repositorium nach § 28 Absatz 3 zu
  erfolgen hat.}} (LHG BaWü) \emph{müssen} die Hochschulangehörigen ihre
wissenschaftlichen Beiträge frei zugänglich zweitveröffentlichen.

Eine rechtliche Verpflichtung zur Open Access-Zweitveröffentlichung
verankerte als bundesweit erste Hochschule im Dezember 2015 die
Universität Konstanz. Sie gibt ihren Hochschulangehörigen seitdem in der
\enquote{Satzung zur Ausübung des wissenschaftlichen
Zweit\-ver\-öffent\-lich\-ungs\-rechts gemäß § 38 Abs. 4 UrhG} ein Open
Access-Mandat vor.

Dagegen sind 17 Professoren/innen der Universität Konstanz vor Gericht
gezogen. Der Verwaltungsgerichtshof Baden-Württemberg hat die
Normenkontrollklage (Aktenzeichen 9 S 2056/16) im September 2016 dem
Bundesverfassungsgericht vorgelegt. Die obersten Verfassungshüter in
Karlsruhe könnten so (endlich) den Rahmen für grundlegende Fragen von
Open Access, (elektronischem) Publizieren und Urheberrecht abstecken.
Zutreffend kommentiert daher \emph{Ulrich Rüdiger}, Rektor der
Universität Konstanz: \enquote{Die rechtliche Prüfung wird eine
entscheidende Weichenstellung für den Bereich Open Access in der
Wissenschaft insgesamt sowie für das Landeshochschulgesetz
Baden-Württemberg darstellen.}\footnote{Vgl. Pressemitteilung Nr.
  90/2016 der Universität Konstanz vom 21.11.2016, abrufbar unter
  \url{https://www.uni-konstanz.de/universitaet/aktuelles-und-medien/aktuelle-meldungen/aktuelles/aktuelles/open-access-satzung-auf-juristischem-pruefstand/}.}

Im Folgenden soll auf die zwölf häufigsten Rechtsfragen eingegangen
werden, welche sich den Hochschulen und in der Wissenschaft insgesamt
dazu stellen.

\hypertarget{was-genau-beinhaltet-die-zweitveruxf6ffentlichungspflicht-in-baden-wuxfcrttemberg}{%
\subsubsection{\texorpdfstring{Was genau beinhaltet die
Zweitveröffentlichungs\emph{pflicht} in
Baden-Württemberg?}{Was genau beinhaltet die Zweit\-ver\-öffent\-lich\-ungs\-pflicht in Baden-Württemberg?}}\label{was-genau-beinhaltet-die-zweitveruxf6ffentlichungspflicht-in-baden-wuxfcrttemberg}}

Nach dem im Frühjahr 2014 eingeführten § 44 Absatz 6 LHG BaWü sollen die
Hochschulen ihre Angehörigen dazu verpflichten, wissenschaftliche
Beiträge\footnote{Um welche Beiträge es sich handelt und in welchem
  Rahmen diese zweitveröffentlicht werden dürfen, siehe unten bei
  \enquote{Was hat die strittige Zweit\-ver\-öffent\-lich\-ungs\-pflicht in
  Baden-Württemberg mit dem Zweit\-ver\-öffent\-lich\-ungs\-recht des Bundes zu
  tun?}.} ein Jahr nach Erscheinen zweitzuveröffentlichen. Betroffen
sind Beiträge, die im Rahmen der Dienstaufgaben erstellt wurden. Mit der
Satzung können die Hochschulen vorsehen, dass die Zweitveröffentlichung
auf dem hochschuleigenen oder einem anderen bestimmten Repositorium
erfolgt. Ferner können die Hochschulen in eigener Regie Ausnahmen von
der Zweit\-ver\-öffent\-lich\-ungs\-pflicht festlegen. Die Universität Konstanz
hat in der strittigen Satzung in einem eigenen Paragraphen solche
Ausnahmefälle und ein Verfahren dafür definiert.\footnote{§ 4 Ausnahmen
  von der Zweitveröffentlichung

  (1) Von einer Zweitveröffentlichung kann abgesehen werden, wenn diese
  berechtigte Interessen der gemäß § 1 Verpflichteten verletzt. Dies ist
  insbesondere dann der Fall, wenn

  a) die erstveröffentlichten Erkenntnisse zwischenzeitlich überholt
  sind oder sich als falsch erwiesen haben,

  b) die Erstpublikation aufgrund gesetzlicher Vorschriften
  zurückgezogen worden ist,

  c) die Publikation Rechte Dritter verletzt oder

  d) die erstveröffentlichten Erkenntnisse bereits auf einem von einer
  Universität oder von einer Forschungseinrichtung betriebenen
  Repositorium zweitveröffentlicht worden sind und eine entsprechende
  Mitteilung gemäß § 3 Abs. 1 erfolgt ist und ein entsprechender Verweis
  im Repositorium KOPS eingetragen ist.

  (2) Ausnahmen von einer Zweitveröffentlichung von wissenschaftlichen
  Beiträgen, in denen der Autor oder die Autorin ein
  Zweit\-ver\-öffent\-lich\-ungs\-recht gemäß § 38 Abs. 4 UrhG hat, sind beim
  Ausschuss für Kommunikation und Information (AKI) mit der Meldung
  gemäß § 3 zu beantragen und zu begründen.

  Aus: Universität Konstanz, Satzung zur Ausübung des wissenschaftlichen
  Zweit\-ver\-öffent\-lich\-ungs\-rechts gemäß § 38 Abs. 4 UrhG. Abrufbar via
  \url{https://www.uni-konstanz.de/universitaet/aktuelles-und-medien/aktuelle-meldungen/aktuelles/aktuelles/open-access-satzung-auf-juristischem-pruefstand/}.}

\hypertarget{was-hat-die-strittige-zweitveruxf6ffentlichungspflicht-in-baden-wuxfcrttemberg-mit-dem-zweitveruxf6ffentlichungsrecht-des-bundes-zu-tun}{%
\subsubsection{\texorpdfstring{Was hat die strittige
Zweitveröffentlichungs\emph{pflicht} in Baden-Württemberg mit dem
Zweitveröffentlichungs\emph{recht} des Bundes zu
tun?}{Was hat die strittige Zweit\-ver\-öffent\-lich\-ungs\-pflicht in Baden-Württemberg mit dem Zweit\-ver\-öffent\-lich\-ungs\-recht des Bundes zu tun?}}\label{was-hat-die-strittige-zweitveruxf6ffentlichungspflicht-in-baden-wuxfcrttemberg-mit-dem-zweitveruxf6ffentlichungsrecht-des-bundes-zu-tun}}

Die Präambel und der Titel der Satzung nehmen ausdrücklich Bezug auf das
seit Januar 2014 nach § 38 Abs. 4 Urheberrechtsgesetz (UrhG)
gewährleistete Recht auf Zweitveröffentlichung. Nach den Worten der
Präambel klingt es nur folgerichtig, wenn Bundesländer und Hochschulen
die Stärkung wissenschaftlicher Autoren/innen des § 38 Abs. 4 UrhG
aufgreifen: \enquote{Die Erkenntnisse wissenschaftlicher Forschung
sollten möglichst frei zugänglich sein. Das Landeshochschulgesetz nimmt
deshalb in § 44 Abs. 6 LHG BaWü den Open Access-Gedanken in der Form
auf, dass die Hochschulen ihre Wissenschaftlerinnen und Wissenschaftler
durch Satzung verpflichten sollen, das Recht auf Zweitveröffentlichung,
das seit Januar 2014 nach § 38 Abs. 4 UrhG gewährleistet ist, auch
tatsächlich in Anspruch zu nehmen.}\footnote{Universität Konstanz,
  Satzung zur Ausübung des wissenschaftlichen
  Zweit\-ver\-öffent\-lich\-ungs\-rechts gemäß § 38 Abs. 4 UrhG, S. 2.}

Wissenschaftspolitisch liegen zwischen dem Zweit\-ver\-öffent\-lich\-ungs\-recht
nach § 38 Abs. 4 UrhG und der Zweit\-ver\-öffent\-lich\-ungs\-pflicht des § 44
Abs. 6 LHG BaWü Welten. Der Bund hat im Urheberrechtsgesetz seit 2014
festgelegt, dass wissenschaftliche Autoren/innen unter bestimmten
Vorgaben wissenschaftliche Beiträge zweitveröffentlichen \emph{können}.
In Baden-Württemberg hingegen \emph{müssen} nun Hochschulangehörige eine
frei zugängliche Zweitveröffentlichung vornehmen.

Ferner verweisen, nicht ohne rechtliche Finesse, das Land
Baden-Württemberg und die Universität Konstanz umfassend auf die
Vorgaben des § 38 Abs. 4 UrhG: Nur falls ein Zweit\-ver\-öffent\-lich\-ungs\-recht
nach § 38 Abs. 4 UrhG besteht, muss es gemäß § 44 Abs. 6 LHG BaWü
beziehungsweise nach den Hochschulsatzungen auch ausgeübt werden. An
entscheidender Stelle aber verkehrt das Land Baden-Württemberg mit § 44
Abs. 6 LHG BaWü den Willen des Bundesgesetzgebers: Denn nach § 44 Abs. 6
LHG BaWü sind vor allem die Professoren/innen dazu verpflichtet, ihre
\enquote{im Rahmen der Dienstaufgaben} entstandenen wissenschaftlichen
Beiträge zweitzuveröffentlichen. Die Bundesregierung hatte indes in die
Gesetzesbegründung\footnote{Vgl. Bundestag-Drucksache 17/13423, S. 9.
  Abrufbar unter
  \url{http://dip21.bundestag.de/dip21/btd/17/134/1713423.pdf}.} zu § 38
Abs. 4 UrhG ausdrücklich aufgenommen, dass das
Zweit\-ver\-öffent\-lich\-ungs\-recht nur für Publikationen aus staatlichen
Projektförderungen, nicht aber für die grundfinanzierte universitäre
Forschung gelten soll.\footnote{Vgl. zur Begründung und rechtlichen
  Wirksamkeit der Differenzierung nach staatlich grundfinanzierter
  Forschung und staatlicher Förderung mit Forschungsprojekten
  insbesondere die Fragen Nr. 15 bis 21 des FAQ zum
  Zweit\-ver\-öffent\-lich\-ungs\-recht der Schwerpunktinitiative
  \enquote{Digitale Information} der Allianz der deutschen
  Wissenschaftsorganisationen, abrufbar unter
  \url{http://www.allianzinitiative.de/handlungsfelder/rechtliche-rahmenbedingungen/faq-zvr.html}.}

Zu den weiteren einzelnen, teils umstrittenen Vorgaben des
Zweit\-ver\-öffent\-lich\-ungs\-recht in § 38 Abs. 4 UrhG hat das
\enquote{Aktionsbündnis Urheberrecht für Bildung und Wissenschaft} einen
Flyer\footnote{Siehe
  \url{http://urheberrechtsbuendnis.de/docs/zvr-folder-2015-a4.pdf}.}
erstellt, zur Vertiefung sei darauf verwiesen, dass zahlreiche
Rechtsfragen in einem ein FAQ\footnote{Siehe
  \url{http://www.allianzinitiative.de/handlungsfelder/rechtliche-rahmenbedingungen/faq-zvr.html}.}
zum Zweit\-ver\-öffent\-lich\-ungs\-recht der Schwerpunktinitiative
\enquote{Digitale Information} der Allianz der deutschen
Wissenschaftsorganisationen behandelt werden.\footnote{Siehe dazu ferner
  Vortragsaufzeichnung \enquote{Neues gesetzliches
  Zweit\-ver\-öffent\-lich\-ungs\-recht -\/- Update zu den Anforderungen an
  Bibliotheken und Wissenschaftseinrichtungen} in Session
  \enquote{Rechtliche Aspekte des Open Access} auf den Open-Access-Tagen
  2013 Hamburg, abrufbar unter \url{https://youtu.be/DmIh_orxRX8}.}

\hypertarget{wer-steht-vor-gericht}{%
\subsubsection{Wer steht vor Gericht?}\label{wer-steht-vor-gericht}}

(Muster-)Kläger sind 17 Professoren/innen der Universität Konstanz. Mit
einer Ausnahme stammen sie alle aus der juristischen Fakultät. Die
Normenkontrollklage wird unterstützt von der Professorenvereinigung
Deutscher Hochschulverband (DHV).\footnote{Siehe
  \url{https://www.boersenblatt.net/artikel-boersenverein_zu_open-access-regelung_.1398729.html}.}
Die Kläger vertritt vor Gericht der Bonner Jura-Professor \emph{Klaus
Gärditz}.

Beklagt wird die Universität Konstanz, die \emph{Alexander Peukert},
Jura-Professor an der Goethe Universität Frankfurt vor Gericht
verteidigt. Zu Peukerts rechtswissenschaftlichen For\-schungs\-schwer\-punkten
im Exzellenzcluster Normative Ordnung zählen Gemeinfreiheit,
Zugangsrecht und das Urheberrecht im Wandel des wissenschaftlichen
Kommunikationssystems.

Brisant also: Einzelne Professoren/innen verklagen ihre eigene
Universität wegen einer Satzung, die in den Universitätsgremien (mehr
dazu unten) verabschiedet wurde. Zudem sind auf beiden Seiten des
Prozesses Jura-Professoren die Hauptakteure, die offenbar auch ein
grundlegend unterschiedliches Verständnis des Publikationssystems in der
Jurisprudenz pflegen.

Eine Reportage zur Verhandlung am Verwaltungsgerichtshof Mannheim ist am
27.09.2017 erschienen bei der Legal Tribune Online.\footnote{\url{https://www.lto.de/recht/hintergruende/h/vgh-mannheim-normenkontrollantrag-9-s-2056-16-professoren-universitaet-konstanz-open-access-wissenschaft-urheberrecht/}.}

\hypertarget{was-motiviert-professoreninnen-gegen-open-access-zu-klagen}{%
\subsubsection{Was motiviert Professoren/innen, gegen Open Access zu
klagen?}\label{was-motiviert-professoreninnen-gegen-open-access-zu-klagen}}

Rechtlich wird das Normenkontrollverfahren von den 17 Professoren/innen
vor allem angestrengt, weil sie einen unzulässigen Eingriff in das
Grundrecht der Wissenschaftsfreiheit (Art. 5 Abs. 3 Grundgesetz)
beanstanden.

Einer der 17 klagenden Professoren/innen ist Christoph Schönberger,
Professor für öffentliches Recht an der Universität Konstanz. Er
begründet die Klage gegenüber dem Deutschlandfunk im Januar 2017
folgendermaßen:

\begin{quote}
\enquote{Wir, also die Kollegen, die sich beteiligt haben, sind nicht
grundsätzlich gegen Open Access. (\ldots{}) Wir entscheiden ja bis jetzt
selbst, in welchen Publikationsformaten, in welchen Zeitschriften, wir
unsere Ergebnisse publizieren. Und nach unserer Meinung gehört es zu
unserer Wissenschaftsfreiheit, dass auch selbst entscheiden zu können.
(\ldots{}) In dem Augenblick, in dem wir gezwungen werden, die
Ergebnisse in irgendeiner Form auf einem Server bereit zu legen, geben
wir diese Hoheit über diese Forschungsergebnisse, über die Art, wie wir
sie präsentieren wollen, auf. Das möchten wir nicht. (\ldots{}) Es ist
für unsere Reputation bedeutsam, in welcher Zeitschrift wir publizieren.
Wir sind darauf angewiesen, dass es diese qualifizierten Fachverlage
gibt. Und der Fachverlag ist natürlich in großen Schwierigkeiten, für
jedes Forschungsvorhaben hinzunehmen, dass es nach einer gewissen Zeit
von der Universität frei veröffentlicht wird.}\footnote{Zitiert nach:
  Deutschlandfunk vom 25.01.2017: \enquote{Open Access: Professoren
  klagen gegen kostenfreie Artikel-Zweitnutzung}. Abrufbar unter
  \url{http://www.deutschlandfunk.de/open-access-professoren-klagen-gegen-kostenfreie-artikel.680.de.html?dram:article_id=377280}.}
\end{quote}

Nachdem die Universitätssatzung im Senat der Universität Konstanz Ende
2015 verabschiedet worden war, hatte der universitätseigene Fachbereich
Rechtswissenschaft \enquote{einmütig} am 01.02.2016 ein Protestschreiben
an den Rektor der Universität Konstanz geschickt. Darin erläutern die
Jura-Professoren ihre Kritik an der satzungsgemäßen Open
Access-Zweit\-ver\-öffent\-lich\-ungs\-pflicht und fordern schließlich ihre
Universität \enquote{als lernende Organisation} dazu auf, die
Möglichkeit einer Selbstkorrektur zu nutzen und \enquote{die Satzung
schnellstmöglich aufzuheben}.\footnote{Das Schreiben des Fachbereichs
  Rechtswissenschaft der Universität Konstanz an den Rektor derselben
  Universität ist vollständig veröffentlicht in: Ordnung der
  Wissenschaft, Nr. 2/2016, S. 135 f. (Editorial). Abrufbar unter
  \url{http://www.ordnungderwissenschaft.de/2016-2/07_Infoteil/14_loewisch_konstanzer-juristenfakultaet_verweigert_zweitveroeffentlichungspflicht_odw_2016.pdf}.}

\hypertarget{weshalb-steht-eine-universituxe4tssatzung-im-zentrum-der-aufmerksamkeit}{%
\subsubsection{\texorpdfstring{Weshalb steht eine
\emph{Universitätssatzung} im Zentrum der
Aufmerksamkeit?}{Weshalb steht eine Universitätssatzung im Zentrum der Aufmerksamkeit?}}\label{weshalb-steht-eine-universituxe4tssatzung-im-zentrum-der-aufmerksamkeit}}

Die Verpflichtung zu einer frei zugänglichen Zweitveröffentlichung
ergibt sich für die Hochschulangehörigen in Konstanz aus der
\enquote{Satzung zur Ausübung des wissenschaftlichen
Zweit\-ver\-öffent\-lich\-ungs\-rechts gemäß § 38 Abs. 4 UrhG}. Eine solche
Universitätssatzung entfaltet für die Universitätsangehörigen eine
vergleichbare, vor allem rechtsverbindliche Wirkung wie Gesetze, darf
also nicht unterschätzt werden.

Die Parallelen zwischen Parlamentsgesetzen und Hochschulsatzungen sind
vielen Wissenschaftler/innen wenig bewusst, an sich aber offenkundig:
Bundesgesetze werden vom Deutschen Bundestag beschlossen und gelten für
das Bundesgebiet. Landesgesetze werden jeweils von den Landtagen
beschlossen und gelten für die jeweiligen Bundesländer.
Hochschulsatzungen werden von den akademischen Senaten der jeweiligen
Hochschulen beschlossen und gelten für die Angehörigen der jeweiligen
Hochschulen. Dies ist übrigens vergleichbar mit Sportvereinen, deren
rechtliche Grundlage die Vereinssatzung (\enquote{Statuten}) bildet, die
von den Vereinsmitgliedern entschieden werden.

Neben dieser \emph{normativen Wirkung} von Satzungen werden die
Selbstverwaltungsgremien in academia ebenso wie Parlamente nach den
allgemeinen Wahlgrundsätzen von den Hochschulangehörigen
gewählt.\footnote{Vgl. Art. 38 Abs. 1 Satz 1 Grundgesetz: \enquote{Die
  Abgeordneten des Deutschen Bundestages werden in allgemeiner,
  unmittelbarer, freier, gleicher und geheimer Wahl gewählt.}}
Regelungen von Bundesgesetzen treten nach Veröffentlichung im
Bundesgesetzblatt in Kraft, vergleichbar geben die Hochschulen
regelmäßig amtliche Bekanntmachungen heraus. In diesem Fall betroffen
ist die \enquote{Amtliche Bekanntmachung der Universität Konstanz Nr.
90/2015}.

\hypertarget{warum-hat-das-land-baden-wuxfcrttemberg-die-veruxf6ffentlichungsverpflichtung-auf-die-einzelnen-hochschulen-abgewuxe4lzt}{%
\subsubsection{Warum hat das Land Baden-Württemberg die
Veröffentlichungsverpflichtung auf die einzelnen Hochschulen
abgewälzt?}\label{warum-hat-das-land-baden-wuxfcrttemberg-die-veruxf6ffentlichungsverpflichtung-auf-die-einzelnen-hochschulen-abgewuxe4lzt}}

Die Vorgabe, eine Open Access-Zweitveröffentlichung einzufordern,
richtet sich gemäß § 44 Abs. 6 LHG BaWü zunächst an die Hochschulen. Sie
sind dazu verpflichtet, ihre eigenen Wissenschaftler/innen mittels einer
Satzung (siehe oben) zur Open Access-Zweitveröffentlichung verbindlich
anzuhalten. Vermutlich wäre auch der direkte Weg möglich gewesen: Ein
Entwurf der Novelle zum Landeshochschulgesetz 2014 hatte noch eine
direkte, landesgesetzliche Pflicht aller Hochschulangehörigen zur Open
Access-Zweitveröffentlichung vorgesehen.

Jenseits politischer Motive spricht juristisch die grundrechtlich
verbürgte Hochschulautonomie dafür, dass die einzelnen Hochschulen
jeweils selbst Satzungen erlassen und nicht ein zentrales,
wissenschaftsexternes Parlament. Die damit verbundene Beratung und
Beschlussfassung in den jeweiligen Universitätsgremien kann zudem die
Akzeptanz in den einzelnen Einrichtungen befördern und
universitätsspezifische Umsetzungsanforderungen berücksichtigen; dabei
ist insbesondere an die jeweilige Infrastruktur wie institutionelle
Repositorien und Dokumentenserver zu denken. Ausdrücklich kann nach dem
letzten Satz des strittigen § 44 Abs. 6 LHG BaWü von den Hochschulen
festgelegt werden, \enquote{dass die Zweitveröffentlichung auf einem
Repositorium (\ldots{}) zu erfolgen hat.}

Dass die Musterklage an der Universität Konstanz ausgetragen wird,
verwundert nicht: Die Universität Konstanz ist seit langem intensiv für
Open Access engagiert. Konsequent sind insoweit die positiven
Kommentierungen der Universitätsleitung\footnote{Zuletzt zum Beispiel:
  \enquote{Dass sich das Bundesverfassungsgericht nun der
  grundsätzlichen Frage annimmt, ob hier das Urheberrecht betroffen und
  damit ob der Bund oder das Land zuständig ist, begrüßen wir voll und
  ganz. Damit wird juristische Klarheit geschaffen. (\ldots{})}, wird
  der Rektor der Universität Konstanz, Ulrich Rüdiger, in einer
  Presseinformation der Universität Konstanz vom 02.11.2017 zitiert
  (siehe
  \url{https://www.uni-konstanz.de/universitaet/aktuelles-und-medien/aktuelle-meldungen/aktuelles/aktuelles/verpflichtendes-zweitveroeffentlichungs-recht/}).}
zu dem langwierigen Gerichtsverfahren und dem umfassenden Medienecho.
Andererseits kann hinterfragt werden, weshalb die zuständige
Wissenschaftsministerin Theresia Bauer ihren politischen Willen nicht
wie bei anderen Themen selbst im Landesgesetz durchsetzt und die
wissenschaftspolitische Verantwortung und rechtlich absehbare Risiken
den einzelnen Hochschulen zuweist.

\hypertarget{welche-bedeutung-hat-die-zweitveruxf6ffentlichungspflicht-in-baden-wuxfcrttemberg-fuxfcr-die-anderen-bundesluxe4nder}{%
\subsubsection{Welche Bedeutung hat die Zweit\-ver\-öffent\-lich\-ungs\-pflicht in
Baden-Württemberg für die anderen
Bundesländer?}\label{welche-bedeutung-hat-die-zweitveruxf6ffentlichungspflicht-in-baden-wuxfcrttemberg-fuxfcr-die-anderen-bundesluxe4nder}}

Die meisten Bundesländer verfolgen mit hoher Aufmerksamkeit, ob eine
Open Access-Zweit\-ver\-öffent\-lich\-ungs\-pflicht im Landeshochschulrecht den
verfassungsrechtlichen Anforderungen genügt. Sollte die Open
Access-Pionierbestimmung aus Baden-Württemberg der
verfassungsrechtlichen Prüfung standhalten, könnte ein Domino-Effekt
eintreten. Andere Bundesländer, die sich (wissenschafts-)politisch auch
klar für Open Access positionieren und engagieren, könnten ebenso wie in
Baden-Württemberg den politischen Handlungsspielraum für Open
Access-Vorgaben nutzen. In Berlin etwa verabschiedeten Ende 2015 Senat
und Abgeordnetenhaus eine landeseigene Open Access-Strategie mit
umfangreichem Maßnahmenpaket.\footnote{Vgl. Regierender Bürgermeister,
  Pressemitteilung vom 13.10.2015: Freier Zugang zu wissenschaftlichen
  Publikationen: Senat beschließt Open-Access-Strategie. Abrufbar unter
  \url{https://www.berlin.de/rbmskzl/aktuelles/pressemitteilungen/2015/pressemitteilung.384964.php}.}
Ein Jahr zuvor legte präsentiere die Landesregierung in Kiel die
\enquote{Strategie 2020 der Landesregierung Schleswig-Holstein für Open
Access}.\footnote{Abrufbar unter
  \url{https://www.schleswig-holstein.de/DE/Landesregierung/VIII/Presse/PI/PDF/2014/141118_msgwg_OpenAccessStrategie.pdf?__blob=publicationFile\&v=3}.}

Rückenwind geben könnte zudem eine der Empfehlungen, welche die
Enquete-Kommission Internet und Digitale Gesellschaft des Deutschen
Bundestages bereits im Jahr 2013 erteilt hat: Die Zuwendung öffentlicher
Mittel für Forschungsprojekte sollte \enquote{an die rechtlich
verpflichtende Bedingung} geknüpft werden, \enquote{dass die daraus
entstehenden, qualitätsgesicherten Publikationen in Periodika,
Sammelbänden (...) zeitnah nach der Erstveröffentlichung frei zugänglich
gemacht werden.}\footnote{Dazu \emph{Hartmann}. Einstimmige Agenda für
  ein innovationsfreundliches Urheberrecht. ZWD-Magazin Bildung,
  Gesellschaft und Politik, Nr. 1/2013, S. 18 f. Abrufbar unter
  \url{http://hdl.handle.net/11858/00-001M-0000-0014-5D38-C}.}

\hypertarget{welche-bedeutung-hat-der-fall-fuxfcr-die-auuxdferuniversituxe4re-wissenschaft}{%
\subsubsection{Welche Bedeutung hat der Fall für die außeruniversitäre
Wissenschaft?}\label{welche-bedeutung-hat-der-fall-fuxfcr-die-auuxdferuniversituxe4re-wissenschaft}}

Zunächst geht es bei dem Konstanzer Fall (nur) um Hochschulrecht, das
für Hochschulangehörige gilt. Die zugrunde liegende Rechtsfrage stellt
sich im Kern aber vergleichbar auch an den Instituten der
Leibniz-Gemeinschaft, der Helmholtz-Gemeinschaft, der Max Planck
Gesellschaft und anderen außeruniversitären Forschungseinrichtungen:
Dürfen staatliche oder staatlich finanzierte Wissenschaftseinrichtungen
ihr wissenschaftlich tätiges Personal zu einer Zweitveröffentlichung
verpflichten, ohne dass dies gegen deren grundgesetzlich verbürgte
Wissenschaftsfreiheit verstößt?

An außeruniversitären Einrichtungen dürften eher entsprechende Vorgaben
in den Arbeitsverträgen des wissenschaftlichen Personals verankert sein.
Deswegen werden diese Forschungseinrichtungen den Konstanzer Fall mit
hohem Interesse verfolgen.

\hypertarget{bestehen-vergleichbare-open-access-publikationspflichten-auuxdferhalb-deutschlands}{%
\subsubsection{Bestehen vergleichbare Open Access-Publikationspflichten
außerhalb
Deutschlands?}\label{bestehen-vergleichbare-open-access-publikationspflichten-auuxdferhalb-deutschlands}}

Ja, die Ausgangslage, das verbreitete Verständnis und der rechtliche
Rahmen für Publikationsvorgaben im Wissenschaftsbereich unterscheiden
sich oftmals gravierend. In den USA etwa wird die
Work-Made-for-Hire-Doktrin auch im Wissenschaftsbereich eingesetzt, die
einen umfänglichen Rechteübergang von Autoren/innen zu Einrichtungen und
Verlagen bedeuten kann. \footnote{Vgl. dazu The Registry of Open Access
  Repository Mandates and Policies (ROARMAP) unter
  \url{http://roarmap.eprints.org/}.} In den direkten Nachbarländern
sind etwa die großen staatlichen Fördereinrichtungen zunehmend zu Open
Access-Mandaten übergegangen, also zu Open Access-Publikationspflichten.
Der Schweizerische Nationalfonds zur Förderung der wissenschaftlichen
Forschung (SNF) gibt prinzipiell vor, dass die Forschungsergebnisse
geförderter Forschung in Open Access-Pub\-li\-ka\-ti\-onen veröffentlicht werden
müssen.\footnote{Vgl.
  \url{http://www.snf.ch/de/derSnf/forschungspolitische_positionen/open_access/Seiten/default.aspx\#Open-Access-Politik\%20und\%20-Bestimmungen\%20des\%20SNF}
  ; mehr dazu bei \emph{Hirschmann/Verdicchio} (2017), Open Access in
  der Schweiz. In: Söllner/Mittermaier (Hrsg.), Praxishandbuch Open
  Access (S. 215-222). Berlin: De Gruyter Saur.} In Österreich
\enquote{verpflichtet und fördert} der Fonds zur Förderung der
wissenschaftlichen Forschung (FWF) \enquote{alle ProjektleiterInnen und
ProjektmitarbeiterInnen, ihre referierten Forschungsergebnisse im
Internet frei zugänglich zu machen.}\footnote{\url{https://www.fwf.ac.at/de/forschungsfoerderung/open-access-policy/}.}
Große Schritte hin zu umfassenden rechtlichen Open
Access-Publikationsvorgaben sind auch in den Niederlanden zu
konstatieren.\footnote{Vgl z.Bsp. \emph{Wijk} (2017), Open access in the
  Netherlands. In: Söllner/Mittermaier (Hrsg.), Praxishandbuch Open
  Access (S. 223-237). Berlin: De Gruyter Saur.}

Als Motor erweist sich die Europäische Union: Sie hat für ihre
Förderprojekte des Forschungsrahmenprogramms Horizon 2020 eine Open
Access-Zweitveröffentlichung wissenschaftlicher Aufsätze zur
verbindlichen Förderauflage gemacht.\footnote{Vgl. dazu z.Bsp. Max
  Planck Digital Library (2014), MPDL Fact Sheets on Open Science: Open
  Access to Scientific Publications in Horizon 2020. Abrufbar unter
  \url{https://openaccess.mpg.de/2076825/Horizon-2020_MPDLFactSheet_Apr2014.pdf}.}

Die besondere rechtliche Brisanz von (Open Access-)Publikationspflichten
speziell in Deutschland ergibt sich aus dem Grundrecht der
Wissenschaftsfreiheit (Art. 5 Abs. 3 Grundgesetz). Dieses mit seiner
nationalen Verfassungshistorie besteht so nicht beziehungsweise
juristisch nicht vergleichbar in anderen Staaten. Diese diffizile
Rechtslage trägt in Deutschland zu einer gewissen Zurückhaltung bei der
Deutschen Forschungsgemeinschaft (DFG) und dem Bundesministerium für
Bildung und Forschung (BMBF) bei:\footnote{\emph{Fournier} (2017),
  Open-Access-Policies und ihre Gestaltung durch Forschungsförderer. In:
  Söllner/Mittermaier (Hrsg.), Praxishandbuch Open Access (S. 21-27).
  Berlin: De Gruyter Saur.} Zwar haben sie sich klar zur Förderung von
Open Access bekannt, zuletzt startete das BMBF am 20.09.2016 eine
umfassende Open Access-Strategie,\footnote{BMBF, Pressemitteilung Nr.
  109/2016: Freier Zugang schafft mehr Wissen. Abrufbar unter
  \url{https://www.bmbf.de/de/freier-zugang-schafft-mehr-wissen-3340.html}.}
allerdings haben sie (noch) keine rechtlich bindenden
Veröffentlichungspflichten etabliert. Die DFG gab im Jahr 2014 ein
Rechtsgutachten\footnote{Eine Zusammenfassung des Rechtsgutachtens von
  \emph{Fehling} ist als Beitrag \enquote{Verfassungskonforme
  Ausgestaltung von DFG-Förderbedingungen zur Open-Access-Pub\-li\-ka\-ti\-on}
  erschienen in: Ordnung der Wissenschaft, Nr. 4/2014, S. 179-214.
  Abrufbar unter
  \url{http://www.ordnungderwissenschaft.de/Print_2014/24_fehling_dfg_odw_ordnung_der_wissenschaft_2014.pdf}
  .} in Auftrag, das die nationale Rechtslage abgestuft einschätzt: Open
Access-Publikationspflichten seien demzufolge nur dann möglich, wenn sie
an die Vergabe zusätzlicher Gelder gekoppelt ist. Der Gutachter knüpft
die Zulässigkeit von (Open Access-)Publikationspflichten zudem an
vielseitige Bedingungen. Nach dieser Rechtsauffassung könnten die DFG
und BMBF in einem gewissen Umfang fördervertraglich Open
Access-Publikationspflichten rechtsbindend vorgeben, Bundesländer und
Universitäten jedoch dürften demnach ihren Professoren/innen bei deren
institutioneller Forschung keine Open Access-Publikationspflichten
auferlegen.

\hypertarget{wie-wird-das-bundesverfassungsgericht-entscheiden}{%
\subsubsection{Wie wird das Bundesverfassungsgericht
entscheiden?}\label{wie-wird-das-bundesverfassungsgericht-entscheiden}}

Wie oftmals kann die Entscheidung des Bundesverfassungsgerichts kaum
prognostiziert werden. Zu wünschen ist allerdings, dass sich das
Bundesverfassungsgericht überhaupt inhaltlich mit der
Normenkontrollklage der 17 Professoren/innen befasst: Diese reklamieren,
dass die Satzung mit der Zweit\-ver\-öffent\-lich\-ungs\-pflicht gegen ihr
Grundrecht der Wissenschaftsfreiheit (Art. 5 Abs. 3 Grundgesetz)
verstoße. Dabei geht es um die bislang wenig ausgeleuchtete Rechtsfrage,
inwieweit Wissenschaftseinrichtungen ihren Wissenschaftlern/innen
rechtlich verpflichtende Publikationsvorgaben auferlegen dürfen.

Leider könnte sich das Bundesverfassungsgericht aber auch \enquote{nur}
mit Zuständigkeitsfragen befassen: Darf ein Bundesland wie
Baden-Württemberg überhaupt § 44 Abs. 6 LHG BaWü erlassen, wenn das
Urheberrecht allein in die Gesetzgebungskompetenz des Bundes\footnote{Art.
  73 Abs. 1 Nr.9 GG, Art. 71 GG} fällt? Diese Rechtsauffassung vertreten
die Richter des Verwaltungsgerichtshofs Baden-Württemberg, die deshalb
den Fall dem Bundesverfassungsgericht vorgelegt haben. In dem
Vorlagebeschluss schreibt der zuständige Senat des Mannheimer Gerichts:
\enquote{Nach der Überzeugung des 9. Senats hat das Land keine Befugnis,
den Hochschullehrern eine Zweit\-ver\-öffent\-lich\-ungs\-pflicht aufzuerlegen.}
Es geht damit grundlegend um unsere staatliche, föderale Ordnung: Die
Bundesländer sind weithin autonom für die Wissenschaft, der Bund ist
allein für das Urheberrecht zuständig.\footnote{Zur Bundeskompetenz
  Urheberrecht schon \emph{Hartmann}, Mantra Rechtssicherheit. In:
  LIBREAS. Library Ideas, 22 (2013). Abrufbar unter
  \url{http://libreas.eu/ausgabe22/01hartmann.htm}.}

Falls sich das Bundesverfassungsgericht demnach der Rechtsauffassung des
VGH Mannheim anschließt, könnte § 44 Abs. 6 LHG BaWü als
verfassungswidrig erklärt werden, ohne dass es zu einer gerichtlichen
Beurteilung der Rechtsfragen rund um Publikationspflichten, Open Access
und der grundrechtlichen Wissenschaftsfreiheit kommt. Ein Vorgeschmack
auf dieses Szenario bietet die Pressemitteilung des VGH Mannheim vom
06.11.2017 zu dessen Vorlagebeschluss an das
Bundesverfassungsgericht.\footnote{Siehe
  \url{http://vghmannheim.de/pb/,Lde/Startseite/Medien/Zweitveroeffentlichungspflicht+von+Hochschullehrern_+Verwaltungsgerichtshof+ruft+Bundesverfassungsgericht+an/}.}

\hypertarget{wie-schuxe4tzen-fachjuristeninnen-die-rechtslage-ein}{%
\subsubsection{Wie schätzen Fachjuristen/innen die Rechtslage
ein?}\label{wie-schuxe4tzen-fachjuristeninnen-die-rechtslage-ein}}

Mittlerweile gibt es doch einige rechtswissenschaftliche
Fachpublikationen. Die hochschulrechtlichen Veröffentlichungs- und
urheberrechtlichen Anbietungspflichten des Hochschulprofessors behandelt
eingehend die an der Technischen Universität Dresden eingereichte
Dissertationsschrift von Nicole Schmidt.\footnote{\emph{N. Schmidt},
  Open Access (2016).} Recht aktuell ist auch ein Rechtsgutachten, das
der Jura-Professor Michael Fehling im Auftrag der Deutschen
Forschungsgemeinschaft (DFG) erstellt hat.\footnote{Vgl. Zusammenfassung
  dazu: \emph{Fehling}, Verfassungskonforme Ausgestaltung von
  DFG-Förderbedingungen zur Open-Access-Pub\-li\-ka\-ti\-on. Erschienen in:
  Ordnung der Wissenschaft, Nr. 4/2014, S. 179-214. Abrufbar unter
  \url{http://www.ordnungderwissenschaft.de/Print_2014/24_fehling_dfg_odw_ordnung_der_wissenschaft_2014.pdf}.}
Es spricht zum Beispiel auch die Rechtsfrage an, ob Forschungsdaten der
Wissenschaftseinrichtung oder den Hochschulprofessoren/innen gehören.
Einen freiheitlicheren Weg zu Open Access anstatt rechtlichen
Veröffentlichungspflichten wünschte sich der Experte für Hochschulrecht,
Manfred Löwisch, schon bei der Verabschiedung von § 44 Abs. 6 LHG
BaWü.\footnote{\emph{Löwisch}, Förderung statt Zwang -- Neue Open Access
  Strategie in Baden-Württemberg. In: Ordnung der Wissenschaft, Nr.
  1/2017, S. 59f. Abrufbar unter
  \url{http://www.ordnungderwissenschaft.de/2017-1/10_Gesamt/08_2017_01_loewisch_foerderung_statt_zwang_odw.pdf}.}

Zum Normenkontrollverfahren an der Universität Konstanz jetzt hat Georg
Sandberger, vormals Rektor der Universität Tübingen, einen Beitrag
verfasst zur \enquote{Zukunft wissenschaftlichen Publizierens, Open
Access und Wissenschaftsschranke. Anmerkungen zu den Kontroversen über
die Weiterentwicklung des Urheberrechts.}\footnote{In: Ordnung der
  Wissenschaft, Nr. 2/2017, S. 75-96. Abrufbar unter
  \url{http://www.ordnungderwissenschaft.de/2017-2/11_Sandberger/11_2017_02_sandberger_die\%20zukunft\%20des\%20wissenschaftlichen\%20publizierens_odw.pdf}.}

Insgesamt ist die Rechtslage jedoch wenig konturiert, was wesentlich
daran liegen mag, dass bislang keine Fälle zu elektronischem Publizieren
im juristischen Kontext von Urheberrecht, Eigentumsgarantie und
Wissenschaftsfreiheit vom Bundesverfassungsgericht verhandelt wurden.
Vor allem hat das Bundesverfassungsgericht in den letzten Jahren keine
der Verfassungsbeschwerden von Wissenschaftsverlagen verhandelt. Schon
am 20.10.2015 hatte beispielsweise der Eugen Ulmer Verlag mit
Unterstützung des Börsenvereins des Deutschen Buchhandels eine
Verfassungsbeschwerde in Karlsruhe eingereicht.\footnote{Vgl. Meldung am
  20.10.2015 auf boersenblatt.net. Abrufbar unter
  \url{http://www.boersenblatt.net/artikel-urheberrecht__oettingers_plaene_und_der_streit_um_paragraf_52b.1038692.html}.}
Ob das Bundesverfassungsgericht diese Klage gegen die geltend gemachte
Enteignung durch den Bibliotheksparagraphen 52b im Urheberrechtsgesetz
aber überhaupt zur Verhandlung zulässt, ist weiterhin ungewiss.

\hypertarget{welche-reaktionen-gibt-es-zum-fall}{%
\subsubsection{Welche Reaktionen gibt es zum
Fall?}\label{welche-reaktionen-gibt-es-zum-fall}}

Die Aufmerksamkeit ist hoch und vielseitig. Die Universität Konstanz
begleitete das Normenkontrollverfahren bislang vergleichsweise offensiv
mit mehreren Presseinformationen, welche die Position der
Universitätsleitung unterstreichen.\footnote{Vgl. z.Bsp. Pressemeldung
  der Universität Konstanz vom 21.11.2016, abrufbar unter
  \url{https://www.uni-konstanz.de/universitaet/aktuelles-und-medien/aktuelle-meldungen/aktuelles/aktuelles/open-access-satzung-auf-juristischem-pruefstand/}.}

Zu beachten sind der Börsenverein des Deutschen Buchhandels und der
Deutsche Hochschulverband (DHV), welche die
Zweit\-ver\-öffent\-lich\-ungs\-pflicht prinzipiell ablehnen. Der Deutsche
Hochschulverband (DHV) unterstützt die nun angestrengte Musterklage. Der
Justitiar des Börsenvereins, Christian Sprang, formulierte schon im
Gesetzgebungsverfahren Anfang 2014 eine elfseitige Stellungnahme, um die
nun eingeklagte Rechtswidrigkeit der Zweit\-ver\-öffent\-lich\-ungs\-pflicht näher
zu begründen.\footnote{Siehe
  \url{http://www.boersenverein.de/sixcms/media.php/976/Stellungnahme_3.HRAG_BaWu_20131128.pdf}.}
Anlässlich des Normenkontrollverfahrens appelliert am 09.11.2017 der
Börsenverein des Deutschen Buchhandels \enquote{an die
baden-württembergische Landesregierung und die Landtagsabgeordneten, die
Zweitveröffentlichungsregelung im Landeshochschulgesetz ersatzlos zu
streichen.}\footnote{\href{https://www.boersenblatt.net/artikel-boersenverein_zu_open-access-regelung_.1398729.html}{https://www.boersenblatt.net/artikel-boersenverein\_zu\_open-access-regelung\_.1398729.html\#}.}
Beim Bundesverfassungsgericht siegesgewiss kommentiert der Vorsitzende
des Verleger-Ausschusses des Börsenverein, Matthias Ulmer: "Die
verantwortlichen Politikerinnen und Politiker in Baden-Württemberg
sollten die Entscheidung des VGH zum Anlass nehmen, ihre bei der
Verabschiedung des Landeshochschulgesetzes getroffene Entscheidung zu
revidieren, noch bevor das Bundesverfassungsgericht die Regelung ohnehin
für nichtig erklären wird."\footnote{\href{https://www.boersenblatt.net/artikel-boersenverein_zu_open-access-regelung_.1398729.html}{https://www.boersenblatt.net/artikel-boersenverein\_zu\_open-access-regelung\_.1398729.html\#}.}

In der Frankfurter Allgemeinen Zeitung (Ausgabe vom 29.11.2017) bezieht
der Literaturwissenschaftler Roland Reuß klar Position: Sein Gastartikel
\enquote{Ende eines Blindflugs} beschreibt, dass 17 Konstanzer
Professoren der Entrechtung durch Open Access die Grenze aufzeige.
\enquote{Karlsruhe muss jetzt entscheiden}, so Reuß einleitend,
\enquote{ob der Staat Wissenschaftler zur Publikation zwingen darf.}

Die Normenkontrollklage wesentlich auch mit Blickwinkeln zu Open Access
aufgegriffen haben regionale Tageszeitungen. Mehrfach berichtete bislang
der Südkurier, am 23.11.2016 mit dem Titel \enquote{Konstanzer
Professoren klagen gegen ihre eigene Universität}\footnote{\url{https://www.suedkurier.de/region/kreis-konstanz/konstanz/Konstanzer-Professoren-klagen-gegen-ihre-eigene-Universitaet;art372448,9013200}.},
dann etwa am 03.11.2017 \enquote{Öffentliches Geld, öffentliches Wissen:
Warum Konstanzer Professoren gegen ihre Uni klagen und der Fall bis vor
das Bundesverfassungsgericht geht}.\footnote{\url{https://www.suedkurier.de/region/kreis-konstanz/konstanz/OEffentliches-Geld-oeffentliches-Wissen-Warum-Konstanzer-Professoren-gegen-ihre-Uni-klagen-und-der-Fall-bis-vor-das-Bundesverfassungsgericht-geht;art372448,9480592}.}
Die Stuttgarter Zeitung brachte am 26.09.2017 einen Beitrag
\enquote{Streit um Recht zur Zweitveröffentlichung: Profs klagen gegen
ihre Uni.}\footnote{\url{https://www.stuttgarter-zeitung.de/inhalt.streit-recht-zur-zweitveroeffentlichung-professoren-klagen-gegen-ihre-uni.7fa75c01-d550-4ef5-9c3a-d3438e80540b.html}.}

\hypertarget{am-rande-i-geht-es-um-open-access-bei-der-satzung-mit-zweitveruxf6ffentlichungspflicht}{%
\subsubsection{\texorpdfstring{Am Rande I: Geht es um \emph{Open Access}
bei der Satzung mit
Zweit\-ver\-öffent\-lich\-ungs\-pflicht?}{Am Rande I: Geht es um Open Access bei der Satzung mit Zweit\-ver\-öffent\-lich\-ungs\-pflicht?}}\label{am-rande-i-geht-es-um-open-access-bei-der-satzung-mit-zweitveruxf6ffentlichungspflicht}}

Wenn Fachpublikationen frei zugänglich sein sollen, geht es natürlich um
Open Access. Dies ist auch das Anliegen der neuen Gesetzesbestimmung im
Landeshochschulrecht Baden-Württemberg und der darauf fußenden
Universitätssatzung in Konstanz. In dieser ist den einzelnen
Satzungsvorschriften eine Präambel vorangestellt, die entsprechend wie
folgt beginnt:

\begin{quote}
\enquote{Die Erkenntnisse wissenschaftlicher Forschung sollten möglichst
frei zugänglich sein. Das Landeshochschulgesetz nimmt deshalb in § 44
Abs. 6 LHG den Open Access-Gedanken in der Form auf, dass die
Hochschulen ihre Wissenschaftlerinnen und Wissenschaftler durch Satzung
verpflichten sollen, das Recht auf Zweitveröffentlichung (\ldots{}) auch
tatsächlich in Anspruch zu nehmen.}\footnote{Universität Konstanz,
  Satzung zur Ausübung des wissenschaftlichen
  Zweit\-ver\-öffent\-lich\-ungs\-rechts gemäß § 38 Abs. 4 UrhG, S. 2.}
\end{quote}

Allerdings bleibt festzuhalten, dass die Zweit\-ver\-öffent\-lich\-ungs\-pflicht
nicht Open Access gemäß der \enquote{Berliner Erklärung über den offenen
Zugang zu wissenschaftlichem Wissen vom 22. Oktober 2003}\footnote{Berliner
  Erklärung über den offenen Zugang zu wissenschaftlichem Wissen vom 22.
  Oktober 2003, abrufbar unter
  \url{https://openaccess.mpg.de/Berliner-Erklaerung}.} ist. Zwar wird
ein freier Zugang zu den wissenschaftlichen Beiträgen gewährt
(Zugangsrecht). Jedoch können die Wissenschaftler/innen lizenzrechtlich
nicht beziehungsweise lizenzrechtlich nicht umfassend\footnote{Vgl.
  insbesondere Frage Nr. 24 des FAQ zum Zweit\-ver\-öffent\-lich\-ungs\-recht der
  Schwerpunktinitiative \enquote{Digitale Information} der Allianz der
  deutschen Wissenschaftsorganisationen, abrufbar unter
  \url{http://www.allianzinitiative.de/handlungsfelder/rechtliche-rahmenbedingungen/faq-zvr.html}.}
erlauben, die Zweitveröffentlichung \enquote{-- in jedem beliebigen
digitalen Medium und für jeden verantwortbaren Zweck -- zu kopieren, zu
nutzen, zu verbreiten, zu übertragen und öffentlich wiederzugeben sowie
Bearbeitungen davon zu erstellen und zu verbreiten, sofern die
Urheberschaft korrekt angegeben wird.}\footnote{Vgl. Berliner Erklärung
  über den offenen Zugang zu wissenschaftlichem Wissen vom 22. Oktober
  2003, abrufbar unter
  \url{https://openaccess.mpg.de/Berliner-Erklaerung}.} So dürfen die
wissenschaftlichen Autoren/innen ihre Zweitveröffentlichung insbesondere
nicht mit der für Open Access empfohlenen\footnote{Vgl. Allianz der
  deutschen Wissenschaftsorganisationen, Appell zur Nutzung offener
  Lizenzen in der Wissenschaft (2014). Abrufbar unter
  \url{http://www.allianzinitiative.de/fileadmin/user_upload/www.allianzinitiative.de/Appell_Offene_Lizenzen_2014.pdf}.}
Creative Commons-Lizenz CC BY etwa in ein Repository einstellen.

\hypertarget{am-rande-ii-welche-bedeutung-fuxfcr-open-access-policy-hat-eine-satzung-mit-zweitveruxf6ffentlichungspflicht}{%
\subsubsection{\texorpdfstring{Am Rande II: Welche Bedeutung für Open
Access \emph{Policy} hat eine Satzung mit
Zweit\-ver\-öffent\-lich\-ungs\-pflicht?}{Am Rande II: Welche Bedeutung für Open Access Policy hat eine Satzung mit Zweit\-ver\-öffent\-lich\-ungs\-pflicht?}}\label{am-rande-ii-welche-bedeutung-fuxfcr-open-access-policy-hat-eine-satzung-mit-zweitveruxf6ffentlichungspflicht}}

Policy und Satzung sind juristisch strikt zu unterscheiden: Satzungen
beinhalten rechtlich verbindliche und (notfalls gerichtlich)
durchsetzbare Pflichten und Rechte. Sie wirken damit gleichsam Gesetzen,
allerdings nur im Geltungsbereich, das heißt eine Satzung der
Universität Konstanz betrifft eben speziell die Angehörige dieser
Universität. Im Unterschied dazu beinhalten Policies typischerweise
Empfehlungen, Richtlinien, Standards, Positionen, Verfahrensabläufe oder
Zuständigkeiten. Sie sollen gerade nicht juristisch wasserdicht
abgeklopft sein, sondern einen flexiblen Handlungsrahmen aufzeigen.

Dennoch ähnelt die Satzung der Universität Konstanz inhaltlich
institutionellen Open Access Policies. Das Anliegen, Open Access zu
befördern, ist Satzung und Policy gemein. Nach einer Phase der Bewährung
und steigenden Akzeptanz in der Wissenschaftseinrichtung liegt es häufig
nahe, Empfehlungen und Leitlinien einer Open Access Policy zu rechtlich
verbindlichen Standards fortzuschreiben. Eine in der Publikationspraxis
gereifte Open Access Policy kann damit als Vorläufer einer
rechtsverbindlichen Satzung mit Open Access-Publikationspflichten
dienen.\footnote{Dazu mehr: Vortragsaufzeichnung
  \enquote{Compliance-Anforderungen für das Forschungs- und
  Publikationsmanagement} in Session \enquote{Umsetzung und Erfahrung
  mit Richtlinien und Guidelines} auf den Open Access-Tagen 2014 Köln,
  abrufbar unter \url{https://youtu.be/BegYmuqD804}.}

\hypertarget{am-rande-iii-welche-bedeutung-fuxfcr-digitale-forschungsdaten-hat-der-fall}{%
\subsubsection{Am Rande III: Welche Bedeutung für digitale
Forschungsdaten hat der
Fall?}\label{am-rande-iii-welche-bedeutung-fuxfcr-digitale-forschungsdaten-hat-der-fall}}

Die Auseinandersetzung an der Universität Konstanz jetzt könnte man
zusammenfassen: Haben allein die Wissenschaftler/innen die Rechte und
damit Publikations- und Verwertungsbefugnisse an ihren Fachaufsätzen
oder (auch) deren Universität beziehungsweise Wissenschaftseinrichtung?
Diese Frage stellt sich vergleichbar auch bei (digitalen)
Forschungsdaten.

Juristisch interessant: An den Forschungsdaten, die im Rahmen der
finanzierten Forschungstätigkeit entstehen, lässt sich schon seit langem
manche Wissenschaftseinrichtung Nutzungen beziehungsweise Rechte im
Arbeits- beziehungsweise Dienstrecht einräumen. Nicht weiter beleuchtet
wurde bislang, inwiefern dies -- mit ähnlicher rechtlicher Argumentation
wie nun in Konstanz -- die Wissenschaftsfreit nach Art. 5 Abs. 3
Grundgesetz berühren könnte.

%autor

\end{document}
