\begin{center}\rule{0.5\linewidth}{\linethickness}\end{center}

\textbf{Thomas Hartmann}, Master of Laws (Informationsrecht und
Rechtsinformation), wissenschaftlicher Mitarbeiter am FIZ Karlsruhe und
an der Europa-Universität Viadrina Frankfurt/Oder, Lehrbeauftragter und
Doktorand an der Humboldt-Universität zu Berlin.

Thomas Hartmann hat Jura, Wirtschaft und Grundlagen der Informatik in
Pforzheim, Mailand und Wien studiert. Er forscht zu
Immaterialgüterrechten, dzt. an der Humboldt-Universität zu Berlin, am
FIZ Karlsruhe - Leibniz-Institut für Informationsinfrastruktur sowie an
der Europa-Universität Viadrina in Frankfurt/Oder (BMBF-Verbundprojekt
FDMentor). Zuvor war er wissenschaftlicher Mitarbeiter mit Arbeitsfeld
Open Access und Recht an der Max Planck Digital Library (MPDL) und am
Max-Planck-Institut für Innovation und Wettbewerb (MPI IP) in München
(2011-2016) sowie an der Georg-August-Universität Göttingen im
EU-Datenschutzprojekt Consent (2011-2012). Daneben ist er seit 2010
Lehrbeauftragter an mehreren Hochschulen und Dozent in der beruflichen
Weiterbildung. Siehe auch
\url{https://www.ibi.hu-berlin.de/de/institut/personen/hartmann}.
