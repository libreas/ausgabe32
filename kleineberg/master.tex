\documentclass[a4paper,
fontsize=11pt,
%headings=small,
oneside,
numbers=noperiodatend,
parskip=half-,
bibliography=totoc,
final
]{scrartcl}

\usepackage{synttree}
\usepackage{graphicx}
\setkeys{Gin}{width=.4\textwidth} %default pics size

\graphicspath{{./plots/}}
\usepackage[ngerman]{babel}
\usepackage[T1]{fontenc}
%\usepackage{amsmath}
\usepackage[utf8x]{inputenc}
\usepackage [hyphens]{url}
\usepackage{booktabs} 
\usepackage[left=2.4cm,right=2.4cm,top=2.3cm,bottom=2cm,includeheadfoot]{geometry}
\usepackage{eurosym}
\usepackage{multirow}
\usepackage[ngerman]{varioref}
\setcapindent{1em}
\renewcommand{\labelitemi}{--}
\usepackage{paralist}
\usepackage{pdfpages}
\usepackage{lscape}
\usepackage{float}
\usepackage{acronym}
\usepackage{eurosym}
\usepackage[babel]{csquotes}
\usepackage{longtable,lscape}
\usepackage{mathpazo}
\usepackage[normalem]{ulem} %emphasize weiterhin kursiv
\usepackage[flushmargin,ragged]{footmisc} % left align footnote
\usepackage{ccicons} 

%%%% fancy LIBREAS URL color 
\usepackage{xcolor}
\definecolor{libreas}{RGB}{112,0,0}

\usepackage{listings}

\urlstyle{same}  % don't use monospace font for urls

\usepackage[fleqn]{amsmath}

%adjust fontsize for part

\usepackage{sectsty}
\partfont{\large}

%Das BibTeX-Zeichen mit \BibTeX setzen:
\def\symbol#1{\char #1\relax}
\def\bsl{{\tt\symbol{'134}}}
\def\BibTeX{{\rm B\kern-.05em{\sc i\kern-.025em b}\kern-.08em
    T\kern-.1667em\lower.7ex\hbox{E}\kern-.125emX}}

\usepackage{fancyhdr}
\fancyhf{}
\pagestyle{fancyplain}
\fancyhead[R]{\thepage}

% make sure bookmarks are created eventough sections are not numbered!
% uncommend if sections are numbered (bookmarks created by default)
\makeatletter
\renewcommand\@seccntformat[1]{}
\makeatother


\usepackage{hyperxmp}
\usepackage[colorlinks, linkcolor=black,citecolor=black, urlcolor=libreas,
breaklinks= true,bookmarks=true,bookmarksopen=true]{hyperref}

%meta
%meta

\fancyhead[L]{M. Kleineberg, B. Kaden \\ %author
LIBREAS. Library Ideas, 32 (2017). % journal, issue, volume.
\href{http://nbn-resolving.de/}
{}} % urn 
% recommended use
%\href{http://nbn-resolving.de/}{\color{black}{urn:nbn:de...}}
\fancyhead[R]{\thepage} %page number
\fancyfoot[L] {\ccLogo \ccAttribution\ \href{https://creativecommons.org/licenses/by/3.0/}{\color{black}Creative Commons BY 3.0}}  %licence
\fancyfoot[R] {ISSN: 1860-7950}

\title{\LARGE{Open Humanities? ExpertInnenmeinungen über Open Access in den Geisteswissenschaften}} % title
\author{Michael Kleineberg, Ben Kaden} % author

\setcounter{page}{1}

\hypersetup{%
      pdftitle={Open Humanities? ExpertInnenmeinungen über Open Access in den Geisteswissenschaften},
      pdfauthor={Michael Kleineberg, Ben Kaden},
      pdfcopyright={CC BY 3.0 Unported},
      pdfsubject={LIBREAS. Library Ideas, 32 (2017).},
      pdfkeywords={Geisteswissenschaften, Digital Humanities, Fu-PusH, Open Publication, Open Research Data, Open Review, Experteninterviews},
      pdflicenseurl={https://creativecommons.org/licenses/by/3.0/},
      pdfcontacturl={http://libreas.eu},
      baseurl={http://libreas.eu},
      pdflang={de},
      pdfmetalang={de}
     }



\date{}
\begin{document}

\maketitle
\thispagestyle{fancyplain} 

%abstracts

%body
\section{Einleitung}\label{einleitung}

Die Debatte um Open Access in der Wissenschaft kann mittlerweile auf
eine beachtliche Geschichte zurück blicken, die in den 1990er Jahren mit
der digitalen Transformation des wissenschaftlichen Publizierens begann
und sich durch eine Reihe von Initiativen (z.\,B. Budapest Open Access
Initiative\footnote{Budapest Open Access Initiative (2002):
  \url{http://www.soros.org/openaccess}.}, Bethesda Statement\footnote{Bethesda
  Statement on Open Access Publishing (2003):
  \url{http://dash.harvard.edu/bitstream/handle/1/4725199/suber_bethesda.htm?sequence=1}.},
Berliner Erklärung\footnote{Berlin Declaration on Open Access to
  Knowledge in the Sciences and Humanities (2003):
  \url{http://oa.mpg.de/lang/en-uk/berlin-prozess/berliner-erklarung}.},
Gemeinsame Erklärung der Wissenschaftsorganisationen\footnote{Allianz
  der deutschen Wissenschaftsorganisationen (2009):
  \url{https://www.humboldt-foundation.de/web/pressemitteilung-2009-08.html}.},
Lyon Declaration\footnote{Lyon Declaration on Access to Information and
  Development (2014): \url{http://www.lyondeclaration.org/}.}, Hague
Declaration\footnote{Hague Declaration on Knowledge Discovery in the
  Digital Age (2015): \url{http://thehaguedeclaration.com/}.}, Hamburger
Note\footnote{Hamburger Note zur Digitalisierung des kulturellen Erbes
  (2015): \url{http://hamburger-note.de/}.}) und -- in weit geringerem
Umfang -- Gegeninitiativen (z.\,B. Heidelberger Appell\footnote{Heidelberger
  Appell (2009): \url{http://www.textkritik.de/urheberrecht/appell.pdf};
  vgl. auch Reuß \& Rieble 2009, Jochum 2009.}) manifestierte.

Das Prinzip der Offenheit bleibt dabei keineswegs auf den freien Zugang
zu Veröffentlichungen von Forschungsergebnissen in Form von klassischen
narrativen Publikationen beschränkt, sondern wird zunehmend im Sinne
einer \enquote{Offenen Wissenschaft} (Open Science bzw. Open
Scholarship) auf weitere Bereiche der Forschung und Lehre übertragen,
wie beispielsweise die Forderungen nach offenen Forschungsdaten (Open
Research Data), offenen Softwareanwendungen (Open Source), offenen
Forschungsverfahren (Open Methodology), offenen
Qualitätssicherungsverfahren (Open Review), offenen Evaluations- bzw.
Kreditierungsverfahren (Open Metrics) oder offenen Bildungsmaterialien
(Open Educational Resources) anzeigen.\footnote{Vgl.
  Helmholtz-Gemeinschaft 2015, Bartling \& Friesike 2014, Herb 2012,
  Kraker et al. 2012, openscienceASAP:
  \url{http://openscienceasap.org/open-science/}; Paris Open Educational
  Resources Declaration (2012):
  \url{http://www.unesco.org/new/en/communication-and-information/access-to-knowledge/open-educational-resources/what-is-the-paris-oer-declaration/};
  Enabling Open Scholarship:
  \url{http://www.openscholarship.org/jcms/c_5012/en/home}.}

Angesichts des vielfach unscharf bestimmten Begriffs der Offenheit
bietet sich als Diskussionsgrundlage zunächst die von der \emph{Open
Knowledge Foundation}\footnote{Open Knowledge Foundation:
  \url{http://okfn.de/}.} vorgeschlagene \enquote{Open Definition} an,
die sowohl den Zugang als auch die Nutzung, Weiterverarbeitung und
Vervielfältigung von Inhalten berücksichtigt:

\begin{quote}
\enquote{Open means anyone can freely access, use, modify, and share for
any purpose (subject, at most, to requirements that preserve provenance
and openness).}\footnote{Open Definition:
  \url{http://opendefinition.org/}.}
\end{quote}

Naturgemäß gehen unterschiedliche Wissenskulturen mit der gesellschafts-
und wissenschaftspolitischen Forderung nach weitreichender Offenheit und
Transparenz der Wissenschaft auf verschiedene Weise um. Beispielweise
haben sich Preprint-Publikationen auf Open-Access-Doku\-mentenservern oder
Forschungsdatenpublikationen in den Naturwissenschaften bzw. dem
STM-Bereich (Science, Technology, Medicine) früher etabliert und sind
gegenwärtig stärker ausgeprägt als in den Geisteswissenschaften.
\footnote{Vgl. Eger et al. 2015, Herb 2012, Dallmeier-Tiessen et al.
  2011, Krönung 2010.}

Aber auch innerhalb der Geisteswissenschaften scheint es einen
heterogenen Umgang mit dem Offenheitsprinzip zu geben.\footnote{Vgl.
  Hogenaar et al. 2011.} Zum einen werden disziplinspezifische
Besonderheiten deutlich, zum anderen unterscheiden sich die Digital
Humanities zum Teil erheblich von den traditionellen
Geisteswissenschaften. Darüber hinaus werden generationale Divergenzen
sichtbar und schließlich spielen auch förderpolitische Rahmenbedingungen
bzw. Anreizsysteme eine Rolle.

In diesem Beitrag soll es im Hinblick auf die Frage nach den
Möglichkeiten und Grenzen des Prinzips der Offenheit in den
Geisteswissenschaften im Sinne von \enquote{Open Humanities}\footnote{Vgl.
  Open Humanities Group der Open Knowledge Foundation:
  \url{http://humanities.okfn.org/}.} um die Innenperspektive der
Forschenden gehen. Zu diesem Zweck werden die Ergebnisse der im Rahmen
des DFG-Projektes \emph{Future Publications in den Humanities}
(Fu-PusH)\footnote{DFG-Projekt Fu-PusH:
  \url{http://www.ub.hu-berlin.de/fu-push}.} durchgeführten
ExpertInneninterviews vorgestellt.

\section{\texorpdfstring{Das DFG-Projekt \enquote{Future Publications in
den
Humanities}}{Das DFG-Projekt Future Publications in den Humanities}}\label{das-dfg-projekt-future-publications-in-den-humanities}

Das an der Universitätsbibliothek der Humboldt-Universität zu Berlin
angesiedelte und von der DFG für zwei Jahre geförderte Projekt Fu-PusH
untersucht die Potenziale der digitalen Transformation für
geisteswissenschaftliche Arbeits- und Veröffentlichungsverfahren. Ziel
des Projektes ist die Formulierung von Handlungsempfehlungen für die
Ausrichtung und Weiterentwicklung von akademischen
Informationsinfrastrukturen zur Unterstützung von Publikationsprozessen.

Ausgehend von der Annahme, dass die Bedingungen in den
Geisteswissenschaften nicht ohne weiteres mit denen des STM-Bereiches
verglichen werden können,\footnote{Vgl. Suber 2005.} stellt sich
zunächst die Frage, inwieweit die Potenziale von informationstechnischen
Innovationen tatsächlich Mehrwerte für dezidiert
geisteswissenschaftliches Forschen, Kommunizieren und Publizieren
bieten. Von besonderem Interesse sind hierbei die Digital Humanities,
die sich sowohl durch eine Erweiterung des Methodenkanons um
computergestützte Analyseverfahren als auch durch den Umgang mit zum
Teil umfangreichen Forschungsdaten wie insbesondere Textkorpora
auszeichnen.\footnote{Vgl. auch den Beitrag \enquote{Zur Epistemologie
  digitaler Methoden in den Geisteswissenschaften} von Ben Kaden in
  diesem Band.}

Die Auswirkungen der digitalen Transformation werden dabei auf drei
Ebenen untersucht. Erstens werden typische Forschungsprozesse
betrachtet, wobei vor allem Besonderheiten der jeweils generierten bzw.
(nach-)genutzten Forschungsdaten sowie die eingesetzten Werkzeuge und
Forschungsinfrastrukturen angefangen von Textverarbeitungsprogrammen
über Kommuni\-kations- und Kollaborationsmedien bis hin zu Virtuellen
Forschungsumgebungen (VFU) erfasst werden sollen.

Zweitens werden typische Formen der Veröffentlichung von
Forschungsergebnissen untersucht. Hierbei stellt sich vor allem die
Frage, welche Arten von Erweiterungen bzw. Anreicherungen sich als
relevant für digitale Publikationen in den jeweiligen
geisteswissenschaftlichen Fachrichtungen erweisen. Als Bezugspunkt dient
hierbei das Konzept der \enquote{Enhanced Publications}\footnote{Vgl.
  Bardi \& Manghi 2014, Vernooy-Gerritsen 2009.} im Sinne digitaler
wissenschaftlicher Publikationen, die über eine bloße Reproduktion von
Printpublikationen mit digitalen Mitteln hinausgehen und durch
erweiterte Funktionalitäten (z.\,B. Multimedialität, Interaktivität,
Semantische Strukturierung, Versionierung, Modularisierung, Integration
von Forschungsdaten) gekennzeichnet sind. Von besonderem Interesse ist
dabei, inwieweit der Prozess der Forschung selbst transparent gemacht
werden soll, indem etwa der zumeist digital dokumentierte
Forschungsverlauf in Form von Teil- oder Negativergebnissen sowie zu
Grunde liegenden Forschungsdaten bzw. Zusatzmaterialien in die
Veröffentlichung eingeht. Zudem wird auch die Rolle von alternativen
Kommunikations- und Publikationsmedien wie Projektwebseiten,
Wissenschaftsblogs oder Twitterstreams untersucht.

Drittens wird schließlich der gesamte Publikationsprozess in den Blick
genommen. Hierbei geht es vor allem um die Frage, inwieweit der
traditionelle Publikationskreislauf durchbrochen wird und sich
gegebenenfalls die Rollen und Aufgaben der jeweiligen Akteure neu
verteilen. Dabei werden insbesondere die Funktion kommerzieller Verlage
sowie die marktstrategische Bedeutung unterschiedlicher Finanzierungs-
bzw. Geschäftsmodelle analysiert.

Auf allen drei Ebenen -- Forschungsprozess, Publikationsform und
Publikationsprozess -- spielt das Prinzip der Offenheit eine bedeutende
Rolle. Beispielsweise zählen zu den zentralen Argumenten der
Open-Access-Befürworter die Forderungen nach Nachvollziehbarkeit und
Nachnutzbarkeit.\footnote{Vgl. Suber 2012.} Bei der Nachvollziehbarkeit
von Forschungsergebnissen handelt es sich um ein epistemologisches und
daher wissenschaftsinternes Argument, das den möglichst freien und
offenen Zugang zu Forschungsdaten und gegebenenfalls Softwareanwendungen
zu begründen versucht. Bei der Nachnutzbarkeit von Forschungsergebnissen
und -daten handelt es sich vielmehr um ein gesellschaftspolitisches und
daher wissenschaftsexternes Argument, das durch einen ebenfalls
möglichst freien und offenen Zugang sowie das Recht auf Vervielfältigung
und Weiterverarbeitung von Inhalten eine Mehrfachfinanzierung von
wissenschaftlicher Forschung mit öffentlichen Geldern vermeiden und den
Umgang mit bereits vorhandenem Wissen effektiver gestalten will.

Um den Themenkomplex des wissenschaftlichen Publizierens in den
Geisteswissenschaften angemessen beurteilen zu können, wurde im Rahmen
des Projektes eine Situations- und Bedarfsanalyse anhand von 43 ein- bis
zweistündigen ExpertInneninterviews mit GeisteswissenschaftlerInnen (23)
sowie VertreterInnen von Infrastruktureinrichtungen (20) vorgenommen, in
denen es unter anderem um Handlungs- und Einstellungsmuster zum
Open-Access-Gedanken geht.

Die Auswahl der befragten GeisteswissenschaftlerInnen bezog sich auf den
gesamten deutschsprachigen Raum und schloss Forschungseinrichtungen wie
Universitäten, Akademien und Stiftungen sowie ein Zentrum für Digital
Humanities ein. Da sich der Fokus der Befragungen auf die Potenziale der
digitalen Transformation für künftige Publikationsszenarien richtete,
wurde unter anderem das Auswahlkriterium \enquote{Erfahrungen mit
digitalem Publizieren} festgelegt. Insofern handelt es sich weniger um
eine die Geisteswissenschaften in ihrer Gesamtheit abbildende
repräsentative Studie als vielmehr um die Erfassung von spezifischem
ExpertInnenwissen. Die Gruppe der GeisteswissenschaftlerInnen, deren
Befragungsergebnisse im Folgenden vorgestellt werden sollen, setzt sich
etwa zu gleichen Teilen aus ProfessorInnen und VertreterInnen des
akademischen Mittelbaus zusammen, wobei folgende Fachrichtungen
vertreten sind:

\begin{itemize}
\item
  Sprachwissenschaft (5)
\item
  Geschichtswissenschaft (5)
\item
  Literaturwissenschaft (5)
\item
  Kunstwissenschaft (1)
\item
  Wissenschaftsgeschichte (1)
\item
  Ägyptologie (1)
\item
  Kulturwissenschaft (1)
\item
  Buchwissenschaft (1)
\item
  Musikwissenschaft (1)
\item
  Filmwissenschaft (1)
\item
  Digital Humanities (1)
\end{itemize}

Thematisch schließen die offenen Leitfadeninterviews neben
infrastrukturellen Desiderata hinsichtlich digitaler Forschungs-,
Kommunikations- und Publikationsumgebungen auch wissenschaftskulturelle
sowie wissenschaftspolitische Anforderungen ausdrücklich ein und
beziehen sich auf den gesamten wissenschaftlichen
Publikationsprozess.\footnote{Der Interview-Leitfaden sowie ein
  vollständiges exemplarisches Interview mit Christian Heise sind
  veröffentlicht in Heise 2015. Für eine erste Trendanalyse zu den
  ExpertInneninterviews vgl. Kaden \& Kleineberg 2015.} Die im folgenden
Abschnitt vorgestellte Auswertung richtet den Fokus auf das Prinzip der
Offenheit in drei Schwerpunktbereichen:

\begin{enumerate}
\def\labelenumi{\arabic{enumi}.}
\item
  Open Publication: Zur Offenheit von klassischen narrativen Publikationen.
\item
  Open Research Data: Zur Offenheit von Forschungsdaten und Zusatzmaterialien.
\item
  Open Review: Zur Offenheit von Qualitätssicherungs- und Kreditierungsverfahren.
\end{enumerate}

\section{Ergebnisse der
ExpertInneninterviews}\label{ergebnisse-der-expertinneninterviews}

Bei der Auswertung der ExpertInneninterviews wurde weniger auf eine
Bestimmung quantitativer Verhältnisse Wert gelegt als vielmehr auf eine
qualitative Inhaltsanalyse von Pro- und Kontraargumenten, um ein
möglichst breites und differenziertes Meinungsbild zu
rekonstruieren.\footnote{Zur qualitativen Inhaltsanalyse anhand
  induktiver Kategorienbildung vgl. Mayring 2008, Kuckartz et al. 2007.}
Im Folgenden werden daher zu den oben genannten Bereichen eine Reihe von
Grundaussagen im Sinne von Argumenten bzw. Statements spezifiziert und
mit signifikanten Protokollauszügen -- zum Teil leicht paraphrasiert --
belegt (siehe Anhang).\footnote{Eine umfassende Sammlung von
  anonymisierten Einzelaussagen sind über das im Projekt entwickelte
  Retrieval-Tool \enquote{Statement Finder} als
  Forschungsdatenpublikation verfügbar:
  \url{https://www2.hu-berlin.de/fupush/statement-finder/\#/statements}.}

\subsection{1. Open Publication: Zur Offenheit von klassischen
narrativen
Publikationen}\label{open-publication-zur-offenheit-von-klassischen-narrativen-publikationen}

Unter einer Open-Access-Veröffentlichung soll zunächst eine offen und
frei zugängliche klassische narrative Ergebnis- bzw. Schlusspublikation
(hier bezeichnet als \enquote{Open Publication}) verstanden werden, die
in den Geisteswissenschaften vorwiegend in Form von Monografien,
Sammelbandbeiträgen bzw. Zeitschriftenaufsätzen erscheint.\footnote{Zur
  gegenwärtigen Rolle von Open-Access-Monografien vgl. Crossick 2015,
  Müller 2012.} Eine spezielle Form stellen edierte bzw. kommentierte
Werkausgaben dar, die in den Geisteswissenschaften traditionell als
eigenständige Publikationen gelten. Insofern bilden digitale Editionen
einen besonderen Anwendungsbereich von Open-Access-Veröffentlichungen
und liegen als Bezugsobjekte für weitere Forschung häufig direkt an der
Schnittstelle zu Open Research Data.

In Übereinstimmung mit anderen Studien zeigt sich, dass die Befragten
eindeutig zwischen der Sicht als Forschende sowie der Sicht als
AutorInnen unterscheiden.\footnote{Vgl. Dupont 2011.} Auf der einen
Seite scheint es einen Konsens zu geben, dass Open Publications für die
eigene Informationsversorgung erhebliche Vorteile (z.\,B. Sichtbarkeit,
Verfügbarkeit, Prozessierbarkeit, Vervielfältigung, Weiterverarbeitung,
Verlinkung, Publikationsgeschwindigkeit) gegenüber traditionellen
Publikationsformen bieten. In diesem Sinne spricht man sich meist
dezidiert für den Open-Access-Gedanken aus. Auf der anderen Seite steht
das Publikationsverhalten der Befragten oftmals in auffälliger
Diskrepanz zu ihrer eigenen Forderung nach mehr Offenheit. Als Gründe
werden vor allem karrierewirksame -- selten finanzielle -- Interessen
angeführt, bei denen die Reputation einschlägiger Verlage sowie die
Aussicht auf Kreditierung beispielsweise anhand von Publikationslisten
in Berufungsverfahren höher wiegen als Offenheit und Transparenz von
Veröffentlichungen. Nur in Ausnahmefällen und vornehmlich von
etablierten WissenschaftlerInnen wird eine ausdrückliche
Open-Access-Strategie verfolgt, bei der mögliche Einschränkungen von
Publikationsoptionen bewusst in Kauf genommen werden. Allerdings scheint
es weder eindeutige Präferenzen für entsprechende Publikationswege (z.\,B.
\enquote{Goldener Weg} im Sinne einer Erstveröffentlichung als
Open-Access-Verlagspublikation oder \enquote{Grüner Weg} im Sinne einer
Zweitveröffentlichung als sogenanntes Self-archiving etwa auf einem
Repositorium)\footnote{Zur Unterscheidung der Open-Access-Strategien des
  Grünen und Goldenen Wegs vgl.
  \url{https://www.open-access.net/informationen-zu-open-access/open-access-strategien/}.}
noch für zu Grunde liegende Finanzierungs- bzw. Geschäftsmodelle (z.\,B.
Author-Pays-Modell, Publikationsfonds, Freemium-Modell)\footnote{Eine
  Übersicht zu Open-Access-Finanzierungsmodellen bietet die
  Informationsplattform Open Access:
  \url{https://open-access.net/informationen-zu-open-access/geschaeftsmodelle/}.}
zu geben.

In der Regel werden Open Publications, wenn diese nicht ohnehin etwa von
Drittmittelgebern vorgeschrieben werden, bislang nur als eine
nachgeordnete Option für Neben- und Zweitveröffentlichungen angesehen.
In diesem Zusammenhang wird die Förderung der Inanspruchnahme des
Zweitveröffentlichungsrechtes unstrittig als wünschenswert angesehen,
teilweise sogar als verpflichtende Forderung empfohlen.\footnote{Zur
  Regelung des Zweitveröffentlichungsrechts vgl. Spielkamp 2015.}
Dagegen rät man NachwuchswissenschaftlerInnen aus karrieretechnischen
Gründen von einer reinen Open-Access-Veröffentlichung insbesondere von
Qualifikationsarbeiten ausdrücklich ab. Dies geschieht sogar in Fällen,
in denen die Befragten prinzipiell Befürworter des Open-Access-Gedankens
sind.

Zum Teil wird das Offenheitsprinzip zwar grundsätzlich als positiv
bewertet, jedoch keineswegs als oberstes Ziel der Wissenschaft
angesehen. Vielmehr findet sich häufig eine gewisse Skepsis hinsichtlich
bestimmter Finanzierungsmodelle (z.\,B. Author-Pays-Modell), bestimmter
Publikationsoptionen (z.\,B. Self-archiving) oder bestimmter
Qualitätssicherungsverfahren (z.\,B. Open Peer Review). Es lassen sich
allerdings auch Vorurteile identifizieren, die sich angesichts
empirischer Forschung kaum aufrecht erhalten lassen, so etwa die
Behauptungen, dass Open Publications weniger rezipiert werden bzw. die
Informationsversorgung der FachwissenschaftlerInnen nicht wesentlich
verbessern oder per se unzureichende Qualitätssicherungsverfahren
durchlaufen würden.\footnote{Vgl. Davis 2011, Weishaupt 2009, vgl. auch
  BioMedCentral: \url{http://www.biomedcentral.com/about/advocacy12}.}

Weiterhin werden bestehende Unterschiede von disziplinspezifischen
Publikationskulturen deutlich. So ist das digitale Publizieren als eine
notwendige Voraussetzung von Open-Access-Ver\-öffentlichungen bereits an
sich unterschiedlich ausgeprägt, wobei sich ein vor allem durch die
angewandten Methoden geprägtes Kontinuum etwa von der Computerlinguistik
mit einer hohen bis zur Alten Geschichte mit einer geringen Ausprägung
ausmachen lässt. Inwieweit die Bereitschaft Open Access oder auch nur
digital zu publizieren eine Generationsfrage ist, wird unterschiedlich
beurteilt. Beispielsweise wird darauf verwiesen, dass die jüngere
Generation der sogenannten Digital Natives zwar in der Konsumentenrolle,
d.\,h. bei der eigenen digitalen Informationsversorgung kompetent ist,
jedoch weniger in der Produzentenrolle, d.\,h. bei der Erstellung und
Veröffentlichung wissenschaftlicher digitaler Publikationen. Dies wird
auf die Unerfahrenheit sowohl mit dem Wissenschaftssystem als auch mit
Publikationsprozessen, vor allem aber auf eine unzureichende
Kompetenzvermittlung in der Lehre zurückgeführt. Zudem sind
NachwuchswissenschaftlerInnen in besonderem Maße auf eine traditionelle
Kreditierung ihrer Veröffentlichungen angewiesen, weshalb oftmals von
einer Open-Access-Veröffentlichung abgesehen wird.

Zu den Motivationen offen zu publizieren zählt teilweise auch, größere
Unabhängigkeit von kommerziellen Verlagen zu erlangen, da deren
ökonomische Interessen nach exklusiven Verwertungsrechten den Interessen
der AutorInnen nach größtmöglicher Verbreitung und Nutzung ihrer Werke
entgegenstehen. Auch HerausgeberInnen von Sammelbänden und
Zeitschriften, aber auch von Monografien\footnote{Ein Verlagsbeispiel
  für Open-Access-Monografien bietet Language Science Press:
  \url{http://langsci-press.org/}.} sind bestrebt, restriktive
Verlagsverträge zu vermeiden und wenden sich zunehmend alternativen
Publikationsoptionen zu wie etwa Open Journal Systems (OJS) oder offenen
Publikationsplattformen (z.\,B. ScienceOpen)\footnote{ScienceOpen:
  \url{https://www.scienceopen.com/home}.}, die allerdings im
STM-Bereich bislang deutlich weiter verbreitet sind.

Als die größten Herausforderungen für Open Access in den
Geisteswissenschaften und zugleich Hauptursachen für die bislang
ausbleibende breitenwirksame Anerkennung des Offenheitsprinzips durch
die jeweiligen Fach-Communities werden übereinstimmend die folgenden
Aspekte angesehen:

\begin{itemize}
\item
  unsicherer Umgang mit rechtlichen Rahmenbedingungen;
\item
  unzureichend entwickelte Finanzierungsmodelle;
\item
  unverlässliche Strategien zur Langzeitarchivierung und -verfügbarkeit;
\item
  unausgereifte Qualitätssicherungs- und Kreditierungsverfahren.
\end{itemize}

Insbesondere wird die Bedeutung adäquater Qualitätssicherungs- und
Kreditierungsverfahren betont, wobei darauf hingewiesen wird, dass
selbst dort, wo gleichwertige Review-Verfahren für Open Publications
Anwendung finden, diese bislang keineswegs eine gleichwertige
Anerkennung gewährleisten. Die vielfach begrüßte wissenschaftspolitische
Forderung nach mehr Open Access wird daher einen Mentalitätswandel
erfordern, dessen Förderung die Befragten -- neben verstärkten
Anreizsystemen etwa in Form von Publikationsfonds -- eben auch in
längerfristigen Lernprozessen durch Best-Practice-Beispiele sowie durch
eine größere Aufmerksamkeit für offenes digitales Publizieren bereits in
der Lehre sehen.

\emph{Grundaussagen zu Open Publications}:\footnote{Die einzelnen
  Grundaussagen können sich zum Teil widersprechen, da es sich lediglich
  um die Abbildung eines vielschichtigen und vielstimmigen
  Meinungsbildes handelt. Die Referenzen in den eckigen Klammern
  verweisen auf konkrete ExpertInnenaussagen, die -- gegebenenfalls
  leicht paraphrasiert -- den Interviewprotokollen entnommen wurden
  (siehe Anhang).}

\emph{Epistemisch/informationsethisch}

\begin{enumerate}
\def\labelenumi{(\arabic{enumi})}
\item
  Prinzip der Offenheit lässt sich wissenschaftsintern begründen. {[}4{]}
\item
  Offenheit sollte ein Standard in der Wissenschaft sein. {[}6, 64{]}
\item
  Es gibt Grenzen für das Offenheitsprinzip. {[}6, 7{]}
\item
  Begriff der Offenheit sollte nicht \enquote{entsubstanzialisiert}
  werden. {[}5{]}
\item
  Open Access verhindert die Privatisierung von Wissen. {[}32, 32{]}
\item
  Open Access ist nicht das höchste Ziel. {[}18, 55{]}
\end{enumerate}

\pagebreak
\emph{Wissenschaftssoziologisch}

\begin{enumerate}
\def\labelenumi{(\arabic{enumi})}
\item
  Open Publication wird generational unterschiedlich bewertet. {[}19{]}
\item
  Open-Access-Kultur ist international unterschiedlich ausgeprägt. {[}57{]}
\item
  In den Geisteswissenschaften gibt es keine Preprint-Kultur. {[}50{]}
\item
  Open Publications sind auch im Digital-Humanities-Bereich nicht vorherrschend. {[}47{]}
\item
  Open Publication für Rezensionen hat sich bereits durchgesetzt. {[}54{]}
\item
  Open Publication ist für NachwuchswissenschaftlerInnen nicht zu empfehlen. {[}20, 21{]}
\item
  Open Publication wird kaum kreditiert. {[}51, 52, 53{]}
\item
  Open Publication gilt oft als unprofessionell. {[}58{]}
\end{enumerate}

\emph{Wissenschaftsökonomisch}

\begin{enumerate}
\def\labelenumi{(\arabic{enumi})}
\item
  Open Publication verhindert eine Mehrfachfinanzierung. {[}1, 2, 3{]}
\item
  Open-Access-Freikauf (beim Goldenen Weg) kann Mehrfachfinanzierung bedeuten. {[}41{]}
\item
  Open Publication erfordert adäquate Finanzierungsmodelle und Ressourcen. {[}17, 36, 37, 38, 39{]}
\item
  Open Publication ermöglicht kaum eine Refinanzierung kostenintensiver Publikationen. {[}29{]}
\item
  Open-Access-Zweitveröffentlichung (Grüner Weg) wirkt sich nicht negativ auf Verkaufszahlen aus. {[}31{]}
\item
  Die Lizenzeinschränkung auf nicht-kommerzielle Nutzung steht dem Open-Access-Ge\-danken entgegen.\footnote{Zur
    Differenzierung von Formen der Open-Content-Lizenzierung vgl.
    Kreutzer 2011.} {[}34{]}
\end{enumerate}

\emph{Technisch}

\begin{enumerate}
\def\labelenumi{(\arabic{enumi})}
\item
  Open Access schließt die Maschinenlesbarkeit von Publikationen ein. {[}14, 35{]}
\item
  Open Publication erfordert Strategien für Langzeitarchivierung bzw. -verfügbarkeit. {[}42, 43, 44, 45{]}
\item
  Eine Versionierung von Publikationen ist nur bedingt sinnvoll. {[}49{]}
\item
  Open Publication auf Repositorien bietet Mehrwerte. {[}30{]}
\end{enumerate}

\emph{Rezeptioniell}

\begin{enumerate}
\def\labelenumi{(\arabic{enumi})}
\item
  Open Publication ist nutzerfreundlich. {[}40{]}
\item
  Open Access erhöht die Sichtbarkeit von Publikationen. {[}8, 9{]}
\item
  Open Access erhöht die Verfügbarkeit von Publikationen. {[}10, 11, 12, 13, 14, 15, 16{]}
\item
  Open-Access-Veröffentlichungen sind zitierfähig. {[}46{]}
\item
  Open Publication bietet die Möglichkeit zur kreativen Nachnutzung. {[}48{]}
\end{enumerate}

\emph{Publikationstrategisch}

\begin{enumerate}
\def\labelenumi{(\arabic{enumi})}
\item
  Open Access ist als Publikationsoption nur bedingt relevant. {[}25, 26{]}
\item
  Open Publication erfordert adäquate Distributionsstrategien. {[}45{]}
\item
  Open Publication bietet sich für Teilpublikationen an. {[}56{]}
\item
  Open Access kann sich negativ auf traditionelle Publikationsformen auswirken. {[}28{]}
\end{enumerate}

\emph{Rechtlich}

\begin{enumerate}
\def\labelenumi{(\arabic{enumi})}
\item
  Für Open Publication bestehen Unsicherheiten in Urheberrechts- und Lizenzierungsfragen. {[}61{]}
\item
  Open Publication erfordert eine nicht-exklusive Rechteübertragung. {[}59{]}
\item
  Open Access scheitert oft an rechtlichen Rahmenbedingungen. {[}27{]}
\end{enumerate}

\emph{Förderpolitisch}

\begin{enumerate}
\def\labelenumi{(\arabic{enumi})}
\item
  Open-Access-Publizieren erfordert neue Anreizsysteme. {[}60, 63{]}
\item
  Open Access sollte durch adäquate Veröffentlichungsplattformen gefördert werden. {[}62{]}
\item
  Die Inanspruchnahme des Zweitveröffentlichungsrechtes sollte gefördert werden. {[}22, 23, 24{]}
\item
  Förderung von Open Publication senkt den Selektionsdruck für potentielle Veröffentlichungen. {[}63{]}
\end{enumerate}

\subsection{2. Open Research Data: Zur Offenheit von Forschungsdaten und
Zusatzmaterialien}\label{open-research-data-zur-offenheit-von-forschungsdaten-und-zusatzmaterialien}

Zu den Potenzialen der digitalen Transformation des wissenschaftlichen
Publizierens zählt insbesondere der Übergang von einer traditionellen
\enquote{Ergebnispublikation} hin zu einer mit funktionalen
Erweiterungen angereicherten Enhanced Publication, die es unter anderem
im Sinne einer \enquote{Prozesspublikation} ermöglicht, den
Forschungsprozess anhand der zu Grunde liegenden Forschungsdaten bzw.
Zusatzmaterialien transparenter zu gestalten.\footnote{Vgl. Degkwitz
  2014, Vernooy-Gerritsen 2009.} Diese Anreicherungen können entweder in
traditionelle Publikationsformen integriert oder als eigenständige
\enquote{Forschungsdatenpublikationen} mit eigenen persistenten
Identifikatoren beispielsweise auf Forschungsdatenrepositorien
veröffentlicht werden.\footnote{Beispiele für
  Forschungsdatenrepositorien in den Geisteswissenschaften sind das
  interdisziplinäre TextGriD Repository:
  \url{http://www.textgridrep.de/}; das Laudatio Repository für die
  historische Korpuslinguistik:
  \url{http://www.laudatio-repository.org/repository/}; oder Arachne für
  die Archäologie:
  \href{http://arachne.uni-koeln.de/drupal/?q=de/node/3}{http://arachne.uni-koeln.de/drupal/?q=node/3}.
  Einen Überblick bietet das Verzeichnis re3data.org:
  \url{http://www.re3data.org/}.}

Bereits die Veröffentlichung solcher Anreicherungen stellt einen
genuinen Zuwachs an Offenheit im Sinne von Transparenz dar, ganz
unabhängig von der Frage, ob dies mittels einer dezidierten
Open-Access-Veröffentlichung oder einer gegebenenfalls kostenpflichtigen
Publikation (z.\,B. als kostenpflichtiger Teil des
Freemium-Modells)\footnote{Ein Beispiel für das
  Freemium-Geschäftsmodell, bei dem eine einfache Version Open Access
  publiziert wird und eine erweiterte Version kostenpflichtig ist,
  bietet die französische Publikationsplattform Open Edition:
  \url{http://www.openedition.org/14043?lang=en}.} erfolgt. Im letzteren
Falle bliebe daher zu klären, ob und unter welchen Bedingungen die
Möglichkeiten zur Vervielfältigung oder Nachnutzung gegeben wären. Unter
offenen Forschungsdaten (Open Research Data im weiteren Sinne) sollen
hier Open-Access-Veröffentlichungen verstanden werden, die entweder eine
epistemische Funktion (Open Research Data im engeren Sinne), eine
illustrative Funktion (Open Extra Material) oder eine evaluative
Funktion (Open Postpublication Data) besitzen.\footnote{Zur
  Unterscheidung von Forschungsdaten, Zusatzmaterial und
  Postpublikationsdaten im Zusammenhang mit Enhanced Publications vgl.
  Vernooy-Gerritsen 2009.}

Im Unterschied zu den Natur- und Sozialwissenschaften bezeichnen
Forschungsdaten in den Geisteswissenschaften vor allem Dokumente der
kulturellen Überlieferung in allen ihren multimedialen Formen, darunter
insbesondere sogenannte Primärtexte, die vorrangig in
Gedächtnisinstitutionen wie Archiven, Bibliotheken oder Museen
aufbewahrt werden und im Zuge großangelegter Digitalisierungsprojekte
oder auch \emph{born digital} zunehmend in maschinenlesbar aufbereiteten
Digitalisaten zur Verfügung stehen. Entsprechende Transkriptionen liegen
zumeist in semantisch strukturierter Form und in zum Teil umfangreichen
Textkorpora vor. Daneben spielen eine Reihe weiterer Formen von
Forschungsdaten eine bedeutende Rolle wie etwa Tonaufnahmen in der
Sprachwissenschaft oder 3D-Simulationen, Geodaten und Vermessungsdaten
in der Archäologie. Eine besondere Form von Forschungsdaten stellen
Fachdatenbanken dar, die oftmals mit standardisierten Normdaten wie der
Gemeinsamen Normdatei (GND)\footnote{Gemeinsame Normdatei:
  http://www.dnb.de/DE/Standardisierung/GND/gnd\_node.html} abgeglichen
sind.

Im weiteren Sinne können auch sämtliche Präpublikationsdaten wie
Exzerpte, Notizen oder Kommentare, aber auch Teilergebnisse und
Negativresultate sowie die interne und externe
Wissenschaftskommunikation (z.\,B. White Paper, Deliverables,
Präsentationen, Projektwebseiten, Wissenschaftsblogs, Wikis) ebenso wie
Postpublikationsdaten (z.\,B. Reviews, Rezensionen, Zitationsindizes,
Impactmessungen, Nutzungsstatistiken) als potentiell relevante
Forschungsdaten bzw. Zusatzmaterialien angesehen werden.

Um zu ermitteln, inwieweit die technischen Möglichkeiten der
Publikationsanreicherung auch Relevanz für geisteswissenschaftliche
Fach-Communities besitzen, wurden im Rahmen der Interviews zunächst
typische Forschungsdaten bzw. Zusatzmaterialien disziplinspezifisch
erhoben, um anschließend zu fragen, in welchem Ausmaß und unter welchen
Bedingungen dieses ohnehin digital vorliegende Material zur
Kontextualisierung einer klassischen narrativen Publikation sinnvoll
erscheint.

Die Ergebnisse der Befragung zeigen, dass dem Thema Forschungsdaten
allgemein eine große Bedeutung zugemessen wird, insbesondere in
Bereichen datenintensiver Forschung wie den Digital Humanities. Ähnlich
wie bei klassischen narrativen Publikationen werden auch bei der
Veröffentlichung von Forschungsdaten bzw. Zusatzmaterialien das
Steuerzahlerargument (freier und offener Zugang ohne
Mehrfachfinanzierung durch die öffentliche Hand) sowie die Möglichkeit
der Nachnutzung hervorgehoben. Es wird betont, dass frei und offen
zugängliche Digitalisate von Dokumenten der kulturellen Überlieferung
als Quellenmaterial -- und somit als Forschungsdaten -- eine erhebliche
Erleichterung für die Recherche, Beschaffung, Bearbeitung und Analyse
bieten sowie gegebenenfalls für die direkte Einbindung in eigene
digitale Publikationen. Zudem verspricht man sich Mehrwerte durch die
Vernetzung von offenen Forschungsdaten (z.\,B.
Linked-Open-Data-Technologie)\footnote{Vgl. Wiljes et al. 2013, Stäcker
  2013.} sowie durch die Entwicklung datenbankübergreifender
Rechercheinstrumente. Ein besonders stichhaltiges Argument wird in der
notwendigen Nachvollziehbarkeit von Forschungsergebnissen gesehen, deren
Datengrundlagen daher offen und bei computergestützten Verfahren sogar
mitsamt den angewandten Algorithmen bzw. Softwareprogrammen (Open
Source) zur Verfügung stehen sollten.

Das gleiche Argument lässt sich dagegen nur bedingt für die Forderung
nach einer Open-Access-Veröffentlichung weiterer Zusatzmaterialen
anwenden, da diese weniger eine genuin epistemische als vielmehr eine
illustrative und im Falle von Postpublikationsdaten eine evaluative
Funktion besitzen. Hierbei stellt sich nicht nur die Frage nach Open
Access als einer Option neben anderen, sondern sehr viel grundlegender
nach der Relevanz einer Veröffentlichung überhaupt. In diesem
Zusammenhang lassen sich deutliche Unterschiede bezogen auf die Arten
des Zusatzmaterials ausmachen.

So sind viele Befragte skeptisch hinsichtlich einer Veröffentlichung von
ursprünglich für den Privatgebrauch bestimmten Präpublikationsdaten wie
persönlichen Exzerpten, Notizen und Kommentaren oder nach eigenen
Standards erstellten Datenbanken, da diese eine Nachbearbeitung bzw.
Dokumentation und folglich einen erheblichen Mehraufwand erfordern
würden. Die Veröffentlichung von Teilergebnissen beispielsweise als
eigenständige Publikation bzw. Preprint sowie von Negativergebnissen,
die als besonders relevant bei quantitativen bzw. algorithmenbasierten
Verfahren angesehen werden, wird dagegen begrüßt.

Bei der Wissenschaftskommunikation wird streng zwischen internen und
externen Kommunikationskanälen unterschieden, wobei die Anreicherung
einer Publikation mit extern kommunizierten Inhalten (z.\,B. Webseiten,
Blogs, offene Wikis) als vergleichsweise unproblematisch angesehen wird.
Generell spricht man sich dafür aus, dass die Entscheidung über die
Integration solcher Zusatzmaterialien in Publikationen, etwa als
Grundlage für eine adäquate Langzeitarchivierung bzw. -verfügbarkeit,
bei den Verantwortlichen selbst liegen sollte.

Die Anreicherung mit Postpublikationsdaten gilt grundsätzlich als ein
informationeller Mehrwert insbesondere als Relevanzfilter bei der
Informationsversorgung. Zugleich wird auf potentielle Gefahren
hingewiesen. So könnten etwa negative oder einseitige Rezensionen
dauerhaft in den Kontext einer Publikation -- nicht zuletzt auch bei
Trefferlisten von Suchmaschinen -- stehen und folglich die Rezeption
negativ beeinflussen. Weiterhin sieht man quantitative Impact-Messungen,
die etwa auf Zitationsindizes oder Nutzungsstatistiken beruhen als nicht
adäquat für die Evaluation geisteswissenschaftlicher Forschung an,
wenngleich solchen Verfahren keineswegs generell ein Mehrwert
abgesprochen wird.\footnote{Ein Beispiel für alternative
  Impact-Messverfahren auf Artikelebene bietet Altmetrics:
  \url{http://www.altmetric.com/}.}

Noch stärker als bei Open Publications wird eine zentrale
Herausforderung für Open Research Data in der Qualitätssicherung
gesehen, für die es bislang kaum standardisierte Dokumentations- bzw.
Evaluationsverfahren gibt. Insbesondere in den Digital Humanities sollte
nach Ansicht der Befragten mehr Aufmerksamkeit auf der Quellenkritik und
damit der Datenqualität liegen, da oftmals ein eher pragmatischer Umgang
mit Primärtexten vorherrscht. Beispielsweise werden mitunter online
verfügbare Textvarianten einer kostenpflichtigen historisch-kritischen
Werkausgabe auf dem neuesten Forschungsstand vorgezogen, oder es wird
auf eine adäquate Datenaufbereitung etwa durch linguistische
Tiefenauszeichnung verzichtet bzw. nicht nach einheitlichen Standards
vorgenommen.

Schließlich betonen die Befragten rechtliche Unsicherheiten etwa
hinsichtlich der Lizenzierung oder der Verantwortung für die
Langzeitarchivierung und -verfügbarkeit. So ist beispielsweise der
Urheber- bzw. Rechtestatus der beteiligten natürlichen bzw. juristischen
Personen (z.\,B. ProjektleiterIn, ProjektmitarbeiterIn, IT-MitarbeiterIn,
Forschungseinrichtung, Förderinstitution) für kollektiv erstellte
Forschungsdaten einschließlich Infrastrukturleistungen (z.\,B.
Softwareanwendungen in VFUs) nicht immer eindeutig geklärt. Zudem gelten
gängige Dokumentenserver bzw. Repositorien von Forschungseinrichtungen
als unzureichend für die Langzeitarchivierung und -verfügbarkeit,
weshalb die zunehmende Etablierung und Standardisierung von
Forschungsdatenrepositorien ausdrücklich begrüßt wird. Hier wird die
Bedeutung von Nachhaltigkeitsaspekten betont, die vor allem angesichts
begrenzter Projektlaufzeiten eine Herausforderung darstellen. Man sieht
an dieser Stelle vor allem Forschungsförderinstitutionen in der Pflicht,
geeignete -- gegebenenfalls disziplinspezifische -- Entwicklungen von
Informationsinfrastrukturen sowie Strategien für das
Forschungsdatenmanagement mit entsprechenden Anreizsystemen zu
forcieren.

\emph{Grundaussagen zu Open Research Data}:

\emph{Epistemisch/informationsethisch}

\begin{enumerate}
\def\labelenumi{(\arabic{enumi})}
\item
  Die durch eine öffentlich finanzierte Wissenschaft erhobenen Forschungsdaten und erzielten Forschungsergebnisse
  sollten der Allgemeinheit zur Verfügung stehen
  (Steuerzahlerargument). {[}65, 71, 125{]}
\item
  Forschungsdaten sollten nachgenutzt werden können. {[}72, 129{]}
\item
  Der Nachvollzug von Forschungsergebnissen erfordert den freien Zugang zu den Forschungsdaten bzw. angewandten
  Softwaretools. {[}66, 67, 68, 69, 70, 106, 112{]}
\item
  Forschungsdaten müssen maschinenverarbeitbar sein. {[}120{]}
\item
  Forschungsdaten sollten mit Kontextinformationen veröffentlicht werden. {[}121{]}
\item
  Die Veröffentlichung von Datenbanken erfordert eine Dokumentation über die Daten (einschließlich Entstehungs- und Bearbeitungsprozess). {[}86, 108, 131{]}
\item
  Forschungsdaten in Form von Digitalisaten haben epistemische Nebeneffekte (z.\,B. auf Grund fehlender Autopsie).
  {[}109, 110{]}
\end{enumerate}

\emph{Forschungsdaten (Open Research Data im engeren Sinn)}

\begin{enumerate}
\def\labelenumi{(\arabic{enumi})}
\item
  Frei zugängliche Digitalisate erleichtern die Quellenarbeit und bieten Mehrwerte. {[}73, 122{]}
\item
  Die Einbindung von offenen Fremdforschungsdaten in Publikationen bietet Mehrwerte. {[}101, 102{]}
\item
  Die Vorabveröffentlichung von Forschungsdaten bietet Mehrwerte. {[}103, 104, 105, 107{]}
\item
  Forschungsdaten können eigenständige Publikationen darstellen. {[}96, 97, 98, 99, 100{]}
\item
  Metadaten als Linked Open Data bieten sich für Tiefenerschließungen an. {[}123{]}
\item
  Die Referenzierung auf digitale Forschungsdaten ist kaum anerkannt. {[}127{]}
\item
  Die Veröffentlichung von Forschungsdaten erfordert einen hohen Pflegeaufwand. {[}87, 88{]}
\item
  Der Veröffentlichung von Forschungsdaten stehen oft rechtliche Probleme im Weg. {[}89, 90, 91, 92, 93, 94, 116, 125{]}
\item
  Strittige Urheberrechtsfragen lassen sich durch eine Bearbeitung mit eigener Schöpfungshöhe umgehen. {[}95{]}
\item
  Das Recht auf Metadatenvergabe sollte nicht unnötig eingeschränkt werden. {[}117{]}
\item
  Die Langzeitverfügbarkeit von Forschungsdaten erfordert geeignete Forschungsdatenrepositorien. {[}73, 74, 75, 76, 77, 78,
  79{]}
\end{enumerate}

\emph{Zusatzmaterialien (Open Extra Material)}

\begin{enumerate}
\def\labelenumi{(\arabic{enumi})}
\item
  Zusatzmaterialien sind nur bedingt für eine Veröffentlichung geeignet. {[}80, 81, 82, 83, 84, 85, 86{]}
\item
  Die Veröffentlichung externer Wissenschaftskommunikation bietet Mehrwerte. {[}118, 119, 126{]}
\item
  Die Veröffentlichung interner Wissenschaftskommunikation bietet nur bedingt Mehrwerte. {[}118, 119{]}
\item
  Die Entscheidung über die Veröffentlichung von Zusatzmaterialien sollte bei den Projektverantwortlichen liegen.
  {[}114{]}
\item
  Offene Software (Open Source) bietet Mehrwerte. {[}124, 130{]}
\item
  Postpublikationsdaten (z.\,B. Nutzungsdaten) bieten Mehrwerte für Relevanzfilterverfahren. {[}111, 113{]}
\item
  Die Veröffentlichung von Teil- und Negativresultaten bietet Mehrwerte. {[}115{]}
\item
  Die Nutzung von offenen Diskussionsforen ist gering. {[}128{]}
\end{enumerate}

\subsection{3. Open Review: Zur Offenheit von Qualitätssicherungs- und
Kreditierungsverfahren}\label{open-review-zur-offenheit-von-qualituxe4tssicherungs--und-kreditierungsverfahren}

Der Qualitätssicherungsprozess sowohl von Forschungsergebnissen als auch
von Forschungsdaten bildet einen zentralen Bereich, in dem die
Geisteswissenschaften mit dem Prinzip der Offenheit konfrontiert werden.
Verfahren zur Qualitätssicherung lassen sich zunächst nach bestimmten
Aspekten differenzieren wie anhand der Gutachtenden (z.\,B. Peers der
Fach-Community, HerausgeberInnen, FachlektorInnen), des Zeitpunktes der
Begutachtung (Pre-Publication/Post-Publication) sowie des Grades der
Transparenz (Double-Blind, Single-Blind, Open). Auf Letzteres bezieht
sich der Begriff des \enquote{Open Review}, bei dem das Prinzip der
Offenheit -- im Gegensatz zu Open Publication und Open Data -- nicht
unmittelbar die Aspekte der Vervielfältigung und Nachnutzung mit
einschließt. In einem weiteren Sinne sollen hier unter Review bzw.
Begutachtung auch Verfahren zur Kreditierung von Veröffentlichungen im
Sinne karrierewirksamer Anerkennung etwa im Zuge von Berufungsverfahren
verstanden werden.

Wie die Befragungen bestätigen, erfolgt die Begutachtung von klassischen
narrativen Publikationen in den meisten geisteswissenschaftlichen
Fachrichtungen traditionell entweder per Editorial Review durch ein
Herausgebergremium oder anhand eines Peer Reviews durch unabhängige und
anonyme Fachgutachten. Das Fachlektorat durch Verlage spielt vor allem
bei Monografien eine -- wenn auch nach Ansicht der Befragten deutlich
abnehmende -- Rolle. Während im deutschsprachigen Raum eher eine
Herausgeberkultur vorherrscht, sind die strenger geregelten
Peer-Review-Verfahren vor allem im angloamerikanischen Raum verbreitet
bzw. dort etabliert, wo die Forschung einen hohen
Internationalisierungsgrad aufweist.

Der Zeitpunkt der Begutachtung findet mit Ausnahme von Rezensionen in
aller Regel vor der Publikation (Pre-Publication-Review) statt. Dieser
Umstand führt zu den in vielen Geisteswissenschaften typischen langen
Publikationszeiten. Im Gegensatz zum STM-Bereich gibt es in den
Geisteswissenschaften keine genuine Preprint-Kultur\footnote{Den
  historischen Fall der Sonderdrucke einmal ausgenommen.}, bei der
digitale Vorabveröffentlichungen nach einem Begutachtungsprozess
schließlich den Status einer vollwertigen Publikation erhalten können.
An dieser Stelle werden zugleich die Grenzen des traditionellen
Publikationsbegriffs deutlich, der für das Phänomen einer
\enquote{unveröffentlichten Veröffentlichung} im Sinne einer formal
nicht begutachteten Online-Publikation keine hinreichende
Differenzierung bietet und bislang durch den für das digitale
wissenschaftliche Publizieren eher unglücklichen Ausdruck
\enquote{Preprint} (Vorabdruck) ersetzt wird. Beispielsweise gelten noch
immer Qualifikationsarbeiten, die lediglich auf einem Repositorium
zugänglich sind, zwar im Sinne einiger Promotionsordnungen als
Veröffentlichungen, dagegen meist nicht im Falle einer Zitation
(\enquote{unveröffentlichte Dissertation}).

Traditionell konstituierten die Verlage durch das \enquote{Erscheinen}
den Status einer Publikation. Gerade dieser Schritt ist jedoch mit der
digitalen Transformation und neuen Infrastrukturen (z.\,B. Open Journal
Systems, Repositorien, Wissenschaftsblogs, offene
Publikationsplattformen) nicht mehr zwingend an Verlage gebunden, zumal
auch diese teilweise keine adäquate Qualitätssicherung mehr
gewährleisten, wie in den Interviews nachdrücklich betont wird. Daher
kommt neuen Formen der Qualitätssicherung beim digitalen Publizieren
eine zentrale Funktion zu. Während im Allgemeinen eine gewisse
Zustimmung zum Experimentieren mit innovativen Review-Verfahren
auszumachen ist, überwiegt die Skepsis hinsichtlich offener
Post-Publication-Reviews. Zum einen wird das Motivationsproblem
angeführt, da die Erstellung von Gutachten ohnehin eine kaum honorierte
Mehrarbeit darstellt, die bei einigen Post-Publication-Review-Verfahren
nicht mehr dezidiert angefragt wird, sondern lediglich als eine
freiwillige und unaufgeforderte Option angeboten wird. Zum anderen
verweist man darauf, dass es kaum die Bereitschaft gibt als AutorIn oder
GutachterIn aus der Anonymität zu treten. Insbesondere wird vor
verzerrenden personenbezogenen Entscheidungen gewarnt, etwa auf Grund
von persönlichen Vorbehalten oder Autoritätsbekundungen. Zudem besteht
ein Schutzbedürfnis gegenüber NachwuchswissenschaftlerInnen. Hingegen
wird als möglicher Vorteil offener Verfahren angeführt, dass auch die
Erstellung von Gutachten als eigenständige Forschungsleistung anerkannt
und gegebenenfalls in eine Publikationsliste mit aufgenommen werden
könnte.

In der gegenwärtigen Praxis reicht der Grad der möglichen Transparenz
vom Double-Blind-Verfahren eines Peer Reviews, wo die AutorInnen und die
GutachterInnen gegenseitig unbekannt sind, zum vollständig transparenten
Open Review, wobei dem Editorial Review eine Zwischenstellung zukommt,
da es sich um ein quasi-offenes Verfahren handelt, bei dem zumindest die
AutorInnen bekannt sind und zumeist auch die Zusammensetzung des
Herausgebergremiums. In jedem Falle erscheinen insbesondere in kleineren
und überschaubaren Fach-Communities für die Befragten
Double-Blind-Reviews als ungeeignet, da AutorInnen anhand ihres
speziellen Forschungsgebietes sowie ihrer individuellen Schreibweise
meist ohnehin leicht erkannt werden können.

Aber auch bei einem Open-Review-Verfahren sind verschiedene Grade der
Transparenz denkbar. Beispielsweise könnten zwar AutorInnen und
GutachterInnen gegenseitig namentlich bekannt sein und gegebenenfalls
auch direkt miteinander kommunizieren, doch der gesamte
Begutachtungsprozess intern, d.\,h. nicht öffentlich stattfinden.
Andererseits könnte der Begutachtungsprozess nicht nur öffentlich
einsehbar sein, sondern auch hinsichtlich der Gutachtenden für die
gesamte (Fach-)Community etwa in spezifischen Review-Foren geöffnet
werden. Unabhängig davon ließe sich auch fragen, inwieweit die
Evaluationskriterien bei der Qualitätssicherung offengelegt werden
können. In der gegenwärtigen Diskussion wird unter Open Review vor allem
ein zugleich transparentes und öffentliches Qualitätssicherungsverfahren
verstanden. Allerdings machen die Befragungen auch deutlich, dass es für
ein Open Review bislang kaum Erfahrungen geschweige denn
Best-Practice-Beispiele gibt.

Wie bereits angedeutet, besteht eine große Herausforderung für
Open-Access-Veröffentlichungen in der adäquaten Kreditierung durch die
Fach-Community. Es wird darauf verwiesen, dass die Reputation einer
Publikation keineswegs am Label des Verlages liegt, sondern vielmehr an
der Zusammensetzung des Herausgebergremiums, was erfolgreiche Beispiele
von Umstellungen auf verlagsunabhängige Open-Access-Zeitschriften
belegen.\footnote{Vgl. Edition TOPOI: \url{http://edition-topoi.org/},
  PloS: \url{https://www.plos.org/}.}

Allgemein gelten jedoch digitale Publikationen, die nur online
erscheinen, bislang als wenig anerkennungswürdig. Laut der Befragten
liegt das vor allem daran, dass in der Tat viele Dokumente offen und
frei zugänglich sind, die keinerlei Qualitätssicherungsprozess
durchlaufen haben und oftmals nicht eindeutig als
\enquote{wissenschaftlich} identifizierbar sind bzw. deren Integrität
und Authentizität angezweifelt werden können. Insofern sollte man
zunächst zwischen einer digitalen Publikation im Allgemeinen und einer
Open-Access-Veröffentlichung im Besonderen unterscheiden, da sich viele
Vorbehalte eben nicht gegen das Offenheitsprinzip wenden, sondern
vielmehr undifferenziert gegen die Unübersichtlichkeit und
Unkontrollierbarkeit des Internets. Gerade an dieser Stelle könnten sich
daher Post-Publication-Review-Verfahren als geeignete Filtermechanismen
erweisen.

\pagebreak
\emph{Grundaussagen zu Open Review}:

\emph{Qualitätssicherungsverfahren}

\begin{enumerate}
\def\labelenumi{(\arabic{enumi})}
\item
  Peer-Review-Verfahren sind in den Geisteswissenschaften allgemein wenig etabliert. {[}140{]}
\item
  Ein Editorial Review übernimmt quasi die Funktion eines Open Reviews. {[}151, 152{]}
\item
  Open Review kann ein Korrektiv für das Editorial Review sein. {[}155{]}
\item
  Verlagsunabhängige Publikationen benötigen eine adäquate Qualitätssicherung. {[}134, 135, 145{]}
\item
  Open Review erhöht die Transparenz. {[}132{]}
\item
  Open Review kann die Fairness erhöhen. {[}148{]}
\item
  Open Review könnte ein Mittel für die Verifikation bzw. Falsifikation sein. {[}133{]}
\item
  Open-Review-Gutachten könnten als eigenständige Publikationsleistung gelten. {[}147{]}
\item
  Post-Publication-Review-Verfahren können die Publikationsgeschwindigkeit erhöhen. {[}153{]}
\item
  Open Review entspricht nur bedingt den geisteswissenschaftlichen Publikationskulturen. {[}137, 150{]}
\item
  Open Review kann zu personenbezogenen Entscheidungen führen. {[}141, 142{]}
\item
  Open Review kann sich nachteilig auf NachwuchswissenschaftlerInnen auswirken. {[}143{]}
\item
  Open Review hat ein Motivierungsproblem. {[}144, 146, 156{]}
\item
  Open-Access-Veröffentlichungen können in den Geisteswissenschaften Anreizwert von gedruckten Rezensionsexemplaren nicht
  ersetzen. {[}139{]}
\item
  Digitale Editionen können auch nach der Veröffentlichung noch korrigiert werden (Post-Publication-Review). {[}154{]}
\end{enumerate}

\emph{Kreditierungsverfahren}

\begin{enumerate}
\def\labelenumi{(\arabic{enumi})}
\item
  Für digitale Publikationen gibt es keine etablierten Kreditierungsverfahren. {[}136{]}
\item
  Postpublikationsdaten bieten evaluative Mehrwerte. {[}157{]}
\item
  Rein quantitative Mess- und Evaluationsverfahren sind ungeeignet für Geisteswissenschaften. {[}159, 160{]}
\item
  Die Publikationsmenge im Digitalen erfordert Relevanzfilterverfahren. {[}138, 149{]}
\item
  Für (Daten-)Repositorien sollten Qualitätssicherungsverfahren etabliert werden. {[}158{]}
\end{enumerate}

\section{Zusammenfassung: Konsens und
Dissens}\label{zusammenfassung-konsens-und-dissens}

Erwartungsgemäß zeigen die Auswertungen der ExpertInneninterviews im
Rahmen des DFG-Projektes Fu-PusH, dass es zum Thema Open Access in den
Geisteswissenschaften sowohl unstrittige als auch streitbare Aspekte
gibt. Zum allgemeinen Konsens gehört der prinzipielle Wunsch nach
weitreichender Offenheit und Transparenz in den Geisteswissenschaften.
Die Vorteile von Open-Access-Veröffentlichungen insbesondere
hinsichtlich der Sichtbarkeit, Verfügbarkeit, Prozessierbarkeit,
Vervielfältigung, Weiterbearbeitung, Verlinkung und
Publikationsgeschwindigkeit werden selbst von Kritikern nicht geleugnet.
Zu den schlagkräftigsten Argumenten zählen das Steuerzahlerargument bzw.
Nachnutzungsargument sowie das Nachvollzugsargument. Weiterhin besteht
Übereinstimmung darüber, dass die Inanspruchnahme des -- obgleich
oftmals als zu restriktiv kritisierten -- Zweitveröffentlichungsrechts
als förderungswürdig anzusehen ist. Bei den Nachteilen besteht Einigkeit
darüber, dass Open-Access-Veröffentlichungen in den jeweiligen
Fach-Communities noch immer einen zweifelhaften Ruf haben und nur
bedingt als gleichwertige wissenschaftliche Leistungen anerkannt werden.
Die größten Herausforderungen werden unisono in sicheren rechtlichen
Rahmenbedingungen, hinreichend entwickelten Finanzierungsmodellen,
verlässlichen Strategien zur Langzeitarchivierung und -verfügbarkeit
sowie ausgereiften Qualitätssicherungs- und Kreditierungsverfahren
gesehen. Die Bedeutung von offenen Forschungsdaten und der Entwicklung
entsprechender Forschungsdatenrepositorien wird gemeinhin anerkannt.

Auf der anderen Seite besteht in vielen Punkten ein Dissens, an dem sich
die gegenwärtigen Debatten entzünden. In erster Linie bleibt die
Finanzierungsfrage ungeklärt, da sich keine eindeutige Präferenz für
bestimmte Geschäftsmodelle ausmachen lässt. Abgesehen von dem
unstrittigen Grünen Weg bei einer Zweitveröffentlichung und der
ebenfalls unstrittigen Vermeidung einer Mehrfachfinanzierung beim
Goldenen Weg, etwa durch Open-Access-Freikauf eines Aufsatzes in einer
subskriptionsfinanzierten Zeitschrift, werden unterschiedliche
Publikationsstrategien verfolgt. Einerseits wird versucht, den
Open-Access-Gedanken mit den etablierten kommerziellen Verlagen
voranzubringen; andererseits werden die verlegerischen Tätigkeiten zum
Teil in Eigenregie und gegebenenfalls in Kooperation mit weiteren
Dienstleistern wahrgenommen. In beiden Fällen werden unterschiedliche
Finanzierungsmodelle wie Publikationsfonds oder Author-Pays-Modelle
erprobt, um etwa die Veröffentlichungskosten wie Article Processing
Charges (APCs) von Open-Access-Aufsätzen zu decken. Hinzu kommen
Freemium-Modelle, institutionell finanzierte Universitätsverlage oder
Anbieter von offenen Publikationsplattformen. In den Interviews wurde
vor allem deutlich, dass die Finanzierungsfrage aus
AutorInnenperspektive nicht die höchste Priorität genießt und neben der
Qualitätssicherung und Kreditierbarkeit sekundär erscheint. Schließlich
gehen die Meinungen auseinander bei der Relevanz von Zusatzmaterialien
zur Anreicherung von Publikationen, bei der Anerkennung alternativer
Kommunikations- und Publikationsformen (z.\,B. Webseiten,
Wissenschaftsblogs oder gar Twitterstreams) oder bei der Einführung von
offenen Review-Verfahren.

In vielen Punkten scheint der Diskussionsstand zum Prinzip der Offenheit
heute nicht signifikant weiter entwickelt zu sein als noch zu Zeiten der
Berliner Erklärung.\footnote{Vgl. Heise 2015.} Betrachtet man jedoch die
bisherige Geschichte der Open-Access-Debatte wiederum in historischer
Perspektive und vergleicht die Prozesse der digitalen Transformation
etwa mit denen des Gutenbergzeitalters, dann können sowohl Befürworter
als auch Kritiker eines Konzeptes wie \enquote{Open Humanities} den
oftmals zäh wirkenden Diskurs gewissermaßen aus der Außenperspektive
relativieren. Mit den Worten aus einem ExpertInneninterview:
\enquote{\emph{Open Science ist ein Fernziel, aber wir sollten uns
bereits heute technisch darauf einstellen}.}

\section{Anhang}\label{anhang}

\subsection{ExpertInnenaussagen über Open
Publication}\label{expertinnenaussagen-uxfcber-open-publication}

\begin{enumerate}
\def\labelenumi{\arabic{enumi}.}
\item
  \enquote{\emph{Ich bin durchaus ein Open-Access-Befürworter und finde,
  wenn wir Forschungsgelder investieren, um bestimmte
  Forschungsergebnisse zu schaffen, dann sollte die Community daran auch
  teilhaben können, ohne dass man das wieder zahlen muss}.} {[}Interview
  27{]}
\item
  \enquote{\emph{Wir wollen die Dinge zugänglich machen. Ich meine: Wozu
  arbeiten wir? Nicht damit es in diesen teuren Bänden irgendwo in der
  Bibliothek verschwindet. Wir wollen, dass das was wir tun und was auch
  teuer bezahlt wird, öffentlich zugänglich ist}.} {[}Interview 33{]}
\item
  \enquote{\emph{Ich finde schon, dass Wissenschaftler, die irgendwie
  öffentlich bestallt sind, die ihren Lebensunterhalt als öffentliche
  Wissenschaftler verdienen, nicht unbedingt Geld verdienen über die
  Publikationen. Da bin ich dafür, dass das Publizieren im
  aufklärerischen Sinne eine gemeinsame Suche nach der Wahrheit ist und
  nicht so sehr persönliche Leistung, die man verkaufen kann. Ob das
  geht, hängt von der Gesamtstruktur ab. Denn natürlich muss
  gewährleistet sein, dass die Leute, die solche Publikationen schaffen,
  davon leben können. Wenn sie darauf angewiesen sind, dass sie aus dem
  Ertrag ihrer Publikationen leben, dann kann man das so nicht machen,
  dann muss man andere Wege finden}.} {[}Interview 27{]}
\item
  \enquote{\emph{Das Open-Access-Modell halte ich für wissenschaftlich
  notwendig, denn es ist Teil der Open Science. Das wird von allen
  Wissenschaftlern geteilt, denn es betrifft den Kern der
  Wissenschaft}.} {[}Interview 09{]}
\item
  \enquote{\emph{Ich als Wissenschaftler habe den Anspruch, wenn ich
  etwas \enquote{offen} nenne, dass es auch den Kriterien der} Open
  Definition\emph{, die von der Open Knowledge Foundation kommt, genüge
  tut. Und damit geht es eben nicht nur um Lizenzen, die es ermöglichen,
  einen Text kostenlos im Internet auf einer Webseite anzeigen zu
  lassen, sondern für mich beinhaltet \enquote{Open} eben auch, dass die
  Weiterverwendung und die Weiterverarbeitung dieser Inhalte unter
  möglichst geringen technischen wie rechtlichen Hürden möglich ist}.}
  {[}Interview 39{]}
\item
  \enquote{\emph{Also meine Hoffnung, auch wenn sie vielleicht utopisch
  ist, wäre natürlich, dass alles so umfassend wie möglich
  veröffentlicht wird und dass die einzige Grenze der Datenschutz
  darstellt oder der Schutz von persönlichen Daten. Es spricht ja nichts
  dagegen, seine Daten nicht zu veröffentlichen, wenn man das nicht
  möchte. Das soll ja auch jedem freistehen. Aber ich würde mir
  wünschen, dass der Standard eher auf offen steht und man schon
  begründen muss, warum man Inhalte zu einer gewissen Situation nicht
  veröffentlicht}.} {[}Interview 39{]}
\item
  \enquote{\emph{Es sollte möglichst alles offen sein mit der Ausnahme
  Open Data bei Qualifikationsarbeiten oder gegebenenfalls mit einer
  Karenzzeit sowie Open Review, da eine anonyme Beurteilung auch mit
  Vorzügen verbunden ist wie zum Beispiel mehr Ehrlichkeit}.}
  {[}Interview 05{]}
\item
  \enquote{\emph{Als Autorin möchte ich alle meine Publikationen frei
  zugänglich haben}.} {[}Interview 07{]}
\item
  \enquote{\emph{Auch ultra-konservative Kollegen nehmen wahr, dass man
  durch Online-Publikationen besser sichtbar ist}.} {[}Interview 14{]}
\item
  \enquote{\emph{Online geht es ja vor allem um Suchen und Finden und
  dann natürlich darum, den Kontext schnell zu begreifen. Und für den
  Kontext in einer wissenschaftlichen Publikation brauche ich auch den
  kompletten Text}.} {[}Interview 13{]}
\item
  \enquote{\emph{Autoren haben in der Wissenschaft ein Interesse an
  einer möglichst breiten Verfügbarkeit ihrer Arbeit}.} {[}Interview
  36{]}
\item
  \enquote{\emph{Gerade global gesehen ist Open Access wichtig, da die
  Literaturversorgung oftmals nicht über Nationallizenzen o.ä. geregelt
  wird}.} {[}Interview 11{]}
\item
  \enquote{\emph{Open Access erleichtert mir oft das Arbeiten. Es gibt
  Sachen, die unsere Bibliothek nicht gekauft hat, was richtig ärgerlich
  ist}.} {[}Interview 22{]}
\item
  \enquote{\emph{Ich würde mir wünschen, dass es prinzipiell jedes Buch
  digital und durchsuchbar gibt}.} {[}Interview 15{]}
\item
  \enquote{\emph{Frei zugängliche digitale Publikationen bieten eine
  schnelle Verfügbarkeit}.} {[}Interview 07{]}
\item
  \enquote{\emph{Die schnelle Verfügbarkeit ist ein großer Mehrwert.
  Dadurch könnte man auch von einer Europa- bzw. Nordamerikazentrik
  wegkommen}.} {[}Interview 19{]}
\item
  \enquote{\emph{Open Access ist niemals kostenlos}.} {[}Interview 29{]}
\item
  \enquote{\emph{Open Science scheint mir eine gute Idee zu sein. Man
  darf sie aber nicht zu einer fixen Idee werden lassen, vor allem, wenn
  sie Geld kostet, das man an anderer Stelle auch gut einsetzen könnte.
  Wenn sich das quasi von selbst ergibt, ist es okay. Ich würde aber
  nicht sagen, dass das das höchste aller denkbaren Ziele ist. Weil
  jeder, der sich wirklich interessiert und wirklich bemüht, so oder so
  drankommt. Die Frage ist, wie niedrig man die Schwelle setzen muss}.}
  {[}Interview 22{]}
\item
  \enquote{\emph{Im Bereich der Editoren und gerade bei den Jüngeren ist
  der Open-Access-Gedanke sehr verbreitet. Bei den Älteren stößt er auf
  Riesenskepsis. Das ist wirklich ein Generationskonflikt}.}
  {[}Interview 33{]}
\item
  \enquote{\emph{Open Access kann man besser praktizieren, wenn man
  bereits im Fach etabliert ist. Ich würde Nachwuchswissenschaftlern
  immer noch die gedruckte, also nicht Open-Access-Publikation
  empfehlen}.} {[}Interview 30{]}
\item
  \enquote{\emph{Es ist leichter, Materialien Open Access zur Verfügung
  zu stellen, wenn man als Wissenschaftler bereits etabliert ist. Es
  wird Promovenden auch von ihren Professoren abgeraten, ihre
  Dissertationen Open Access zu publizieren. Das sei ein Karriererisiko.
  Creative Commons wird mit Skepsis und Angst gesehen, weil eine solche
  Nutzbarkeit als potentiell schädlich für die eigene Reputation
  bewertet wird}.} {[}Interview 36{]}
\item
  \enquote{\emph{Die institutionell geförderte Nutzung des
  Zweitveröffentlichungsrechts finde ich richtig}.} {[}Interview 12{]}
\item
  \enquote{\emph{Jeder Autor sollte sein Zweitveröffentlichungsrecht
  nutzen und zumindest eine PDF auf einem Repositorium ablegen}.}
  {[}Interview 11{]}
\item
  \enquote{\emph{Das Zweitveröffentlichungsrecht sollte sogar als
  Pflicht angesehen werden}.} {[}Interview 11{]}
\item
  \enquote{\emph{Ob meine eigenen Publikationen Open Access erscheinen, hängt von
  der Zeitschrift ab}.} {[}Interview 04{]}
\item
  \enquote{\emph{Wo ich publiziere hängt vom Thema ab und nicht von
  Open-Access-Strategien oder Publikationsformen}.} {[}Interview 06{]}
\item
  \enquote{\emph{Open Access als solches sollte überhaupt nicht zur
  Debatte stehen. Also es muss Open Access publiziert werden. Aber so
  ist ja nun mal nicht die Realität. Es gibt Verlagsverträge, die
  unterschiedlich ausgestaltet sind, es gibt Workflows in Projekten, die
  man nicht sofort publizieren kann. Das kann man erstmal gar nicht
  ändern}.} {[}Interview 13{]}
\item
  \enquote{\emph{Ich bin kein Verfechter von Open Access um jeden Preis,
  da bestimmte Textsorten dann verschwinden würden (z.\,B. Lehrbücher)}.}
  {[}Interview 04{]}
\item
  \enquote{\emph{Multimediale Inhalte machen bestimmte Publikationen
  sehr teuer, darum ist Open Access oft nicht geeignet}.} {[}Interview
  04{]}
\item
  \enquote{\emph{Wir publizieren beispielsweise Sammelbände auf der
  Plattform unserer Universitätsbibliothek}.} {[}Interview 30{]}
\item
  \enquote{\emph{Die Erfahrungen der} Amsterdam University Press
  \emph{zeigen, dass parallele Open-Access-Angebote von Publikationen
  sich nicht negativ auf die Verkaufszahlen auswirken}.} {[}Interview
  36{]}
\item
  \enquote{\emph{Um Potenziale des Digitalen nutzen zu können muss so
  viel wie möglich frei zugänglich sein. Demzufolge sollte die
  Privatisierung} {[}von Wissen{]} \emph{durch Verlage ausgeschaltet
  werden}.} {[}Interview 20{]}
\item
  \enquote{\emph{Auf Dauer stellt sich auch die Frage, wem das
  Erzeugnis, zum Beispiel eine Edition bzw. Codierung gehört. Eigentlich
  nicht dem Verlag. Man könnte dann dem Verlag Rechte zur Nutzung
  einräumen, aber eben nicht als ausschließliche Nutzung, sondern man
  räumt ihm bestimmte Rechte ein}.} {[}Interview 33{]}
\item
  \enquote{\emph{Wenn ich einen Text nehme, der unter einer
  Open-Access-Lizenz veröffentlicht wurde, fälschlicherweise, weil er
  nämlich die kommerzielle Nutzung ausschließt, und ich fange an diesen
  Text über eine Annotationssoftware und ein SPSS}
  {[}Statistikprogramm{]} \emph{reinzuladen, und damit quasi in einem
  kommerziellen Umfeld mit diesen Texten zu arbeiten, dann verstoße ich
  gegen die Non-Commercial-Regel zum Beispiel einer
  Creative-Commons-Lizenz. Ein schönes Beispiel ist auch die Nutzung von
  Lehrmaterialien an teilprivatisierten Schulen oder Ähnliches. Es gibt
  einfach viel zu viele Fälle, wo eben das \enquote{Nicht-Kommerzielle}
  die Weiterverarbeitung und Weiternutzung von Inhalten behindert. Von
  daher finde ich die Debatte um NC völlig nichtig}.} {[}Interview 39{]}
\item
  \enquote{\emph{Volltextsuche ist wichtig}.} {[}Interview 04{]}
\item
  \enquote{\emph{Open-Access-Geschäftsmodelle kosten immer Geld, deshalb
  sollten akademische Einrichtungen dafür auch wenigstens eine halbe
  Stelle einrichten}.} {[}Interview 11{]}
\item
   \enquote{\emph{Wenn die Forschungsgeldgeber nicht so viel finanzieren, dass
  sie etwa die Nachhaltigkeitsprobleme} {[}für Digitale Editionen{]}
  \emph{lösen können, sondern dass wir da dann Dienstleister brauchen, die
  das machen, dann muss man auch über kostenpflichtigen Zugang
  nachdenken. Das ist wirklich eine Frage, wie wir das verteilen. Es ist
  ja klar, dass die ganze Sache Geld kostet. Es ist eine
  Milchmädchen-Rechnung, zu glauben, digitales Edieren sei irgendwie
  günstiger als gedrucktes Edieren. Das ist nicht so}.} {[}Interview
  27{]}
\item
  \enquote{\emph{Bei unserem Open-Access-Journal ist vieles auf Basis
  von Ehrenamt (z.\,B. Korrekturlesen und Layout). In Arbeitszeit
  umgerechnet wären das 1000 oder 2000 bis 3000 Euro pro Artikel}.}
  {[}Interview 19{]}
\item
  \enquote{\emph{Die Finanzierung stellt natürlich über kurz oder lang
  eine Herausforderung für die Preisgestaltung der Open-Access-Verlage
  dar. Aber was den Autor angeht, sehe ich den da nicht in der Pflicht.
  Ich sehe da den Forschungsförderer und letztendlich auch den
  Forschungsservice, aber auch die Bibliothek in der Aufgabe, dem
  Wissenschaftler zu helfen, die besten Möglichkeiten für die
  Publikation seines Textes zu evaluieren}.} {[}Interview 39{]}
\item
  \enquote{\emph{Auf der Kostenseite ändert sich durch Open Access nicht
  viel, aber auf der Nutzungsseite, denn die ist dann für jeden
  kostenlos}.} {[}Interview 36{]}
\item
  \enquote{\emph{Kommerzialisierung und vor allem Double-Dipping}
  {[}Mehrfachfinanzierung{]} \emph{wird als große Gefahr für Open Access
  gesehen}.} {[}Interview 36{]}
\item
  \enquote{\emph{Was passiert aber nach Auslaufen des Projektes? Bei uns
  kümmert sich die Universität noch ein paar Jahre darum. Aber was kommt
  danach? Gerade bei großen Editionsunternehmen müssen Vereinbarungen
  getroffen werden: Entweder geht es tatsächlich an die großen
  Bibliotheken. Oder die Akademien richten noch etwas ein. Oder es gibt
  sonstige Infrastrukturen, in denen das integriert werden kann. Dafür
  ist derzeit noch überhaupt keine Lösung in Sicht}.} {[}Interview 33{]}
\item
  \enquote{\emph{Ein Kollege von einem Verlag sagte, dass sie nur noch
  die PDFs aufheben. Wenn Korrekturen kommen, setzen sie diese grafisch
  in die PDFs. Man kann sich doch nur an den Kopf fassen. Im Prinzip
  sind so viele Dinge hinterher verschwunden und müssen neu erstellt
  werden. Das ist eine Katastrophe}.} {[}Interview 33{]}
\item
  \enquote{\emph{Für uns ist da neben der eigentlichen Publikation noch
  die Frage: Wie gewährleisten wir denn überhaupt die Aufrechterhaltung
  der Dienste, die hier entstanden sind? Das ist überhaupt noch nicht
  gelöst, weder forschungspolitisch noch technisch. Es gibt einen
  Horizont von drei Jahren, und weiter blickt man nicht}.} {[}Interview
  13{]}
\item
  \enquote{\emph{Wenn wir dem Internet Langfristigkeit und Dauer
  zugestehen wollen, und nicht nur ein kurzes Aufblitzen und dann
  Vergessen, dann müsste man dafür sorgen, dass auch in der
  Internet-Vermittlung und der Nach-Bewerbung darauf hingewiesen wird,
  dass bestimmte Sachen, die mal da waren, immer noch da sind und dass
  man sie aktualisiert}.} {[}Interview 12{]}
\item
  \enquote{\emph{Meine eigene Digitale Edition (Open Access) wurde
  bereits zitiert, obwohl es auch eine Printversion gibt. Es
  funktioniert wirklich!}} {[}Interview 07{]}
\item
  \enquote{\emph{Die führende Fachzeitschrift in den Digital Humanities,
  LLC -- Literacy and Linguistic Computing,}\footnote{Die Zeitschrift
    LCC wurde im Dezember 2014 umbenannt in DSH -- \emph{Digital
    Scholarship in the Humanities},
    \url{http://llc.oxfordjournals.org/}. Für Open-Access-Journals im
    DH-Bereich siehe Brill Open Humanties:
    \url{http://booksandjournals.brillonline.com/content/journals/23527064};
    Journal of Digital Humanities:
    \url{http://journalofdigitalhumanities.org/}; Zeitschrift für
    digitale Geisteswissenschaften: \url{http://zfdg.de/}.} \emph{ist
  nicht Open Access. Man kann die Metadaten über Suchmaschinen finden,
  aber nicht mit den Inhalten arbeiten}.} {[}Interview 19{]}
\item
  \enquote{\emph{Man kann durch die Neuanordnung von schon
  veröffentlichten Artikeln eine neue Story erzählen für eine neue
  Zielgruppe. Ich kann aber natürlich auch durch getypte Links mit
  Referenzen aus einem Werk sagen, warum ich das Werk zitiere. Ich kann
  also sagen, ob ein Zitat ein Beleg ist, ein nicht zureichender
  Gegengrund usw. Und ich kann auch aus Teilargumentationen in Form von
  einem Mash-up eine Neuargumentation erstellen und die einzelnen Teile
  sind dann keine ganzen Artikel sondern nur Artikelteile. Allerdings
  sehe ich in meiner Community keine Leute, die Mash-Ups machen}.}
  {[}Interview 22{]}
\item
  \enquote{\emph{Entstehungsprozesse abzubilden ist nur sinnvoll für
  Editionen, aber nicht für Monografien oder Aufsätze}.} {[}Interview
  07{]}
\item
  \enquote{\emph{In den Geisteswissenschaften gibt es kaum Preprints wie
  in den Naturwissenschaften, dafür ist die Monografie noch immer sehr
  bedeutend, was wohl auch so bleiben wird, dazu kommen Sammelbände und
  Zeitschriftenartikel}.} {[}Interview 09{]}
\item
  \enquote{\emph{In der Geschichtswissenschaft wird der Wandel zu Open
  Access lange dauern, denn bei einer Bewerbung müssen heute die
  Referenzen zwingend gedruckt sein}.} {[}Interview 15{]}
\item
  \enquote{\emph{Ich habe noch nie bei einer reinen Online-Zeitschrift
  veröffentlicht, weil es einfach nicht zählt und keiner liest. Wenn ich
  etwas Gutes habe, reiche ich es bei einer guten Zeitschrift ein}.}
  {[}Interview 05{]}
\item
  \enquote{\emph{Es gibt immer mehr Dissertationen, die auf
  Universitätsservern als PDFs abgelegt werden, die haben aber nach wie
  vor leider den Ruf \enquote{obwohl nur online erschienen}. Also der
  höhere Grad der Reichweite wird nicht unbedingt als positiv
  eingeschätzt}.} {[}Interview 21{]}
\item
  \enquote{\emph{In der Geschichtswissenschaft ist das große
  Rezensionsorgan \enquote{Sehepunkte} digital und anerkannt. Es ist
  Open Access und hat eine gute Qualität}.} {[}Interview 19{]}
\item
  \enquote{\emph{Online-Publikationen werden bei den Zeitschriften
  zunehmen. Aber es sollte nicht darum gehen, etwas im Internet zu
  machen, sondern darum eine gute Zeitschrift zu machen. Es muss ein
  hohes Qualitätslevel erreicht werden}.} {[}Interview 12{]}
\item
  \enquote{\emph{Bei einem Buchprojekt habe ich mit dem Verlag
  besprochen, dass ich die einzelnen Kapitel online veröffentlichen
  konnte, um Feedback zu erhalten}.} {[}Interview 16{]}
\item
  \enquote{\emph{In Ländern, in denen die Buch- und Druckkultur nicht so
  ausgeprägt ist, ist man gegenüber Online-Publikationen offener (z.\,B.
  Russland, Frankreich, Libanon, Türkei). Dort ist Open Access viel
  ausgeprägter.}} {[}Interview 32{]}
\item
  \enquote{\emph{Das Internet ist mit dem Makel des Amateurhaften
  behaftet. Definitiv. Aus dem eigenen Studienbetrieb erlebt man, das
  die Studenten die Internetquellen zusammenklauben, alles andere nicht
  mehr wahrnehmen und dabei auf unglaublich viel Halbgares stoßen und
  das dann auch verwenden}.} {[}Interview 12{]}
\item
  \enquote{\emph{Unsere mühevollen Verhandlungen mit Verlagen sind
  gescheitert, denn Verlage versuchen ihre Geschäftsmodelle auf Basis
  von Exklusivrechten aufzubauen anstatt auf Publikationen. Kein Autor
  hat ein Interesse daran, seine Rechte abzugeben. Das ist
  unkontrovers}.} {[}Interview 09{]}
\item
  \enquote{\emph{Das alte Druckkostenzuschusssystem ist ja auch wettbewerbsmäßig
  organisiert gewesen, indem es verschiedene Institutionen und
  Fördereinrichtungen gibt, bei denen man Druckkostenzuschüsse
  beantragen kann. So etwas müsste im größeren Stil umorganisiert werden
  hin zu den Anbietern von wissenschaftlichen Publikationen, die nicht
  Verlage sind. Die also nicht wie Verlage gewinnorientiert arbeiten}.}
  {[}Interview 21{]}
\item
  \enquote{\emph{Wir arbeiten mit CC-BY-SA, aber viele meiner Kollegen
  wissen nicht einmal was CC} {[}Creative Commons{]} \emph{eigentlich
  ist}.} {[}Interview 15{]}
\item
  \enquote{\emph{Wissen muss so frei wie möglich zur Verfügung gestellt
  werden. Es müssen Infrastrukturen, etwa Veröffentlichungsplattformen,
  geschaffen werden, die kollaboratives Arbeiten, Transparenz und offene
  Forschungsdaten (Korpora, Datenbanken) bieten}.} {[}Interview 20{]}
\item
  \enquote{\emph{Open Access als Wunschszenario ist natürlich mit einer
  marktwirtschaftlich arbeitenden Verlagswirtschaft nur beschränkt
  vereinbar. Das führt zu einem gesellschaftlichen Sozialismus, dass
  also die Tätigkeiten der Verlage keinen Wert mehr haben. Deshalb kann
  es also nur in einem lokalen Sozialismus, alles innerhalb der
  Wissenschaft, funktionieren, was heißen würde, dass der Staat über die
  Institutionen alles bezahlt. Es könnten dann viele Incentives
  wegfallen, zum Beispiel besonders schlechte Publikationen gar nicht
  erst auf den Markt zu bringen}.} {[}Interview 22{]}
\item
  \enquote{\emph{Open Science ist ein Fernziel, aber wir sollten uns
  bereits heute technisch darauf einstellen}.} {[}Interview 06{]}
\end{enumerate}

\subsection{ExpertInnenaussagen über Open Research
Data}\label{expertinnenaussagen-uxfcber-open-research-data}

\begin{enumerate}
\def\labelenumi{\arabic{enumi}.}
\item
  \enquote{\emph{Bei der Frage nach einer Open Science bzw. Open
  Scholarship wäre zu klären, wie man \enquote{Open} definiert. Wir
  werden ja durch öffentliche Gelder, also Open Finance finanziert.
  Daher finde ich es richtig, wenn die DFG darauf drängt, dass Tools und
  Daten, die in DFG-Projekten erzeugt wurden, auch öffentlich zugänglich
  sein sollen. Öffentlich heißt in diesem Zusammenhang, dass man sie auf
  Anfrage weitergibt}.} {[}Interview 29{]}
\item
  \enquote{\emph{Es gehört für mich einfach zu einer guten
  wissenschaftlichen Praxis, wenn ich Aussagen treffe, es dem
  zukünftigen Rezipienten zu ermöglichen, die zu verifizieren oder zu
  falsifizieren. Wenn ich da die Daten nicht mit dazu liefere, dann
  nehme ich ihm die Möglichkeit, das zu tun}.} {[}Interview 13{]}
\item
  \enquote{\emph{Es dürfen ja eigentlich nicht nur die Forschungsdaten
  bereitgehalten, es muss ja auch die Infrastruktur gesichert sein}}.
  {[}Interview 33{]}
\item
  \enquote{\emph{Wir setzen 3D-Visualisierungen und Geodaten ein, um
  Nachvollziehbarkeit zu gewährleisten, darum müssen diese Dinge digital
  publiziert werden. Es ist also nicht nur eine weitere multi-mediale
  Beschreibungsweise, sondern der Kern der Datenpublikation}}.
  {[}Interview 09{]}
\item
  \enquote{\emph{Es wäre ein großer Mehrwert von der Publikation direkt
  in den Korpus bzw. die Datenbank zu kommen, um die Ergebnisse besser
  nachvollziehen zu können -- auch für die Lehre}.} {[}Interview 11{]}
\item
  \enquote{\emph{Arbeiten in der Literaturwissenschaft sind oftmals
  nicht nachvollziehbar, da die Quellen nur exklusiv für den Autor
  zugänglich sind}.} {[}Interview 20{]}
\item
  \enquote{\emph{Es kann nicht sein, dass Wissenschaft aus öffentlichen
  Geldern finanziert wird, und dass diese Daten, die dabei erschlossen
  werden (da spreche ich noch nicht von Publikationen) dann nicht sofort
  der Allgemeinheit zur Verfügung stehen}.} {[}Interview 13{]}
\item
  \enquote{\emph{Wenn man die Forschungsdaten kombiniert mit
  entsprechender Software, ergibt sich auch die Möglichkeit, dass sich
  die Leute aufgrund der Forschungsdaten auch andere Visualisierungen
  wünschen können oder auch mit den Forschungsdaten auf ihre Weise
  spielen können. Man hat in einem Prozess methodisch sauber für einen
  bestimmten Zweck Daten erhoben und ausgewertet und jemand anderes kann
  mit den Daten weiterarbeiten und mit ganz anderen Fragestellungen}.}
  {[}Interview 22{]}
\item
  \enquote{\emph{Digitale Publikationen von mittelalterlichen
  Handschriften verbessern den Zugang zu den Quellen immens, was die
  Bereitstellung von Materialien betrifft und die Bereitstellung von
  Basisinformationen, von Metadaten zu diesen Materialien}.}
  {[}Interview 12{]}
\item
  \enquote{\emph{Es gibt große Datenfriedhöfe. Viele Forschungsdaten aus
  Projekten sind irgendwann nicht mehr lesbar. Das passiert auch mit
  Daten in größeren Forschungsverbünden. Wir brauchen ein
  geisteswissenschaftliches Forschungsdatenrepositorium. Ein Angebot wie
  GESIS in den Sozialwissenschaften gibt es noch nicht}.} {[}Interview
  21{]}
\item
  \enquote{\emph{Theoretisch wäre eine Forschungsplattform am besten, wo
  alle ihre Projekte einspeisen, die aber immer nur den einzelnen
  Bearbeitern zugänglich sind. Im Fall des Ausscheidens werden diese
  Dinge an den Nächsten weitergegeben. Vielleicht eine
  Gesamt-Nachlassverwaltung von allen toten oder revitalisierbaren
  Projekten. Einer müsste da die Übersicht bekommen}.} {[}Interview
  12{]}
\item
  \enquote{\emph{Daten können auf unterschiedlichen Repositorien liegen
  (z.\,B. Github, TextGrid) und dafür gibt es bereits gute Lösungen, das
  sollte man nicht duplizieren}.} {[}Interview 16{]}
\item
  \enquote{\emph{Eine einzige Umgebung für alle Arbeitsprozesse halte
  ich nicht für sinnvoll. Ich könnte mir aber interdisziplinäre
  \enquote{Erfassungsumgebungen} vorstellen, die sowohl die
  Forschungsdaten als auch Aussagen darüber speichern. Eine Art
  Datenbank, die aber als Linked Open Data im Sinne des Semantic Web
  funktioniert. CIDOC-CRM} {[}Referenzmodell für die Dokumentation von
  Gütern des kulturellen Erbes{]} \emph{ist allerdings so komplex, dass
  es niemand anwendet. Gewisse Grundprinzipien sind aber sinnvoll wie
  die Verbindung einer Storage-Umgebung, die mit Front-Ends dynamisch
  Inhalte für Publikationen generieren können. Mikroaussagen müssen
  speicherbar sein und persistent identifizierbar}.} {[}Interview 14{]}
\item
  \enquote{\emph{Die Frage nach einem zentralen und nachhaltigen
  Forschungsdatenrepositorium für die Geisteswissenschaften ist auch
  Wissenschaftspolitik. Nach Lage der Dinge kann das nicht in den
  Hochschulen laufen, solange der Bund da nicht längerfristig fördern
  darf. Das müssten die Länder bereitstellen, was sie momentan nicht
  tun}.} {[}Interview 21{]}
\item
  \enquote{\emph{Mit der Förderpolitik hinsichtlich der Publikationen
  bin ich mit der DFG zufrieden, es müssen heute in jedem Antrag Angaben
  zur nachhaltigen Nutzung von Forschungsdaten stehen, das finde ich
  nachvollziehbar und richtig}.} {[}Interview 04{]}
\item
  \enquote{\emph{Exzerpte von vornherein so anzulegen, dass sie auch
  jemand anderes versteht, ist schwierig. Erstens, dass es für andere
  verständlich ist und zweitens nicht total desavouiert, ist eine heikle
  Geschichte. Insofern gibt es da Grenzen von Open Access}.}
  {[}Interview 21{]}
\item
  \enquote{\emph{Einige Aspekte der Präpublikationsphase sind wichtig,
  diese sollten aber nur strukturiert mitpubliziert werden, denn
  Rohdaten bleiben Rohdaten. Aber sonst spielen sie kaum eine Rolle}.}
  {[}Interview 04{]}
\item
  \enquote{\emph{Präpublikationsdaten zu veröffentliche hieße sich in
  die Karten schauen zu lassen. Das würde eine ganz andere Kultur
  erfordern. Für die Germanistik wäre das auch schwer vorstellbar, wenn
  jetzt Exzerpte veröffentlicht würden, dann würden die Autoren wieder
  daran feilen und nicht zu ihren eigentlichen Aufgaben kommen}.}
  {[}Interview 06{]}
\item
  \enquote{\emph{Teilergebnisse zu publizieren ist fraglich, oft gibt es
  bei Projekten einen gewissen Druck. Negative Resultate sind eine echte
  Marktlücke, das ist in der Naturwissenschaft, wo stärker empirisch
  gearbeitet wird, weiter verbreitet. Präsentationsfolien können
  sinnvoll als Visualisierung und Zusammenfassung sein, aber hier fehlt
  oft die Kontroll- und Filterfunktion.}} {[}Interview 11{]}
\item
  \enquote{\emph{Teile der Wissenschaftskommunikation gehen niemanden
  etwas an (Persönlichkeitsrechte)}.} {[}Interview 16{]}
\item
  \enquote{\emph{Aspekte der Präpublikationsphase können relevant sein, aber erhöhen
  dramatisch den \enquote{Datenmüll}, deshalb sollte man das nur sehr
  selektiv anwenden}.} {[}Interview 05{]}
\item
  \enquote{\emph{Wenn man von Anfang an weiß, dass man das mit
  verschiedenen Leuten teilt, dann sieht die Situation anders aus als
  wenn man sich eine private Datenbank zusammenstrickt, wo oft kein
  Unterschied zwischen Daten aus der Literatur und selbstgesammelten
  Daten gemacht wird. Wenn man das ohne dies zu kommentieren nach außen
  gibt, denken die Nutzer, dass der da selbst hingefahren ist und das
  ausgemessen hat}.} {[}Interview 12{]}
\item
  \enquote{\emph{Wir haben nur ganz wenig offen zugängliche Datenbanken.
  Weil wir keine Zeit und kein Geld haben, das zu machen. Wir haben ganz
  wenige Leute. Das ist eine reine Zeit- und Geldfrage, keine politische
  Frage}.} {[}Interview 27{]}
\item
  \enquote{\emph{Die Datenqualität ist oftmals nicht gut und erfordert
  einen großen Aufbereitungsaufwand. Zum Beispiel reicht es nicht die
  reinen Wörter zu haben, wichtig ist die Grammatik, daher müssen
  Wortformen ausgezeichnet werden. Dabei kommt es weniger auf die Masse
  der Korpora an}.} {[}Interview 05{]}
\item
  \enquote{\emph{Open Software ist gut, Open Data ist gut, aber es gibt
  da oft rechtliche Probleme}.} {[}Interview 11{]}
\item
  \enquote{\emph{Bei den Nutzungsrechten sperren sich Archive und
  Bibliotheken zum Teil immer noch. Selbst wenn sie es nicht täten, muss
  es erstmal digitalisiert werden. Deutsche und englische Archive zum
  Beispiel erlauben keine eigenen Digitalisate}.} {[}Interview 21{]}
\item
  \enquote{\emph{Multimediale Inhalte nutze ich weniger, eher Bilder. Es
  gibt das Problem der Rechte z.\,B. für Filmausschnitte. Bei Bildern und
  Filmen haben wir kein eigenes} {[}uneingeschränktes{]}
  \emph{Zitationsrecht, man hat die technischen Möglichkeiten, aber
  nicht die rechtlichen}.} {[}Interview 06{]}
\item
  \enquote{\emph{Die Rechtslage ist schwierig, da es oft nur einfache
  Nutzungsrechte an Forschungsdaten gibt, dass läßt sich dann gar nicht
  Open Access publizieren}.} {[}Interview 04{]}
\item
  \enquote{\emph{Die rechtliche Situation für die Veröffentlichung von
  Forschungsdaten muss geklärt werden; z.T. werden Datenbanken nach
  amerikanischem Recht gebraucht (Fair use), während es in Deutschland
  keine Bildungs- und Wissenschaftsschranke gibt}.} {[}Interview 15{]}
\item
  \enquote{\emph{Das große Hindernis sind da die rechtlichen Vorgaben
  der Archive, die ein einfaches Austauschen von kopierten oder
  digitalisierten Quellen in der Regel untersagen. Ich habe zwar
  zigweise DVDs mit digitalisierten Quellen aus dem Geheimen
  Staatsarchiv in Berlin. Aber ich darf sie keinem meiner Kollegen
  zeigen}.} {[}Interview 21{]}
\item
  \enquote{\emph{Wir speichern unsere Forschungsdaten in einer eigenen
  Tabelle (daher mit eigener Schöpfungshöhe), damit wir das problemlos
  Open Access veröffentlichen können}.} {[}Interview 15{]}
\item
  \enquote{\emph{Mikroformen und Datenbanken sind eigentlich die
  angemessenere Form des Austausches, da ellenlange Narrative auch kaum
  als Gesamtheit wahrgenommen werden}.} {[}Interview 14{]}
\item
  \enquote{\emph{Digitale Publikationen senken die Schwelle, damit
  werden z.\,B. Aufsätze und Working Papers publikabel}.} {[}Interview
  15{]}
\item
  \enquote{\emph{Präpublikation ist für Primärdaten geeignet}.}
  {[}Interview 30{]}
\item
  \enquote{\emph{Wir können uns im Digitalen nicht auf das Narrative als
  eigentliche Publikation beschränken, die Brücke vom Narrativ zu den
  Forschungsdaten muss geschlagen werden. Das dürfen vorgegebene Formate
  nicht verhindern. Forschungsdaten sollten gegebenenfalls auch
  veröffentlicht werden ohne das Narrativ}.} {[}Interview 14{]}
\item
  \enquote{\emph{Selbst wenn ich einen Code publiziere, den ich für eine
  Applikation in der Geschichtswissenschaft geschrieben habe, ist das
  für mich eine geisteswissenschaftliche Publikation. Weil ich als
  Geisteswissenschaftler, auch wenn ich hier programmiert habe, hier
  geisteswissenschaftlich reflektiert habe: Mit welchem Programm und
  welchen Prozessen kann ich abbilden, was ich hier gerade
  geisteswissenschaftlich rausfinden will. Der} Programming Historian
  \emph{ist genauso gleichberechtigt wie ein digital wissenschaftlich
  Publizierender}.} {[}Interview 13{]}
\item
  \enquote{\emph{In der Linguistik wäre es sinnvoll Datenbanken (z.\,B.
  mit Sprachbeispielen) direkt in Publikationen zu integrieren.
  Interessant ist der Ansatz mit Linked Open Data, aber das steckt noch
  sehr in den Kinderschuhen}.} {[}Interview 11{]}
\item
  \enquote{\emph{Digitale Editionen binde ich direkt bei dem
  entsprechenden Zitat in digitale Publikationen ein. Das ist ein
  Riesen-Luxus}.} {[}Interview 06{]}
\item
  \enquote{\emph{Es sollten auch \enquote{Work-in-progress-Datenbanken}
  veröffentlicht werden. Man kann sehr viel früher mit der globalen
  Wissenschafts-Community in Dialog treten. Das steht natürlich in
  diametralem Gegensatz zu der eigentlich über Jahrzehnte antrainierten
  Verhaltensweise von Geisteswissenschaftlern}.} {[}Interview 13{]}
\item
  \enquote{\emph{Forschungsdaten können auch für andere Communities
  interessant sein, deswegen sollten sie so schnell wie möglich (ggf.
  auch ohne vollständige Dokumentation) frei zugänglich sein, am besten
  als Open Data mit einer Creative-Commons-Lizenz}.} {[}Interview 13{]}
\item
  \enquote{\emph{Wir haben uns entschieden, Forschungsdaten sofort zu
  publizieren, gegebenenfalls mit dem Hinweis, dass die Annotationen
  noch nicht geprüft sind. Andere geben auf Anfrage ihre Daten heraus,
  aber das schreckt manche (z.\,B. Studierende) schon ab}.} {[}Interview
  11{]}
\item
  \enquote{\emph{Man kann vielleicht sagen, dass heute zu einer
  wissenschaftlichen Publikation auch die Zugänglichkeit der
  Forschungsdaten gehört}.} {[}Interview 29{]}
\item
  \enquote{\emph{Es muss so sein, dass Daten -- egal in welchem Zustand
  -- mitpubliziert werden. Denn es wird ja oft so gemacht, dass Daten
  erst dann von Institutionen mitveröffentlicht werden, wenn sie
  sinnvoll beschrieben sind. Wenn ich mich auf den Standpunkt stelle und
  sage: Die Daten sind nicht sauber, dann habe ich faktisch schon eine
  inhaltliche Bewertung abgegeben. Das schließt ja schon wieder komplett
  aus, dass das, was da steht, auch in anderen Kontexten verwendet
  werden könnte}.} {[}Interview 13{]}
\item
  \enquote{\emph{Die Historie der Forschungsdaten sollte sichtbar
  gemacht werden}.} {[}Interview 04{]}
\item
  \enquote{\emph{Der Zugang zu digitalisierten raren Büchern verändert
  auch die Art der Autopsie}.} {[}Interview 06{]}
\item
  \enquote{\emph{Durch die freie Verfügbarkeit der kulturellen
  Überlieferung findet auch eine Abwertung statt (kostenlos = nicht
  wertvoll).}} {[}Interview 16{]}
\item
  \enquote{\emph{Postpublikationsdaten sind wichtig. Zum Beispiel bei
  Journals kann es die Reviews bzw. Rezensionen geben und auch
  Zugriffsstatistiken. Dadurch kann auch eine Diskrepanz sichtbar werden
  zwischen Gutachtern und Fach-Community}.} {[}Interview 06{]}
\item
  \enquote{\emph{Auch die Bearbeitung von Forschungsdaten selbst kann
  eine intellektuelle und wissenschaftliche Leistung sein, die mittels
  Publikation zugänglich werden sollte}.} {[}Interview 29{]}
\item
  \enquote{\emph{Es wäre wünschenswert, die Rezeption meiner eigenen
  Publikationen besser nachvollziehen zu können}.} {[}Interview 04{]}
\item
  \enquote{\emph{Jedes Projekt sollte selbst entscheiden, inwieweit die
  Kommunikation (z.\,B. Wikis, Emails) mitveröffentlicht wird (oft ist die
  Zeit im Projekt zu knapp zur Selektion).}} {[}Interview 16{]}
\item
  \enquote{\emph{Es gibt leider keine Kultur, negative Resultate zu
  veröffentlichen. Gerade im DH-Bereich scheitern 9 von 10 Experimenten.
  Das ist fatal, denn es entsteht gegebenenfalls Doppel- oder
  Mehrfachforschung}.} {[}Interview 19{]}
\item
  \enquote{\emph{Die Daten in unserem Wörterbuchprojekt stellen wir auf
  Anfrage zur Verfügung. Sie sind nur teilweise offen, weil es
  Verlagsvereinbarungen gibt und die Daten nur für Forschungszwecke zur
  Verfügung stehen dürfen. Unsere Daten stehen aber z.\,B. allen
  DFG-Projekten zur Nachnutzung zur Verfügung}.} {[}Interview 29{]}
\item
  \enquote{\emph{Forschungsdaten wie Quellenmaterial sollte nicht nur
  von der hostenden Institution annotierbar sein, sondern auch
  \enquote{von außen}, so dass viele Forscher die Daten anreichern
  können. Dafür sind Standards erforderlich.}} {[}Interview 14{]}
\item
  \enquote{\emph{Wir diskutieren manchmal, inwieweit unser Wiki
  zugänglich sein sollte. Wir müssten dann dies und jenes
  herausschmeißen, denn ansonsten müsste man sich ja von vornherein
  öffentlich verhalten, und das tut man eben nicht in so einem
  Arbeits-Wiki. Und gleichzeitig sind da natürlich Diskussionsprozesse,
  ausführliche Dokumentationen über die Transkriptionsregeln usw.
  geführt. Es wäre gut, wenn auch die geführte Debatte der Wissenschaft
  zugänglich gemacht werden könnte, aber die Redaktion des Wikis wäre
  wieder viel Arbeit}.} {[}Interview 27{]}
\item
  \enquote{\emph{Mit der Zustimmung der Autoren können auch
  Präpublikationsdaten veröffentlicht werden und
  Infrastruktureinrichtungen sollten dann dafür zuständig sein (beim
  CERN gibt es einen Medien-Server zur schnellen Kommunikation, dann
  müssen die Autoren aber auch wissen in welchem Kontext sie das
  \enquote{publizieren}}.} {[}Interview 09{]}
\item
  \enquote{\emph{Mein Lieblingsindex bei meiner eigenen Datenbank ist
  der Volltext-Index. Gegenüber diesem Volltext-Index ist die
  Durchsuchbarkeit von PDFs zweite Wahl. Es ist gut, dass es sie gibt,
  aber sie gibt mir ja kein Angebot, was ich finden werde. Aber bei
  diesem Index sehe ich: Da steht XYZ beispielsweise siebzig mal. Und
  dann weiß ich: Das ist so eine große Treffermenge. Dann kann ich auch
  kombiniert suchen}.} {[}Interview 12{]}
\item
  \enquote{\emph{Publikationsobjekte müssen in ihrem Kontext
  veröffentlicht werden (z.\,B. Interviewaufnahmen)}.} {[}Interview 09{]}
\item
  \enquote{\emph{Also wenn ich mit einer historisch-kritischen Ausgabe
  ins Archiv gehe und mir die Handschriften anschaue, dann hilft sie mir
  sehr viel. Aber wenn ich nur diese Ausgabe habe und nicht die
  Materialien dazu, dann komme ich nicht sehr weit. Das heißt, es fehlt
  immer das Material, an dem eigentlich diskutiert wird. Das steht dem
  Forscher eigentlich nicht zur Verfügung. Die Idee ist, die weltweit
  verteilten Archive der Forschung zugänglich zu machen. Die erste
  Aufgabe unserer Digitalen Edition} {[}Open Access{]} \emph{ist, dieses
  Material aus der ganzen Welt in digitalen Bilddateien
  zusammenzutragen. Dann werden diese Bilddateien transkribiert, in
  unterschiedlichen Transkriptionsformaten, die dann auch TEI-kodiert}
  {[}Text Encoding Initiative{]} \emph{sind, um die Sache nachhaltig zu
  machen. So kann man sie dann auch entsprechend in die
  unterschiedlichsten Umgebungen migrieren, die wir uns in digitaler
  Gegenwart und Zukunft vorstellen können}.} {[}Interview 27{]}
\item
  \enquote{\emph{In einer dieser Ausführungen zum Thema'
  Weiterentwicklung des Semantic Web -- erweiterte Metadaten' steht eben
  drin, dass es auch Triple} {[}Linked (Open) Data nach RDF{]}
  \emph{geben könnte, die die Methode beschreiben. Es könnten natürlich
  auch Triple oder Gruppen von verschachtelten Triplen geben, die die
  Grundstruktur der Argumentation wiederholen}.} {[}Interview 22{]}
\item
  \enquote{\emph{Digitale Editionen sollten offenen Quellcode haben}.}
  {[}Interview 07{]}
\item
  \enquote{\emph{Hinsichtlich der Frage, ob man bei Creative Commons für
  wissenschaftliche Daten und Ergebnisse auch eine kommerzielle Nutzung
  zulassen soll, gibt es geteilte Meinungen. Manche glauben kaum, dass
  die Wirtschaft mit ihren Daten etwas anfangen könnte. Andere wollen
  nicht, dass ihre Materialien wirtschaftlich durch Dritte verwertet
  werden, während sie selbst mit einem kleinen Etat wirtschaften
  müssen}.} {[}Interview 29{]}
\item
  \enquote{\emph{Das ist ein sehr valider Ansatz zu sagen: Soziale
  Medien (Twitter etc.) in Kombination mit Bloggen und dem Anspruch,
  eben mal nicht monographisch zu publizieren, sondern tatsächlich
  einfach zu schildern: Ich beschäftige mich gerade mit dem und dem und
  tue das gerade mal kund}.} {[}Interview 13{]}
\item
  \enquote{\emph{Mein eigener Handschriftenzensus wird nicht so oft
  zitiert. Ich habe öfter die mündliche Rückmeldung: Das ist toll und
  praktisch, ich gehe damit den Weg. Aber dass das irgendwie in den
  Fußnoten erscheinen würde, ist reichlich selten. Weil es eine
  Online-Quelle ist und weil die ältere Generation nicht so genau weiß,
  wie sie das zitieren soll}.} {[}Interview 12{]}
\item
  \enquote{\emph{Wir haben bei unserem Portal auch ein internes
  Diskussionsforum, wo etwa 300 Leute angemeldet sind. Aber es wird kaum
  benutzt}.} {[}Interview 21{]}
\item
  \enquote{\emph{Es ist nicht klar, wie viel von den Tools und Daten,
  die Open Source gestellt wurden, überhaupt nachgenutzt wird. Trotz der
  Tatsache, dass es vieles frei zugänglich gibt, neigt man in der
  Wissenschaft dazu, das Rad neu zu erfinden und lieber ein eigenes
  Werkzeug zu entwickeln}.} {[}Interview 29{]}
\item
  \enquote{\emph{Unsere Digitale Edition basiert auf Open Source. Die
  bei uns entwickelte Software wird mit der Creative-Commons-Lizenz
  veröffentlicht. Das war unsere erklärte Absicht. Das ist tatsächlich
  auch im Interesse unserer IT-Mitarbeiter. Die möchten gerne, dass die
  von ihnen entwickelte Software nachgenutzt und weiterentwickelt wird.
  Und gelegentlich kann man dann auch Weiterentwickeltes irgendwo
  abholen und weiternutzen}.} {[}Interview 27{]}
\item
  \enquote{\emph{Es müsste eine Art Not-Fond geben, aus dem
  Universitäten abgeschlossenen Projekten Mittel bereitstellen, um ihre
  Forschungsdaten zu dokumentieren}.} {[}Interview 12{]}
\end{enumerate}

\subsection{ExpertInnenaussagen über Open
Review}\label{expertinnenaussagen-uxfcber-open-review}

\begin{enumerate}
\def\labelenumi{\arabic{enumi}.}
\item
  \enquote{\emph{Wer ist die beste Person für die Auswahl von
  Gutachtern? Es ist der Autor selbst. Allerdings muss ein offener und
  transparenter Prozess gewährleistet sein im Sinne des Open Peer
  Reviews. Damit es funktioniert, sollte immer alles online stattfinden
  und sichtbar sein, sowohl die Dokumente als auch die Namen der
  Gutachter sowie die Gutachten selbst}.} {[}Interview 34{]}
\item
  \enquote{\emph{Offene Peer-Review-Verfahren oder offene
  Wissenskommunikationen in wissenschaftlichen Prozessen könnten es ja
  auch ermöglichen, noch besser herauszustellen, was richtig und falsch
  ist. Und ich finde, man sollte es probieren}.} {[}Interview 39{]}
\item
  \enquote{\emph{Bei digitalen Publikationen stellt sich die Frage nach
  Qualitätssicherung neu, da hier meist keine Verlage dahinter stehen}.}
  {[}Interview 11{]}
\item
  \enquote{\emph{Beim letzten Historikertag fragt man beim Thema
  digitale Publikationen nicht, wohin man aufbrechen könnte. Nein, man
  unterhält sich nur darüber, was denn ein denkbarer Peer-Review-Prozess
  wäre, wenn wir von dem alten Verlagsmodell abkommen}.} {[}Interview
  13{]}
\item
  \enquote{\emph{Für digitale Publikationen gibt es noch keine
  angemessenen Verfahren zur Präsentation und zur Kreditierung. Das wäre
  wichtig vor allem für den Nachwuchs.}} {[}Interview 15{]}
\item
  \enquote{\emph{Man hält Paper lieber bis zur absoluten Fertigstellung
  zurück. Open Peer Review ist daher keine Option}.} {[}Interview 36{]}
\item
  \enquote{\emph{Früher war die Publikationslage das Problem, heute ist
  es die Verwaltung der Masse}.} {[}Interview 12{]}
\item
  \enquote{\emph{Ein großes Problem ist, für Online-Publikationen
  Rezensionen zu bekommen, weil der Tausch der Gegengabe, ich hab dann
  das Buch als Rezensionsexemplar im Regal stehen, wegfällt. Die Frage
  ist, bekomme ich für Online-Publikationen, die nicht von Verlagen
  organisiert sind, Rezensionen?}} {[}Interview 21{]}
\item
  \enquote{\emph{Es ist ein offenes Geheimnis, dass auch renommierte
  Verlage jede Publikation annehmen, weil sie dafür hohe
  Druckkostenzuschüsse erhalten. Bei vielen Verlagen gibt es kein
  Lektorat mehr}.} {[}Interview 36{]}
\item
  \enquote{\emph{Ich habe schlechte Erfahrungen mit Open Peer Review, da
  ich auch schon einmal aus persönlichen Gründen abgelehnt wurde}.}
  {[}Interview 07{]}
\item
   \enquote{\emph{Open Review ist schwierig, weil es nicht anonym ist und
  vielleicht ein großer Name durchgewunken wird. Außerdem sind auch
  nicht alle Reviewer unabhängig, zum Beispiel Doktoranden. Double Blind
  Review ist der Standard}.} {[}Interview 11{]}
\item
  \enquote{\emph{Es gibt es ein großes Schutzbedürfnis gegenüber
  Nachwuchswissenschaftlern beim Thema Open Peer Review}.} {[}Interview
  06{]}
\item
  \enquote{\emph{Open Post Peer Review ist nur etwas für Trolls, da es
  keine ausreichende Motivation gibt}.} {[}Interview 07{]}
\item
  \enquote{\emph{Man könnte sich das Open-Peer-Review-System auch als
  eine disziplinäre Sammelstelle vorstellen. Ein Beitrag würde nicht nur
  rezensiert, sondern es könnten sich auch Redakteure melden und ihre
  Bereitschaft, den Text zu publizieren, angeben. Später würde dann
  entschieden, wo der Text publiziert wird}.} {[}Interview 29{]}
\item
  \enquote{\emph{Gutachten als Peer Review sind Zusatzleistungen, wenn das
  offen wäre, wäre auch die Motivation geringer}.} {[}Interview 06{]}
\item
   \enquote{\emph{Ich fände es einen Anreiz, wenn Reviews als Publikationen gelten,
  das tun sie aber heute nicht. Das dauert vielleicht noch 20 Jahre}.}
  {[}Interview 06{]}
\item
  \enquote{\emph{Es bleibt eine offene Frage, ob Open Peer Review zu
  einer besseren Qualität führt. Es gibt auch die Gefahren, dass die
  Gutachter angreifbar werden und die Vertrauenswährung inflationiert
  wird. Ich glaube bei Open Peer Review nicht an einen Qualitätsgewinn,
  jedoch an einen Fairness-Gewinn}.} {[}Interview 04{]}
\item
   \enquote{\emph{Die Masseneinreichungen -- Masse ist auch schlecht, wenn die
  Qualität gut ist -- führen zu Gutachten, die nur noch formal und rein
  mechanisch die Qualität prüfen. Auch als Herausgeber kann man sich
  nicht mehr einige Tage Zeit lassen, um zu entscheiden, was man
  publizieren will. Als Autor geht man auch lieber auf Nummer sicher und
  lässt eine vielleicht interessante These zugunsten einer nur sauberen
  Methode fallen}.} {[}Interview 05{]}
\item
  \enquote{\emph{Peer Review ist eher für Naturwissenschaften geeignet
  weniger für Geisteswissenschaften, da diese nicht so
  verfahrensorientiert sind (z.\,B. Experimente). Ich sehe keinen
  besonderen Bedarf für Open Peer Review. Ein Preprint-Journal mit Open
  Peer Review hat nicht funktioniert, weil sich da kein Gutachter outen
  wollte}.} {[}Interview 14{]}
\item
  \enquote{\emph{Von Open Peer Review halte ich nicht soviel. Bei einem
  Herausgebergremium ist es aber sowieso eher offen. Das Review sollte
  bleiben wie es ist, es hat sich bewährt}.} {[}Interview 15{]}
\item
  \enquote{\emph{Als Herausgeber einer Reihe nutzen wir ein
  Double-Blind-Review-Verfahren und fordern unsere Beiträger auf,
  gegebenenfalls zu überarbeiten. Das ist derzeit die beste Praxis, die
  wir haben. Beim Digitalen Publizieren sind Experimente mit neuen
  Reviewformen (z.\,B. Open Review) ziemlich wichtig}.} {[}Interview 16{]}
\item
  \enquote{\emph{Jetzt schreibe ich das und es ist gerade
  forschungsmäßig interessant, aber es dauert zwei oder drei Jahre bis
  es veröffentlicht ist}.} {[}Interview 13{]}
\item
  \enquote{\emph{Die traditionellen Editions-Typen sind so, dass die
  Qualitätssicherung durch wiederholtes Korrekturlesen der
  Transkriptionen stattfindet. Bei uns wird alles x-mal Korrektur
  gelesen und von Redakteuren geprüft, die dafür sorgen, dass das dann
  auch stimmt, was wir da ins Netz stellen. Wir merken jetzt, dass im
  digitalen Medium eigentlich nicht bereits alles x-mal Korrektur
  gelesen werden muss, sondern dass es da ja die Möglichkeit gibt, eine
  Beta-Version schon mal zu veröffentlichen und dann zu sagen: Jetzt
  sammeln wir ein, was die Community an Fehlern findet, und verbessern
  das}.} {[}Interview 27{]}
\item
  \enquote{\emph{Bewertung der Qualität von Publikationen sollte nicht
  von Herausgebern übernommen werden, sondern Open Peer Review durch
  Kommentare aus der Community erfolgen}.} {[}Interview 36{]}
\item
  \enquote{\emph{Für Open-Peer-Review-Plattformen ist es notwendig, die
  Reviewer vorher eingeworben zu haben. Je hochkarätiger eine solche
  Plattform sein will, desto renommierter müssen die Wissenschaftler
  sein, die sich beteiligen. Man müsste als Gutachter eingeladen werden,
  Teil eines Reviewer-Teams einer Zeitschrift zu sein}.} {[}Interview
  29{]}
\item
  \enquote{\emph{Soziale Wissenschaftsnetzwerke nutze ich auch. Ich
  freue mich einfach unglaublich, wenn auf bestimmte Dokumente
  zugegriffen wird, und ich sehe, das hat sich jetzt jemand angesehen.
  Das ist auch eine Hauptmotivation gewesen}.} {[}Interview 21{]}
\item
  \enquote{\emph{Für Universitätsbibliotheken kritiklos alles online zu
  publizieren ist möglicherweise gar kein guter Weg. Eine
  Universitätsbibliothek könnte sich für jeden Fachbereich ein Gremium
  zusammensuchen (oder vielleicht könnten da Universitäten auch
  zusammenarbeiten), das sagt: Das nehmen wir auf und das nicht. Das
  müsste anonym funktionieren, die Sachverständigen müssen ja nicht an
  der eigenen Universität sein}.} {[}Interview 12{]}
\item
  \enquote{\emph{Impact-Messungen sind ungeeignet für die
  Geisteswissenschaften; auch Rezensionen sagen nichts über den
  wissenschaftlichen Wert aus, sondern sind eher Literaturanzeiger}.}
  {[}Interview 09{]}
\item
  \enquote{\emph{Bei Impact-Messungen müssen wir aufpassen. Das sind
  rein quantitative Informationsmöglichkeiten und die dürfen nicht die
  hauptsächlichen werden. Wenn die das ersetzen, was wir an qualitativer
  Information haben, dann machen wir einen großen Fehler. Das merkt man
  ja beim Fernsehen: Wenn alles nur noch nach Quote läuft, läuft gar
  nichts mehr.}} {[}Interview 27{]}
\end{enumerate}

%autor
\begin{center}\rule{0.5\linewidth}{\linethickness}\end{center}

\textbf{Michael Kleineberg} ist wissenschaftlicher Mitarbeiter an der
Universitätsbibliothek der Hum\-boldt-Universität zu Berlin.

\textbf{Ben Kaden} ist wissenschaftlicher Mitarbeiter an der
Universitätsbibliothek der Humboldt-Uni\-versität zu Berlin. Er ist
Mitbegründer und -herausgeber von LIBREAS. Library Ideas.

\end{document}
