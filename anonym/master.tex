\documentclass[a4paper,
fontsize=11pt,
%headings=small,
oneside,
numbers=noperiodatend,
parskip=half-,
bibliography=totoc,
final
]{scrartcl}

\usepackage{synttree}
\usepackage{graphicx}
\setkeys{Gin}{width=.4\textwidth} %default pics size

\graphicspath{{./plots/}}
\usepackage[ngerman]{babel}
\usepackage[T1]{fontenc}
%\usepackage{amsmath}
\usepackage[utf8x]{inputenc}
\usepackage [hyphens]{url}
\usepackage{booktabs} 
\usepackage[left=2.4cm,right=2.4cm,top=2.3cm,bottom=2cm,includeheadfoot]{geometry}
\usepackage{eurosym}
\usepackage{multirow}
\usepackage[ngerman]{varioref}
\setcapindent{1em}
\renewcommand{\labelitemi}{--}
\usepackage{paralist}
\usepackage{pdfpages}
\usepackage{lscape}
\usepackage{float}
\usepackage{acronym}
\usepackage{eurosym}
\usepackage[babel]{csquotes}
\usepackage{longtable,lscape}
\usepackage{mathpazo}
\usepackage[normalem]{ulem} %emphasize weiterhin kursiv
\usepackage[flushmargin,ragged]{footmisc} % left align footnote
\usepackage{ccicons} 

%%%% fancy LIBREAS URL color 
\usepackage{xcolor}
\definecolor{libreas}{RGB}{112,0,0}

\usepackage{listings}

\urlstyle{same}  % don't use monospace font for urls

\usepackage[fleqn]{amsmath}

%adjust fontsize for part

\usepackage{sectsty}
\partfont{\large}

%Das BibTeX-Zeichen mit \BibTeX setzen:
\def\symbol#1{\char #1\relax}
\def\bsl{{\tt\symbol{'134}}}
\def\BibTeX{{\rm B\kern-.05em{\sc i\kern-.025em b}\kern-.08em
    T\kern-.1667em\lower.7ex\hbox{E}\kern-.125emX}}

\usepackage{fancyhdr}
\fancyhf{}
\pagestyle{fancyplain}
\fancyhead[R]{\thepage}

% make sure bookmarks are created eventough sections are not numbered!
% uncommend if sections are numbered (bookmarks created by default)
\makeatletter
\renewcommand\@seccntformat[1]{}
\makeatother


\usepackage{hyperxmp}
\usepackage[colorlinks, linkcolor=black,citecolor=black, urlcolor=libreas,
breaklinks= true,bookmarks=true,bookmarksopen=true]{hyperref}
%URLs hart brechen
\makeatletter 
\g@addto@macro\UrlBreaks{ 
  \do\a\do\b\do\c\do\d\do\e\do\f\do\g\do\h\do\i\do\j 
  \do\k\do\l\do\m\do\n\do\o\do\p\do\q\do\r\do\s\do\t 
  \do\u\do\v\do\w\do\x\do\y\do\z\do\&\do\1\do\2\do\3 
  \do\4\do\5\do\6\do\7\do\8\do\9\do\0} 
% \def\do@url@hyp{\do\-} 
\makeatother 

%meta
%meta

\fancyhead[L]{A. Czymborskat \\ %author
LIBREAS. Library Ideas, 32 (2017). % journal, issue, volume.
\href{http://nbn-resolving.de/}
{}} % urn 
% recommended use
%\href{http://nbn-resolving.de/}{\color{black}{urn:nbn:de...}}
\fancyhead[R]{\thepage} %page number
\fancyfoot[L] {\ccLogo \ccAttribution\ \href{https://creativecommons.org/licenses/by/3.0/}{\color{black}Creative Commons BY 3.0}}  %licence
\fancyfoot[R] {ISSN: 1860-7950}

\title{\LARGE{Open-Access-Ideologie und nachteilige Systemwirkungen. Einige Überlegungen}} % title
\author{Anita Czymborska} % author

\setcounter{page}{1}

\hypersetup{%
      pdftitle={Open-Access-Ideologie und nachteilige Systemwirkungen. Einige Überlegungen},
      pdfauthor={Anita Czymborska},
      pdfcopyright={CC BY 3.0 Unported},
      pdfsubject={LIBREAS. Library Ideas, 32 (2017).},
      pdfkeywords={Open Access},
      pdflicenseurl={https://creativecommons.org/licenses/by/3.0/},
      pdfcontacturl={http://libreas.eu},
      baseurl={http://libreas.eu},
      pdflang={de},
      pdfmetalang={de}
     }



\date{}
\begin{document}

\maketitle
\thispagestyle{fancyplain} 

%abstracts

%body
Open Access ist die konsequente Fortsetzung der digitalen Transformation
der Wissenschaften im Bereich des wissenschaftlichen Publikationswesens.
Die Open-Access-Bewegung ist von der Grundannahme ausgegangen, dass der
freie Zugang zu und die Nachnutzbarkeit von wissenschaftlichen Texten
der Wissenschaft zweifelsohne und in jeder Hinsicht nutzen kann. Open
Access sollte Zugangsschranken überwinden, strukturelle Ungleichheiten
bezüglich der Wahrnehmung von Forschungsinhalten ausgleichen,
effizientere Forschung ermöglichen, zu neuen Auswertungsmöglichkeiten
und damit Erkenntnissen beitragen. Die Möglichkeiten der digitalen
Publikation sollten vollumfänglich ausgeschöpft werden. Dabei galt als
willkommener Effekt, dass neben dem wissenschaftlichen Diskurs auch die
allgemeine Öffentlichkeit und die Industrie von öffentlich finanzierten
Forschungsergebnissen profitieren kann. Nicht zuletzt hat die letzte
Reform des Urheberrechts, inklusive des zuvor kodifizierten
Zweitveröffentlichungsrechts, prinzipiell die Richtung verfolgt, die
Rechtslage an die digitalen Möglichkeiten anzupassen und dabei einen
Interessensausgleich herzustellen.

In der gegenwärtigen Situation sind jedoch dringend
Richtungsbestimmungen vorzunehmen und strategische Entscheidungen zu
treffen, die sowohl auf dialektische Entwicklungen, welche sich aus der
Open-Access-Bewegung selbst, als auch aus äußeren Systemumständen
ergeben, reagieren.

Die Dialektik der Digitalisierung ist in bisherigen
Open-Access-Diskursen nicht ausreichend aufgenommen und reflektiert
worden. Insgesamt haben diese den Anschein appellativer, affirmativer
und positivistischer Ideologieproduktion vermittelt, das heisst Open
Access als Zweck konstituiert, Maßnahmen diesem Zweck untergeordnet und
nachteilige Effekte nicht systematisch analysiert.

Dialektik wird hier verstanden als Existenz faktisch widersprüchlicher
Tendenzen im Rahmen einer Entwicklung, unter Einbezug auch
antagonistisch agierender Akteure. Im Open-Access-Bereich ist zudem --
wie in anderen gesellschaftlichen Teilbereichen, in denen es um die
Definition zukünftiger Strategien und damit um Machtfragen geht, zum
Beispiel beim Begriff \enquote{Nachhaltigkeit} oder
\enquote{greengrowth} -- auch zu beobachten, dass eine Dialektik im
altgriechischen Sinne stattfindet. Großverlage versuchen, den Begriff
von Open Access gegen ihn selbst zu wenden und in einer Weise
auszulegen, dass er de facto unterwandert und gegen die Ziele der
Wissenschaft gewendet wird.

Wenn das Ziel von Open Access war, der Zeitschriftenkrise zu begegnen,
ging es dabei um die Kostenkontrolle bei Preisen von Einzeltiteln und
vor allem von \enquote{big deals}. Open Access wollte eine bessere
digitale Rezeptions- und Nutzungssituation bei gleichzeitiger
Kostenkontrolle erreichen.

Die Preissteigerungen waren das Ergebnis von Oligopolen, die vornehmlich
durch drei Tendenzen ermöglicht wurden:

\begin{itemize}
\item
  Zunahme der Publikationen, damit auch Zunahme von Zeitschriften;
\item
  Steigende Notwendigkeit von Auswahlmechanismen und damit
  Stratifizierungseffekten;
\item
  Abhängigkeit der Wissenschaft von den Inhalten bei gleichzeitiger
  Abgabe von den Verwertungsrechten für die Inhalte.
\end{itemize}

Im Endergebnis dieser Entwicklung haben kommerzielle Akteure, deren
Imperativ die Gewinnmaximierung ist, die Hoheit erlangt über die
Infrastrukturen zur Produktion von Gütern, die Güter selbst sowie die
Kriterien für die Werthaftigkeit der Güter. Es in ein asymmetrischer
Markt mit nicht-substituierbaren Gütern entstanden, der lock-in-Effekte
und Pfadabhängigkeiten hervorgebracht hat.

Die Open-Access-Bewegung war eine Reaktion auf diese Situation, die sich
verselbstständigt hat. Von Anfang an wäre die Verbindung von
Lizenzierungsexpertise und Open-Access-Diskursen auf internationaler
Ebene sinnvoll gewesen, um Ziele zu erreichen, die nicht mit
vernunftorientiertem Diskurs, sondern nur mit Marktmacht zu erreichen
sind. Zugleich hätte Open Access von Anfang an für alle
Publikationsformen gleichermaßen propagiert werden sollen, um nicht
strukturelle Nachteile in einzelnen Disziplinen zu perpetuieren, sondern
die Wissenschaftskommunikation insgesamt gleichberechtigt zu öffnen.
Nach wie vor generieren die Großverlage und die STM-Disziplinen Gewinne
im Bereich der öffentlichen Aufmerksamkeitsökonomie, die sie in einen
Bedeutungsgewinn reinvestieren.

Das Ziel gegenwärtiger Situationsbeschreibungen aus Sicht der
Wissenschaft kann daher nicht sein, zu fragen: Is OA doing the job?, das
heisst lediglich zu analysieren, in welchem Maße die freie
Zugänglichkeit umgesetzt wird, und ob sie faktisch auch genutzt wird
beziehungsweise welche Evidenz es dafür gibt, dass die bessere
Zugänglichkeit auch bessere Forschung ermöglicht. Vielmehr muss,
aufbauend auf einem Diskurs, der auch den Erfolg oder das
Versagen\footnote{Toby Green: We've failed: Pirate black open access is
  trumping green and gold and we must change our approach: How can
  publishers see off the pirates? In: Learned Publishing, Wiley, Volume
  30, Issue 4, October 2017, Pages 325--329.
  \url{https://doi.org/10.1002/leap.1116}.} bisheriger OA-Strategien im
Kontext obiger Frage systematisch aufarbeitet, gefragt werden: Welche
negativen Auswirkungen können Open-Access-Strategien auf das
Gesamtsystem der Wissenschaftskommunikation und damit auf die
Wissenschaft haben? Welche Aspekte von Open-Access-Strategien sind
besonders kritisch? Ist das Ziel Open Access angesichts der Nebeneffekte
überhaupt noch richtig? Wenn, ja, in welcher Form?

Man kann in dieser Hinsicht zwei Kategorien unterscheiden:

\begin{enumerate}
\def\labelenumi{\arabic{enumi}.}
\item
  Systemstabilisierende Strategien, Maßnahmen und Effekte
\item
  Systemerneuernde Strategien, Maßnahmen und Effekte
\end{enumerate}

In der Tat können auch systemstabilisierende und systemerneuernde
Maßnahmen und Effekte in Open-Access-Strategien zusammenkommen, so dass
sich beide neutralisieren oder blockieren. Zudem kann man auch im Rahmen
der Lizenzierung und Literaturbeschaffung insgesamt
systemstabilisierende und systemerneuernde Strategien verfolgen oder
Maßnahmen umsetzen. Auch im System der wissenschaftlichen Bewertung gilt
dies analog. Publikationswesen und Bewertungssystem sind in
stabilisierenden und destabilisierenden Weisen miteinander verbunden,
auch wenn sie grundsätzlich unterschiedlichen Systemlogiken folgen.

Die asymmetrische und hypertrophe Situation im wissenschaftlichen
Publikationswesen ist dem wissenschaftlichen Diskurs dann abträglich,
wenn er durch gegebene Rahmenbedingungen oder Interventionen
fremdbestimmt wird und zum Beispiel nicht die
Publikationsdienstleistungen den Erfordernissen in der jeweiligen
Wissenschaftsdisziplin folgen, sondern die Erfordernisse durch die
Marktbedingungen geprägt werden. Er wird fremdbestimmt durch die
strukturelle, nicht intrinsisch bedingte Ungleichheit der
Finanzvoluminatransfers zur Aufrechterhaltung des Systems, durch die
Abhängigkeit des wissenschaftlichen Wertungssystems von
externalisierten, das heisst nicht mehr auf die konkreten
wissenschaftlichen Inhalte gerichteten Bewertungskriterien und durch den
Verlust der Inhaltskontrolle, also dem Verlust der Bestimmungsgewalt
über die Nutzung und Verwertung der Inhalte durch die Urheber. Zudem
wird er prinzipiell eingeschränkt durch die digitale (insbesondere die
angestrebte deanonymisierte) Wissenschaftsüberwachung, wie sie durch die
Anhäufung und Auswertung von Zugriffs-, Nutzungs-, Netzwerks- und
Weitergabespuren bei globalen, keiner wissenschaftlichen oder
rechtsstaatlichen Kontrolle unterworfenen, kommerziell agierenden und
privatrechtlich organisierten Firmen ermöglicht wird. Dies gilt sowohl
für digitale Spuren im Rahmen subskribierter wie im Open Access
verbreiteter Inhalte.\footnote{Siehe
  \url{https://ra21.org/index.php/what-is-ra21/}.}

Diese Rahmenbedingungen schränken die prinzipiell existierende
Wissenschaftsautonomie faktisch ein, auch wenn nicht jeder einzelne
Wissenschaftler, nicht jede Wissenschaftlerin diese Einschränkungen
wahrnimmt.

Open Access kann nur systemerneuernde und befreiende Wirkungen haben,
wenn diese Rahmenbedingungen verändert werden. Open Access kann in
diesem Fall daher kein Selbstzweck sein. Es kann nur Mittel zum Zweck
sein. Heute muss man sich immer fragen: Zu welchem Zweck wird Open
Access eingesetzt?

Systemstabilisierende Open Access-Strategien tragen dazu bei, dass diese
Rahmenbedingungen sich nicht ändern. Goldene Open-Access-Strategien bei
der Kooperationen mit dem kommerziellen Sektor haben diesen Effekt, wenn
sie die Zahlung von Veröffentlichungsgebühren nicht an systemändernde
Bedingungen knüpfen. Insbesondere kann die Finanzierung von APCs in
Abhängigkeit von den in einem Bereich vorhandenen Forschungsgeldern oder
traditionell verausgabten Geldern für die Literaturversorgung, die
Privilegierung einzelner Verhandlungspartner bei der Lenkung von
Finanzströmen, die Orientierung der Finanzierung an traditionellen
Kriterien wie dem JIF und damit der Erhalt der Markenbildung,
systemstabilisierende und weiter oligopolisierende Effekte haben, die
dem ursprünglichen Ziel von Open Access schaden.

Nicht alle Effekte sind intendiert, können aber -- auch wenn Maßnahmen
andere Ziele verfolgen -- Systemwirkung entfalten. Dies gilt auch im
Bereich systemerneuernder Strategien.

Systemerneuernde Strategien verfolgen den Zweck, durch Einzelmaßnahmen
die oben genannten Rahmenbedingungen zu ändern. Systemerneuernde
Strategien können in produktive und destruktive Kategorien unterteilt
werden. Produktive Strategien bauen Strukturen auf oder etablieren
Mechanismen, die parallel und alternativ zu den bisherigen gelten und
diese zu überwinden beziehungsweise durch die Erhöhung der Vielfalt die
Marktmechanismen zu ändern und eine Wettbewerbsbalance herzustellen in
der Lage sind.

Destruktive Strategien oder Maßnahmen unterwandern das bisherige System.
Destruktive Effekte gehen zurzeit maßgeblich von auch illegalen
Aktivitäten aus (SciHub und Sharing von geschützten Inhalten über
Academic Social Networks). Problematisch wird es, wenn solcherart
destruktive Strategien von den Hauptakteuren der
Wissenschaftskommunikation, das heisst den Wissenschaftlerinnen und
Wissenschaftlern, als wissenschaftsadäquat gesehen, aufgenommen und
verfolgt werden. Die Wissenschaftlerinnen und Wissenschaftler machen
sich nicht nur persönlich strafbar unter den jetzigen rechtlichen
Bedingungen, sondern sie verdeutlichen, dass die legalen Strategien der
Zugriffsorganisation nicht ausreichend Wirkung entfalten. Zugleich wird
deutlich, dass das Ziel der kommerziellen Verlage, welche gegen diese
Verbreitungsweisen vorgehen, es nicht ist, die Wissenschaft zu
unterstützen, sondern den Umsatz zu schützen, den Gewinnerhalt und die
Gewinnmaximierung in jeder Arena -- wie es jeher ihrer Rolle auch
entspricht -- zu verfolgen.

Man kann auch einzelne Sekundäreffekte identifizieren, die
wissenschaftsfeindlich sind und durch gewisse Maßnahmen im
Open-Access-Bereich hervorgerufen werden. Diese sind aber nicht an und
für sich mit Open Access verbunden, sondern können generell eintreten
und werden nur verstärkt.

An dieser Stelle gilt es, auch immer trennscharf zwischen Effekten der
digitalen und der offenen Wissenschaft zu unterscheiden. Prinzipiell
wissenschaftsfeindliche Effekte sind immer solche, welche auf einen
extrinsischen und sachfremd motivierten Systemzwang zurückgehen, in die
Wissenschaftsautonomie intervenieren und Entscheidungen beeinflussen
können, welche rein wissenschaftsgeleitet sein sollten. Solche Effekte
werden heutzutage vor allem vom kommerziellen Publikationssystem
generiert. Im Open Access prägen sie sich beispielhaft folgendermaßen
aus: Artikelgebühren steigen mit der Länge oder Komplexität von Texten.
Artikelgebühren berechnen sich nach Art der CC-BY-Lizenz.
Artikelgebühren steigen mit dem vermeintlichen Prestige der Zeitschrift.
Renommierte Verlage verlangen höhere Buchveröffentlichungspreise.
Aspekte wie slicing, predatory publishing und so weiter können
prinzipiell auch ohne Open Access auftreten und tun dies auch.

Nicht jede Veränderung epistemischer Strukturen ist natürlich
wissenschaftsfeindlich. Wissenschaftsfeindlich sind allerdings Eingriffe
in epistemische Logiken und Diskursstrukturen, die nicht nur extrinsisch
generiert sondern auch extrinsisch rechtfertigt werden. An dieser Stelle
werden mich Wissenschaftstheoretiker der Naivität bezichtigen und damit
Recht haben. Ich versuche lediglich, solche Effekte pragmatisch zu
umschreiben. Möglich ist, dass die Qualitätsfrage in ganz neuen
Dimensionen gelöst werden muss. Allerdings ist sie ein spezifisch
wissenschaftsimmanentes Betätigungsfeld. Wenn Open Access-Zeitschriften
oder digitale Publikationen sie aufwerfen, dann muss man sagen: ein
funktionierendes System kann sie lösen und sogar besser lösen als
bisher, und wenn nicht, ist das Gesamtsystem der Qualitätssicherung
fragwürdig und muss ebenfalls verbessert werden. Qualitätssicherung,
eine wissenschaftliche \enquote{Filterfunktion}, wird angesichts der
Masse an Material immer wichtiger und damit auch ein
Distinktionskriterium der Wissenschaft gegenüber anderen Diskursen. Sie
erfordert höchste Aufmerksamkeit, Verantwortungsbereitschaft und
Sorgfalt der Communities. Die Verlage waren nicht und sind nicht
durchweg in der Lage, diese Funktion zu übernehmen.

An dieser Stelle muss ein Wort zur vermeintlichen Obligatorik gesagt
werden, welche mit Open-Access-Verpflichtungen einhergeht. Es gibt
rechtliche Untersuchungen\footnote{Michael Fehling: Verfassungskonforme
  Ausgestaltung von DFG-Richtlinien zur Open-Access-Publikation:
  \url{http://www.ordnungderwissenschaft.de/pdf/2014-4/PDFs_Gesamtpdf/04_01_fehling_dfg.pdf}}
dazu, wie sie sich in Deutschland zum Artikel 5 des Grundgesetzes
verhält. Mir geht es nicht darum, sie moralisch zu rechtfertigen,
sondern wissenschaftsimmanent. Die Wissenschaftskommunikation
funktioniert nur, wenn Wissenschaft auch kommuniziert und rezipiert
wird. Open Access bietet die beste Möglichkeit zur Erreichung dieses
Ziels. Nur ein offenes System dient der Zirkulation und Generierung von
Erkenntnissen. Es ist an sich widersprüchlich, die Freiheit der
Wissenschaft anzuführen, um gegen Open-Access-Verpflichtungen zu
argumentieren. Die Wissenschaft ist momentan angesichts der Systemzwänge
nicht frei. Die Publikationsfreiheit wird durch Open Access überhaupt
nicht tangiert. Tangiert werden Geschäftsmodelle von Verlagen, die nicht
bereit sind oder nicht in der Lage sind, sich anzupassen. Es kann daher
nicht darum gehen, ob eine Verpflichtung zum Open-Access die
Wissenschaftsfreiheit einschränkt. Es muss vielmehr darum gehen, wie
Open-Access-Verpflichtungen auf das gesamte Publikationssystem wirken.
Dadurch muss eine Begründung dafür oder dagegen legitimiert werden.

Open-Access-Verpflichtungen können auch nachteilige Effekte haben, wenn
sie gegensätzliche Maßnahmen befürworten und Zielkonflikte produzieren.
Beim grünen und goldenen Weg des Open Access kann es sich zunehmend um
solche Gegensätze handeln, insbesondere, wenn Verlage die
unterschiedlichen Wege und damit verschiedene Länderstrategien
gegeneinander ausspielen. So wird eine Stabilisierung des Systems durch
ein Patt erreicht, von dem vor allem große globale Verlage profitieren
können. Grüner Open Access, der von einer Erstveröffentlichung in einer
Subskriptionszeitschrift abhängt, kann systemstabilisierend wirken.
Goldener Open Access überwindet prinzipiell eine nicht-freie
Erstpublikation. Goldener Open Access kann jedoch in anderer Hinsicht
auch wiederum systemstabilisierend wirken. Grüner Open Access kann
kostensenkend wirken. Goldener Open Access kann für die Nutzung in der
Wissenschaft geeigneter sein.

Open-Access-Verpflichtungen können zudem Open Access schwächen, indem
sie eine Komplexität der Publikations-, Auffindbarkeits- und
Rezeptionssituation aus Wissenschafts- und Verwaltungssicht erzeugen,
welche sowohl der Umsetzung als auch der Akzeptanz schadet und zu
höheren Transaktionskosten führt. Es geht für die Wissenschaft darum,
Mittel im Publikationssystem so effizient wie möglich einzusetzen, um
Freiheiten in Form von Überfluss auszunutzen und nicht statt bei
Aktionären in hypertrophen und übersichtlichen (Infra-)Strukturenmittel
zu vergeuden.

Zur Verbesserung der Situation insgesamt gehört auch eine Eindämmung der
Publikationsflut, sofern sie extrinsisch motiviert ist. Die
Konzentration auf Wesentliches ist auch hier eine Tugend. Dazu müssen
Bewertungssysteme angepasst beziehungsweise andere Spielregeln im System
der Wissenschaftsgeltung insgesamt erfunden werden. Eine zu große
Komplexität der Regelungen sowie der Such- und Zugangswege im Bereich
der Publikation und der Literaturversorgung insgesamt -- Open Access
oder und Lizenzen -- kann zum Systemkollaps führen. Es ist aber nicht
ausgemacht, wer von diesem Kollaps profitiert und wer Schaden nimmt.

Jede Evolution ist dialektisch; jede Revolution ist es auch. Bei
Revolutionen wenden sich in einer späteren Phase oftmals die
Revolutionäre gegeneinander. Bei Evolutionen gewinnt eine überlegene
Fraktion. Es ist offen, wie die Evolution im System der
Wissenschaftskommunikation weiterverlaufen wird. Man sollte aber
wenigstens wissen, wo die Fronten verlaufen und warum man Kriege
verliert. Insbesondere droht die Gefahr, dass die Open-Access-Bewegung
sich in Scharmützeln verzettelt.

Open Access darf aus Sicht der Wissenschaft nicht ideologisch, sondern
muss funktional gesehen werden. Dazu gehört es, marginalisierende,
stabilisierende und disruptive Effekte zu erkennen.

Der Humanist würde sagen: Dieses Erkennen mündet optimalerweise in
Handlungen. Um das ursprüngliche Ziel von Open Access noch erreichen zu
können, muss man möglicherweise die Frage nach dem Publikationsformat
und -modus zurückstellen, jedenfalls wenn sie mit höheren Kosten und
stabilisierenden Effekten verbunden ist, und Strategien verfolgen, die
systemerneuernd wirken.

Eine völlige Entkopplung von Publikationsdienstleistungen und
Bewertungsmechanismen scheint insgesamt als einziger Weg, um das
Publikationswesen systemerneuernd aufzustellen. Dabei werden zahlreiche
Akteure, Dienstleistungen und Marktteilnehmer nötig bleiben, jedoch
unter dem Primat, dass die Hoheit und Entscheidungsgewalt über die
Inhalte, die Kriterien von deren Wertschätzung (assessment, appraisal)
und die Infrastrukturen durch die Wissenschaft ausgeübt wird. Die Frage
muss immer sein: Wer kontrolliert Publikationsopportunität und -orte,
Zugang zu Publikationen, deren Archivierung, Verbreitung und Nutzung?
Wer gibt die Kriterien der wissenschaftlichen Bewertung vor? Ein
grundsätzlicher Antagonismus der Ziele von Wissenschaft und Verlagen
kann dabei nicht außer Acht gelassen werden. Pfadabhängigkeiten müssen
schnell erkannt, lock-in und sell-out-Effekte vermieden werden.

Insgesamt muss auch die Gesamtheit der Wissenschaftlerinnen und
Wissenschaftler sich dazu aufgerufen fühlen, ein Bewusstsein dafür zu
entwickeln, unter welchen Bedingungen und wie das System ausgestaltet
werden soll. Bewusstsein bestimmt dann das Werden.

%autor
\begin{center}\rule{0.5\linewidth}{\linethickness}\end{center}

\textbf{Anita Czymborska}, Wissenschaftlerin, veröffentlicht hier unter
Pseudonym.

\end{document}
