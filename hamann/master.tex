\documentclass[a4paper,
fontsize=11pt,
%headings=small,
oneside,
numbers=noperiodatend,
parskip=half-,
bibliography=totoc,
final
]{scrartcl}

\usepackage{synttree}
\usepackage{graphicx}
\setkeys{Gin}{width=.4\textwidth} %default pics size

\graphicspath{{./plots/}}
\usepackage[ngerman]{babel}
\usepackage[T1]{fontenc}
%\usepackage{amsmath}
\usepackage[utf8x]{inputenc}
\usepackage [hyphens]{url}
\usepackage{booktabs} 
\usepackage[left=2.4cm,right=2.4cm,top=2.3cm,bottom=2cm,includeheadfoot]{geometry}
\usepackage{eurosym}
\usepackage{multirow}
\usepackage[ngerman]{varioref}
\setcapindent{1em}
\renewcommand{\labelitemi}{--}
\usepackage{paralist}
\usepackage{pdfpages}
\usepackage{lscape}
\usepackage{float}
\usepackage{acronym}
\usepackage{eurosym}
\usepackage[babel]{csquotes}
\usepackage{longtable,lscape}
\usepackage{mathpazo}
\usepackage[normalem]{ulem} %emphasize weiterhin kursiv
\usepackage[flushmargin,ragged]{footmisc} % left align footnote
\usepackage{ccicons} 

%%%% fancy LIBREAS URL color 
\usepackage{xcolor}
\definecolor{libreas}{RGB}{112,0,0}

\usepackage{listings}

\urlstyle{same}  % don't use monospace font for urls

\usepackage[fleqn]{amsmath}

%adjust fontsize for part

\usepackage{sectsty}
\partfont{\large}

%Das BibTeX-Zeichen mit \BibTeX setzen:
\def\symbol#1{\char #1\relax}
\def\bsl{{\tt\symbol{'134}}}
\def\BibTeX{{\rm B\kern-.05em{\sc i\kern-.025em b}\kern-.08em
    T\kern-.1667em\lower.7ex\hbox{E}\kern-.125emX}}

\usepackage{fancyhdr}
\fancyhf{}
\pagestyle{fancyplain}
\fancyhead[R]{\thepage}

% make sure bookmarks are created eventough sections are not numbered!
% uncommend if sections are numbered (bookmarks created by default)
\makeatletter
\renewcommand\@seccntformat[1]{}
\makeatother


\usepackage{hyperxmp}
\usepackage[colorlinks, linkcolor=black,citecolor=black, urlcolor=libreas,
breaklinks= true,bookmarks=true,bookmarksopen=true]{hyperref}

%meta
%meta

\fancyhead[L]{N. Hamann \\ %author
LIBREAS. Library Ideas, 32 (2017). % journal, issue, volume.
\href{http://nbn-resolving.de/}
{}} % urn 
% recommended use
%\href{http://nbn-resolving.de/}{\color{black}{urn:nbn:de...}}
\fancyhead[R]{\thepage} %page number
\fancyfoot[L] {\ccLogo \ccAttribution\ \href{https://creativecommons.org/licenses/by/3.0/}{\color{black}Creative Commons BY 3.0}}  %licence
\fancyfoot[R] {ISSN: 1860-7950}

\title{\LARGE{Offenheit als demokratisches Prinzip}} % title
\author{Nikolaus Hamann} % author

\setcounter{page}{1}

\hypersetup{%
      pdftitle={Offenheit als demokratisches Prinzip},
      pdfauthor={Nikolaus Hamann},
      pdfcopyright={CC BY 3.0 Unported},
      pdfsubject={LIBREAS. Library Ideas, 32 (2017).},
      pdfkeywords={Open Access},
      pdflicenseurl={https://creativecommons.org/licenses/by/3.0/},
      pdfcontacturl={http://libreas.eu},
      baseurl={http://libreas.eu},
      pdflang={de},
      pdfmetalang={de}
     }



\date{}
\begin{document}

\maketitle
\thispagestyle{fancyplain} 

%abstracts

%body

\hypertarget{offenheit}{%
\section*{Offenheit}\label{offenheit}}

Offenheit -- das ist ein Wort, das Erwartungen weckt und Möglichkeiten
verspricht. Denken Sie nur an die \enquote{offene Gesellschaft}, wie sie
Karl Popper in den 50er Jahren gezeichnet hat, denken Sie an die
\enquote{offene Zweierbeziehung} der sexuellen Revolution in der Folge
von 1968, denken Sie an den Begriff \enquote{offenes Haus}, das Besuche
ohne vorige Vereinbarung verspricht.

Auch heute sind wir mit einer Vielzahl von Begriffspaaren konfrontiert,
die das Wort \enquote{offen} bzw. \enquote{open} beinhalten:
\enquote{open culture}, \enquote{open data}, \enquote{open government},
\enquote{open source}, \enquote{open access}, \enquote{open archiving}
und viele viele mehr. Es gibt aber auch Auslegungen von \enquote{open},
an die man nicht sofort denken würde. So versteht zum Beispiel der
multinationale Konzern Procter \& Gamble \enquote{open innovation}
dahingehend, dass er die Weiterentwicklung bestehender bzw. die
Erfindung neuer Produkte großteils auf die Konsument*innen auslagert und
sich damit hohe Innovationskosten erspart.

Letztlich ist auch \enquote{Offenheit} oder \enquote{Openness} nur eine
Worthülse, die mit Inhalten gefüllt werden muss. Welche Inhalte das sein
können und werden, hängt auch davon ab, wer in der Gesellschaft seine
Interessen besser durchsetzen können wird: Die großen Konzerne der
Informationsbranche wie Google, Facebook etc., die mehr darauf setzen,
dass {wir} offen (also gläsern) werden sowie die staatlichen wie
privaten Informationssammler und Geheimdienste. Oder die Menschen in
Wissenschaft und Kunst, die Studierenden, insgesamt die Bürgerinnen und
Bürger aller Länder, also wir, deren Bedürfnis es ist, auf Wissen,
Information und Werke der Kultur ohne Zeitverlust sowie ohne technische
und finanzielle Beschränkungen zugreifen zu können. Es geht also auch
hier um eine Machtfrage, und da kommen Politik und daher auch die Frage
nach dem Wesen der Demokratie ins Spiel.

\hypertarget{demokratie}{%
\section*{Demokratie}\label{demokratie}}

Das Wesen der Demokratie besteht darin, dass das Recht und die Macht vom
Volk ausgehen. Wie dieses Volk oder -- besser -- die Gesellschaft
strukturiert ist, fällt dabei nicht ins Gewicht. So wird zum Beispiel
das alte Athen gerne als Wiege der Demokratie bezeichnet, obwohl wir
alle wissen, dass nur ein kleiner Teil der Bevölkerung, nämlich die
freien, wohlsituierten männlichen Bürger, stimmberechtigt war.

Die Länder des sowjetisch dominierten Systems bezeichneten sich in
sinnloser Verdoppelung des Begriffs als Volksdemokratien, was formal
richtig gewesen sein mag, aber durch das Diktat der kommunistischen
Parteien konterkariert wurde. Ähnliche Kritik kann man zu Recht auch
heute gegenüber sogenannten demokratischen Systemen im arabischen oder
asiatischen Raum äußern.

Aber auch unsere westlichen repräsentativen Demokratien, mit ihrem
System von Wahlen und Parlamenten, kommen dem Idealbild von Demokratie
bestenfalls nahe, sind sie doch dadurch gekennzeichnet, dass die
Bürger*innen ihre Vertretungen zwar frei wählen können, die wahre Macht
aber doch in den Händen von nationalen und multinationalen Konzernen
liegt. Die aktuellen -- geheimen -- Verhandlungen über die
Handelsabkommen TTIP und CETA zeichnen ein klares Bild davon.

Die Entwicklung der digitalen Gesellschaften könnte allerdings eine
Disruption der gängigen Auffassungen hinsichtlich Demokratie mit sich
bringen. Würde die Digitalität doch unmittelbare, zeitnahe und
weltumspannende Mitbestimmung der Menschen an allen sie betreffenden
Fragen möglich machen. Die Einführung von World Wide Web und Internet
mit allen damit verbundenen Möglichkeiten waren eine noch viel größere
Disruption, als es die Erfindung des Buchdrucks mit beweglichen Lettern
seinerzeit war.

Disruptionen in der gesellschaftlichen Wirklichkeit können also die
Verhältnisse transformieren. Darauf müssen Politik und Recht reagieren.
Wie darauf reagiert wird, ist wieder ein Ergebnis der gesellschaftlichen
Machtverhältnisse, womit sich der Kreis zum Feld Demokratie schließt.

\textbf{Verknappung versus gemeinsames geistiges Eigentum}

\begin{quote}
``Privatisierung und Verknappung sind hinderlich und haben spätestens
dann keine Berechtigung mehr, wenn es ums Überleben geht. Stattdessen
sollten die Parolen gelten: Aufhebung aller Patente! Freiheit für den
Erfindergeist! Kopiert global und schrankenlos! Macht Euch zu Diensten,
was Euch nützlich ist! Schleift die Mautstellen! Entmachtet die
Wegelagerer, die von Euch Zölle verlangen, wenn Ihr Euch bilden wollt!
\end{quote}

\begin{quote}
Solche Appelle wären noch vor 20 Jahren ultraradikale Phrasen gewesen.
Heute dagegen stehen die technischen Mittel bereit, um die Gesamtheit
des {[}\ldots{}{]} wichtigen Wissens so aufzubereiten, dass es möglichst
viele verstehen, es so zu verbreiten, dass es alle Erdenmenschen
unentgeltlich nutzen können, und so zu präsentieren, dass die
Veränderung für eigene Zwecke möglich ist. Solche Darstellungen und
Verbreitungen liegen im gemeinsamen Interesse der Menschheit. Deshalb
sollten sie ausdrücklich mit dem Zweck des Kopierens und Adaptierens zur
Verfügung stehen. Was man heute mit großem Aufwand zu verhindern
versucht oder nur stillschweigend duldet, die Verbreitung ohne Zahlung,
sollte zum expliziten Ziel werden. Was vom bürgerlichen Standpunkt wie
ein gigantischer Raubzug aussieht, wäre tatsächlich eine Wallfahrt für
das gemeinsame Wohl.``\footnote{Hans Thie: Rotes Grün. Pioniere und
  Prinzipien einer ökologischen Gesellschaft. Eine Veröffentlichung der
  Rosa-Luxemburg-Stiftung,
  \url{http://www.rosalux.de/fileadmin/rls_uploads/pdfs/sonst_publikationen/VSA_Thie_Rotes-Gruen.pdf},
  S. 99 f. (abgerufen 22.09.2017)}
\end{quote}

Diese von Hans Thie, dem Wirtschaftsreferenten der Fraktion DIE LINKE im
deutschen Bundestag, ursprünglich auf den \enquote{ökologischen Geist}
gemünzten Sätze können unverändert auch für den \enquote{Geist der
Offenheit} in Wissenschaft und Kunst verwendet werden, denn \enquote{bei
rein geistiger Produktion, also dort, wo geistige Betätigung nicht nur
Mittel, sondern Selbstzweck des Wirtschaftens ist, kommen die
bürgerlichen Verhältnisse grundlegend ins Rutschen. Hier wanken die
bisherigen Gesetze der Ökonomie und zugleich ihre bürgerlichen Formen.
Hier ist tendenziell alles anders. Der Geist ist freier Commonist, nicht
elitärer Bürger}.\footnote{Hans Thie: Rotes Grün. Pioniere und
  Prinzipien einer ökologischen Gesellschaft. Eine Veröffentlichung der
  Rosa-Luxemburg-Stiftung,
  \url{http://www.rosalux.de/fileadmin/rls_uploads/pdfs/sonst_publikationen/VSA_Thie_Rotes-Gruen.pdf},
  S. 100 f. (abgerufen 22.09.2017)}

In der Ökonomie des Geistes gilt laut Thie der Grundsatz, dass dieser
umso produktiver sein kann, je allgemeiner und je mehr er der
bürgerlichen Eigentumsform entrissen werden kann. Offenheit als
politisches Prinzip bedeutet also, Ergebnisse der Wissenschaft und der
Kunst nicht mehr primär in ihrer Warenform, sondern als weltweites
gesellschaftliches Eigentum zu betrachten.

Dies bedeutet aber per se noch nicht, dass damit quasi automatisch die
bürgerlichen Verhältnisse in eine neue, wie auch immer bezeichnete
Gesellschaftsform transformiert werden würden, denn -- wie die
Initiativen vieler Verlage in Richtung Open Access zeigen -- es lassen
sich solche Modelle durchaus in die kapitalistische Verwertungslogik
einbauen. Auch das intensive Engagement des US-Investors George Soros
weist in diese Richtung. Auf jeden Fall aber hat das Prinzip Offenheit
eine emanzipatorische Funktion und bedeutet eine Demokratisierung der
gesellschaftlichen Verhältnisse, da sich Wissen und Werke der Kunst auf
diese Art leichter und schneller verbreiten lassen, weniger leicht der
Willkür der Verlage oder staatlicher Zensur unterworfen werden können,
die weltweite Kommunikation fördern und daher auch den weniger
entwickelten Ländern neue Chancen eröffnen. Bezieht sich das Prinzip
Offenheit auch auf politische Entscheidungen und Akte der Verwaltung, so
befördert dies die Transparenz derselben und gibt den Citoyens größere
Chancen auf ihre Lebensumstände Einfluss zu nehmen.

\hypertarget{open-access-als-teilmenge-des-prinzips-offenheit}{%
\section*{Open Access als Teilmenge des Prinzips
Offenheit}\label{open-access-als-teilmenge-des-prinzips-offenheit}}

Geistige Produkte via Internet und WWW als öffentliches Gut zur
Verfügung zu stellen ist der wesentliche Inhalt des Prinzips Offenheit
in einer digitalisierten Welt. Diese Offenheit kann auf verschiedenen
Ebenen und mit unterschiedlichen Wirkungen erreicht werden (Open Access,
Open Data, Open Science, Open Source \ldots{}). Für diesen Artikel soll
vor allem Open Access beleuchtet werden, da dies für Bibliotheken die
größte Relevanz besitzt.

\begin{quote}
\enquote{Die Idee des Open Access entspricht ganz den traditionellen
Werten und Zielen akademischen Arbeitens, das auf Kollegialität, dem
Austausch von Ideen und Ergebnissen, der gemeinsamen Suche nach
Erkenntnis und der Verbreitung von Wissen zum Wohle der Gesellschaft
insgesamt beruhen. Erst das digitale Zeitalter ermöglicht einen
gemeinsamen freien Zugang zu wissenschaftlicher Erkenntnis und zu
Forschungsdaten, wie er zuvor unter den Bedingungen des Druckzeitalters
nicht denkbar war. Entscheidend dabei ist die Auswirkung dieses neuen
Instruments auf die Entwicklung einer Gesellschaft, die den
Leitbegriffen von Open Scholarship und Open Knowledge verpflichtet
ist.}\footnote{Hans-Jochen Schiwer: Es wird Zeit, alle alles lesen zu
  lassen, in: FAZ 05.06.2011,
  \url{http://www.faz.net/aktuell/beruf-chance/campus/open-access-es-wird-zeit-alle-alles-lesen-zu-lassen-1639084.html}
  (abgerufen 22.09.2017)}
\end{quote}

Ziel aller Open-Access-Initiativen ist es, Wissen und Information in
digitaler Form für jedermann unentgeltlich und ohne technische oder
rechtliche Barrieren dauerhaft zugänglich und nachnutzbar zu machen, was
dem Fortschritt sowohl der Wissenschaft als auch der Menschheit
insgesamt dient. Unterschiede gibt es in Bezug auf die Ausdehnung des
Geltungsbereiches. Minimalforderung ist das Öffentlichmachen von
Forschungsergebnissen, die mit Unterstützung der Öffentlichkeit
erarbeitet wurden, die Maximalforderung bezieht auch Ergebnisse der
Forschungsabteilungen der Wirtschaft bzw. solche, die in öffentlichen
Institutionen (vor allem Universitäten und Fachhochschulen) im Auftrag
der Wirtschaft erstellt wurden. In Hinblick auf die Maximalforderung
werden sehr schnell die Grenzen der gegenwärtigen Open-Access-Politik
sichtbar, denn Forschungsdaten zum Beispiel der Pharmaindustrie werden
von den Wenigsten gedanklich -- geschweige denn tatsächlich -- dem
Prinzip Offenheit unterworfen.

\emph{Exkurs:}

\emph{Open Access (wie auch alle anderen Kategorien von Offenheit) kann
nicht in gleicher Weise für Wissenschaft und Kunst gelten. Während
Wissenschaftler*innen in der Regel ihre Arbeit in (mehr oder weniger)
gesicherter Position vollbringen, sind freischaffende Künstler*innen
sehr wohl darauf angewiesen, ihre Werke vermarkten zu können. Daher wird
es -- dies nur als Nebenbemerkung -- notwendig sein, das
Urheber*innenrecht nach Sparten getrennt zu reformieren oder die
Einkommensfrage der Künstler*innen generell, etwa durch ein
bedingungsloses Grundeinkommen, auf eine neue Basis zu stellen.}

\hypertarget{open-access-und-nord-suxfcd-beziehungen}{%
\section*{Open Access und
Nord-Süd-Beziehungen}\label{open-access-und-nord-suxfcd-beziehungen}}

\enquote{The close link between scientific development and the social
and economic wellbeing of a nation has long been recognised. Jawaharlal
Nehru, India's first prime minister, said: \enquote{It is science alone
that can solve the problems of hunger and poverty, of insanitation and
illiteracy\ldots{} Who indeed can ignore science today? We need it at
every turn.} Likewise, a 1982 UNESCO report states that
\enquote{assimilation of scientific and technological information is an
essential precondition for progress in developing countries}. The
InterAcademy Council says: \enquote{In a world moving rapidly toward the
knowledge-based economies of the 21st century, capacity building in
science and technology (S\&T) is necessary everywhere. But the need is
greatest for the developing countries.}}\footnote{Leslie Chan, Barbara
  Kirsop, Subbiah Arunachalam: Open access archiving: the fast track to
  building research capacity in developing countries. In: SciDevNet
  (11.02.2005),
  \url{http://www.scidev.net/global/communication/feature/open-access-archiving-the-fast-track-to-building-r.html}
  (abgerufen 22.09.2017)}

Wie bereits im vorigen Abschnitt erwähnt, spielt Offenheit und damit
auch Open Access in der Diskussion um gerechte Nord-Süd-Beziehungen eine
große Rolle, oft aber in dem Sinn, dass der Norden in
(post-)imperialistischer Großzügigkeit die ärmeren Länder an seinem
Wissen teilhaben lassen will. Die sich entwickelnden Länder mögen diese
Sichtweise naturgemäß nicht sehr gerne und stellen ihre ganz andere
Argumente entgegen, warum die Zeit für Open Access bzw. Open Archiving
reif sei.

Natürlich ist es wahr, dass die Länder des Südens auf Grund ihrer Armut
noch viel weniger in der Lage sind als wir im globalen Norden, die
exorbitanten Kosten für traditionell oder digital gespeichertes Wissen
aufzubringen. Auch die Forschungsetats sind verständlicherweise viel
spärlicher bedacht. Aber die Länder des Südens haben auch eine Menge
Wissen anzubieten, das bisher viel zu wenig beachtet wurde, vor allem im
Gesundheits- oder Agrarsektor und Klimafragen. Das Ermöglichen eines
gerechten Austauschs über Open Access würde also auch dem Norden großen
Nutzen bringen.

Die Länder des Südens haben oft -- auch das eine Folge von Imperialismus
und Kolonialismus -- sehr konservative Strukturen für ihre Bildungs- und
Forschungslandschaft übernommen. Dennoch haben diese Länder oft einen
prozentuell wesentlich höheren Ausstoß an Open-Access-Journalen als
manche Länder des Nordens. Das heißt auch strukturell kann gegenseitiges
Lernen durchaus erfolgversprechend und lohnend sein.

\hypertarget{grenzen-und-kritik-von-open-access}{%
\section*{Grenzen und Kritik von Open
Access}\label{grenzen-und-kritik-von-open-access}}

Kritik an Open-Access-Strategien kam hauptsächlich aus den Reihen der
Verleger und aus der Wirtschaft. Erstere war eindeutig vom möglichen
Wegfall gewinnträchtiger Geschäftsfelder geleitet und somit von einer
gesellschaftspolitischen Position widerlegbar, wird aber von manchen
Wissenschaftlerinnen entweder aus einer traditionellen Denkweise heraus
oder mit dem Argument unterstützt, Open-Access-Publikationen brächten
nicht die gleiche Reputation wie Veröffentlichungen in herkömmlichen
Journalen mit hohem Impact-Faktor und behinderten dadurch die
wissenschaftliche Karriere.

Die Unternehmen sind oftmals Auftraggeberin für angewandte Forschung und
fürchten aus Konkurrenz- und Profitgründen natürlich, durch Open Access
Ergebnisse der von ihnen beauftragten Forschung Mitbewerbern gegenüber
nicht mehr geheim halten zu können. Gleichzeitig profitiert die
Wirtschaft aber auch von schnellerem und kostenfreiem Zugang zu
Information. Hier ist genau der Punkt erreicht, wo die Diskussion von
der Ebene der allgemeinen Sinnhaftigkeit zur Ebene der
Gesellschaftsordnung führt, wo also die Debatte zwischen
Weiterentwicklung der bürgerlichen Demokratie oder einem anderen
Gesellschaftsmodell geführt werden muss.

Kritik an der Verortung von Open Access innerhalb der bürgerlichen
Ordnung kommt aus linken Gruppen. Moniert wird einerseits die
ideologische Begründung für Open Access im imperialistischen Diskurs
einer \enquote{Entwicklungshilfe} -- dazu siehe oben -- aber auch die
Tatsache, dass Open Access als technische Lösung allein bestehende
Ungleichheiten nicht zu vermindern vermag. \enquote{{[}Es{]} zeigt die
aktuelle Definition der Golden Road des Open Access, dass es gerade die
großen Wissenschaftsverlage sind, die es sehr genau verstanden haben,
Open Access für sich und damit für eine weitergehende Vormachtstellung
zu nutzen. War mit der Golden Road ursprünglich nur gemeint,
Wissenschaftliche Beiträge offen und frei erstzuveröffentlichen (im
Gegensatz zur Green Road, der freien und offenen Zweitveröffentlichung
nach einer exklusiven Vermarktung der Beiträge durch
privatwirtschaftliche Verlage), wird \enquote{golden} mittlerweile
mehrheitlich so verstanden, dass privatwirtschaftliche Verlage nach
Zahlung einer Publikationsgebühr die offene und freie
Erstveröffentlichung der wissenschaftlichen Beiträge
organisieren.}\footnote{Joerg Braun: Ohne Gleichberechtigung und
  sozialen Ausgleich bleibt Open dicht. In: Digitale Linke (10.07.2012),
  \url{http://blog.die-linke.de/digitalelinke/ohne-gleichberechtigung-und-sozialen-ausgleich-bleibt-open-dicht/}
  (abgerufen 22.09.2017)} Eingefordert wird eine \enquote{faire{[}n{]}
Teilhabe der Fabrikbelegschaften, Konsument*innen oder anderen
Ideengeber*innen an den Ergebnissen.}\footnote{Joerg Braun: Ohne
  Gleichberechtigung und sozialen Ausgleich bleibt Open dicht. In:
  Digitale Linke (10.07.2012),
  \url{http://blog.die-linke.de/digitalelinke/ohne-gleichberechtigung-und-sozialen-ausgleich-bleibt-open-dicht/}
  (abgerufen 22.09.2017)}

\hypertarget{geistiges-eigentum}{%
\section*{\texorpdfstring{\enquote{Geistiges
Eigentum}}{``Geistiges Eigentum''}}\label{geistiges-eigentum}}

\enquote{The debate on the value of open access to publicly funded
research information is now migrating from 'whether' to
'how'.}\footnote{Leslie Chan, Barbara Kirsop, Subbiah Arunachalam: Open
  access archiving: the fast track to building research capacity in
  developing countries. In: SciDevNet (11.02.2005),
  \url{http://www.scidev.net/global/communication/feature/open-access-archiving-the-fast-track-to-building-r.html}
  (abgerufen 22.09.2017)} Die Open-Access-Bewegung hat eine Breite
erreicht, die eine Implementierung auf jeden Fall erforderlich macht,
unabhängig davon, ob damit eine Reform des Publikationsverhaltens
innerhalb der bürgerlichen Ordnung gemeint ist oder darüber hinaus
dieser Prozess als Teilbereich einer Transformation der
gesellschaftlichen Verhältnisse dienen soll.

Hinter der Debatte um Open Access steht die grundsätzliche Frage, wem
Wissen eigentlich gehört, und das führt schnurgerade zur Diskussion um
\enquote{geistiges Eigentum}, die eine zutiefst ethische und daher
natürlich auch politische ist.

Es gibt eine ganze Reihe von Gründen, die darauf hinweisen, dass
Institutionen zum Schutz sogenannten \enquote{geistigen Eigentums}

\begin{itemize}
\item
  das Wirtschaftswachstum insgesamt eher hemmen und daher nur ein
  suboptimales Wohlstandsniveau ermöglichen, da durch den Schutz von
  sogenanntem \enquote{geistigen Eigentum} nur eine suboptimale
  Allokation von Wissen zur Steigerung von Effektivität und Effizienz im
  Umgang mit knappen Gütern möglich ist;
\item
  zur Reproduktion oder gar Verstärkung sozialer Ungleichheit beitragen;
\item
  aufgrund der technischen Voraussetzungen, die zu ihrer Durchsetzung
  notwendig sind, antiliberale bzw. totalitäre Entwicklungen
  begünstigen.``\footnote{Eissler, Stephan: Eine Kritik der Institution
    des so genannten \enquote{geistigen Eigentums} im digitalen
    Zeitalter aus Perspektive liberaler Theorien. Vortrag auf der 3.
    Oekonux-Konferenz in Wien im Mai 2004,
    \url{http://dritte.oekonux-konferenz.de/dokumentation/texte/Eissler.pdf}
    , (abgerufen am 30.09.2017), S. 46}
\end{itemize}

Insgesamt lässt sich aus der Entwicklung der Schutzrechte für
\enquote{geistiges Eigentum} ablesen, dass es eine stetige Tendenz hin
zu den Rechten von Verwertungsorganisationen gibt, wohingegen die Rechte
der Urheber*innen, der Nutzer*innen und insgesamt der Öffentlichkeit
immer weniger Beachtung finden. Das im 18. Jahrhundert als gerechter
Ausgleich zwischen den verschiedenen Interessen von Urheber*innen,
Verwertern, Nutzer*innen und Öffentlichkeit angelegte
Kräfteparallelogramm ist aus dem Gleichgewicht geraten. Gerechtigkeit
ist aber eines der Grundthemen der Ethik, genauso wie die Ermöglichung
eines \enquote{guten Lebens} für alle, was hier sowohl philosophisch als
auch sozial-ökonomisch gemeint ist. Dabei ist aber -- wie bereits
erwähnt -- zu beachten, dass Open-Access-Strategien nicht eins zu eins
auf künstlerische Produktion umgelegt werden können. Hier muss vielmehr
das Urheber*innenrecht (\enquote{das Arbeitsrecht der
Kreativen}\footnote{Daniel Leisegang: Die Zukunft des Wissens Google
  Books, Open Access und die Informationsgesellschaft von morgen. In:
  Blätter für deutsche und internationale Politik 53 (11) 2009,
  \url{http://www.blaetter.de/archiv/jahrgaenge/2009/november/die-zukunft-des-wissens}
  (abgerufen 22.09.2017)}) so geändert werden, dass Künstler*innen
erstmals (!) in ihrer Gesamtheit die Chance auf ein \enquote{gutes
Leben} bekommen.

%autor
\begin{center}\rule{0.5\linewidth}{\linethickness}\end{center}

\textbf{Nikolaus Hamann}. Bibliothekar i.R., Vorstandsmitglied der
Vereinigung österreichischer Bibliothekarinnen und Bibliothekare (VÖB),
Koordinator des Arbeitskreises kritischer Bibliothekarinnen und
Bibliothekare (KRIBIBI), \url{http://www.kribibi.at}

\end{document}
