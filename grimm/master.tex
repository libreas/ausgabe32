\documentclass[a4paper,
fontsize=11pt,
%headings=small,
oneside,
numbers=noperiodatend,
parskip=half-,
bibliography=totoc,
final
]{scrartcl}

\usepackage{synttree}
\usepackage{graphicx}
\setkeys{Gin}{width=.4\textwidth} %default pics size

\graphicspath{{./plots/}}
\usepackage[ngerman]{babel}
\usepackage[T1]{fontenc}
%\usepackage{amsmath}
\usepackage[utf8x]{inputenc}
\usepackage [hyphens]{url}
\usepackage{booktabs} 
\usepackage[left=2.4cm,right=2.4cm,top=2.3cm,bottom=2cm,includeheadfoot]{geometry}
\usepackage{eurosym}
\usepackage{multirow}
\usepackage[ngerman]{varioref}
\setcapindent{1em}
\renewcommand{\labelitemi}{--}
\usepackage{paralist}
\usepackage{pdfpages}
\usepackage{lscape}
\usepackage{float}
\usepackage{acronym}
\usepackage{eurosym}
\usepackage[babel]{csquotes}
\usepackage{longtable,lscape}
\usepackage{mathpazo}
\usepackage[normalem]{ulem} %emphasize weiterhin kursiv
\usepackage[flushmargin,ragged]{footmisc} % left align footnote
\usepackage{ccicons} 

%%%% fancy LIBREAS URL color 
\usepackage{xcolor}
\definecolor{libreas}{RGB}{112,0,0}

\usepackage{listings}

\urlstyle{same}  % don't use monospace font for urls

\usepackage[fleqn]{amsmath}

%adjust fontsize for part

\usepackage{sectsty}
\partfont{\large}

%Das BibTeX-Zeichen mit \BibTeX setzen:
\def\symbol#1{\char #1\relax}
\def\bsl{{\tt\symbol{'134}}}
\def\BibTeX{{\rm B\kern-.05em{\sc i\kern-.025em b}\kern-.08em
    T\kern-.1667em\lower.7ex\hbox{E}\kern-.125emX}}

\usepackage{fancyhdr}
\fancyhf{}
\pagestyle{fancyplain}
\fancyhead[R]{\thepage}

% make sure bookmarks are created eventough sections are not numbered!
% uncommend if sections are numbered (bookmarks created by default)
\makeatletter
\renewcommand\@seccntformat[1]{}
\makeatother


\usepackage{hyperxmp}
\usepackage[colorlinks, linkcolor=black,citecolor=black, urlcolor=libreas,
breaklinks= true,bookmarks=true,bookmarksopen=true]{hyperref}

%meta
%meta

\fancyhead[L]{S. Grimm \& D. Schobert \\ %author
LIBREAS. Library Ideas, 32 (2017). % journal, issue, volume.
\href{http://nbn-resolving.de/}
{}} % urn 
% recommended use
%\href{http://nbn-resolving.de/}{\color{black}{urn:nbn:de...}}
\fancyhead[R]{\thepage} %page number
\fancyfoot[L] {\ccLogo \ccAttribution\ \href{https://creativecommons.org/licenses/by/3.0/}{\color{black}Creative Commons BY 3.0}}  %licence
\fancyfoot[R] {ISSN: 1860-7950}

\title{\LARGE{Open Access an der Universität verankern: Ein Praxisbericht aus dem Jahr 2017}} % title
\author{Steffi Grimm \& Dagmar Schobert} % author

\setcounter{page}{1}

\hypersetup{%
      pdftitle={Open Access an der Universität verankern: Ein Praxisbericht aus dem Jahr 2017},
      pdfauthor={Steffi Grimm \& Dagmar Schobert},
      pdfcopyright={CC BY 3.0 Unported},
      pdfsubject={LIBREAS. Library Ideas, 32 (2017).},
      pdfkeywords={Open Access},
      pdflicenseurl={https://creativecommons.org/licenses/by/3.0/},
      pdfcontacturl={http://libreas.eu},
      baseurl={http://libreas.eu},
      pdflang={de},
      pdfmetalang={de}
     }



\date{}
\begin{document}

\maketitle
\thispagestyle{fancyplain} 

%abstracts

%body
Anfang 2017 wurden die Open-Access-Beauftragte der Technischen
Universität Berlin (TU Berlin), Prof.~Vera Meyer\footnote{Prof.~Vera
  Meyer, Fachgebiet Angewandte und Molekulare Mikrobiologie, Fakultät
  III, TU Berlin. URL:
  \url{https://www.mikrobiologie.tu-berlin.de/menue/cv_prof_vera_meyer/}}
und die Universitätsbibliothek vom Präsidium beauftragt, den Entwurf
einer Open-Access-Policy in der Universität bekannt zu machen und einen
Diskussionsprozess über Open Access in Gang zu bringen. Was folgte waren
Vorträge in verschiedenen Gremien der Universität. Die Veranstaltungen
boten die Gelegenheit, den Entwurf der Policy vorzustellen und den
Argumenten für und gegen Open Access Raum zu geben. Der vorliegende
Beitrag dokumentiert diese Diskussionen und zeigt auf, welche Positionen
im Umfeld von Open Access für Studierende, Doktorand/innen, PostDocs und
Professor/innen der TU Berlin von Bedeutung sind. Ziel des
Diskussionsprozesses ist die Verabschiedung einer Open-Access-Policy,
welche die Interessen aller an der TU vertretenen Disziplinen
berücksichtigt und Open Access als strategisches Ziel der Universität
verankert\footnote{Die Open-Access-Policy der TU Berlin wurde am
  06.12.2017 durch den Akademischen Senat der Universität einstimmig
  verabschiedet \url{http://www.tu-berlin.de/?191164}. Die
  Unterzeichnung der Berlin Declaration on Open Access to Knowledge in
  the Sciences and Humanities durch die TU Berlin erfolgte 2016.}.

\hypertarget{hochschulpolitischer-rahmen}{%
\section*{Hochschulpolitischer
Rahmen}\label{hochschulpolitischer-rahmen}}

Vorhaben wie das Verankern eines strategischen Ziels der Universität und
die Verabschiedung einer Policy bedürfen einigen diplomatischen
Geschicks und erfordern zunächst die Auseinandersetzung mit
hochschulpolitischen Fragen: Wie sind die politischen Abläufe an der
Hochschule? Wie funktioniert akademische Selbstverwaltung in der Praxis?
Welche Statusgruppen sollten in die Diskussion einbezogen, welche
Gremien entsprechend berücksichtigt werden? Für die
Universitätsbibliothek war es wichtig, im Vorfeld einen Konsens mit dem
Präsidium herzustellen und den Diskussionsprozess über den Entwurf der
Open-Access-Policy ausdrücklich im Auftrag des Präsidiums der
Universität anzustoßen. Nachdem dieser Auftrag erteilt war, wurde
schließlich entschieden, den Diskussionsprozess auf Ebene der
Fakultätsräte anzusiedeln, ergänzt um den Allgemeinen
Studierendenausschuss (AStA) und ein Forum für internationale
Gastwissenschaftler/innen an der Universität. Mit der Gliederung der TU
Berlin in sieben Fakultäten\footnote{Geistes- und
  Bildungswissenschaften; Mathematik und Naturwissenschaften;
  Prozesswissenschaften; Elektrotechnik und Informatik; Verkehrs- und
  Maschinensysteme; Planen Bauen Umwelt; Wirtschaft und Management}
fanden insgesamt neun Veranstaltungen statt, bei denen die
Open-Access-Beauftragte jeweils kurze Vorträge zur Policy hielt und die
anschließenden Diskussionen moderierte. Anwesend waren stets auch der
Direktor der Universitätsbibliothek (UB) und Vertreterinnen des
Open-Access-Teams der UB. Die anwesenden Mitglieder der Gremien wurden
gebeten, zugleich als Multiplikator/innen zu fungieren, indem sie
proaktiv den Entwurf der Policy und die Debatten in die Fakultäten, die
Institute und die Studierendenschaft tragen sollten.

\hypertarget{aufgabenteilung-und-vorbereitung}{%
\section*{Aufgabenteilung und
Vorbereitung}\label{aufgabenteilung-und-vorbereitung}}

Seit 2016 ist Prof.~Vera Meyer Open-Access-Beauftragte der TU Berlin.
Mit ihr unterstützt eine Vertreterin der Wissenschaft das TU-Präsidium
bei strategischen Entscheidungen zu Open Access. Die Verantwortung für
die operativen Aufgaben liegt bei der Universitätsbibliothek: Jürgen
Christof übernimmt in seiner Funktion als Direktor der UB die
langfristige Zielsetzung und Planung, während das Open-Access-Team für
die Beratung von TU-Angehörigen und damit verbundene Dienstleistungen
verantwortlich ist. Analog gestaltete sich auch die Aufgabenteilung beim
Rundgang durch die Gremien der Universität: Die
inhaltlich-konzeptionellen Vorarbeiten erfolgten in enger Zusammenarbeit
aller Beteiligten, die Präsentation des Entwurfs der Open-Access-Policy
oblag der OA-Beauftragten, in die sich anschließenden Diskussionen waren
alle drei Akteure eingebunden.

Um die Rahmenbedingungen der verschiedenen Disziplinen im Vorfeld besser
einschätzen zu können, wurde in Vorbereitung auf die einzelnen Sitzungen
das Publikationsverhalten unter Berücksichtigung der unterschiedlichen
Publikationskulturen der Fächer betrachtet: Welchen Stellenwert haben
Journals, Konferenz- und Sammelbände sowie Monografien für einzelne
Disziplinen? Welche TU-Angehörigen sind bereits in Editorial Boards von
OA-Journals aktiv und/oder stellen ihre Publikationen auf dem grünen Weg
zur Verfügung? Wie sieht es mit der Gründung von OA-Journals an der TU
aus? Ziel der Betrachtung des Publikationsverhaltens war es,
OA-erfahrene TU-Angehörige in den Sitzungen direkt anzusprechen und
durch deren Erfahrungsberichte die Debatten um den Policy-Entwurf zu
bereichern.

31,2 Prozent der 2016 veröffentlichten Zeitschriftenartikel von
Wissenschaftler/innen der publikationsstärksten Einrichtungen in Berlin
sind laut einer aktuellen Erhebung frei verfügbar.\footnote{Voigt,
  Michaela; Winterhalter, Christian (2017): Wie viel Open Access steckt
  in Berlin? \url{https://doi.org/10.5281/zenodo.1035138}} Die Erhebung
machte deutlich, dass durchaus zahlreiche Wissenschaftler/innen der TU
Berlin bereits im Bereich Open Science beziehungsweise Open Access aktiv
sind.

\hypertarget{durchfuxfchrung-und-diskussion}{%
\section*{Durchführung und
Diskussion}\label{durchfuxfchrung-und-diskussion}}

Der Rundgang durch neun Gremien erfolgte zwischen April und Juli 2017
und sah einen Zeitraum von jeweils 60 bis 90 Minuten für Vortrag und
Diskussionen vor. \enquote{Was ist Open Access?}, \enquote{Warum
brauchen wir Open Access?} und \enquote{Was hindert uns an Open Access?}
-- die Präsentationen stellten zunächst grundlegende Rahmenbedingungen
vor, bevor auf Open-Access-Policies anderer Universitäten, die Förderung
von Mittelgebern und Wege der Umsetzung, inklusive das Projekt
DEAL\footnote{\href{https://www.projekt-deal.de/}{Projekt DEAL --
  Bundesweite Lizenzierung von Angeboten großer Wissenschaftsverlage};
  siehe auch \enquote{Aus dem DEAL-Maschinenraum -- ein Gespräch mit
  Bernhard Mittermaier} in dieser Ausgabe,
  \url{http://libreas.eu/ausgabe32/mittermaier/}} als Ansatz zur
großflächigen Open-Access-Transformation, eingegangen wurde.

Zusammenfassend lässt sich in Bezug auf die Diskussionen festhalten,
dass Open Access als Zukunft des Publizierens in allen Gremien anerkannt
wurde. Der Entwurf der Open-Access-Policy wurde grundsätzlich positiv
reflektiert. Mit Blick auf die Umsetzung des Systemwechsels wurden
jedoch kritische Stimmen laut: Fragen, Anregungen, Bedenken und
Kritikpunkte der Diskussionen lassen sich den sechs Kategorien
Transformationsprozess und Finanzierung von Open Access,
Publikationskulturen, Qualitätssicherung, Impact Factor als
Bewertungskriterium, Unterstützung durch Universitäten und
Forschungsförderer sowie Fragen zur praktischen Umsetzung zuordnen.

\hypertarget{transformationsprozess-und-finanzierung-von-open-access}{%
\subsection*{Transformationsprozess und Finanzierung von Open
Access}\label{transformationsprozess-und-finanzierung-von-open-access}}

\enquote{Warum sollte man aus moralischer Sicht Open-Access-Unternehmen
finanzieren, wenn sich der Publikationsmarkt durch das neue
Geschäftsmodell nicht signifikant ändert?} Mehrfach wurde in den
Veranstaltungen kritisch angemerkt, dass Verlage Open Access zunehmend
als gewinnträchtiges Geschäftsmodell entdecken. Gefordert wurden
nicht-kommerzielle Publikationsinfrastrukturen in der Hand der
Wissenschaft. Zugleich wurden mit Blick auf 345 Lehrstühle an der TU
Berlin Zweifel geäußert, wie realisierbar die Finanzierung von Gold Open
Access tatsächlich sei. Die Sorge, dass im neuen System Einzelne wegen
fehlender Mittel nicht mehr publizieren können, wurde immer wieder
thematisiert. Anerkannt wurde hingegen, dass der Transformationsprozess
von Seiten der Universität zeitweise höhere Ausgaben erfordern wird. In
nahezu allen Sitzungen wurde der Direktor der Universitätsbibliothek
schließlich auf den aktuellen Stand der DEAL-Verhandlungen angesprochen.
Die Universitätsbibliothek hatte die DEAL-Problematik auf mehreren Wegen
in die Universität getragen. Auch der Präsident der TU Berlin hatte sich
öffentlichkeitswirksam dazu geäußert.\footnote{Tagesspiegel
  (20.02.2017): Streit um teure Wissenschaftsjournale -- Showdown
  zwischen Bücherregalen
  (\url{http://www.tagesspiegel.de/wissen/streit-um-teure-wissenschaftsjournale-showdown-zwischen-buecherregalen/19412772.html});
  Berliner Zeitung (08.03.2017): Open Access -- Forscher der TU Berlin
  sollen ihre Ergebnisse im Netz veröffentlichen
  (\url{http://www.berliner-zeitung.de/berlin/open-access-forscher-der-tu-berlin-sollen-ihre-ergebnisse-im-netz-veroeffentlichen-26148112});
  Tagesspiegel (19.03.2017): Urheberrecht in der Wissenschaft --
  Studierende zurück in den Copyshop?
  (\url{http://www.tagesspiegel.de/wissen/urheberrecht-in-der-wissenschaft-studierende-zurueck-in-den-copyshop/19539196.html});
  Open-Access-Sonderseiten: Hochschulzeitung TU intern 2-3/2017
  (\url{http://www.ub.tu-berlin.de/fileadmin/pdf/Verlag/TUintern_Open_Access_20170217_pdf.pdf});
  Tagesspiegel (30.06.2017): Berliner Unis kündigen "Verlagsriesen"
  (\url{http://www.tagesspiegel.de/wissen/streit-um-teure-zeitschriften-abos-berliner-unis-kuendigen-verlagsriesen/20003678.html});
  Berliner Zeitung (06.07.2017): Berliner Universitäten kündigen Vertrag
  mit Wissenschaftsverlag Elsevier
  (\url{http://www.berliner-zeitung.de/berlin/hohe-preise-berliner-universitaeten-kuendigen-vertrag-mit-grossem-wissenschaftsverlag-27926974}).}
Es zeigte sich, dass der DEAL-Prozess mit seinem Ringen um die
Open-Access-Komponente und die landesweite Kooperation von Universitäten
und Hochschulen eine breite Unterstützung an der Universität hat.

Die Deckelung von Article Processing Charges (APC) bei 2.000 Euro nach
den Kriterien der DFG-Publikationsfonds wurde mit Sicht auf das
Gesamtsystem als strategisch nachvollziehbar und wünschenswert
eingeordnet. Dennoch gingen die Meinungen hier auseinander:
Vertreter/innen eines nicht-kommerziellen Open Access schätzten den
Betrag als wesentlich zu hoch ein und plädierten für Journals in der
Hand von Fachgesellschaften mit niedrigen Publikationsgebühren. Von
anderen Wissenschaftsdisziplinen wurde die Obergrenze als nicht
praktikabel kritisiert: Viele TU-Angehörige mit einem Interesse an Gold
Open Access konstatierten resigniert, dass die APC für den eigenen
Forschungsschwerpunkt bei mehreren tausend Euro starten. Auch die
Nichtfinanzierung von Artikeln in hybriden Journals\footnote{\enquote{Hybrid}
  nicht im Sinne von Print/Online, sondern im Sinne einer Zeitschrift,
  die sowohl \enquote{Closed-Access-Artikel} als auch
  Open-Access-Artikel bereitstellt. Hybride Geschäftsmodelle werden mit
  \enquote{double dipping} umschrieben, da sie die Etats der
  Bibliotheken (Lizenzgebühr) und der veröffentlichenden Universitäten
  (Publikationsgebühr) doppelt belasten. Beispiele für solche Modelle
  sind \enquote{Sponsored Article} (Elsevier), \enquote{Open Choice}
  (Springer) und \enquote{OnlineOpen} (Wiley).} wurde von diesen
Vertreter/innen kritisiert. Hier zeigte sich, wie viel Aufklärungsbedarf
hinsichtlich der Möglichkeiten des Green Open Access besteht. Analog zu
den Kriterien der DFG-geförderten Publikationsfonds lautete und lautet
die Botschaft des Open-Access-Teams ganz klar: Open-Access-Optionen in
hybriden Journals und APC über 2.000 Euro vermeiden und Open Access in
diesen Fällen stets über den grünen Weg realisieren, das
Open-Access-Team unterstützt dabei.

\hypertarget{publikationskulturen}{%
\subsection*{Publikationskulturen}\label{publikationskulturen}}

Unterschiedliche Fachkulturen bevorzugen bekanntermaßen verschiedene
Publikationsformen. An den sieben Fakultäten der TU sind dies neben
Journals insbesondere Konferenzbände und Workshop-Proceedings, ferner
Sammelbände und Monografien. Infrastrukturen und Unterstützungsangebote
für Gold Open Access sind in der Wahrnehmung von TU-Angehörigen
überwiegend auf Journale ausgerichtet. Angebote wie Language Science
Press\footnote{\url{http://langsci-press.org/}}, Knowledge
Unlatched\footnote{\url{http://www.knowledgeunlatched.org/}}, Open
Journal Systems\footnote{\url{https://pkp.sfu.ca/ojs/}} oder der
Universitätsverlag der TU Berlin\footnote{\url{http://www.ub.tu-berlin.de/publizieren/universitaetsverlag/}}
sind nicht durchgängig bekannt. In nahezu allen Gremien wurde ein Bedarf
an weiteren Unterstützungsangeboten formuliert, beispielsweise mit Blick
auf die Einrichtung eines Publikationsfonds für andere
Publikationsformen als Zeitschriftenartikel. Einzelne TU-Angehörige
naturwissenschaftlicher und technischer Disziplinen bereicherten die
Diskussionen um Erfahrungsberichte darüber, wie Editorial Boards von
Konferenz- und Sammelbänden erfolgreich ein Zweitveröffentlichungsrecht
als Bedingung für ihre Herausgeberschaft verhandeln können. Zugleich
wurde deutlich, dass in den Fakultäten Mathematik und
Naturwissenschaften bzw. Elektrotechnik und Informatik die
Preprint-Kultur besonders stark ausgeprägt ist und eine breite
Verankerung in der Forschungspraxis hat: Artikel werden hier
selbstverständlich als Preprints Open Access auf arXiv publiziert.

\enquote{Kleinverlage haben Angst!} und \enquote{Stoßen wir in Zukunft
auf den Upload einer PDF an?} -- in den überwiegend textbasiert
publizierenden Geistes- und Sozialwissenschaften wurden von Einzelnen
die Brisanz der Diskussionen für die eigene Community und die Sorge um
den Niedergang der Printkultur formuliert. Zugleich wurde kritisiert,
dass die Geistes- und Sozialwissenschaften in den letzten Jahren
zunehmend aufgefordert waren, sich an den Naturwissenschaften zu
orientieren und ihre Forschung zum Beispiel über die Zahl der Zitationen
bewerten zu lassen. Open Access wird aus dieser Perspektive als
Transformationsversuch von außen betrachtet, der auf elektronische
Publikationen ausgerichtet ist, jedoch nicht zur traditionellen
Printkultur dieser Fächer passt. Die Universitätsbibliothek nimmt diese
Sorgen ernst, weist jedoch darauf hin, dass Green Open Access, das heißt
Zweitveröffentlichungen, nach Embargofristen für Printpublikationen ein
veritabler Weg sind, um Open Access zu erreichen. Zugleich argumentiert
sie, dass es mit Blick auf elektronische Publikationen und Gold Open
Access letztlich darum geht, bestehende wissenschaftsfeindliche
Geschäftsmodelle wissenschaftsfreundlich zu gestalten und sich gerade
hier Chancen für kleinere Fachverlage bieten, die offen für
Veränderungen sind und im Austausch mit der Zielgruppe neue
Finanzierungsmodelle und Dienstleistungsportfolios entwickeln.
Vertreter/innen aus den Digital Humanities betonten abschließend die
Notwendigkeit von Open Access für die Anwendung neuer
Forschungsmethoden.

\hypertarget{qualituxe4tssicherung}{%
\subsection*{Qualitätssicherung}\label{qualituxe4tssicherung}}

Gold Open Access und den Aspekt der Qualitätssicherung betreffend wurde
in allen Fakultäten die Herausforderung deutlich, qualitativ hochwertige
Journale für den eigenen Forschungsschwerpunkt zu finden. Der
unüberschaubaren Anzahl neu gegründeter Open-Access-Journale stehen
viele reserviert gegenüber. Einige TU-Angehörige berichteten von beinahe
täglichen E-Mail-Anfragen mit der Bitte um Mitwirkung in Editorial
Boards, als Reviewer oder Submitter, sowie von langen
Peer-Review-Prozessen wegen fehlender Mitwirkung der jeweiligen
Fachcommunity. Einzelne Stimmen monierten, dass den Gutachter/innen für
das Reviewing von Artikeln in Open-Access-Zeit\-schrif\-ten nicht genug Zeit
zugestanden wird. Die Open-Access-Bewegung wurde in diesem Kontext
aufgefordert, nicht nur politisch zu agieren, sondern ganz praktisch
Qualitätssicherung (mit) zu organisieren. Ergänzt wurden die
Diskussionen durch Erfahrungsberichte von Editor/innen, die Neu- oder
Ausgründungen von Open-Access-Zeit\-schrif\-ten erfolgreich begleitet haben
und deren Impact Factor nach mehreren Jahren an etablierte und
renommierte Journals heran reicht.

Der Rundgang durch die Gremien machte sehr deutlich, dass weiterhin
Aufklärungsbedarf über das Angebot und die Auswahl geeigneter
Open-Access-Journale besteht: Einerseits sollte auf Angebote wie
Think-Check-Submit\footnote{\url{http://thinkchecksubmit.org/}} oder
Plattformen wie openaccess.net\footnote{\url{https://openaccess.net/}}
hingewiesen werden. Andererseits ist es hilfreich, auf die Möglichkeiten
von erfolgreichen Neu- und Ausgründungen von Journalen hinzuweisen und
Instrumente wie Open Journal Systems (OJS), ORCiD\footnote{\url{https://orcid.org/}},
BASE\footnote{\url{https://www.base-search.net/}} und
Unpaywall\footnote{\url{http://unpaywall.org/}} zu bewerben. Auf die
Erfahrungen von Wissenschaftler/innen an der eigenen Institution
verweisen zu können und die Akteure zu vernetzen, unterstützt die
Beratungsarbeit natürlich wesentlich.

\hypertarget{impact-factor-als-bewertungskriterium}{%
\subsection*{Impact Factor als
Bewertungskriterium}\label{impact-factor-als-bewertungskriterium}}

\enquote{Neugründungen eigener Open-Access-Journale sind möglich} und
\enquote{Professor/innen sind in der Verantwortung, Vorbild zu sein}.
Äußerungen wie diese waren beim Rundgang durch die Gremien an vielen
Stellen zu hören. Neben dem Aspekt der Qualitätssicherung wurde im
Hinblick auf Gold Open Access vor allem der Journal Impact Factor (JIF)
als Bewertungskriterium diskutiert. Abhängig von der Disziplin stellt
dieser häufig das wichtigste Argument bei der Entscheidung für oder
gegen eine Zeitschrift dar, da wissenschaftliche Wirksamkeit, Reputation
und Karriere in der Wissenschaft noch immer eng damit verbunden sind.
Viele Open-Access-Zeit\-schrif\-ten erreichen beim JIF (noch) keine hohen
Werte. TU-Angehörige mit Erfahrungen bei der Gründung von
Open-Access-Journalen berichteten jedoch, dass der offizielle JIF erst
ungefähr fünf bis sechs Jahre nach Journalgründung kalkuliert und
veröffentlicht werden kann. Da Open Access mit einer erhöhten
Sichtbarkeit und höherer Zitationsrate einhergeht (sofern rigoroses peer
reviewing sichergestellt wird), erzielen diese Journale sehr gute bis
exzellente JIF-Werte nach ungefähr zehn Jahren. Diskutiert wurde zudem,
inwiefern der Journal Impact Factor langfristig durch andere
Bewertungskriterien ersetzt werden kann.

\hypertarget{unterstuxfctzung-durch-universituxe4t-und-forschungsfuxf6rderer}{%
\subsection*{Unterstützung durch Universität und
Forschungsförderer}\label{unterstuxfctzung-durch-universituxe4t-und-forschungsfuxf6rderer}}

Im Zusammenhang mit Diskussionen um die Beurteilung wissenschaftlicher
Leistungen wurde auf Berufungs- und Leistungsbewertungsverfahren an der
Universität eingegangen, bei denen der Journal Impact Factor von
zentraler Bedeutung ist. Mehrere TU-Angehörige forderten nachdrücklich,
Open Access als Kriterium in Berufungsleitfäden aufzunehmen und bei der
Erfassung von Leistungsaktivitäten in Forschung und Lehre sowie der
damit verbundenen leistungsbezogenen Mittelverteilung zu
berücksichtigen. Auch von Förderern wird erwartet, eine klare
Open-Access-Policy zu verabschieden und Open Access etwa bei der
Bewertung von Projektanträgen einzubeziehen: \enquote{Solange ein
Artikel in einem traditionellen Journal von Förderern höher bewertet
wird, besteht kein Anreiz zur Publikation in Open-Access-Journalen.}
Einzelne Wissenschaftler/innen berichteten von der Erfahrung, dass in
Projektanträgen veranschlagte Publikationskosten häufig als erstes
gekürzt werden.

Auf operativer Ebene wurde vielfach der Wunsch nach Unterstützung durch
die Universitätsbibliothek geäußert. In Bezug auf Gold Open Access
betraf dies den Ausbau des bestehenden Publikationsfonds um
Fördermöglichkeiten für Conference Proceedings, Sammelbände und
Monografien. Dabei wurden die Förderkriterien diskutiert. Über den Fonds
hinaus besteht Interesse an technischen Infrastrukturen und an
Unterstützung bei der Herausgabe und Publikation digitaler
Open-Access-Zeit\-schrif\-ten. Der grüne Weg des Open Access war vielen
TU-Angehörigen nicht bekannt, Hinweise auf den
Zweitveröffentlichungsservice der Universitätsbibliothek\footnote{\url{https://www.ub.tu-berlin.de/publizieren/oa/erst-und-zweitveroeffentlichungen/services-fuer-zweitveroeffentlichungen/}.}
wurden entsprechend positiv aufgenommen. Die zu erwartende erhöhte
Nachfrage dieses Angebots ist mit der Anforderung an das
Open-Access-Team verbunden, mittelfristig adäquate und skalierbare
Workflows zu entwickeln.\footnote{Angedacht wird eine
  (Teil-)Automatisierung des bestehenden Workflows unter
  Berücksichtigung von Tools und Diensten wie OpenRefine
  (\url{http://openrefine.org/}), Crossref
  (\url{http://api.crossref.org}), oaDOI (\url{https://oadoi.org/}) und
  SHERPA/RoMEO (\url{http://sherpa.ac.uk/romeo}).}

\hypertarget{fragen-zur-praktischen-umsetzung}{%
\subsection*{Fragen zur praktischen
Umsetzung}\label{fragen-zur-praktischen-umsetzung}}

Praktische Fragen, die sich im Laufe des Rundgangs durch die Gremien
ergaben, richteten sich vor allem auf die Möglichkeiten der Umsetzung
von Gold und Green Open Access an der Universität. Großes Interesse
bestand beispielsweise an den Erfahrungen von TU-Angehörigen, die
bereits Open-Access-Journale gegründet haben. Insbesondere wurden hier
Fragen zum personellen Aufwand, zur technischen Umsetzung und zum
aktuellen Impact Factor gestellt. Die Universitätsbibliothek wiederum
wurde auf Dienstleistungen rund um Open Access angesprochen:
Förderbedingungen des Publikationsfonds, unverbindliche Rechtsberatung
für Verlagsverträge oder rechtliche Rahmenbedingungen von kumulativen
Dissertationen und Zweitveröffentlichungen waren in diesem Zusammenhang
Thema.

\hypertarget{fazit}{%
\section*{Fazit}\label{fazit}}

Die TU Berlin ist in Bezug auf die Verabschiedung einer
Open-Access-Policy im Vergleich zu anderen Universitäten und Hochschulen
spät dran: Sicher liegt hierin ein Grund, warum der Entwurf der Policy
auf überwiegend positive Rückmeldungen, Sympathien und Unterstützung
stieß. Als vorteilhaft erwies sich eindeutig die Beteiligung der
Open-Access-Beauftragten aus den Reihen der Wissenschaft, die mit
entsprechendem Standing bei den Forschenden und ihrer Identifikation mit
dem Thema Open Access Argumentationen auf Augenhöhe liefern konnte, die
anders klingen als die üblichen bibliothekarischen
Erläuterungen.\footnote{Lektüreempfehlung: Cirasella, Jill (2017): Open
  access outreach. SMASH vs.~Suasion. URL:
  \url{http://crln.acrl.org/index.php/crlnews/article/view/16681/18150}.}
Ergebnis des Prozesses ist eine grundsätzliche Zustimmung zur Policy,
aber auch die Aufnahme eines zusätzlichen Punktes\footnote{\enquote{Die
  TU Berlin bittet alle Universitätsangehörigen, ihre Mitarbeit bei der
  Begutachtung, Redaktion und Herausgabe von Publikationen hinsichtlich
  der jeweiligen Open-Access-Politik zu überdenken, über ihre Funktion
  auf Verlage und Fachgesellschaften einzuwirken und nach Möglichkeit
  ihre Mitarbeit bevorzugt Open-Access-Publikationen zukommen zu lassen.
  Das Engagement für nicht-kommerzielle Angebote wird besonders
  befürwortet.}}, der insbesondere den geäußerten Bedenken zur
fortschreitenden Kommerzialisierung des wissenschaftlichen
Publikationsmarktes Rechnung tragen soll.

Der Rundgang in den Gremien der Universität hat gezeigt, dass
Universitätsbibliotheken mit Unterstützung der Präsidien einen breiten
Diskussionsprozess über Open Access an der eigenen Institution
initiieren können. Open Access ist ein wichtiges Thema für Bibliotheken,
vor allem aber ist es ein Thema, das die Wissenschaft besetzen muss.
Universitätsbibliotheken können hierfür Motor sein, Open Access an den
richtigen Stellen auf die Agenda bringen und sich als kompetente
Ansprechpartnerinnen rund um das Thema Publizieren präsentieren.

%autor
\begin{center}\rule{0.5\linewidth}{\linethickness}\end{center}

\textbf{Steffi Grimm} studierte Bibliotheks- und
Informationswissenschaft (M.A.) an der Humboldt-Universität zu Berlin
und arbeitet im Open-Access-Team der Technischen Universität Berlin
(ORCiD: \url{https://orcid.org/0000-0001-5055-9492}).

\textbf{Dagmar Schobert} ist Leiterin der Abteilung
Universitätsverlag/Hochschulschriften/Open Access der
Universitätsbibliothek der Technischen Universität Berlin (ORCiD:
\url{https://orcid.org/0000-0002-1792-3077}).

\end{document}
