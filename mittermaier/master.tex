\documentclass[a4paper,
fontsize=11pt,
%headings=small,
oneside,
numbers=noperiodatend,
parskip=half-,
bibliography=totoc,
final
]{scrartcl}

\usepackage{synttree}
\usepackage{graphicx}
\setkeys{Gin}{width=.4\textwidth} %default pics size

\graphicspath{{./plots/}}
\usepackage[ngerman]{babel}
\usepackage[T1]{fontenc}
%\usepackage{amsmath}
\usepackage[utf8x]{inputenc}
\usepackage [hyphens]{url}
\usepackage{booktabs} 
\usepackage[left=2.4cm,right=2.4cm,top=2.3cm,bottom=2cm,includeheadfoot]{geometry}
\usepackage{eurosym}
\usepackage{multirow}
\usepackage[ngerman]{varioref}
\setcapindent{1em}
\renewcommand{\labelitemi}{--}
\usepackage{paralist}
\usepackage{pdfpages}
\usepackage{lscape}
\usepackage{float}
\usepackage{acronym}
\usepackage{eurosym}
\usepackage[babel]{csquotes}
\usepackage{longtable,lscape}
\usepackage{mathpazo}
\usepackage[normalem]{ulem} %emphasize weiterhin kursiv
\usepackage[flushmargin,ragged]{footmisc} % left align footnote
\usepackage{ccicons} 

%%%% fancy LIBREAS URL color 
\usepackage{xcolor}
\definecolor{libreas}{RGB}{112,0,0}

\usepackage{listings}

\urlstyle{same}  % don't use monospace font for urls

\usepackage[fleqn]{amsmath}

%adjust fontsize for part

\usepackage{sectsty}
\partfont{\large}

%Das BibTeX-Zeichen mit \BibTeX setzen:
\def\symbol#1{\char #1\relax}
\def\bsl{{\tt\symbol{'134}}}
\def\BibTeX{{\rm B\kern-.05em{\sc i\kern-.025em b}\kern-.08em
    T\kern-.1667em\lower.7ex\hbox{E}\kern-.125emX}}

\usepackage{fancyhdr}
\fancyhf{}
\pagestyle{fancyplain}
\fancyhead[R]{\thepage}

% make sure bookmarks are created eventough sections are not numbered!
% uncommend if sections are numbered (bookmarks created by default)
\makeatletter
\renewcommand\@seccntformat[1]{}
\makeatother


\usepackage{hyperxmp}
\usepackage[colorlinks, linkcolor=black,citecolor=black, urlcolor=libreas,
breaklinks= true,bookmarks=true,bookmarksopen=true]{hyperref}
%URLs hart brechen
\makeatletter 
\g@addto@macro\UrlBreaks{ 
  \do\a\do\b\do\c\do\d\do\e\do\f\do\g\do\h\do\i\do\j 
  \do\k\do\l\do\m\do\n\do\o\do\p\do\q\do\r\do\s\do\t 
  \do\u\do\v\do\w\do\x\do\y\do\z\do\&\do\1\do\2\do\3 
  \do\4\do\5\do\6\do\7\do\8\do\9\do\0} 
% \def\do@url@hyp{\do\-} 
\makeatother 

%meta
%meta

\fancyhead[L]{B. Mittermaier \\ %author
LIBREAS. Library Ideas, 32 (2017). % journal, issue, volume.
\href{http://nbn-resolving.de/}
{}} % urn 
% recommended use
%\href{http://nbn-resolving.de/}{\color{black}{urn:nbn:de...}}
\fancyhead[R]{\thepage} %page number
\fancyfoot[L] {\ccLogo \ccAttribution\ \href{https://creativecommons.org/licenses/by/3.0/}{\color{black}Creative Commons BY 3.0}}  %licence
\fancyfoot[R] {ISSN: 1860-7950}

\title{\LARGE{Aus dem DEAL-Maschinenraum -- ein Gespräch mit Bernhard Mittermaier}} % title
\author{Bernhard Mittermaier} % author

\setcounter{page}{1}

\hypersetup{%
      pdftitle={Aus dem DEAL-Maschinenraum -- ein Gespräch mit Bernhard Mittermaier},
      pdfauthor={Bernhard Mittermaier},
      pdfcopyright={CC BY 3.0 Unported},
      pdfsubject={LIBREAS. Library Ideas, 32 (2017).},
      pdfkeywords={Open Access, DEAL, Transformation, Elsevier, Springer Nature, Wiley},
      pdflicenseurl={https://creativecommons.org/licenses/by/3.0/},
      pdfcontacturl={http://libreas.eu},
      baseurl={http://libreas.eu},
      pdflang={de},
      pdfmetalang={de}
     }



\date{}
\begin{document}

\maketitle
\thispagestyle{fancyplain} 

%abstracts

%body
Was vor wenigen Jahren nicht vorstellbar war (jedenfalls nicht aus Sicht
von OA-Aktivist*innen und womöglich auch nicht aus Sicht von
Erwerbungsleiter*innen), ist seit gut zwei Jahren in aller Munde.
Universitätsleitungen, Wissenschaftler*innen, Bibliothekar*innen,
Wissenschaftsjournalist*innen -- wer an \enquote{DEAL} denkt, denkt
nicht mehr primär an den \enquote{New Deal} aus den USA der 1930er oder
eine TV-Spielshow aus den frühen 2000ern. Sondern an ein Projekt, das
für ganz Deutschland Lizenzverträge für den Zugriff auf
Verlagspublikationen abschließen will -- zunächst mit den drei
Platzhirschen Elsevier, Springer Nature und Wiley, und später auch
weiteren Wissenschaftsverlagen.

Was ist besonders daran? Der Vertrag soll nicht nur den Zugang zu
wissenschaftlichen Zeitschriften regeln und öffentlich einsehbar
bepreisen, sondern auch eine Open-Access-Komponente für alle
\enquote{deutschen} Artikel beinhalten, das heißt Publizierende der an
DEAL teilnehmenden Einrichtungen können ihren Beitrag unter den
Bedingungen von Open Access veröffentlichen. Publish and Read (PAR), ein
Vertrag mit einem transparenten Preis. Ein großer Schritt in Sachen
Open-Access-Transformation des Zeitschriftenmarktes.

In der nationalen und internationalen Öffentlichkeit sind vor allem die
(zähen) Verhandlungen mit Elsevier über die Subskriptionsgebühren
präsent. Seit Anfang 2017 befinden sich 76 Einrichtungen in einem
vertragslosen Zustand. Konkret heißt das für die Forschenden der
Einrichtungen, dass sie nicht mehr direkt auf Elsevier-Zeitschriften
zugreifen können.\footnote{Zwischenzeitlich hat Elsevier allerdings bei
  den meisten Einrichtungen die Zugänge wieder freigeschaltet. Vgl.
  \url{http://www.sciencemag.org/news/2017/02/elsevier-journals-are-back-online-60-german-institutions-had-lost-access}}
Der von einigen befürchtete Aufstand aus der Wissenschaft blieb
aus.\footnote{Vgl. z.B.
  \url{http://www.nature.com/news/german-scientists-regain-access-to-elsevier-journals-1.21482},
  \url{https://www.helmholtz.de/aktuell/presseinformationen/artikel/artikeldetail/helmholtz_zentren_kuendigen_die_vertraege_mit_elsevier/}
  oder
  \url{https://www.berliner-zeitung.de/berlin/hohe-preise-berliner-universitaeten-kuendigen-vertrag-mit-grossem-wissenschaftsverlag-27926974}}
Bis Ende Oktober 2017 haben 109 wissenschaftliche Einrichtungen
angekündigt, einen individuellen Vertrag mit Elsevier zum Jahresende
nicht zu erneuern. Für insgesamt 185\footnote{Ohne Vertrag sind seit
  Anfang 2017 insgesamt 76 Einrichtungen: 30 Universitäten, 16
  Hochschulen, 27 Forschungseinrichtungen sowie 3 Regionalbibliotheken.
  Insgesamt 109 Einrichtungen haben bislang angekündigt, ihre
  Einzelverträge mit Elsevier über 2017 hinaus nicht zu verlängern,
  darunter 29 Universitäten, 57 Hochschulen und 23
  Forschungseinrichtungen. Vgl.
  \url{https://www.projekt-deal.de/vertragskundigungen-elsevier-2017/}
  (Stand 12.11.2017)} Einrichtungen gilt ab Januar 2018 also: Entweder
es gibt einen deutschlandweiten DEAL, oder es gibt keinen Zugang zu
Elsevier-Zeitschriften.

Über die Ziele des DEAL-Projektes und das (Nicht-)Vorankommen der
Verhandlungen wurde bereits viel geschrieben.\footnote{Siehe
  Pressespiegel auf der Projektwebseite
  \url{https://www.projekt-deal.de/pressespiegel/}} Doch uns
interessiert, wie man ein solches Projekt mit einer derartigen Relevanz
für die derzeitige und zukünftige Publikationslandschaft in der
deutschen Wissenschaft konkret organisiert. Welche Räder müssen
ineinandergreifen, damit es (hoffentlich erfolgreich) abgeschlossen
wird? Welche Anforderungen stellen die Verhandlungen an die beteiligten
Personen und Einrichtungen? Wir haben mit Bernhard Mittermaier als
Mitglied des DEAL-Verhandlungsteams gesprochen. Die Fragen stellten
Michaela Voigt und Maxi Kindling.

\hypertarget{chronologie-und-team}{%
\section*{Chronologie und Team}\label{chronologie-und-team}}

\emph{\textbf{LIBREAS}: Wann kam die Idee zum DEAL-Projekt auf? Gab es
einen bestimmten Zeitpunkt, der als Startpunkt angesehen werden kann?}

\textbf{BM}: Im Sommer 2013 trat die Rektorin der Universität Leipzig an
die Hochschulrektorenkonferenz (HRK) mit dem Vorschlag heran,
Lizenzverträge auf nationaler Ebene mit den großen Zeitschriftenverlagen
zu verhandeln\footnote{\url{http://www.tagesspiegel.de/wissen/teure-fachzeitschriften-nationallizenzen-fuer-uni-bibliotheken-gefordert/8624114.html}}.
Die HRK trug das Anliegen in die Allianz der
Wissenschaftsorganisationen, die schließlich die AG Lizenzen mit der
Erstellung einer Expertise beauftragt. In dieser sollte ausgeführt
werden unter welchen Rahmenbedingungen auch Angebote großer
Wissenschaftsverlage national lizenziert werden können. Seitens der AG
Lizenzen nahmen sich Anne Lipp (DFG), Hildegard Schäffler (BSB München)
und ich selbst des Themas an, da wir seinerzeit die AG Lizenzen
gemeinsam geleitet haben. Auf Grundlage dieser Expertise hat die Allianz
der Wissenschaftsorganisationen dann eine Projektgruppe eingesetzt, die
mit einigen Ergänzungen bis heute besteht\footnote{\url{https://www.projekt-deal.de/aktuelles/}}.
Zum Projektteam kamen später ein Projektlenkungsausschuss sowie ab 2016
das Verhandlungsteam hinzu. Seit 2015 werden außerdem zwei
Projektstellen finanziert.

\emph{\textbf{LIBREAS}: Das klingt relativ geradlinig. Wer
Bibliothekswesen und die Hochschullandschaft in Deutschland kennt, ahnt
jedoch, dass dies kein einfaches Unterfangen ist. Was musste
zusammenkommen, damit aus der Idee ein Projekt wurde?}

\textbf{BM}: Essentiell ist die Bereitschaft von sehr vielen
Einrichtungen aus dem Hochschulbereich ebenso wie aus dem
außeruniversitären Bereich zu zentralen Verhandlungen in einem ganz
neuen Kontext und auch, diese Verhandlungen mit der nötigen Härte zu
führen. Und diese Bereitschaft musste sowohl auf Leitungsebene als auch
in der Wissenschaft und bei den Bibliotheken vorhanden sein.

\textbf{LIBREAS:} \emph{Das Team ist ja durchaus heterogen in der
Zusammenstellung. Zufall oder Absicht?}

\textbf{BM}: Das Team als solches besteht aus drei Gruppen -- dem
eigentlichen Projektteam, dem Verhandlungsteam und dem
Projektlenkungsausschuss.\footnote{Siehe Abbildung
  \enquote{Projektstruktur} unter
  \url{https://www.projekt-deal.de/aktuelles/}} Bei der Zusammensetzung
aller Gremien wurde darauf geachtet, dass Bibliotheken und Wissenschaft,
die verschiedenen Fachdisziplinen und verschiedene Einrichtungstypen
angemessen vertreten sind. Die breiteste Repräsentanz aller Perspektiven
ist im Projektlenkungsausschuss gegeben. Im Projektteam sind
ausschließlich Bibliothekar*innen tätig, im Verhandlungsteam dominiert
die Wissenschaft. Um ein Bild aus der Seefahrt zu verwenden: Im
Maschinenraum sind Bibliothekar*innen, auf der Brücke sind
Wissenschaftler*innen.

\emph{\textbf{LIBREAS}: Und sicher tauschen Sie sich auch mit
Kolleg*innen aus dem Ausland aus?}

\textbf{BM}: Es gibt natürlich einen gewissen Austausch, zum Beispiel
wenn man sich bei Konferenzen begegnet. Die Verhandlungen werden jedoch
getrennt geführt.

\emph{\textbf{LIBREAS}: Es gibt zwei Projektstellen -- \enquote{nur}
oder \enquote{immerhin}, wie man es nimmt. Arbeiten alle anderen quasi
im Rahmen ihrer sonstigen dienstlichen Pflichten im DEAL-Team mit? Wie
viel Zeit investieren Sie beispielsweise in das Projekt?}

\textbf{BM}: DEAL bindet im Projekt- und Verhandlungsteam massiv
Ressourcen. Bei mir persönlich schätze ich es auf nicht unter 20 Stunden
pro Woche.

\hypertarget{die-verlage}{%
\section*{Die Verlage}\label{die-verlage}}

\emph{\textbf{LIBREAS}: Wie wurde die Initiative DEAL seitens der
Verlage aufgenommen?}

\textbf{BM}: Die Verlage versicherten sich zunächst, dass DEAL
tatsächlich das Mandat der wissenschaftlichen Einrichtungen in
Deutschland zur Verhandlung hat. Nachdem dies geklärt war, waren sie für
Verhandlungen offen.

\emph{\textbf{LIBREAS}: Können Sie grob schätzen, wie viele Personen auf
Seite der Verlage mit DEAL beschäftigt sind?}

\textbf{BM}: Bei Verhandlungen sind auf Verlagsseite im Schnitt fünf
Personen anwesend, stets der Verkauf in mindestens zwei Hierarchieebenen
sowie dezidierte Expert*innen, auch aus dem Bereich Open Access. Wie
viele Personen im Hintergrund tätig sind, können wir nicht abschätzen.

\emph{\textbf{LIBREAS}: Aus Presseberichten lässt sich erahnen, dass mit
harten Bandagen gekämpft wird. Wie würden Sie das Auftreten der
Verlagsvertreter*innen beschreiben -- sowohl während der
Verhandlungssitzungen als auch in der Öffentlichkeit?}

\textbf{BM}: Insgesamt ist das Auftreten korrekt und professionell. Dem
steht nicht entgegen, dass man sich in den Verhandlungen nichts schenkt.
In der Außenkommunikation haben wir sowohl mit Wiley als auch mit
Springer Nature ein gutes gemeinsames Verständnis gefunden: Wir stimmen
Informationen an Einrichtungen und den Handel vor Veröffentlichung
miteinander ab; inzwischen gab es ja sogar gemeinsame
Pressemitteilungen. Details werden nur äußerst zurückhaltend
kommuniziert. Das befriedigt zwar nicht das verständliche
Informationsbedürfnis der Öffentlichkeit. Es hilft aber die
Verhandlungen ohne größere Störungen weiterzuführen.

Bei Elsevier ist die Situation schwieriger: Wir hatten zwar auch hier
vereinbart, keine Details nach außen zu kommunizieren. Elsevier rückte
in seiner Kommunikation mit Einrichtungen und dann sogar mit
Editor*innen peu à peu davon ab. Inzwischen wird das vorliegende Angebot
ziemlich offen dargelegt, allerdings ohne manche \enquote{schmutzige
Details} und vor allem ohne dazu die finanziellen Vorstellungen zu
nennen.

\emph{\textbf{LIBREAS}: Wie ist die allgemeine Resonanz in der
(nationalen und internationalen) Verlagslandschaft? Stehen Sie bereits
mit weiteren Verlagen in Kontakt?}

\textbf{BM}: Andere Verlage verfolgen das Projekt mit größtem Interesse.
Es ist ja gut nachvollziehbar, dass Abschlüsse mit den drei größten
Verlagen in Deutschland, die über die Hälfte des Marktes auf sich
vereinen, sowohl Auswirkungen auf andere Verlage in Deutschland als auch
für das internationale Geschäft der Verlage hätten. Am Anfang stand auch
eine kartellrechtliche Beschwerde des Börsenvereins. In dieser kam die
Sorge zum Ausdruck, dass der Beitritt zu einem DEAL-Vertrag den gesamten
Etat auffressen würde und Bibliotheken kein Geld mehr für andere Verlage
hätten. Dem lag offenbar die Ansicht zugrunde, man werde für einen
DEAL-Vertrag mehr ausgeben müssen als bisher. Das Bundeskartellamt ist
dieser Logik nicht gefolgt, scheinbar unterliegen aber einige Verlage
dieser Propaganda des eigenen Verbandes: Sie wollen nun auch
DEAL-Verhandlungen führen, um etwas vom (vermeintlich größeren) Kuchen
abzubekommen. Das DEAL-Verhandlungsteam kann derzeit schon allein aus
Kapazitätsgründen nicht noch mit weiteren Verlagen Verhandlungen führen.
Es laufen aber einige Verhandlungen im Rahmen der DFG-Ausschreibung
\enquote{Open-Access-Transformationsverträge}. Diese werden von
einzelnen Verhandlungsführern analog zu den Allianz-Lizenzen verhandelt.

\hypertarget{vorbereitung-und-verhandlung}{%
\section*{Vorbereitung und
Verhandlung}\label{vorbereitung-und-verhandlung}}

\emph{\textbf{LIBREAS}: Wie ist die DEAL-Geschäftsstelle organisiert?}

\textbf{BM}: Eine Geschäftsstelle im eigentlichen Sinn gibt es nicht.
Die Allianz der Wissenschaftsorganisationen finanziert zwei
Projektstellen: Eine zur Datenerhebung und -analyse, angesiedelt bei der
MPDL und eine zur Kommunikation und Öffentlichkeitsarbeit, angesiedelt
an der UB Freiburg. Weitere Unterstützung haben wir durch die
Geschäftsstelle des Konsortiums Baden-Württemberg, die ebenfalls an der
UB Freiburg angesiedelt ist.

\emph{\textbf{LIBREAS}: Das DEAL-Team bereitet sich auf eine neue
Verhandlungsrunde vor. Wie kann man sich das vorstellen -- gibt es eine
Flut an E-Mails, Papieren, \ldots{}?}

\textbf{BM}: Anfangs gab es viele Treffen, physisch und per
Videokonferenz, beispielsweise um die DEAL-Verhandlungsziele
festzulegen. Inzwischen findet die Vorbereitung vor allem per E-Mail und
dann unmittelbar vor Verhandlungsterminen in einer Vorbesprechung statt.
Das Verhandlungsteam ist mittlerweile sehr gut eingespielt.

\emph{\textbf{LIBREAS}: Wie kommen Verhandlungen organisatorisch
zustande?}

\textbf{BM}: Das Büro von Professor Hippler macht Terminvorschläge, die
dann in der Verhandlungsgruppe koordiniert, also gedoodelt, werden. Wir
versuchen immer möglichst viele Wissenschaftler*innen dabei zu haben und
mindestens eine*n Bibliothekar*in. Dieses Ergebnis wird dann mit den
Verlagen abgestimmt. Dass die Terminfindung nicht einfach ist, kann man
sich vorstellen, auch weil bei den Verlagen Personen aus den
Niederlanden, Großbritannien und zum Teil USA involviert sind.

\emph{\textbf{LIBREAS}: Wie kann man sich die Stimmung im
Verhandlungsraum vorstellen?}

\textbf{BM}: Die Sitzungen als solche dauern zwischen zwei und vier
Stunden und finden in der Regel in den Räumen der
Hochschulrektorenkonferenz in Bonn oder Berlin statt. Der
Verhandlungsführer auf DEAL-Seite ist Professor Hippler, der Präsident
der HRK. Das Verhandlungsteam besteht -- wie schon erwähnt -- aus
Wissenschaftler*innen und Bibliothekar*innen und ist sehr gut
eingespielt. Sicher, die Stimmung ist häufiger angespannt als gelöst,
oft fallen sehr deutliche Worte. Aber einen Handschlag hat es bislang
immer gegeben -- auch bei der Verabschiedung.

\emph{\textbf{LIBREAS}: Wie häufig fällt der Begriff \enquote{SciHub}
während einer solchen Sitzung?}

\textbf{BM}: Inzwischen nur noch selten. Die Verlage wissen, dass wir
ausschließlich an einem Vertragsabschluss interessiert sind und nicht
daran, am Ende zwischen einem DEAL-Vertrag und Sci-Hub zu wählen. Falls
aber umgekehrt behauptet wird, die Einrichtungen könnten ohne die
Zeitschriften des jeweiligen Verlags nicht leben, dann weisen wir darauf
hin, dass die Erfahrung der Aussteiger bei Elsevier etwas Anderes lehrt:
Sie nutzen verschiedene legale Wege der alternativen Dokumentlieferung
und können so die Informationsversorgung ihrer Einrichtung
sicherstellen.

\hypertarget{kritik}{%
\section*{Kritik}\label{kritik}}

\emph{\textbf{LIBREAS}: Sie verhandeln für DEAL mit klassisch
ausgerichteten Verlagen. Was hat das eigentlich mit Open Access zu tun?
}

\textbf{BM}: Inzwischen jede Menge. Sicher, das ursprüngliche Anliegen
bezog sich lediglich auf die Verhandlung von Subskriptionsverträgen.
Aber alle Beteiligten realisierten sehr schnell, dass es damit nicht
getan sein könne. Ziel der Verhandlungen ist neben dem Zugriff auf alle
Zeitschriften des Verlags für alle teilnehmenden Einrichtungen auch die
Publikation aller Artikel von Wissenschaftler*innen der teilnehmenden
Einrichtungen, sofern sie Corresponding Author sind, im Open Access und
mit CC BY unter einer Open-Access-konformen Lizenz. Darüber hinaus soll
die Finanzierung nicht nur fair und zukunftsorientiert sein, sondern
sich auch an der Zahl der Publikationen orientieren. Mit anderen Worten:
Das DEAL-Projekt bringt die teilnehmenden Einrichtungen bei allen
Subskriptionszeitschriften der betroffenen Verlage annähernd in eine
Position als wären die Zeitschriften Open-Access-Zeitschriften: Die
eigenen Artikel sind Gold-OA und stehen unter CC BY. Andere Artikel kann
man lesen, allerdings nicht CC-gemäß nachnutzen. Für das Publizieren im
Goldenen Open Access fallen keine weiteren Gebühren an. Und gleiches
gilt auch für alle Gold-Open-Access-Zeitschriften der Verlage. Sie
werden in die DEAL-Verträge einbezogen und auch für diese müssen die
Autor*innen keine Publikationsgebühren bezahlen.

\emph{\textbf{LIBREAS}: OA-Aktivist*innen kritisieren, dass DEAL zur
weiteren Kommerzialisierung des wissenschaftlichen Publikationsmarktes
führt und die \enquote{Großen} bevorzugt -- während
Grassroots-Initiativen und neu gegründete OA-Verlage, insbesondere im
Bereich OA-Monographien, immer wieder in Finanznot geraten. Wie stehen
Sie dazu?}

\textbf{BM}: DEAL will erklärtermaßen einen Beitrag zur Transformation
des Publikationswesens in den Open Access beitragen. Zugegeben, die
DEAL-Verhandlungen im engeren Sinn sind auf diese drei Verlage
beschränkt. Es finden aber auch mit anderen Verlagen Gespräche im
Zusammenhang der DFG-Ausschreibung
\enquote{Open-Access-Transformationsverträge} statt. Alle Verlage, die
bereit sind, sich auf den Weg zur Transformation zu begeben, sind
eingeladen sich zu beteiligen. Im Übrigen maßt sich DEAL nicht an, in
die im Grundgesetz gesicherte Freiheit von Forschung und Lehre
einzugreifen. Wir wollen Wissenschaftler*innen auch nicht im Ansatz
vorschreiben, wo sie publizieren sollen, sondern wir wollen die frei
wählbaren Publikationsorte wissenschaftsadäquat gestalten. Aus unserer
Sicht gehört dazu, dass sie Open Access werden, wenn sie es nicht schon
sind.

\hypertarget{prognosen}{%
\section*{Prognosen}\label{prognosen}}

\emph{\textbf{LIBREAS}: Wann kommt der DEAL?}

\textbf{BM}: Die Gespräche mit Springer Nature und Wiley sind auf einem
guten Weg. Als sich im September 2017 abzeichnete, dass bis Jahresende
kein Abschluss möglich war, musste zur Vermeidung eines vertragslosen
Zustands Anfang 2018 eine Übergangslösung gefunden werden.\footnote{\url{https://www.projekt-deal.de/vertragskundigungen-elsevier-2017/}}
Die war mit beiden Verlagen in hochkonzentrierter Arbeit in einigen
Abstimmungen per E-Mail, in Telefonkonferenzen und zuletzt bei der
Buchmesse möglich. Die gute Atmosphäre in diesen Gesprächen und die nun
erstmals vertraglich manifest gewordene Bereitschaft der Verlage sich
auf den DEAL-Weg einzulassen, stimmen mich optimistisch, dass ein
Abschluss erreicht werden kann, wohl im ersten oder zweiten Quartal
2018.

Bei Elsevier ist die Situation deutlich anders. Die Verhandlungen dauern
zwar schon fast ein Jahr länger, sind aber weniger weit fortgeschritten
als bei Springer Nature und Wiley. Elsevier hat den Ansatz,
ausschließlich das Publizieren zu bezahlen, bisher nicht akzeptiert.
Auch ist die wechselseitige Aufstellung wesentlich offensiver:
Einrichtungen haben ihre auslaufenden Verträge nicht verlängert. Anfang
2017 waren dies knapp 70 Einrichtungen, inzwischen sind es über 180.
Anfang Oktober wurden die ersten Rücktritte von Editor*innen an Elsevier
übergeben und veröffentlicht.\footnote{\url{https://www.projekt-deal.de/herausgeber_elsevier/}}
Elsevier wiederum sucht das Gespräch mit einzelnen Einrichtungen --
obwohl gar keine Verhandlungen anstehen. Man schreibt Editor*innen an
und lädt zu \enquote{Editors Dinners} ein, sucht auch den Kontakt zu
Rektoraten und zu Ministerien. Mutmaßlicher Zweck ist das Aufbrechen der
Front auf Seiten von DEAL -- bislang ohne Erfolg. Man kann hoffen, dass
die Zwischenlösungen mit Wiley und Springer Nature, die in der
Öffentlichkeit auf großes Interesse gestoßen sind, bei Elsevier Bewegung
auslösen. Sollte dies nicht geschehen, werden weitere
Eskalationsschritte folgen: Weitere Einrichtungen werden kündigen, in
regelmäßigen Abständen werden Herausgeber*innen ihre Tätigkeit
niederlegen, eventuell wird DEAL das letzte Angebot von Elsevier den
Einrichtungen auch hinsichtlich der finanziellen Details mitteilen.
Allerspätestens wenn die Verträge mit Wiley und Springer Nature
abgeschlossen sind, wird Elsevier Farbe bekennen müssen. Sollte dann
immer noch keine Bewegung festzustellen sein, muss man davon ausgehen,
dass Elsevier lieber auf die Umsätze in Deutschland verzichtet, als sein
Geschäftsmodell in Frage stellen zu lassen. Aber selbst das wäre hoch
riskant für den Verlag: Schließlich handelt es sich um einen großen
Feldversuch zur Frage, ob man ohne Elsevier-Zeitschriften leben kann.

\emph{\textbf{LIBREAS}: Was ist aus Sicht des DEAL-Teams wünschenswert?}

\textbf{BM}: Dass die Einrichtungen die Nerven behalten. Eine derartige
Aktion hat es wohl noch nie gegeben. Sie findet international große
Beachtung, um nicht zu sagen Bewunderung. Und sie hat beste Aussichten
auf Erfolg.

\emph{\textbf{LIBREAS}: Lieber Herr Mittermaier, wir danken Ihnen
herzlich für das Gespräch!}

%autor
\begin{center}\rule{0.5\linewidth}{\linethickness}\end{center}

\textbf{Dr.~Bernhard Mittermaier} hat Chemie (Diplom) und Bibliotheks-
und Informationswissenschaft (M.A.) studiert und in Analytischer Chemie
promoviert. Er leitet die Zentralbibliothek des Forschungszentrums
Jülich und ist Mitglied in den Projektgruppen von DEAL und im Nationalen
Open Access Kontaktpunkt OA2020-DE sowie im Steuerungsgremium der
Allianz-Initiative ``Zukunft der Digitalen Informationsversorgung''.
(ORCiD: \url{https://orcid.org/0000-0002-3412-6168})

\textbf{Michaela Voigt}, Open-Access-Team der TU Berlin, Redakteurin
LIBREAS. Library Ideas. (ORCiD:
\url{https://orcid.org/0000-0001-9486-3189})

\textbf{Maxi Kindling} ist wissenschaftliche Mitarbeiterin am Institut
für Bibliotheks- und Informationswissenschaft (IBI) der HU Berlin. Sie
ist Mitbegründerin und -herausgeberin von LIBREAS. Library Ideas.
(ORCiD: \url{https://orcid.org/0000-0002-0167-0466})

\end{document}
